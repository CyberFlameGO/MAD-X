%%\title{the mad program}
%  Changed by: Chris ISELIN, 27-Jan-1997 
%  Changed by: Oliver Bruning, 19-Jun-2002 
 
\section{Define Variable Parameter}
\label{sec:match_vary}

\subsection{\href{vary}{VARY: Define Variable Parameter}} 

A parameter to be varied is specified by the command 

\begin{verbatim}
VARY,NAME=variable,STEP=real,LOWER=real,UPPER=real;
\end{verbatim}

It has four attributes: 
\begin{itemize}
	\item NAME: The \href{../Introduction/variable.html}{name of the
          parameter or attribute} to be varied,  
	\item STEP: The approximate initial step size for varying the
          parameter. If the step is not entered, MAD tries to find a
          reasonable step, but this may not always work.  
	\item LOWER: Lower limit for the parameter (optional), 
	\item UPPER: Upper limit for the parameter (optional). 
	\item SLOPE: allowed change rate (optional, available only using
          \href{match_xeq.html#jacobian}{JACOBIAN} routine). Limit the
          parameter to increase (SLOPE=1) decrease (SLOPE=-1) only.  
	\item OPT: optimal value for the parameter (optional, available
          only using \href{match_xeq.html#jacobian}{JACOBIAN} routine).  
\end{itemize} 

Examples: 
\begin{verbatim}
VARY,NAME=PAR1,STEP=1.0E-4;                         ! vary global parameter PAR1 
VARY,NAME=QL11->K1,STEP=1.0E-6;                     ! vary attribute K1 of the QL11 
VARY,NAME=Q15->K1,STEP=0.0001,LOWER=0.0,UPPER=0.08; ! vary with limits
\end{verbatim}

If the upper limit is smaller than the lower limit, the two limits are
interchanged. If the current value is outside the range defined by the
limits, it is brought back to range. If a parameter comes outside the
limits during the matching process the matching module resets the
parameter to a value inside the limits and informs the user with a
message. If such a 'rescaling' occurs more than 20 times the matching
process terminates. The user should either eliminate the corresponding
parameters from the list of varied parameters or change the
corresponding upper and lower limits before restarting the matching
process. After a matching operation all varied attributes retain their
value after the last successful matching iteration. Using
\href{match_xeq.html#jacobian}{JACOBIAN} routine, STRATEGY=3, in case
the number of parameters is greater the the number of constraint, if a
parameter comes outside the limits, it is excluded automatically from
the set of variables and a new solution is searched.  

%\href{http://bruening.home.cern.ch/bruening/}{Oliver Br\"uning}, June, 2002. 
%\href{http://rdemaria.home.cern.ch/rdemaria/}{Riccardo de Maria}, February, 2006.
