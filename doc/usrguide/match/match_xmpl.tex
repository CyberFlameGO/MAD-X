%%\title{the mad program}
%  Changed by: Chris ISELIN, 16-Sep-1997 

%  Changed by: Oliver Bruning, 27-Jun-2002 

%  Changed by: Hans Grote, 25-Sep-2002 

%%\usepackage{hyperref}
% commands generated by html2latex


%%\begin{document}
%%\begin{center}
 %%EUROPEAN ORGANIZATION FOR NUCLEAR RESEARCH 
%%\includegraphics{http://cern.ch/madx/icons/mx7_25.gif}

\subsection{Matching Examples}
%%\end{center}

 All matching examples and the related files for executing the MADX sample jobs can be found in the examples directory. 


\begin{itemize}
	\item 

\subsection{\href{cell}{Simple Periodic Cell}} Match a simple cell to given phase advances: 

%\begin{verbatim}
\href{http://cern.ch/madx/madX/examples/match/5cell/job.5cell.madx}{FIVE-CELL}%\end{verbatim}
	\item 

\subsection{\href{cell}{Simple Periodic Cell}} Match the matrix elements of the linear transfer matrix at the end of a sequence 5 periodic cells: 

%\begin{verbatim}
\href{http://cern.ch/madx/madX/examples/match/r-matrix/job.r-matrix.madx}{RMATRIX}%\end{verbatim}
	\item 

\subsection{\href{cell}{Transfer line with initial conditions}} Match a sequence of 5 periodic cells with initial conditions  to given beta-functions at the end of the sequence: 

%\begin{verbatim}
\href{http://cern.ch/madx/madX/examples/match/line/job.line.madx}{Transfer line}%\end{verbatim}
	\item 

\subsection{\href{cell}{Global tune matching in a sequence of 5 periodic cells}} Match the global tune of a sequence of 5 periodic cells: 

%\begin{verbatim}
\href{http://cern.ch/madx/madX/examples/match/global-tune/job.global-tune.madx}{Global tune}%\end{verbatim}
	\item 

\subsection{\href{cell}{Global tune matching for the LHC}} Match the global tune for beam1 of the LHC: 

%\begin{verbatim}
\href{http://cern.ch/madx/madX/examples/match/lhc.tune/job.lhc.tune.madx}{Global tune for the LHC}%\end{verbatim}
	\item 

\subsection{\href{cell}{Global chromaticity matching for the LHC}} Match the global chromaticity for beam1 of the LHC: 

%\begin{verbatim}
\href{http://cern.ch/madx/madX/examples/match/lhc.chromaticity/job.lhc.chromaticity.madx}{Global chromaticity for the LHC}%\end{verbatim}
	\item 

\subsection{\href{cell}{Global chromaticity matching for both beams of the LHC}} Match the global chromaticity for beam1 and beam2 of the LHC: 

%\begin{verbatim}
\href{http://cern.ch/madx/madX/examples/match/lhc.2chromaticity/job.lhc.2chromaticity.madx}{Global chromaticity for both beams of the LHC}%\end{verbatim}
	\item 

\subsection{\href{cell}{IR8 insertion matching for beam1 of the LHC}} Match the insertion IR8 with initial conditions to given values of the optics  functions at the IP and the end of the insertion: 

%\begin{verbatim}
\href{http://cern.ch/madx/madX/examples/match/lhc.insertion/job.lhc.insertion.madx}{IR8 insertion matching} for beam1 of the LHC
%\end{verbatim}
	\item 

\subsection{\href{cell}{IR8 insertion matching for beam1 of the LHC with upper limits on the optics functions}} Match the insertion IR8 with initial conditions to given values of the optics  functions at the IP and the end of the insertion while limiting the maximum acceptable beta functions over the whole insertion: 

%\begin{verbatim}
\href{http://cern.ch/madx/madX/examples/match/lhc.insertion-upper/job.lhc.insertion-upper.madx}{IR8 insertion matching} for beam1 of the LHC with upper limits for all beta functions inside the insertion
%\end{verbatim}
	\item 

\subsection{\href{cell}{Simultaneous orbit matching at IP8 for beam1 and beam2 of the LHC}} Match simultaneously the orbit of beam1 and beam of the LHC at IP8  with initial conditions to the same given values at the IP: 

%\begin{verbatim}
\href{http://cern.ch/madx/madX/examples/match/lhc.iporbit/job.lhc.iporbit.madx}{Orbit matching at IP8} for beam1 and beam2 of the LHC
%\end{verbatim}
	\item 

\subsection{\href{cell}{IR8 beta squeeze for beam1 using JACOBIAN matching routine}} Try to find a beta squeeze for IR8 starting from 10 meters. 

%\begin{verbatim}
\href{http://cern.ch/madx/madX/examples/match/lhcV65.ir8squeeze/job.lhcV65.ir8squeeze.madx}{Beta squeeze for IR8}%\end{verbatim}
	\item 

\subsection{\href{cell}{Mathching first and second order chromaticity of the LHC using USE\_MACRO option.}} Match simultaneously the first and second order chromaticity by defining macros which compute them using the TWISS command or PTC. 

%\begin{verbatim}
\href{http://cern.ch/madx/madX/examples/match/lhc.qpp/job.lhc.qpp.madx}{Second order chromaticity}%\end{verbatim}
	\item 

\subsection{\href{cell}{Mathching s position using VLENGTH flag.}} match the positions of elements and the total sequence length for a simple sample sequence. 

%\begin{verbatim}
\href{http://cern.ch/madx/madX/examples/match/s-match/job.s-match.madx}{s position matching}%\end{verbatim}
	\item 

\subsection{\href{cell}{Mathching s position using USE\_MACRO.}} match the positions of elements and the total sequence length for a simple sample sequence using USE\_MACRO. 

%\begin{verbatim}
\href{http://cern.ch/madx/madX/examples/match/s-match-usemacro/job.s-match-usemacro.madx}{s position matching}%\end{verbatim}
\end{itemize}\href{http://bruening.home.cern.ch/bruening/}{Oliver Br\"uning}, June, 2002; \href{http://rdemaria.home.cern.ch/rdemaria/}{Riccardo de Maria}, August, 2007. 

%%\end{document}
