%%\title{Wildcards}
%  Changed by: Chris ISELIN, 27-Mar-1997 
%  Changed by: Hans Grote, 31-Jul-2002 

\section{Regular Expressions}

Some commands allow selection of items via "regular expression"
strings. Such a pattern string \textbf{must} be enclosed in single or
double quotes. MAD-X follows regexp (Unix regular expression patterns)
for matching. The following features are implemented:  

A "search string" below is the string containing the pattern, a "target
string" is the string being searched for a possible match with the
pattern. 

\begin{itemize}
   \item "\textasciicircum" at the start of the search string: Match
     following search string at the start of the target string;
     otherwise the search string can start anywhere in the target
     string. To search for a  genuine "\textasciicircum" anywhere, use
     "$\backslash$\textasciicircum".  
   \item "\$" at the end of the search string: Match preceding search
     string at the end of the target string; otherwise the search string
     can end anywhere in the target string. To search for a  genuine
     "\$" anywhere, use "$\backslash$\$".  
   \item ".": Stands for an arbitrary character; to search for a genuine
     ".", use "$\backslash$." 
   \item "[xyz]": Stands for one character belonging to the string
     contained in brackets (example: "[abc]" means one of a, b, c).  
   \item "[a-ex-z]": Stands for ranges of characters (example:
     "[a-zA-Z]" means any letter).  
   \item "[\textasciicircum xyz]" (i.e. a "\textasciicircum" as first
     character in a square bracket): Stands for exclusion of all
     characters in the list, i.e. "[\textasciicircum a-z]" means "any
     character but a lower case letter". 
   \item "*": Allows zero or more repetitions of the preceding
     character, either specified directly, or from a list. (examples:
     "a*" means zero or more occurrences of "a",  "[A-Z]*" means zero or
     more upper-case letters).  
   \item "backslash-c" (e.g. "$\backslash$."): Removes the special
     meaning of character c.  
\end{itemize} 
All other characters stand for themselves. 


Example: 
\begin{verbatim}
select, flag=twiss, pattern="^d..$" ;
select, flag=twiss, pattern="^k.*qd.*\.r1$" ;
\end{verbatim}
 
The first command selects all elements whose names have exactly three
characters and begin with the letter "D". The second command selects
elements beginning with the letter "K", containing the string "QD", and
ending with the string ".R1". The two occurrences of ".*" each stand for
an arbitrary number (including zero) of any character, and the
occurrence "$\backslash$." stands for a literal period.  

%\href{http://www.cern.ch/Hans.Grote/hansg_sign.html}{hansg}, May 8, 2001 

