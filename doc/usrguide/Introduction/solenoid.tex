%\documentclass[a4paper,11pt]{article}
%%\usepackage{ulem}
%%\usepackage{a4wide}
%%\usepackage[dvipsnames,svgnames]{xcolor}
%%\usepackage[pdftex]{graphicx}
%%\title{SOLENOID}
%  Changed by: Chris ISELIN, 27-Jan-1997 

%  Changed by: Hans Grote, 30-Sep-2002 

%  Changed by: Alexander Koschik, 16-May-2006 

%%\usepackage{hyperref}
% commands generated by html2latex


%%\begin{document}
%%\begin{center}
 %%EUROPEAN ORGANIZATION FOR NUCLEAR RESEARCH 
%%\includegraphics{http://cern.ch/madx/icons/mx7_25.gif}

\subsection{Solenoid}
%%\end{center}

\texttt{label: SOLENOID, L=real, KS=real;           } (\textbf{thick} version) 
\\\texttt{label: SOLENOID, L=0,    KS=real, KSI=real; } (\textbf{thin} version) 

  A SOLENOID has two (three) real attributes: 
\begin{itemize}
	\item L: The length of the solenoid (default: 0 m) 
	\item KS: The solenoid strength \textit{K$_s$} (default: 0 rad/m). For positive KS and positive particle charge, the solenoid field points in the direction of increasing \textit{s}. 
	\item KSI: The solenoid integrated strength \textit{K$_s$*L} (default: 0 rad).  This additional attribute is needed only when using the thin solenoid,  where \textit{L=0}!    
	\item \textit{ KNL \& KSL:  Take note that one can specify multipole coefficients but they have no effect in MAD-X proper but are used for solenoids with multipoles in PTC.}
\end{itemize}

 Example: 
\begin{verbatim}

SOLO: SOLENOID, L=2., KS=0.001;
THINSOLO: SOLENOID, L=0, KS=0.001, KSI=0.002;
\end{verbatim}

 The \href{local_system.html#straight}{straight reference system} for a solenoid is a cartesian coordinate system. 

\href{http://www.cern.ch/Hans.Grote/hansg_sign.html}{hansg}, January 27, 1977 

%%\end{document}
