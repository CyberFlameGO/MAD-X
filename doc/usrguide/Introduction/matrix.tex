%\documentclass[a4paper,11pt]{article}
%%\usepackage{ulem}
%%\usepackage{a4wide}
%%\usepackage[dvipsnames,svgnames]{xcolor}
%%\usepackage[pdftex]{graphicx}
%%\title{MATRIX}
%  Changed by: Frank Schmidt, 25-June-2003 

%%\usepackage{hyperref}
% commands generated by html2latex


%%\begin{document}

\section{MATRIX: Arbitrary Element}
\begin{verbatim}
label: MATRIX,TYPE=name,L=real,KICK1=real,...,KICK6=real,
               RM11=real,...,RM66=real,
               TM111=real,...,TM666=real;
\end{verbatim} The MATRIX permits the definition of an arbitrary transfer matrix. It has four real array attributes: 
\begin{itemize}
	\item L: Length of the element, which may be zero. 
	\item KICKi: Defines the kick of the element acting on the six phase space coordinates. 
	\item RMik: Defines the linear transfer matrix (6*6) of the element. 
	\item TMikl: Defines the second-order terms (6*6*6) of the element. 
\end{itemize} Data values not entered are taken from the identity transformation, kick and second order terms are zero as default. In the thin-lens tracking module the length of an arbitrary matrix is accepted, however no second order are allowed to avoid non symplectic tracking runs. In the latter case the tracking run will be aborted.
\\
   \href{http://www.cern.ch/Frank/Schmidt/frs_sign.html}{frs}, June 25, 2003 

%%\end{document}
