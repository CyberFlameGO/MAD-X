%%\title{the mad program}
%  Changed by: Chris ISELIN, 27-Jan-1997 

%  Changed by: Hans Grote, 30-Sep-2002 

%  Changed by: Frank Schmidt, 28-Aug-2003 

%%\usepackage{hyperref}
% commands generated by html2latex


%%\begin{document}
%%\begin{center}
 %%EUROPEAN ORGANIZATION FOR NUCLEAR RESEARCH 
%%\includegraphics{http://cern.ch/madx/icons/mx7_25.gif}

\subsection{Octupole}
%%\end{center}
\begin{verbatim}

label: OCTUPOLE,L=real,K3=real,K3S=real,TILT=real;
\end{verbatim} An OCTUPOLE has four real attributes: 
\begin{itemize}
	\item L: The octupole length (default: 0 m). 
	\item K3: The normal octupole coefficient 

\textit{K}$_3$ = 1/(\textit{B} rho) ($\partial$$^3$\textit{B$_y$}/$\partial$\textit{x}$^3$). 

 (default: 0 m**(-4)). 
	\item K3S: The skew octupole coefficient 

\textit{K}$_3S$ = 1/(2 \textit{B} rho) ($\partial$$^3$\textit{B$_x$}/$\partial$\textit{x}$^3$ - $\partial$$^3$\textit{B$_y$}/$\partial$\textit{y}$^3$). 

 where (x,y) is now a coordinate system rotated by -22.5$^o$ around s with respect to the normal one. (default: 0 m**(-4)). A positive skew octupole strength implies defocussing (!) of positively charged particles in the (x,s) plane rotated by 22.5$^o$ around s (particles in this plane have x $>$ 0, y $>$ 0). 


	\item TILT: The roll angle about the longitudinal axis (default: 0 rad, i.e. a normal octupole). A positive angle represents a clockwise rotation. A TILT=pi/8 turns a positive normal octupole into a negative skew octupole. 

\textbf{  Please note that contrary to MAD8 one has to specify the desired TILT angle, otherwise it is taken as 0 rad. This was needed to avoid the confusion in MAD8 about the actual meaning of the TILT attribute for various elements. }
\end{itemize}

\textbf{  Note also that K$_3$/K$_3s$ can be considered as the normal or skew quadrupole components of the magnet on the bench, while the TILT attribute can be considered as an tilt alignment error in the machine. In fact, a positive K$_3$ with a tilt=0 is equivalent to a positive K$_3s$ with positive tilt=+pi/8.  }

Example: 
\begin{verbatim}

O3: OCTUPOLE,L=0.3,K3=0.543;
\end{verbatim} The \href{local_system.html#straight}{straight reference system} for a octupole is a cartesian coordinate system. Octupoles are normally treated as thin lenses, except when tracking by Lie-algebraic methods.  

\href{http://www.cern.ch/Hans.Grote/hansg_sign.html}{hansg}, \href{http://www.cern.ch/Frank.Schmidt/frs_sign.html}{frs}, August 28, 2003  

%%\end{document}
