%%\title{Logical Attributes}
%  Changed by: Chris ISELIN, 27-Jan-1997 

%  Changed by: Hans Grote, 10-Jun-2002 

%%\usepackage{hyperref}
% commands generated by html2latex


%%\begin{document}
%%\begin{center}
 %%EUROPEAN ORGANIZATION FOR NUCLEAR RESEARCH 
%%\includegraphics{http://cern.ch/madx/icons/mx7_25.gif}

\subsection{Logical Attributes}
%%\end{center} 
 Many commands in MAD require the setting of logical values (flags) to represent the on/off state of an option. A logical value "flag" can be set in two ways: 
\begin{verbatim}

flag | flag = true
\end{verbatim} It can be reset by: 
\begin{verbatim}

-flag | flag=false
\end{verbatim} Example: 
\begin{verbatim}

option,-echo;  // switch off copying the input to the standard output
\end{verbatim} The default for a logical flag is normally false, but can be found e.g. for options by the command  
\begin{verbatim}

help,option;
\end{verbatim}\href{http://www.cern.ch/Hans.Grote/hansg_sign.html}{hansg}, May 8, 2001 

%%\end{document}
