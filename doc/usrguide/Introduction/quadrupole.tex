%%\title{QUADRUPOLE}
%  Changed by: Chris ISELIN, 27-Jan-1997 
%  Changed by: Hans Grote, 30-Sep-2002 
%  Changed by: Frank Schmidt, 28-Aug-2003 
%  Changed by: Andrea Latina, 6-May-2013 

\section{Quadrupole}

\begin{verbatim}
label: QUADRUPOLE, L = real, K1 = real, K1S = real, TILT = real;
\end{verbatim}    

A QUADRUPOLE has four real attributes:     
\begin{itemize}
   \item L: The quadrupole length (default: 0 m). 
   \item K1: The normal quadrupole coefficient \\        
     \textit{K}$_1$ = 1/(\textit{B} rho) ($\partial$\textit{B$_y$}/$\partial$\textit{x}).\\ 
     The default is 0 m**(-2). A positive normal quadrupole strength
     implies horizontal focussing of positively charged particles.  
   \item K1S: The skew quadrupole coefficient \\        
     \textit{K}$_{1s}$ = 1/(2 \textit{B} rho)
     ($\partial$\textit{B$_x$}/$\partial$\textit{x} -
     $\partial$\textit{B$_y$}/$\partial$\textit{y})\\  
     where (x,y) is now a coordinate system rotated by -45$^o$ around s
     with respect to the normal one. The default is 0  m**(-2). A
     positive skew quadrupole strength implies defocussing (!) of
     positively charged particles in the (x,s) plane rotated by 45$^o$
     around s (particles in this plane have x = y $>$ 0). 
   \item TILT: The roll angle about the longitudinal axis (default: 0
     rad, i.e. a normal quadrupole). A positive angle represents a
     clockwise rotation. A TILT=pi/4 turns a positive normal quadrupole
     into a negative skew quadrupole.          

\textbf{ Please note that contrary to MAD8 one has to
  specify the desired TILT angle, otherwise it is taken as
  0 rad. This was needed to avoid the confusion in MAD8
  about the actual meaning of the TILT attribute for
  various elements. } 

    \item THICK: If this flag is set to 1 the quadrupole will be tracked
      through as a thick-element, instead of being converted into
      thin-lenses.  
\end{itemize}

\textbf{ Note also that K$_1$/K$_{1s}$ can be considered as
  the normal or skew quadrupole components of the magnet on
  the bench, while the TILT attribute can be considered as an
  tilt alignment error in the machine. In fact, a positive
  K$_1$ with a tilt=0 is equivalent to a positive K$_{1s}$
  with positive tilt=+pi/4. } 

Example: 
\begin{verbatim}
QF: QUADRUPOLE, L = 1.5, K1 = 0.001, THICK = 1;
\end{verbatim}     

The \href{local_system.html#straight}{straight reference system} for
a quadrupole is a cartesian coordinate system.

%\href{http://www.cern.ch/Hans.Grote/hansg_sign.html}{hansg},
%\href{http://www.cern.ch/Frank.Schmidt/frs_sign.html}{frs},
%\href{https://phonebook.cern.ch/phonebook/?id=PE525753}{al},       May 6, 2013 
