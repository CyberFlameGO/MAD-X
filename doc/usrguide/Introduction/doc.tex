%%\title{The MAD Program}
%  Changed by: Chris ISELIN, 17-Jul-1997 

%  Changed by: Hans Grote, 10-Jun-2002 

%%\usepackage{hyperref}
% commands generated by html2latex


%%\begin{document}
\begin{center}
 %%EUROPEAN ORGANIZATION FOR NUCLEAR RESEARCH 
%%\includegraphics{http://cern.ch/madx/icons/mx7_25.gif}

\section{The MAD-X Program, Version 2.11}  
Hans Grote and Frank Schmidt 
\end{center}

\line(1,0){300}
\\
 MAD is a tool for charged-particle optics in alternating-gradient accelerators and beam lines. It can handle very large and very small accelerators and solves various problems on such machines. 

\textbf{{ This new version supersedes MAD-8 which has been  frozen.}}

 The input format has been modified slightly, mainly in order to make it safer, and to allow the introduction of new features (WHILE loop, MACROs, and others). 

 The MAD-X framework should make it easy to add new features in the form of program modules. A \href{module_doc.html}{ Module Writer's Guide} is available and contains guidelines and examples. The authors of MAD hope that such modules will also be contributed and documented by others. As required, the table of contents and the indices will be revised. The contributions of other authors are acknowledged in the relevant chapters. 

 Last but not least it is only fair to mention that a large part of the documentation provided here is based on the MAD-8 documentation that was originally written and published by F. C. Iselin. 

 Comments from readers are most welcome. They may be sent to the following internet addresses: 
\begin{verbatim}
   Frank.Schmidt@cern.ch
\end{verbatim}

\line(1,0){300}


\subsection{\href{contents-X}{MAD-X Contents:}}
\begin{itemize}
	\item \href{conventions.html}{Conventions}
	\item \href{format.html}{Command and Statement Format}


	\item \href{tfs.html}{TFS File Format}


	\item \href{bibliography.html}{References}
\end{itemize}

\line(1,0){300}


%%\end{document}
