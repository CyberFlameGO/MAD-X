%%\title{Sign Conventions}
%  Changed by: Chris ISELIN, 17-Jul-1997 
%  Changed by: Hans Grote, 10-Jun-2002 

\section{Sign Conventions for Magnetic Fields}

The MAD program uses the following Taylor expansion for the field on the
mid-plane \textit{y}=0, described in
\href{bibliography.html#slac75}{SLAC-75}:  


%%\includegraphics{null}
%Taylor_field
$$
B_y(x,0)=\sum_{n=0}^{\infty} \frac{B_n\,x^n}{n!}
$$

Note the factorial in the denominator. The field coefficients have the following meaning: 
\begin{itemize}
   \item \textit{B}$_0$: Dipole field, with a positive value in the
     positive \textit{y} direction; a positive field bends a positively
     charged particle to the right.  
   \item \textit{B}$_1$: Quadrupole coefficient \\
     \textit{B}$_1$ = (del \textit{B$_y$} / del \textit{x});\\ 
     a positive value corresponds to horizontal focussing of a
     positively charged particle. 
   \item \textit{B}$_2$: Sextupole coefficient \\
     \textit{B}$_2$ =  (del$^2$\textit{B$_y$} / del \textit{x}$^2$). 
   \item \textit{B}$_3$: Octupole coefficient \\ 
     \textit{B}$_3$ =  (del$^3$\textit{B$_y$} / del \textit{x}$^3$). 
\end{itemize} 

Using this expansion and the curvature \textit{h} of the reference orbit, the longitudinal component of the vector potential to order 4 is: 
%%\includegraphics{null}
%Taylor_A_s
% this is problematic, {align}, {eqnarray} do not work 
% EDIT : Laurent fixed that.
\[
\begin{aligned}
A_x =
&+ B_0\,\Big(x-\frac{hx^2}{2(1+hx)}\Big)&
&+ B_1\,\Big(\frac{1}{2}(x^2-y^2) - \frac{h}{6}x^3 + \frac{h^2}{24}(4x^4-y^4)+\cdots\Big) \\
&+ B_2\,\Big(\frac{1}{6}(x^3-3xy^2) - \frac{h}{24}(x^4-y^4)+\cdots \Big)&
&+ B_3\,\Big(\frac{1}{24}(x^4-6x^2y^2+y^4) \cdots \Big)+\cdots
\end{aligned}
\]

Taking curl \textit{A} in curvilinear coordinates, the field components can be computed as 

%%\includegraphics{null}
%Taylor_B
\[
\begin{aligned}
B_x(x,y) =
&+ B_1\,\Big(y+\frac{h^2}{6}y^3+\cdots\Big)&  &  \\
&+ B_2\,\Big(xy - \frac{h^3}{6}y^3+\cdots \Big)&+B_3\,\Big(\frac{1}{6}(3x^2y-y^3)+ \cdots \Big)+\cdots\\
B_y(x,y)=
&+ B_0 & + B_1\,\Big(x-\frac{h}{2}y^2+\frac{h^2}{2}xy^2+\cdots \Big)\\
&+ B_2\,\Big(\frac{1}{2}(x^2-y^2)-\frac{h}{2}xy^2+\cdots \Big) & + B_3\,\Big(\frac{1}{6}(x^3-3xy^2)+ \cdots \Big)+\cdots
\end{aligned}
\]

It can be easily verified that both curl \textit{B} and div \textit{B}
are zero to the order of the \textit{B}$_3$ term. Introducing the
magnetic rigidity \textit{B}rho, the multipole coefficients are computed
as  

\textit{K$_n$} = \textit{e B$_n$ / p$_s$} =  \textit{B$_n$ / B} rho. 


%\href{http://www.cern.ch/Hans.Grote/hansg_sign.html}{hansg}, June 17, 2002 

