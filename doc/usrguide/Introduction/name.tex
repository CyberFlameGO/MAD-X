%%\title{ name or string attributes}
%  Changed by: Chris ISELIN, 27-Jan-1997 
%  Changed by: Hans Grote, 10-Jun-2002 

\section{Name or String Attributes}

A name or string attribute often selects one of a set of options: 
\begin{verbatim}
use, period=lhc;    // expand the LHC sequence
\end{verbatim} 

It may also refer to a user-defined object: 
\begin{verbatim}
twiss, file=optics;    // specifies the name of the OPTICS output file
\end{verbatim} 

It may also define a string: 
\begin{verbatim}
title, "LHC version 6.2";
\end{verbatim} 

The case of letters is only significant if a string is enclosed in
quotes, otherwise all characters are converted to lower at reading. On
the other hand, strings that do not contain blanks do not need to be
enclosed in quotes. 

Example:
\begin{verbatim}
call, file="my.file";
call, file=my.file;
call, file=MY.FILE;
call, file="MY.FILE";
call, file='MY.FILE';
\end{verbatim} 
In the first three cases, MAD-X will try to read a file my.file, in the
last two it will try to read MY.FILE.  

%\href{http://www.cern.ch/Hans.Grote/hansg_sign.html}{hansg}, May 8, 2001 
