%%\title{MULTIPOLE}
%  Changed by: Chris ISELIN, 27-Jan-1997 
%  Changed by: Hans Grote, 17-Jun-2002 
%  Changed by: Frank Schmidt, 28-Aug-2003 

\section{RFMULTIPOLE: Thin Radio-Frequency Multipole}
\begin{verbatim}
Label: RFMULTIPOLE, VOLT = real, LAG = real,
       HARMON = integer, FREQ = real,
       LRAD = real, TILT = real,
       KNL := {k0nl, k1nl, k2nl, ... },  ! Normal coefficients
       KSL := {k0sl, k1sl, k2sl, ... },  ! Skew coefficients
       PNL := {p0n,  p1n,  p2n,  ... },  ! Normal phases [2pi]
       PSL := {p0s,  p1s,  p2s,  ... };  ! Skew phases [2pi]
\end{verbatim} 

A RFMULTIPOLE is a thin-lens element which exhibits the properties
of an RF-cavity and of a magnet of arbitrary order oscillating the
a certain frequency:        
 
\begin{itemize}
   \item L: The length of the rfmultipole (DEFAULT: 0 m) 
   \item LAG: The phase lag [2pi] (DEFAULT: 0) 
   \item FREQ: The frequency [MHz] (no DEFAULT). Note that if the RF
     frequency is not given, it is computed from the harmonic
     number and the revolution frequency f0 as before. However, for
     accelerating structures this makes no sense, and the frequency
     is mandatory.  
   \item HARMON: The harmonic number h (no DEFAULT). Only if the
     frequency is not given. 
   \item LRAD: A fictitious length, which was originally just used to
     compute synchrotron radiation effects. A non-zero LRAD in
     conjunction with the OPTION thin\_foc set to a true logical value
     takes into account of the weak focussing of bending magnets.  
   \item TILT: The roll angle about the longitudinal axis (default: 0
     rad). A positive angle represents a clockwise rotation of the
     multipole element.            

     \textbf{Please note that contrary to MAD8 one has to specify the
       desired TILT angle, otherwise it is taken as 0 rad. We
       believe that the MAD8 concept of having individual TILT
       values for each component and on top with default values
       led to considerable confusion and allowed for excessive
       and unphysical freedom. Instead, in MAD-X the KNL/KSL
       components can be considered as the normal or skew
       multipole components of the magnet on the bench, while the
       TILT attribute can be considered as an tilt alignment
       error in the machine.} 

   \item KNL: The normal rfmultipole coefficients from order zero to
     the maximum; the parameters are positional, therefore zeros
     must be filled in for components that do not exist. Example of
     a thin-lens sextupole: ms:rfmultipole, knl:=\{0, 0, k2l\}; 
   \item KSL: The skew rfmultipole coefficients from order zero to
     the maximum; the parameters are positional, therefore zeros
     must be filled in for components that do not exist. Example of
     a thin-lens skew octupole:
\end{itemize}

\begin{verbatim}
ms: rfmultipole, ksl := {0, 0, 0, k3sl};
\end{verbatim}

Both KNL and KSL may be specified for the same multipole.  


\begin{itemize}
   \item VOLT: The peak RF voltage (DEFAULT: 0 MV). The effect of the
     cavity is \\ 
     delta(E) = VOLT * sin(2 pi * (LAG - HARMON * f0 t)). 
   \item PNL: The phase for each normal rfmultipole coefficients from
     order zero to the maximum; the parameters are positional,
     therefore zeros must be filled in for components that do not
     exist.  
   \item PSL: The phase for each skew rfmultipole coefficients from
     order zero to the maximum; the parameters are positional,
     therefore zeros must be filled in for components that do not
     exist.  
\end{itemize}       

A rfmultipole requires the particle energy (ENERGY) and the
particle charge (CHARGE) to be set by a BEAM command before any
calculations are performed. Notice that, contrary to the regular
multipole where the dipole component has no effect on the
reference orbit, an RF-Multipole that includes a dipole component
bends also the reference orbit.      


%\href{https://phonebook.cern.ch/foundpub/Phonebook/index.html?search=latina#id=PE525753}{Andrea
%Latina}, September 28, 2012  
