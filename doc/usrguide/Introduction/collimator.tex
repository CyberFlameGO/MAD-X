%%\title{[RE]COLLIMATOR}
%  Changed by: Chris ISELIN,  3-Mar-1998 
%  Changed by: Hans Grote, 30-Sep-2002 
%  Changed by: Ghislain Roy, 23-May-2013 

\section{Collimators}
 
The definition of collimators in MAD-X has been inherited from MAD8 and two types of collimators are defined: 
\begin{itemize}
   \item \href{ecol}{ECOLLIMATOR}. 
   \item \href{rcol}{RCOLLIMATOR}. 
\end{itemize}  
Note that although two types are defined, both behave in exactly the
same way within MAD-X and the type name does NOT imply elliptic or
rectangular aperture.    

\begin{verbatim}
label: ECOLLIMATOR, TYPE = name, L = real, XSIZE = real, YSIZE = real;
label: RCOLLIMATOR, TYPE = name ,L = real, XSIZE = real, YSIZE = real;
\end{verbatim}  

Either type has several real attributes: 
\begin{itemize}
   \item L: The collimator length (default: 0 m). 
   \item XSIZE: The horizontal half-aperture (default:
     unlimited). \textbf{OBSOLETE : Will be parsed but not used.} 
   \item YSIZE: The vertical half-aperture (default:
     unlimited). \textbf{OBSOLETE : Will be parsed but not used.} 
\end{itemize}  

The actual definition of the aperture type and aperture parameters must
be done in accordance with \href{aperture.html}{Defining aperture in
  MAD-X}.   

Optically a collimator behaves like a drift space.  However during
tracking the aperture is checked at the entrance of the collimator,
provided that the aperture type is one of the predefined RECTANGLE,
ELLIPSE, RECTELLIPSE or LHCSCREEN types.  In particular an aperture
model defined in an external file will not be used during tracking.  

If the length is not zero, the aperture is also checked at the
exit. \textbf{ TO BE CHECKED } 

Example: 
\begin{verbatim}
COLLIM: ECOLLIMATOR, L = 0.5, APERTYPE = ELLIPSE, APERTURE = {0.01,0.005};
\end{verbatim}

The \href{local_system.html#straight}{straight reference system} for a
collimator is a cartesian coordinate system.  

\textbf{NOTE:} When a collimator is displaced transversally in order to
model  an asymmetric collimator, particle losses in tracking are
reported with respect to the \textbf{displaced} reference system, not
with respect to the surrounding beam line.  


%  <a href="http://www.cern.ch/Hans.Grote/hansg_sign.html">hansg</a>, January 24, 1997 
%  Ghislain Roy, May 23, 2013 
