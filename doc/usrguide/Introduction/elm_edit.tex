%%\title{Editing Element Definitions}
%  Changed by: Chris ISELIN, 24-Jan-1997 
%  Changed by: Hans Grote, 25-Sep-2002

\section{Editing Element Definitions}  

An element definition can be changed in two ways: 
\begin{itemize}
   \item \textbf{Entering a new definition:} The element will be
     replaced in the main beam line expansion.  
   \item \textbf{Entering the element name together with new
     attributes:} The element will be updated in place, and the new
     attribute values will replace the old ones.  
\end{itemize} 

This example shows two ways to change the strength of a quadrupole: 
\begin{verbatim}
QF: QUADRUPOLE, L = 1, K1 = 0.01;     ! Original definition of QF
QF: QUADRUPOLE, L = 1, K1 = 0.02;     ! Replace whole definition of QF
QF, K1 = 0.02;                        ! Replace value of K1
\end{verbatim} 

When the type of the element remains the same, replacement of an
attribute is the more efficient way.  

Element definitions can be edited freely. The changes do not affect
already defined objects which belong to its
\href{elm_class.html}{element class}.  

%\href{http://www.cern.ch/Hans.Grote/hansg_sign.html}{hansg}, January 24, 1997 
