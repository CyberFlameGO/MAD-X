%%\title{DRIFT}
%  Changed by: Chris ISELIN, 24-Jan-1997 

%  Changed by: Hans Grote, 30-Sep-2002 

%%\usepackage{hyperref}
% commands generated by html2latex


%%\begin{document}
%%\begin{center}
 %%EUROPEAN ORGANIZATION FOR NUCLEAR RESEARCH 
%%\includegraphics{http://cern.ch/madx/icons/mx7_25.gif}

\subsection{Drift Space}
%%\end{center}
\begin{verbatim}

label: DRIFT,L=real;
\end{verbatim} A DRIFT space has one real attribute: 
\begin{itemize}
	\item L: The drift length (default: 0 m) 
\end{itemize} Examples: 
\begin{verbatim}

DR1:   DRIFT,L=1.5;
DR2:   DRIFT,L=DR1[L];
\end{verbatim} The length of DR2 will always be equal to the length of DR1. The \href{../Introduction/local_system.html#straight}{straight  reference system} for a drift space is a cartesian coordinate system. 

\href{http://www.cern.ch/Hans.Grote/hansg_sign.html}{hansg}, January 24, 1997 

%%\end{document}
