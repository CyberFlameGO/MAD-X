%%\title{SELECT}
%  Changed by: Hans Grote, 11-Sep-2002 

\subsection{Sectormap output}
\label{subsec:sectormap}
% Begin New version: Jean-Luc Nougaret, 18-Dec-2008 


The flag "sectormap" on the Twiss command (together with an element
selection via select,flag=sectormap,...) causes a file "sectormap" to be
written.   

For each user-selected element, it contains the user-selected coefficients of the kick vector 
\texttt{K} (6 values), of the first-order map 
\texttt{R} (6 x 6 values) and of the second-order map 
\texttt{T} (6 x 6 x 6 values)

The sector file is the output of a standard TFS table, which means that
the set of columns of interest may be selected through a SELECT command
as in the following example:  


\begin{verbatim}
select, flag=my_sect_table, column=name, pos, k1, r11, r66, t111;
\end{verbatim}


The sectormap file contains for each selected element, one element per line, the
set of chosen K, R, and T matrix coefficients: 
\\
\\
\begin{tabular}{l|l|l}
@ NAME &              \%13s &  "MY\_SECT\_TABLE" \\ 
@ TYPE &              \%09s &  "SECTORMAP" \\ 
@ TITLE &             \%08s &  "no-title" \\ 
@ ORIGIN &           \%19s &  "MAD-X 3.04.62 Linux" \\ 
@ DATE &              \%08s &  "18/12/08" \\ 
@ TIME &              \%08s &  "10.33.58"
\end{tabular}
\\
\\
\begin{tabular}{l | l | l | l | l | l }
* NAME & POS & K1 & R11 & R66 & T111 \\ 
\$ \%s & \%le & \%le & \%le  & \%le & \%le \\ 
 "FIVECELL\$START"  & 0 & 0 & 1 & 1 & 0 \\ 
 "SEQSTART"  & 0 &  0  &  1 &  1  &  0 \\ 
 "QF.1"  & 3.1 & -1.305314637e-05 & 1.042224745 & 1 & 0 \\ 
 "DRIFT\_0" & 3.265 & 7.451656548e-21 & 1 & 1 & 0 \\ 
 "MSCBH" & 4.365 & -1.686090613e-15 & 0.9999972755 & 1 & 0.006004411526 \\ 
 "CBH.1" & 4.365 & 0 & 1 & 1 & 0 \\ 
 "DRIFT\_1" & 5.519992305 & -6.675347543e-21 & 1 & 1 & 0 \\ 
 "MB" & 19.72000769 & 2.566889547e-18 & 1.000000091 & 1 & -4.135903063e-25 \\ 
 "DRIFT\_2" & 21.17999231 & -1.757758802e-20 & 1 & 1 & 0 \\ 
 "MB" & 35.38000769 & 2.822705549e-18 & 1.000000091 & 1 & -4.135903063e-25 \\ 
 "DRIFT\_2" & 36.83999231 & 2.480880093e-20 & 1 & 1 & 0 \\ 
 "MB" & 51.04000769 & 3.006954115e-18 & 1.000000091 & 1 & -4.135903063e-25 \\ 
 "DRIFT\_3" & 52.21 & -4.886652187e-20 & 1 & 1 & 0 \\ 
... & ... & ... & ... & ... & ... \\ 
... & ... & ... & ... & ... & ... \\ 
... & ... & ... & ... & ... & ...
\end{tabular}
\\
\\ 
Of course, the \texttt{select} statement can be combined with additional
options to filter-out the list of elements, such as in the following
statement, which for instance only retains drift-type elements:  

\begin{verbatim}
select, flag=my_sect_table, class=drift, column=name, pos, k1, r11, r66, t111;
\end{verbatim}



\texttt{K} coefficients range: 
\texttt{K1}... 
\texttt{K6}


\texttt{R} coefficients range: 
\\
\begin{tabular}{ccc}
\texttt{R11} & ... & \texttt{R61} \\ 
\texttt{R12} & ... & \texttt{R62} \\ 
... & ... & ... \\ 
\texttt{R61} & ... & \texttt{R66}
\end{tabular}


\texttt{T} coefficients range: 
\\
\begin{tabular}{ccc}
\texttt{T111} & ... &\texttt{T611} \\ 
\texttt{T121} & ... & \texttt{T621} \\ 
... & ... & ... \\ 
\texttt{T161} & ... & \texttt{T661} \\ 
\texttt{T112} & ... & \texttt{T612} \\ 
... & ... & ... \\ 
\texttt{T166} & ... & \texttt{T666}
\end{tabular}

 In the above notation, 
\texttt{Rij} stands for "effect on the 
\texttt{i}-th coordinate of the 
\texttt{j}-th coordinate in phase-space", whereas 
\texttt{Tijk} stands for "combined effect on the 
\texttt{i}-th coordinate of both the 
\texttt{j}-th and 
\texttt{k}-th coordinates in phase-space" along the lattice. 
% End New Version 

%  Commented by jluc, on 18 December 2008
% The flag "sectormap" on the Twiss command (together with an element
% selection via select,flag=sectormap,...) causes a file "sectormap" to
% be written. This is a fixed format file; per selected element it
% contains:
% 
% <pre>
% end_position   element_name
% kick vector (6 values)
% first order map (6 lines with 6 values each), column-wise
% second order map (36 lines with 6 columns each, column-column-wise)
% </pre>
% 
% Or more explicitly:
% 
% <pre>
% The first line is:
% K[1] ... K[6]
% 
% Then: 
% R[1,1] ... R[6,1]
% R[1,2] ... R[6,2]
% ...
% R[1,6] ... R[6,6]
% 
% 
% Then:
% T[1,1,1] ... T[6,1,1]
% T[1,2,1] ... T[6,2,1]
% ...
% T[1,6,1] ... T[6,6,1]
% T[1,1,2] ... T[6,1,2]
% ...
% T[1,6,6] ... T[6,6,6]
% </pre>
% 
   
The maps are the accumulated maps between the selected elements. They
contain both the alignment, and field errors present. Together with the
starting value of the closed orbit (which can be obtained from the
standard twiss file) this allows the user to track particles over larger
sectors, rather than element per element. A typical usage therefore lies
in the interface to other programs, such as TRAIN.  

%\href{http://www.cern.ch/Hans.Grote/hansg_sign.html}{hansg}, May 8, 2001 

