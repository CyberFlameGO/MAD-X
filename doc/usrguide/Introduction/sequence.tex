%%\title{SEQUENCE}
%  Changed by: Chris ISELIN, 27-Jan-1997 
%  Changed by: Hans Grote, 30-Sep-2002 

\chapter{Beam Line Sequences}
\label{chap:sequence}
 
MAD-X accepts two forms of an accelerator definition: sequences and
\href{line.html}{lines}. However, the sequence definition is the only
one used internally; lines are converted into sequences when they are
USEd. Consequently, only sequences can be saved (written onto a file)
by MAD-X.  

The corresponding sequence of statements defining a sequence is 
\begin{verbatim}
name: SEQUENCE, REFER=keyword, REFPOS=name, L=real, AT=real, FROM=string
                ADD_PASS=integer,
                NEXT_SEQU='seq_name';
...
label: class, AT=real{,attributes}; 
class, AT=real;
sequ_name, AT=real;
...
ENDSEQUENCE;
\end{verbatim} 
where "real" means a real number, variable, or expression. 

The first line declares a sequence with the given name and has several
optional arguments:
\begin{itemize}
   \item {\bf REFER}=flag in \{entry, centre, exit\} (Default: centre) \\
     The flag specifies at which part of the element its position along
     the beam line will be given
   \item {\bf REFPOS}=string (Default: none)\\
     argument used for sequence insertion
   \item {\bf L}=real (Default: 0.)\\
     the total length of the sequence in meters. 
   \item {\bf AT}=real (Default: 0)\\
     ???
   \item {\bf FROM=}string (Default: none)\\
     ???
   \item {\bf ADD\_PASS}=integer (Default: 0)\\ 
     specifies a number of additionnal passes (max. 5) through the
     structure; in case of an RBEND the angle will be overwritten in  survey
     using the i-th component (1 \textless = i \textless = add\_pass
     \textless= 5) of its array\_of\_angles (see \href{bend.html}{RBEND}.
   \item {\bf NEXT\_SEQU}=string (Deafult: none)\\
     concatenates the sequence with the name given to the end of the
     specified sequence. 
\end{itemize}
 
Inside the ``sequence ... endsequence'' bracket three types of commands may
be placed:  
\begin{itemize}
   \item \verb( label: class, AT=real{,attributes}; (\\
     an element declaration as usual, with an additional "at"
     attribute giving the element position relative to the start of the
     sequence; CAUTION: an existing definition for an element with the
     same name will be replaced, however, defining the same element
     twice inside the same sequence results in a fatal error, since a
     unique object (this element) would be placed at two different
     positions. 
   \item \verb( class, AT=real; (\\
     a class name with a position; this causes an instance of the
     class to be placed at the position given. For uses inside ranges,
     instances of the same class can be accessed with an occurrence
     count. 
   \item \verb( sequ_name, AT=real; (\\
     a sequence name with a position; this causes the sequence with
     that name to be placed at the position indicated. The entry,
     centre, or exit of the inserted sequence are placed at the position
     given, UNLESS a "refpos" (the name of an element in the inserted
     sequence) is given, in which case the sequence is inserted such
     that the refpos element is at the insertion point. 
\end{itemize} 

When the sequence is expanded in a
\href{../control/general.html#use}{USE} command, MAD generates the
missing drift spaces. At this moment, overlapping elements will cause
"negative drift length" errors.  

For efficiency reasons MAD-X imposes an \textbf{important restriction}
on element lengths and positions: once a sequence is expanded, the
element positions and lengths are considered as fixed; in order to vary
a position or element length, a re-expansion of the sequence becomes
necessary. The MATCH command contains a special flag "vlength" to
\href{../match/match.html}{match element lengths}.  

\href{example}{Example:}
\begin{verbatim}
! define a default beam (otherwise fatal error)
beam;

! Define element classes for a simple cell:
b:   sbend, l = 35.09, angle = 0.011306116;
qf:  quadrupole, l = 1.6,  k1 = -0.02268553;
qd:  quadrupole, l = 1.6,  k1 =  0.022683642;
sf:  sextupole,  l = 0.4,  k2 = -0.13129;
sd:  sextupole,  l = 0.76, k2 =  0.26328;

! define the cell as a sequence:
sequ:  sequence, l = 79;
   b1:    b,      at = 19.115;
   sf1:   sf,     at = 37.42;
   qf1:   qf,     at = 38.70;
   b2:    b,      at = 58.255, angle = b1->angle;
   sd1:   sd,     at = 76.74;
   qd1:   qd,     at = 78.20;
   endm:  marker, at = 79.0;
endsequence;
\end{verbatim}

%\href{http://www.cern.ch/Hans.Grote/hansg_sign.html}{hansg}, June 17, 2002 
