%%\title{RFCAVITY}
%  Changed by: Chris ISELIN, 27-Jan-1997 
%  Changed by: Hans Grote, 30-Sep-2002 

\section{RF Cavity}


\begin{verbatim}
label: RFCAVITY, L = real, VOLT = real, LAG = real, HARMON = integer, FREQ = real;                  
\end{verbatim} 

%  HARMON=integer, BETRF=real,PG=real,
%                  FREQ=real,SHUNT=real,TFILL=real; 


An RFCAVITY has eight real attributes and one integer attribute: 
\begin{itemize}
   \item L: The length of the cavity (DEFAULT: 0 m) 
   \item VOLT: The peak RF voltage (DEFAULT: 0 MV). The effect of the cavity is \\
     delta(\textit{E}) = VOLT * sin(2 pi * (LAG - HARMON * \textit{f$_0$ t})). 
   \item LAG: The phase lag [2pi] (DEFAULT: 0). 
   \item FREQ: The frequency [MHz] (no DEFAULT). Note that if the RF
     frequency is not given, it is computed from the harmonic number and
     the revolution frequency \textit{f$_0$} as before. However, for
     accelerating structures this makes no sense, and the frequency is
     mandatory.  
   \item HARMON: The harmonic number \textit{h} (no DEFAULT). Only if the frequency is not given. 

   \item \textit{ Please take note, that the following MAD8 attributes:
     BETRF, PG, SHUNT and TFILL are currently not implemented in MAD-X!}    
%  \item BETRF: RF coupling factor (DEFAULT: 0).
%  \item PG: The RF power per cavity (DEFAULT: 0 MW).
%  \item SHUNT: The relative shunt impedance (DEFAULT: 0 MOhm/m).
%  \item TFILL: The filling time of the cavity $T_{fill}$ (DEFAULT: 0 microseconds). 

   \item \textit{ Note as well that twiss is 4D only. As a consequence
     the TWISS parameters in the plane of non-zero dispersion may not
     close as expected. Therefore, it is best to perform TWISS in 4D
     only, i.e. with cavities switched off. If 6D is needed one has to
     use the \href{../ptc_twiss/ptc_twiss.html}{ptc\_twiss} command. } 
\end{itemize}  

The RFCAVITY has attributes that will only become active in PTC: 
\begin{itemize}
   \item n\_bessel (DEFAULT: 0): \\
     Transverse focussing effects are typically ignored in the cavity in
     MAD-X or even PTC. This effect is being calculated to order n\_bessel,
     with n\_bessel=0 disregarding this effect and with a correct treatment
     when n\_bessel goes to infinty.
   \item no\_cavity\_totalpath (DEFAULT: no\_cavity\_totalpath=false): \\
     flag to choose if in a cavity the transit time factor is considered
     (no\_cavity\_totalpath=false) or if the particle is kept on the
     crest of RF voltage (no\_cavity\_totalpath=true).  
\end{itemize}  

A cavity requires the particle energy (\href{beam.html#energy}{ENERGY})
and the particle charge (\href{beam.html#charge}{CHARGE}) to be set by a
\href{beam.html}{BEAM} command before any calculations are performed.  

 Example: 
\begin{verbatim}
BEAM, PARTICLE = ELECTRON, ENERGY = 50.0;
CAVITY: RFCAVITY, L = 10.0, VOLT = 150.0, LAG = 0.0, HARMON = 31320;
\end{verbatim} 

The \href{local_system.html#straight}{straight reference system} for a
cavity is a cartesian coordinate system.  

%\href{http://www.cern.ch/Hans.Grote/hansg_sign.html}{hansg}, January 24, 1997 
