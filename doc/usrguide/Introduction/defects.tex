%%\title{Known Defects of MAD8 and MAD-X}
%  Changed by: Chris ISELIN, 24-Jan-1997 

%  Changed by: Hans Grote, 25-Sep-2002 

%%\usepackage{hyperref}
% commands generated by html2latex


%%\begin{document}
%%\begin{center}
 %%EUROPEAN ORGANIZATION FOR NUCLEAR RESEARCH 
%%\includegraphics{http://cern.ch/madx/icons/mx7_25.gif}

\subsection{Known Differences to Other Programs}
%%\end{center}

\subsection{Definitions} MAD uses full 6-by6-matrices to allow coupling effects to be treated, and the canonical variable set (\textit{x}, \textit{p$_x$ / p$_0$}), (\textit{y}, \textit{p$_y$ / p$_0$}), (\textit{-ct}, delta(\textit{E}) / \textit{p$_0$ c}), as opposed to other programs most of which use the set (\textit{x}, \textit{x}'), (\textit{y}, \textit{y}'),  (-delta(\textit{s}), delta(\textit{p})/\textit{p$_0$}). Like \href{bibliography.html#dragt}{[Dragt]}, MAD uses the relative energy error \textit{p$_y$ / p$_0$}, which is equal the relative momentum error  delta = delta(\textit{p})/\textit{p$_0$} multiplied by beta = v/c. 

 As from Version 8.13, MAD8 uses an additional \textbf{constant} momentum error delta$_\textit{s}$ in all optical calculations. The transfer maps contain the \textbf{exact} dependence upon this value; therefore the tunes for large deviations can be computed with high accuracy as opposed to previous versions. 

 The choice of canonical variables in MAD still leads to slightly different definitions of the lattice functions. In MAD the Courant-Snyder invariants in \href{bibliography.html#courant}{[Courant and Snyder]} take the form 

 W$_x$ = gamma$_\textit{x}$\textit{x}$^2$ - 2 alpha$_\textit{x}$\textit{x p$_x$} + beta$_\textit{x}$\textit{p$_x$}$^2$

 Comparison to the original form 

 W$_x$ = gamma$_\textit{x}$\textit{x}$^2$ - 2 alpha$_\textit{x}$\textit{x x}' + beta$_\textit{x}$\textit{x}'$^2$

 shows that the orbit functions cannot be the same. A more detailed analysis, using 

\textit{x}' = \textit{p$_x$} / (1 + delta) 

 shows that all formulas can be made consistent by defining the MAD orbit functions as 

 beta$_\textit{x}M$ = beta$_\textit{x}C$ * (1 + delta), alpha$_\textit{x}M$ = alpha$_\textit{x}C$, gammaa$_\textit{x}M$ = gamma$_\textit{x}C$ / (1 + delta), 

 For constant delta$_\textit{s}$ along the beam line and delta = 0, the lattice functions are the same. In a machine where delta varies along the circumference, e.g. in a linear accelerator or in an electron-positron storage ring, the definition of the Courant-Snyder invariants must be generalised. The MAD invariants have the advantage that they remain invariants along the beam line even for variable delta. 

 With the new method this problem occurs in  \href{../twiss/twiss.html}{Twiss module} only for non-constant delta.  

\subsection{Treatment of Energy Error in TWISS} It has been noted in \href{bibliography.html#ruggiero}{[Milutinovic and Ruggiero]} that MAD returned tunes which are too low for non-zero delta. The difference was found to be quadratic in delta with a negative coefficient. This problem has been eliminated thanks to the new treatment  of momentum errors from MAD8 Version 8.13 onwards.  

\href{http://www.cern.ch/Hans.Grote/hansg_sign.html}{hansg}, January 24, 1997 

%%\end{document}
