%%\title{SET}
%  Changed by: Hans Grote, 09-Jun-2003 

\subsection{Set Statements}

\begin{verbatim}
set, format="...", sequence="...";
\end{verbatim} 

The set command allows 2 actions: 

\subsubsection{1) Format} 
The first command lets you vary the output precision. 
\begin{verbatim}
parameter: format = s1, s2, s3
\end{verbatim} 
(up to) three strings defining the integer, floating, and string output
format for the save, show, value, and table output. The formats can be
given in any order and stay valid until replaced. The defaults are:  
\begin{verbatim}
"10d","18.10g","-18s".
\end{verbatim} 
They follow the C convention. The quotes are mandatory. The allowed formats are: 
\begin{verbatim}
"nd" for integer with n = field width.
\end{verbatim}
\begin{verbatim}
"m.nf" or "m.ng" or "m.ne" for floating, m field width, n precision.
\end{verbatim}
\begin{verbatim}
"ns" for string output.
\end{verbatim} 
The default is "right adjusted", a "-" changes it to "left adjusted".  Example: 
\begin{verbatim}
set,format="22.14e";
\end{verbatim} 
changes the current floating point format to 22.14e; the other formats remain untouched. 
\begin{verbatim}
set,format="s","d","g";
\end{verbatim} 
sets all formats to automatic adjustment according to C conventions. 

\subsubsection{2) Sequence} The second command lets you choose the
current sequence without having to use the "USE" command, which would
bring you back to a bare lattice without errors. The command only works
if the chosen sequence had been activated before with the "USE" command,
otherwise a warning will be issued and MAD-X will continue with the
unmodified current sequence. This command is particularly useful for
commands that do not have the sequence as an argument like "EMIT" or
"IBS". 

%\href{http://www.cern.ch/Hans.Grote/hansg_sign.html}{hansg}, 
%\href{http://www.cern.ch/Frank.Frank/frs_sign.html}{frs}, June 18, 2003 

