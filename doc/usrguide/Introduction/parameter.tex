%%\title{Parameters}
%  Changed by: Hans Grote, 17-Jun-2002 


\chapter{Parameter Statements}
\label{chap:parameter}

\section{\href{relation}{Relations between Variable Parameters}}
 A relation is established between variables by one of two statements 
\madbox{
parameter-name = \= expression; \\
parameter-name := \> expression;
}
The first form evaluates the expression on the right immediately and
assigns its value to the parameter. It is an immediate assignment.

The second form assigns the value by evaluating  the expression on the
right every time the parameter is actually used. It is a deferred
assignment.  

This mechanism holds as well for element parameters that can be defined
with either immediate or deferred assignments.

Attention! If you want
to modify e.g. the strength of a quadrupole later (e.g. in a match,  or
by entering a new value for a parameter on which it depends) then the
defition has to be  
\madxmp{QD: QUADRUPOLE, K1 := ak1;}
and not 
\madxmp{QD: QUADRUPOLE, K1 = ak1;}
In the latter case, K1 will be set to the current value of ak1 at the
time of declaration, and will not change when ak1 later changes.  

Parameters that have not yet been defined at time of evaluation have a
zero value. 

Example: 
\madxmp{
gev = 100; \\
BEAM, ENERGY=gev;
}

The parameter on the left may appear on the right as well in the
computer science form of assignments: 
\madxmp{x = x+1;}
increases the value of x by 1. \\
As a result, the SET statement of \madeight is no longer necessary and
is not implemented in \madx.   


Successive definitions are allowed in the first form of relations or 
immediate assignments:
\madxmp{
a = b + 2; \\
b = a * b;
}
But circular definitions in the second form of relations, or
deferred assignments, are forbidden: 
\madxmp{
a := b + 2; \\
b := a * b;
}
will result in an error.


\section{\href{par_output}{VALUE: Output of Parameters}}
The VALUE statement evaluates the current value of all listed expressions, and
prints the result on the standard output file.
\madbox{
VALUE = expression {, expression2};
}

Example:
\madxmp{
p1 = 5; \\
p2 = 7; \\
VALUE, p1*p2-3;
}
After echoing the command, this prints:
\madxmp{
p1*p2-3 = 32       ;
}

%% Another example:
%% \begin{verbatim}
%% option,-warn;
%% while (x 
%% \end{verbatim}
%% ?????

%% After echoing the command, this prints:
%% \begin{verbatim}
%% x = 100        ;       
%% x^2 = 10000      ;       
%% log10(x) = 2      ;            
%% \end{verbatim}

%\href{http://www.cern.ch/Hans.Grote/hansg_sign.html}{hansg} 11.9.2000

