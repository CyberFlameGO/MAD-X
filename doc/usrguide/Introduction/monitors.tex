%%\title{Monitors}
%  Changed by: Chris ISELIN, 27-Jan-1997 
%  Changed by: Hans Grote, 30-Sep-2002 
%  Changed by: G. Roy, 17 Oct 2013: added PLACEHOLDER

\section{Beam Position Monitors}
\label{sec:monitors}

A beam monitor acts on the beam like a drift space. In addition it
serves to record the beam position for closed orbit corrections. Four
different types of beam position monitors are recognised:  

\begin{itemize}
   \item \href{hmon}{HMONITOR}. Monitor for the horizontal beam position, 
   \item \href{vmon}{VMONITOR}. Monitor for the vertical beam position, 
   \item \href{mon}{MONITOR}. Monitor for both horizontal and vertical beam position. 
   \item \href{inst}{INSTRUMENT}. A place holder for any type of beam
     instrumentation. Optically it behaves like a drift space; it
     returns \emph{no beam observation}. It represent a class of
     elements which is completely independent from drifts and monitors.  
   \item \href{plac}{PLACEHOLDER}. A place holder for any type of
     element. Internally it is equivalent to an INSTRUMENT: optically it
     behaves as a drift space, it returns \emph{no beam observation}. It
     represent a class of elements which is completely independent from
     drifts and monitors. 
\end{itemize}

\begin{verbatim}
label: HMONITOR,    L = real;
label: VMONITOR,    L = real;
label: MONITOR,     L = real;
label: INSTRUMENT,  L = real;
label: PLACEHOLDER,  L = real;
\end{verbatim} 

A beam position monitor has one real attribute: 
\begin{itemize}
   \item L: The length of the monitor (default: 0 m). If the length is
     different from zero, the beam position is recorded in the centre of
     the monitor.  
\end{itemize} 

Examples: 
\begin{verbatim}
MH: HMONITOR, L = 1;
MV: VMONITOR;
\end{verbatim} 

The \href{local_system.html#straight}{straight reference system} for a
monitor is a cartesian coordinate system.  

%\href{http://www.cern.ch/Hans.Grote/hansg_sign.html}{hansg}, June 17, 2002 
