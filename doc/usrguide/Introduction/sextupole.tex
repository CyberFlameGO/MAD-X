%%\title{SEXTUPOLE}
%  Changed by: Chris ISELIN, 27-Jan-1997 
%  Changed by: Hans Grote, 30-Sep-2002 
%  Changed by: Frank Schmidt, 28-Aug-2003 

\section{Sextupole}
\label{sec:sextupole}

\begin{verbatim}
label: SEXTUPOLE, L = real, K2 = real, K2S = real, TILT = real;
\end{verbatim} 

A SEXTUPOLE has four real attributes: 
\begin{itemize}
    \item L: The sextupole length (default: 0 m). 
    \item K2: The normal sextupole coefficient \\
      \textit{K}$_2$ = 1/(\textit{B} rho)
      ($\partial$$^2$\textit{B$_y$}/$\partial$ \textit{x}$^2$). \\       
      (default: 0 m**(-3)). 
    \item K2S: The skew sextupole coefficient \\
%% 2013-Jul-05  17:58:30  ghislain: error reported by Christian Carli
%      \textit{K}$_{2S}$ = 1/(2 \textit{B} rho)
%      ($\partial$$^2$\textit{B$_x$}/$\partial$ \textit{x}$^2$ -
%      $\partial$$^2$\textit{B$_y$}/$\partial$ \textit{y}$^2$).   
      \textit{K}$_{2S}$ = 1/(\textit{B} rho)
      ($\partial$$^2$\textit{B$_x$}/$\partial$ \textit{x}$^2$)
      \\
      where (x,y) is now a coordinate system rotated by -30$^o$ around s with
      respect to the normal one. (default: 0 m**(-3)). A positive skew
      sextupole strength implies defocussing (!) of positively charged
      particles in the (x,s) plane rotated by 30$^o$ around s (particles in
      this plane have x $>$ 0, y $>$ 0).  


    \item TILT: The roll angle about the longitudinal axis (default: 0
      rad, i.e. a normal sextupole). A positive angle represents a
      clockwise rotation. A TILT=pi/6 turns a positive normal sextupole
      into a negative skew sextupole.
      
      \textbf{  Please note that contrary to MAD8 one has to specify the
        desired TILT angle, otherwise it is taken as 0 rad. This was needed to
        avoid the confusion in MAD8 about the actual meaning of the TILT
        attribute for various elements. } 
\end{itemize}

\textbf{  Note also that K$_2$/K$_{2s}$ can be considered as the normal
  or skew sextupole components of the magnet on the bench, while the
  TILT attribute can be considered as an tilt alignment error in the
  machine. In fact, a positive K$_2$ with a tilt=0 is equivalent to a
  positive K$_{2s}$ with positive tilt=+pi/6.  } 

Example: 
\begin{verbatim}
S: SEXTUPOLE, L = 0.4, K2 = 0.00134;
\end{verbatim} 

The \href{local_system.html#straight}{straight reference system} for a
sextupole is a cartesian coordinate system.   

%\href{http://www.cern.ch/Hans.Grote/hansg_sign.html}{hansg}, 
%\href{http://www.cern.ch/Frank.Schmidt/frs_sign.html}{frs}, August 28, 2003  
