%%\title{MARKER}
%  Changed by: Chris ISELIN, 27-Jan-1997 

%  Changed by: Hans Grote, 10-Jun-2002 

%%\usepackage{hyperref}
% commands generated by html2latex


%%\begin{document}
%%\begin{center}
 %%EUROPEAN ORGANIZATION FOR NUCLEAR RESEARCH 
%%\includegraphics{http://cern.ch/madx/icons/mx7_25.gif}

\subsection{Marker.}
%%\end{center}
\begin{verbatim}

label: MARKER;
\end{verbatim} The simplest element which can occur in a beam line is the MARKER. It has no effect on the beam, but it allows one to identify a position in the beam line, for example to apply a matching constraint. 

 Example: 
\begin{verbatim}

m27: marker;
\end{verbatim}\href{http://www.cern.ch/Hans.Grote/hansg_sign.html}{hansg}, June 6, 2002 

%%\end{document}
