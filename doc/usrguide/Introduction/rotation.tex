%%\title{Rotation, SROTATION, YROTATION}
%  Changed by: Chris ISELIN, 27-Jan-1997 

%  Changed by: Hans Grote, 30-Sep-2002 

%%\usepackage{hyperref}
% commands generated by html2latex


%%\begin{document}
%%\begin{center}
 %%EUROPEAN ORGANIZATION FOR NUCLEAR RESEARCH 
%%\includegraphics{http://cern.ch/madx/icons/mx7_25.gif}

\subsection{Coordinate Transformations}
%%\end{center}

\paragraph{\href{yrotation}{YROTATION: Rotation About the Vertical Axis}}
\begin{verbatim}

label: YROTATION,TYPE=name,ANGLE=real;
\end{verbatim} The element YROTATION rotates the \href{local_system.html#straight}{straight reference system} about the vertical (\texttt{y}) axis. YROTATION has no effect on the beam, but it causes the beam to be referred to the new coordinate system 

\textit{x}$_2$=\textit{x}$_1$cos(theta)-\textit{s}$_1$sin(theta), \textit{y}$_2$=\textit{x}$_1$sin(theta)+\textit{s}$_1$cos(theta), 

 It has one real attribute: 
\begin{itemize}
	\item ANGLE: The rotation angle theta (default: 0 rad). It must be a \emph{small} angle, i.e. an angle comparable to the transverse angles of the orbit. 
\end{itemize} A positive angle means that the new reference system is rotated clockwise about the local \texttt{y}-axis with respect to the old system. 

 Example: 
\begin{verbatim}

KINK: YROTATION,ANGLE=0.0001;
\end{verbatim}

\paragraph{\href{srotation}{SROTATION: Rotation Around the Longitudinal Axis}}
\begin{verbatim}

label: SROTATION,ANGLE=real;
\end{verbatim} The element SROTATION rotates the \href{local_system.html#straight}{straight reference system} about the longitudinal (\texttt{s}) axis. SROTATION has no effect on the beam, but it causes the beam to be referred to the new coordinate system 

\textit{x}$_2$=\textit{x}$_1$cos(psi)-\textit{y}$_1$sin(psi), \textit{y}$_2$=\textit{x}$_1$sin(psi)+\textit{y}$_1$cos(psi), 

 It has one real attribute: 
\begin{itemize}
	\item ANGLE: The rotation angle psi (default: 0 rad) 
\end{itemize} A positive angle means that the new reference system is rotated clockwise about the \texttt{s}-axis with respect to the old system. 

 Example: 
\begin{verbatim}

ROLL1: SROTATION,ANGLE=PI/2.;
ROLL2: SROTATION,ANGLE=-PI/2.;
HBEND: SBEND,L=6.0,ANGLE=0.01;
VBEND: LINE=(ROLL1,HBEND,ROLL2);
\end{verbatim} The above is a way to represent a bend down in the vertical plane, it could be defined more simply by 
\begin{verbatim}

VBEND: SBEND,L=6.0,K0S=0.01/6;
\end{verbatim}\href{http://www.cern.ch/Hans.Grote/hansg_sign.html}{hansg}, June 17, 2002 

%%\end{document}
