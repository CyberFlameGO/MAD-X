%%\title{Dipedge}
%  Changed by: Frank Schmidt, 27-FEB-2005 

%%\usepackage{hyperref}
% commands generated by html2latex


%%\begin{document}
%%\begin{center}
 %%EUROPEAN ORGANIZATION FOR NUCLEAR RESEARCH 
%%\includegraphics{http://cern.ch/madx/icons/mx7_25.gif}

\subsection{Dipedge Element}
%%\end{center}

 A thin element describing the edge focusing of a dipole has been  introduced in order to make it possible to track trajectories in the  presence of dipoles with pole face angles. Only linear terms are considered since the higher order terms would the tracking non-symplectic. The transformation of the machine elements into thin lenses leaves dipedge untouched and splits correctly the SBENDS's. 

 It does not make sense to use it alone. It can be specified at the  entrance and the exit of a SBEND.  They are defined by the commands: 
\begin{verbatim}

label : dipedge, h=real, e1=real, fint=real, hgap=real, tilt=real;
\end{verbatim} It has zero length and five attributes. 
\begin{itemize}
	\item H: Is angle/length or 1/rho (default: 0 m$^{-1}$ - for the default the dipedge element has no effect). (must be equal to that of the associated SBEND) 
	\item E1: The rotation angle for the pole face.The sign convention is as for a SBEND \href{bend.html}{Bending Magnet}. Note that it is different for an entrance and an exit. (default: 0 rad). 
	\item FINT: field integral as for SBEND \href{local_system.html#sbend}{sector bend}. Note that each dipedge has its own fint, so fintx is no longer necessary. 
	\item HGAP: half gap height of the associated SBEND \href{bend.html}{Bending Magnet}.  


	\item TILT: The roll angle about the longitudinal axis (default: 0 rad, i.e. a horizontal bend). A positive angle represents a clockwise rotation. 
\end{itemize}

\href{http://www.cern.ch/Frank.Schmidt/frs_sign.html}{frs}, February 27, 2005  

%%\end{document}
