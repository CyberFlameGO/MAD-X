%%\title{ELSEPARATOR}
%  Changed by: Chris ISELIN, 27-Jan-1997 

%  Changed by: Hans Grote, 30-Sep-2002 

%  Changed by: Frank Schmidt, 28-Aug-2003 

%%\usepackage{hyperref}
% commands generated by html2latex


%%\begin{document}
%%\begin{center}
 %%EUROPEAN ORGANIZATION FOR NUCLEAR RESEARCH 
%%\includegraphics{http://cern.ch/madx/icons/mx7_25.gif}

\subsection{ELSEPARATOR: Electrostatic Separator}
%%\end{center}
\begin{verbatim}

label: ELSEPARATOR,L=real,EX=real,EY=real,TILT=real;
\end{verbatim} An ELSEPARATOR (electrostatic separator) has four real attributes: 
\begin{itemize}
	\item L: The length of the separator (default: 0 m). 
	\item EX: The horizontal electric field strength (default: 0 MV/m). A positive field increases \textit{p$_x$} for positive particles. 
	\item EY: The vertical electric field strength (default: 0 MV/m). A positive field increases \textit{p$_y$} for positive particles. 
	\item TILT: The roll angle about the longitudinal axis (default: 0 rad). A positive angle represents a clockwise of the electrostatic separator. 
\end{itemize} A separator requires the particle energy (\href{beam.html#energy}{ENERGY}) and the particle charge (\href{beam.html#charge}{CHARGE}) to be set by a \href{beam.html}{BEAM} command before any calculations are performed. 

 Example: 
\begin{verbatim}

BEAM,PARTICLE=POSITRON,ENERGY=50.0;
SEP:    ELSEPARATOR,L=5.0,EY=0.5;
\end{verbatim} The \href{local_system.html#straight}{straight reference system} for a separator is a cartesian coordinate system.  

\href{http://www.cern.ch/Hans.Grote/hansg_sign.html}{hansg}, \href{http://www.cern.ch/Frank.Schmidt/frs_sign.html}{frs}, August 28, 2003  

%%\end{document}
