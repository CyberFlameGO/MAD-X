%%\title{MAD-X PITFALLS}

\chapter{\madx PITFALLS}

Find a loose collection of pitfalls that may be difficult to avoid in
particular for new users but also experienced user might profit from
this list.  


\begin{description}

\item[Twiss calculation is 4D only!]
  The Twiss command will calculate an approximate 6D closed orbit when
  the accelerator structure includes an active
  \href{../Introduction/cavity.html}{cavity}. However, the calculation
  of the Twiss parameters are 4D only. This may result in apparently
  non-closure of the beta values in the plane with non-zero
  dispersion. The full 6D Twiss parameters can be calculated with the
  \href{../ptc_twiss/ptc_twiss.html}{ptc\_twiss} command. Presently,
  the \href{../thintrack/thintrack.html}{Thinlens Tracking} module
  suffers from this deficiency since it requires the true 6d closed
  orbit and not the approximate one as calculated by Twiss. In this
  context one has to mention that the coordinate system for the Twiss
  module is not x, px in the horizontal plane as the advertised
  canonical coordinates instead x, x' have been used (same for the
  vertical plane).  
  
  Mind you that for TWISS with the "CENTRE" attribute activated,
  i.e. looking     inside the element, the closed orbit includes the
  misalignment of the element.  


  
\item[Dispersion for machines with small relativistic beta] 
  \madx uses the PT coordinate as the canonical momentum in the
  longitudinal plane. The derivative of e.g. dispersion is therefore
  not taken wrt delta-p over p but PT. Therefore one unfortunately finds the
  dispersion being divided by the relativistic beta which is annoying
  for low energy machines. PTC allows to change the coordinate system
  to  delta-p over p with the "time=false" option of the
  \href{../ptc_general/ptc_general.html#PTC_CREATE_LAYOUT}{PTC\_CREATE\_LAYOUT}
  command which delivers the proper dispersion with the
  \href{../ptc_twiss/ptc_twiss.html}{ptc\_twiss} command.  



\item[Non-standard definition of DDX, DDPX, DDY, DDPY] 
  The \madx proper defintion of DDX, DDPX, DDY, DDPY is not the second
  order derivative with respect to deltap/p but multiplied by a factor
  of 2. The corresponding values from ptc\_normal and in ptc\_twiss
  are the proper derivaties to all orders.  
  

\item[Chromaticity calculation in presence of coupling] 
  Chromaticity calculations are typically in order and agree with PTC
  and other codes. However, it was recently discovered that in
  presences of coupling MAD-X simply seems to ignore coupling when the
  chromaticity is calculated. This is surprising since the eigentunes
  Q1, Q2 are properly calculated for a given (small!) dp/p. The issue
  is under investigation.  


\item[Field errors in thick elements]
  Only a very limited number of field error components are
  considered in TWISS calculations for some thick elements. Find below
  a complete list of all those field error components that
  are taking into account for a particular thick element. It
  should be mentioned that BENDs also allow a skew quadrupole
  component k1s but NOT in the body of the magnet. It is only
  active in the edge effect for radiation (expert use only). 


{\renewcommand{\arraystretch}{2}
  \begin{tabular}{c | c | c}
    \hline 
    \textbf{Magnet Type} & \textbf{Normal Field Components} & \textbf{Skew Field Components} \\ 
    \hline
    & Dipole & ---\\
    Bend & Quadrupole & ---\\
    & Sextupole & ---\\
    \hline
    HKicker & Dipole & ---\\
    \hline
    VKicker & --- & Dipole\\
    \hline
    Quadrupole & Quadrupole & Quadrupole \\
    \hline
    Sextupole & Sextupole & Sextupole \\
    \hline
    Octupole & Octupole & Octupole \\
    \hline
  \end{tabular}
}

\item[MAD-X versus PTC] 
  The user has to understand that PTC exists inside of MAD-X as a
  library. MAD-X offers the interface to PTC, i.e. the MAD-X input
  file is used as input for PTC. Internally, both PTC and MAD-X have
  their own independent databases which are linked via the
  interface. With the
  \href{../ptc_general/ptc_general.html#PTC_CREATE_LAYOUT}{PTC\_CREATE\_LAYOUT}
  command, only numerical numbers are transferred from the MAD-X
  database to the PTC database. Any modification to the MAD-X
  database is ignored in PTC until the next call to
  \href{../ptc_general/ptc_general.html#PTC_CREATE_LAYOUT}{PTC\_CREATE\_LAYOUT}.
  For example, a deferred expression of MAD-X after a
  \href{../ptc_general/ptc_general.html#PTC_CREATE_LAYOUT}{PTC\_CREATE\_LAYOUT}
  command is ignored within PTC.  
  
  When introducing a cavity with the "harmon" instead of the "freq"
  attribute (highly discouraged!) a problem arises for ptc\_twiss due to
  the fact that internally "harmon" is transferred to "freq" too late. A
  simple "twiss" command executed before PTC start-up will help. However,
  avoiding "harmon" is advantageous.  
    

\item[SLOW attribute in matching] 
  The "slow" attribute enforces the old matching procedure and is
  considerably slower. Therefore we did not make it the default
  option. Recently a number of parameters, like "RE56", have been
  added to the list of matchable parameters in the default and fast
  version. Nevertheless, some parameters are only available when using
  the "slow" attribute. Therefore it is advisable to check with the
  "slow" attribute if there are doubts about the matching procedure.  



\item[Validity of Twiss parameters]
  The standard Teng-Edwards Twiss parameters suffer from a deficiency near
  full coupling: i.e. the "donuts" of linear motion in x-x' and y-y' phase
  space have no hole anymore. This means that all energy is transfered
  from one plane to the other. In this case the Twiss parameters and the
  coupling matrix (R11, R12, R21, R22) become large or even infinite or
  the beta functions might become negative. The Ripken-Mais Twiss
  parameters are always well defined (they are the "average" amplitude
  functions of their proper phase space region), i.e. at full coupling we
  have: beta11 $\sim$ beta12 and beta21 $\sim$ beta22. Using the "RIPKEN"
  flag Twiss calculates the Mais-Ripken parameters via a transformation
  from the Teng-Edwards Twiss parameters. Obviously this fails when the
  Teng-Edwards Twiss parameters are ill defined. In this case one has to
  rely on ptc\_twiss.    
  
\end{description}

%\href{mailto:Frank.Schmidt@cern.ch}{  F.\nolinebreak Schmidt}, November  2008

