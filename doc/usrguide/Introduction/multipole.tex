%%\title{MULTIPOLE}
%  Changed by: Chris ISELIN, 27-Jan-1997 
%  Changed by: Hans Grote, 17-Jun-2002 
%  Changed by: Frank Schmidt, 28-Aug-2003 

\section{General Thin Multipole}
\label{sec:multipole}

\begin{verbatim}
label: MULTIPOLE, LRAD = real, TILT = real,
                  KNL := {..,..,..}, KSL := {..,..,..};
\end{verbatim} 

A MULTIPOLE is a thin-lens magnet of arbitrary order, including a dipole: 
\begin{itemize}
    \item LRAD: A fictitious length, which was originally just used to
      compute synchrotron radiation effects. \\
      A non-zero \textbf{ LRAD } in conjunction  with the
      \href{../control/general.html#option}{OPTION}\textbf{ thin\_foc }
      set to a \textbf{ true } logical value takes into account of the
      weak focussing of bending magnets.  
    \item TILT: The roll angle about the longitudinal axis (default: 0
      rad). A positive angle represents a clockwise rotation of the
      multipole element.   

      \textbf{  Please note that contrary to MAD8 one has to specify the
        desired TILT angle, otherwise it is taken as 0 rad. We believe
        that the MAD8 concept of having individual TILT values for each
        component and on top with default values led to considerable
        confusion and allowed for excessive and unphysical
        freedom. Instead, in MAD-X the KNL/KSL components can be
        considered as the normal or skew multipole components of the
        magnet on the bench, while the TILT attribute can be considered
        as an tilt alignment error in the machine. } 

    \item KNL: The normal multipole coefficients from order zero to the
      maximum; the parameters are positional, therefore zeros must be
      filled in for components that do not exist. Example of a thin-lens
      sextupole:  
\begin{verbatim}
ms: multipole, knl := {0, 0, k2l};
\end{verbatim}

   \item KSL: The skew multipole coefficients from order zero to the
     maximum; the parameters are positional, therefore zeros must be
     filled in for components that do not exist. Example of a thin-lens
     skew octupole:  
\begin{verbatim}
ms: multipole, ksl := {0, 0, 0, k3sl};
\end{verbatim}

\end{itemize} 

Both KNL and KSL may be specified for the same multipole. 

A multipole with no dipole component has no effect on the reference
orbit, i.e. the reference system at its exit is the same as at its
entrance. If it includes a dipole component, it has the same effect on
the reference orbit as a dipole with zero length and deflection angle
K0L, being the first component of KNL above.  

%\href{http://www.cern.ch/Hans.Grote/hansg_sign.html}{hansg}, 
%\href{http://www.cern.ch/Frank.Schmidt/frs_sign.html}{Frank.Schmidt}, August 28, 2003  
