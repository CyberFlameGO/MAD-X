%%\title{the mad program}
%  Changed by: Chris ISELIN, 27-Mar-1997 
%  Changed by: Hans Grote, 25-Sep-2002 

\section{Statements and Comments}

Input for MAD-X follows in broad lines the new
\href{http://cern.ch/mad9}{MAD-9} format, i.e. free format with commas
"," as separators; however, outside \{...\} enclosures blanks may be
used as separators. Blank input lines do not affect program
execution. The input is not case sensitive except for strings enclosed
in " ". 

The input file consists of a sequence of commands, also known as
statements. A statement may occupy any number of  input lines. It must
be terminated by a semicolon, except if it contains a block of
statements itself, like in  

\begin{verbatim}
if (a < 3) {a=b^2; b=2*b+4;}
\end{verbatim}

Several statements may be placed on the same line.
When a "!" or "//" is found on an input line,
the remaining characters of the line are skipped.
A line "/*" starts a comment region, it ends with a "*/" line.
The general format for a command is (items enclosed in /rep/ ... /rep/
can be repeated any number of times, including zero):

\begin{verbatim}
label: keyword /rep/,attribute/rep/ ;
\end{verbatim}

It has three parts:
\begin{itemize}
   \item A \href{label.html}{label}
     is required for a definition statement.
     It gives a name to the stored command.
     
   \item A \href{keyword.html}{keyword}
     identifies the action desired.
     
   \item The \href{attribute HREF=attribute.html}{attributes}
     are normally entered in the form
     "attribute-name=attribute-value"
     and serve to define data for the command, where:
     \begin{itemize}
       \item \href{label.html}{attribute-name} selects the attribute,
       \item \href{attribute.html}{attribute-value} gives it a value.       
     \end{itemize}
\end{itemize}

If a value has to be assigned to an attribute, the
attribute name is mandatory.
For logical attributes it is sufficient to enter the name only.
The attribute is then given a default value taken from the
command dictionary.


Example: TILT attribute for various magnets.

The command attributes can have one of the following types:
\begin{itemize}
  \item \href{string.html}{String attribute},
  \item \href{logical.html}{Logical attribute},
  \item \href{integer.html}{Integer attribute},
  \item \href{real.html}{Real attribute},
  \item \href{expression.html}{Expression},
  \item \href{select.html}{Range selection},
\end{itemize}

Any integer or real attribute can be replaced by
a \href{expression.html}{real expression}; expressions are
normally deferred (see \href{expression.html#defer}{deferred
  expression}), except in the 
definition of constants where they are evaluated immediately.

When a command has a \href{label.html}{label},
MAD-X keeps it in memory.
This allows repeated execution of the same command
by just entering EXEC label. This construct may be nested.
For an exhaustive list of valid declarations of constants or variables
see \href{declarations.html}{declarations}.

%\href{http://www.cern.ch/Hans.Grote/hansg_sign.html}{hansg}, May 8, 2001



% add other files to the end of this file

%%\title{Identifiers}
%  Changed by: Chris ISELIN, 24-Jan-1997 
%  Changed by: Hans Grote, 10-Jun-2002 

%\subsection{Identifiers or Labels}

\subsection{Keywords}

A keyword begins with a letter and consists of letters and digits. 

The MAD-X keywords are protected; using one of them as a label results
in a fatal error.   

% \href{http://www.cern.ch/Hans.Grote/hansg_sign.html}{hansg}, May 8, 2001 


%%\title{Variable Declarations}
%  Changed by: Chris ISELIN, 24-Jan-1997 
%  Changed by: Hans Grote, 10-Jun-2002 

\subsection{Variable Declarations}

In the following, "=" means that the variable at the left receives the
current value of the expression at right, but does not depend on it any
further, whereas ":=" makes it depend on this expression, i.e. every
time the expression changes, the variable is re-evaluated, except for
"const" variables.  

\begin{verbatim}
var = expression;
var := expression;
real var = expression;        // identical
real var := expression;       // to above
int var = expression;         // truncated if expression is real
int var := expression;
const var = expression;
const var := expression;
const real var = expression;        // identical
const real var := expression;       // to above
const int var = expression;         // truncated if expression is real
const int var := expression;
\end{verbatim}

%\href{http://www.cern.ch/Hans.Grote/hansg_sign.html}{hansg}, May 8, 2001 


%%\title{SELECT}
%  Changed by: Hans Grote, 16-Jan-2003 

\subsection{Selection Statements}

The elements, or a range of elements, in a sequence can be selected for
various purposes. Such selections remain valid until cleared (in
difference to MAD-8); it is therefore recommended to always start with a  

\begin{verbatim}
select, flag =..., clear;
\end{verbatim} 
before setting a new selection. 
\begin{verbatim}
SELECT, FLAG=name, RANGE=range, CLASS=class, PATTERN=pattern [,FULL] [,CLEAR];
\end{verbatim} 
where the name for FLAG can be one of ERROR, MAKETHIN, SEQEDIT or the
name of a twiss table which is established for all sequence positions in
general.  

Selected elements have to fulfill the \href{ranges.html#range}{RANGE},
\href{ranges.html#class}{CLASS}, and \href{wildcard.html}{PATTERN}
criteria.  

Any number of SELECT commands can be issued for the same flag and are
accumulated (logically ORed). In this context note the following:  

\begin{verbatim}
SELECT, FLAG=name, FULL;
\end{verbatim} 
selects all positions in the sequence for this flag. This is the default
for all tables and makethin, whereas for ERROR and SEQEDIT the default
is "nothing selected".  

\href{save_select}{}SAVE: A SELECT,FLAG=SAVE statement causes the
selected sequences, elements, and variables to be written into the save
file. A class (only used for element selection), and a pattern can be
specified. Example:  
\begin{verbatim}
select, flag=save, class=variable, pattern="abc.*";
save, file=mysave;
\end{verbatim} 
will save all variables (and sequences) containing "abc" in their name,
but not elements with names containing "abc" since the class "variable"
does not exist (astucieux, non ?).  

SECTORMAP: A SELECT,FLAG=SECTORMAP statement causes sectormaps to be
written into the file "sectormap" like in MAD-8. For the file to be
written, a flag SECTORMAP must be issued on the TWISS command in
addition.  

TWISS: A SELECT,FLAG=TWISS statement causes the selected rows and
columns to be written into the Twiss TFS file (former OPTICS command in
MAD-8). The column selection is done on the same select. See as well
example 2.  

Example 1:  
\begin{verbatim}
TITLE,'Test input for MAD-X';

option,rbarc=false; // use arc length of rbends
beam; ! sets the default beam for the following sequence
option,-echo;
call file=fv9.opt;  ! contains optics parameters
call file="fv9.seq"; ! contains a small sequence "fivecell"
OPTION,ECHO;
SELECT,FLAG=SECTORMAP,clear;
SELECT,FLAG=SECTORMAP,PATTERN="^m.*";
SELECT,FLAG=TWISS,clear;
SELECT,FLAG=TWISS,PATTERN="^m.*",column=name,s,betx,bety;
USE,PERIOD=FIVECELL;
twiss,file=optics,sectormap;
stop;
\end{verbatim} 

This produces a file \href{sectormap.html}{sectormap}, and a
\href{tfs}{}twiss output file (name = optics):  
\begin{verbatim}
@ TYPE             %05s "TWISS"
@ PARTICLE         %08s "POSITRON"
@ MASS             %le          0.000510998902
@ CHARGE           %le                       1
@ E0               %le                       1
@ PC               %le           0.99999986944
@ GAMMA            %le           1956.95136738
@ KBUNCH           %le                       1
@ NPART            %le                       0
@ EX               %le                       1
@ EY               %le                       1
@ ET               %le                       0
@ LENGTH           %le                   534.6
@ ALFA             %le        0.00044339992938
@ ORBIT5           %le                      -0
@ GAMMATR          %le           47.4900022541
@ Q1               %le           1.25413071556
@ Q2               %le           1.25485338377
@ DQ1              %le           1.05329608302
@ DQ2              %le           1.04837000224
@ DXMAX            %le           2.17763211131
@ DYMAX            %le                       0
@ XCOMAX           %le                       0
@ YCOMAX           %le                       0
@ BETXMAX          %le            177.70993499
@ BETYMAX          %le           177.671582415
@ XCORMS           %le                       0
@ YCORMS           %le                       0
@ DXRMS            %le           1.66004270906
@ DYRMS            %le                       0
@ DELTAP           %le                       0
@ TITLE            %20s "Test input for MAD-X"
@ ORIGIN           %16s "MAD-X 0.20 Linux"
@ DATE             %08s "07/06/02"
@ TIME             %08s "14.25.51"
* NAME               S                  BETX               BETY               
$ %s                 %le                %le                %le                
 "MSCBH"             4.365              171.6688159        33.31817319       
 "MB"                19.72              108.1309095        58.58680717       
 "MB"                35.38              61.96499987        102.9962313       
 "MB"                51.04              34.61640793        166.2227523       
 "MSCBV.1"           57.825             33.34442808        171.6309057       
 "MB"                73.18              58.61984637        108.0956006       
 "MB"                88.84              103.0313887        61.93159422       
 "MB"                104.5              166.2602486        34.58939635       
 "MSCBH"             111.285            171.6688159        33.31817319       
 "MB"                126.64             108.1309095        58.58680717       
 "MB"                142.3              61.96499987        102.9962313       
 "MB"                157.96             34.61640793        166.2227523       
 "MSCBV"             164.745            33.34442808        171.6309057       
 "MB"                180.1              58.61984637        108.0956006       
 "MB"                195.76             103.0313887        61.93159422       
 "MB"                211.42             166.2602486        34.58939635       
 "MSCBH"             218.205            171.6688159        33.31817319       
 "MB"                233.56             108.1309095        58.58680717       
 "MB"                249.22             61.96499987        102.9962313       
 "MB"                264.88             34.61640793        166.2227523       
 "MSCBV"             271.665            33.34442808        171.6309057       
 "MB"                287.02             58.61984637        108.0956006       
 "MB"                302.68             103.0313887        61.93159422       
 "MB"                318.34             166.2602486        34.58939635       
 "MSCBH"             325.125            171.6688159        33.31817319       
 "MB"                340.48             108.1309095        58.58680717       
 "MB"                356.14             61.96499987        102.9962313       
 "MB"                371.8              34.61640793        166.2227523       
 "MSCBV"             378.585            33.34442808        171.6309057       
 "MB"                393.94             58.61984637        108.0956006       
 "MB"                409.6              103.0313887        61.93159422       
 "MB"                425.26             166.2602486        34.58939635       
 "MSCBH"             432.045            171.6688159        33.31817319       
 "MB"                447.4              108.1309095        58.58680717       
 "MB"                463.06             61.96499987        102.9962313       
 "MB"                478.72             34.61640793        166.2227523       
 "MSCBV"             485.505            33.34442808        171.6309057       
 "MB"                500.86             58.61984637        108.0956006       
 "MB"                516.52             103.0313887        61.93159422       
 "MB"                532.18             166.2602486        34.58939635       
\end{verbatim}

 Example 2: 

 Addition of variables to (any internal) table: 
\begin{verbatim}
select, flag=table, column=name, s, betx, ..., var1, var2, ...; ! or
select, flag=table, full, column=var1, var2, ...; ! default col.s + new
\end{verbatim} 
will write the current value of var1 etc. into the table each time a new
line is added; values from the same (current) line can be accessed by
these variables, e.g.  
\begin{verbatim}
var1 := sqrt(beam->ex*table(twiss,betx));
\end{verbatim} 
in the case of table above being "twiss". The plot command accepts the
new variables.  

Remark: this replaces the "string" variables of MAD-8. 

\href{ucreate}{} This example demonstrates as well the usage of a user defined table. 
\begin{verbatim}
beam,ex=1.e-6,ey=1.e-3;
// element definitions
mb:rbend, l=14.2, angle:=0,k0:=bang/14.2;
mq:quadrupole, l:=3.1,apertype=ellipse,aperture={1,2};
qft:mq, l:=0.31, k1:=kqf,tilt=-pi/4;
qf.1:mq, l:=3.1, k1:=kqf;
qf.2:mq, l:=3.1, k1:=kqf;
qf.3:mq, l:=3.1, k1:=kqf;
qf.4:mq, l:=3.1, k1:=kqf;
qf.5:mq, l:=3.1, k1:=kqf;
qd.1:mq, l:=3.1, k1:=kqd;
qd.2:mq, l:=3.1, k1:=kqd;
qd.3:mq, l:=3.1, k1:=kqd;
qd.4:mq, l:=3.1, k1:=kqd;
qd.5:mq, l:=3.1, k1:=kqd;
bph:hmonitor, l:=l.bpm;
bpv:vmonitor, l:=l.bpm;
cbh:hkicker;
cbv:vkicker;
cbh.1:cbh, kick:=acbh1;
cbh.2:cbh, kick:=acbh2;
cbh.3:cbh, kick:=acbh3;
cbh.4:cbh, kick:=acbh4;
cbh.5:cbh, kick:=acbh5;
cbv.1:cbv, kick:=acbv1;
cbv.2:cbv, kick:=acbv2;
cbv.3:cbv, kick:=acbv3;
cbv.4:cbv, kick:=acbv4;
cbv.5:cbv, kick:=acbv5;
!mscbh:sextupole, l:=1.1, k2:=ksf;
mscbh:multipole, knl:={0,0,0,ksf},tilt=-pi/8;
mscbv:sextupole, l:=1.1, k2:=ksd;
!mscbv:octupole, l:=1.1, k3:=ksd,tilt=-pi/8;

// sequence declaration

fivecell:sequence, refer=centre, l=534.6;
   qf.1:qf.1, at=1.550000e+00;
   qft:qft, at=3.815000e+00;
!   mscbh:mscbh, at=3.815000e+00;
   cbh.1:cbh.1, at=4.365000e+00;
   mb:mb, at=1.262000e+01;
   mb:mb, at=2.828000e+01;
   mb:mb, at=4.394000e+01;
   bpv:bpv, at=5.246000e+01;
   qd.1:qd.1, at=5.501000e+01;
   mscbv:mscbv, at=5.727500e+01;
   cbv.1:cbv.1, at=5.782500e+01;
   mb:mb, at=6.608000e+01;
   mb:mb, at=8.174000e+01;
   mb:mb, at=9.740000e+01;
   bph:bph, at=1.059200e+02;
   qf.2:qf.2, at=1.084700e+02;
   mscbh:mscbh, at=1.107350e+02;
   cbh.2:cbh.2, at=1.112850e+02;
   mb:mb, at=1.195400e+02;
   mb:mb, at=1.352000e+02;
   mb:mb, at=1.508600e+02;
   bpv:bpv, at=1.593800e+02;
   qd.2:qd.2, at=1.619300e+02;
   mscbv:mscbv, at=1.641950e+02;
   cbv.2:cbv.2, at=1.647450e+02;
   mb:mb, at=1.730000e+02;
   mb:mb, at=1.886600e+02;
   mb:mb, at=2.043200e+02;
   bph:bph, at=2.128400e+02;
   qf.3:qf.3, at=2.153900e+02;
   mscbh:mscbh, at=2.176550e+02;
   cbh.3:cbh.3, at=2.182050e+02;
   mb:mb, at=2.264600e+02;
   mb:mb, at=2.421200e+02;
   mb:mb, at=2.577800e+02;
   bpv:bpv, at=2.663000e+02;
   qd.3:qd.3, at=2.688500e+02;
   mscbv:mscbv, at=2.711150e+02;
   cbv.3:cbv.3, at=2.716650e+02;
   mb:mb, at=2.799200e+02;
   mb:mb, at=2.955800e+02;
   mb:mb, at=3.112400e+02;
   bph:bph, at=3.197600e+02;
   qf.4:qf.4, at=3.223100e+02;
   mscbh:mscbh, at=3.245750e+02;
   cbh.4:cbh.4, at=3.251250e+02;
   mb:mb, at=3.333800e+02;
   mb:mb, at=3.490400e+02;
   mb:mb, at=3.647000e+02;
   bpv:bpv, at=3.732200e+02;
   qd.4:qd.4, at=3.757700e+02;
   mscbv:mscbv, at=3.780350e+02;
   cbv.4:cbv.4, at=3.785850e+02;
   mb:mb, at=3.868400e+02;
   mb:mb, at=4.025000e+02;
   mb:mb, at=4.181600e+02;
   bph:bph, at=4.266800e+02;
   qf.5:qf.5, at=4.292300e+02;
   mscbh:mscbh, at=4.314950e+02;
   cbh.5:cbh.5, at=4.320450e+02;
   mb:mb, at=4.403000e+02;
   mb:mb, at=4.559600e+02;
   mb:mb, at=4.716200e+02;
   bpv:bpv, at=4.801400e+02;
   qd.5:qd.5, at=4.826900e+02;
   mscbv:mscbv, at=4.849550e+02;
   cbv.5:cbv.5, at=4.855050e+02;
   mb:mb, at=4.937600e+02;
   mb:mb, at=5.094200e+02;
   mb:mb, at=5.250800e+02;
   bph:bph, at=5.336000e+02;
end:marker, at=5.346000e+02;
endsequence;

// forces and other constants

l.bpm:=.3;
bang:=.509998807401e-2;
kqf:=.872651312e-2;
kqd:=-.872777242e-2;
ksf:=.0198492943;
ksd:=-.039621283;
acbv1:=1.e-4;
acbh1:=1.e-4;
!save,sequence=fivecell,file,mad8;

s := table(twiss,bpv[5],betx);
myvar := sqrt(beam->ex*table(twiss,betx));
use, period=fivecell;
select,flag=twiss,column=name,s,myvar,apertype;
twiss,file;
n = 0;
create,table=mytab,column=dp,mq1,mq2;
mq1:=table(summ,q1);
mq2:=table(summ,q2);
while ( n < 11)
{
  n = n + 1;
  dp = 1.e-4*(n-6);
  twiss,deltap=dp;
  fill,table=mytab;
}
write,table=mytab;
plot,haxis=s,vaxis=aper_1,aper_2,colour=100,range=#s/cbv.1,notitle;
stop;
\end{verbatim}
prints the following user table on output:

\begin{verbatim}
@ NAME             %05s "MYTAB"
@ TYPE             %04s "USER"
@ TITLE            %08s "no-title"
@ ORIGIN           %16s "MAD-X 1.09 Linux"
@ DATE             %08s "10/12/02"
@ TIME             %08s "10.45.25"
* DP                 MQ1                MQ2                
$ %le                %le                %le                
 -0.0005            1.242535951        1.270211135       
 -0.0004            1.242495534        1.270197018       
 -0.0003            1.242452432        1.270185673       
 -0.0002            1.242406653        1.270177093       
 -0.0001            1.242358206        1.270171269       
 0                  1.242307102        1.27016819        
 0.0001             1.242253353        1.270167843       
 0.0002             1.242196974        1.270170214       
 0.0003             1.24213798         1.270175288       
 0.0004             1.242076387        1.270183048       
 0.0005             1.242012214        1.270193477       
\end{verbatim}
and produces a twiss file with the additional column myvar, as well as a plot
file with the aperture values plotted.


\href{screate}{}

Example of joing 2 tables with different length into a third table
making use of the length of either table as given by
table("your\_table\_name", tablelength) and adding names by the "\_name"
attribute.

\begin{verbatim}
title,   "summing of offset and alignment tables";
set,    format="13.6f";

readtable, table=align,  file="align.ip2.b1.tfs";   // mesured alignment
readtable, table=offset, file="offset.ip2.b1.tfs";  // nominal offsets

n_elem  =  table(offset, tablelength);

create,  table=align_offset, column=_name,s_ip,x_off,dx_off,ddx_off,y_off,dy_off,ddy_off;

calcul(elem_name) : macro = {
    x_off = table(align,  elem_name, x_ali) + x_off;
    y_off = table(align,  elem_name, y_ali) + y_off;
}


one_elem(j_elem) : macro = {
    setvars, table=offset, row=j_elem;
    exec,  calcul(tabstring(offset, name, j_elem));
    fill,  table=align_offset;
}


i_elem = 0;
while (i_elem < n_elem) { i_elem = i_elem + 1; exec,  one_elem($i_elem); }

write, table=align_offset, file="align_offset.tfs";

stop;
\end{verbatim}

% \href{http://www.cern.ch/Hans.Grote/hansg_sign.html}{hansg}, May 8, 2001


%%\title{SET}
%  Changed by: Hans Grote, 09-Jun-2003 

\subsection{Set Statements}

\begin{verbatim}
set, format="...", sequence="...";
\end{verbatim} 

The set command allows 2 actions: 

\subsubsection{1) Format} 
The first command lets you vary the output precision. 
\begin{verbatim}
parameter: format = s1, s2, s3
\end{verbatim} 
(up to) three strings defining the integer, floating, and string output
format for the save, show, value, and table output. The formats can be
given in any order and stay valid until replaced. The defaults are:  
\begin{verbatim}
"10d","18.10g","-18s".
\end{verbatim} 
They follow the C convention. The quotes are mandatory. The allowed formats are: 
\begin{verbatim}
"nd" for integer with n = field width.
\end{verbatim}
\begin{verbatim}
"m.nf" or "m.ng" or "m.ne" for floating, m field width, n precision.
\end{verbatim}
\begin{verbatim}
"ns" for string output.
\end{verbatim} 
The default is "right adjusted", a "-" changes it to "left adjusted".  Example: 
\begin{verbatim}
set,format="22.14e";
\end{verbatim} 
changes the current floating point format to 22.14e; the other formats remain untouched. 
\begin{verbatim}
set,format="s","d","g";
\end{verbatim} 
sets all formats to automatic adjustment according to C conventions. 

\subsubsection{2) Sequence} The second command lets you choose the
current sequence without having to use the "USE" command, which would
bring you back to a bare lattice without errors. The command only works
if the chosen sequence had been activated before with the "USE" command,
otherwise a warning will be issued and MAD-X will continue with the
unmodified current sequence. This command is particularly useful for
commands that do not have the sequence as an argument like "EMIT" or
"IBS". 

%\href{http://www.cern.ch/Hans.Grote/hansg_sign.html}{hansg}, 
%\href{http://www.cern.ch/Frank.Frank/frs_sign.html}{frs}, June 18, 2003 


