%%\title{Element Classes}
%  Changed by: Chris ISELIN, 24-Jan-1997 
%  Changed by: Hans Grote, 25-Sep-2002

\section{Element Classes}  
The concept of element classes solves the problem of addressing
instances of elements in the accelerator in a convenient manner. It will
first be explained by an example. All the quadrupoles in the accelerator
form a class QUADRUPOLE. Let us define three subclasses for the
focussing quadrupoles, the defocussing quadrupoles, and the skewed
quadrupoles:  
\begin{verbatim}
MQF: QUADRUPOLE, L = LQM, K1 = KQD;     ! Focussing quadrupoles
MQD: QUADRUPOLE, L = LQM, K1 = KQF;     ! Defocussing quadrupoles
MQT: QUADRUPOLE, L = LQT;               ! Skewed quadrupoles
\end{verbatim} 

These classes can be thought of as new keywords which define elements
with specified default attributes. We now use theses classes to define
the real quadrupoles:  
\begin{verbatim}
QD1: MQD;           ! Defocussing quadrupoles
QD2: MQD;
QD3: MQD;
 ...
QF1: MQF;           ! Focussing quadrupoles
QF2: MQF;
QF3: MQF;
 ...
QT1: MQT, K1S = KQT1;   ! Skewed quadrupoles
QT2: MQT, K1S = KQT2;
 ...
\end{verbatim} 

These quadrupoles inherit all unspecified attributes from their
class. This allows to build up a hierarchy of objects with a rather
economic input structure.  

The full power of the class concept is revealed when object classes are
used to select instances of elements for various purposes. Example:  
\begin{verbatim}
select, flag = twiss, class = QUADRUPOLE; ! Select all quadrupoles for the
                                          ! Twiss TFS file
\end{verbatim}

More formally, for each element keyword MAD maintains a class of
elements with the same name. A defined element becomes itself a class
which can be used to define new objects, which will become members of
this class. A new object inherits all attributes from its class; but its
definition may override some of those values by new ones. All class and
object names can be used in range selections, providing a powerful
mechanism to specify positions for matching constraints and printing.  

 %\href{http://www.cern.ch/Hans.Grote/hansg_sign.html}{hansg}, January 24, 1997 
