%%\title{Element Input Format}
%  Changed by: Chris ISELIN, 24-Jan-1997 

%  Changed by: Hans Grote, 22-Jan-2003 

%%\usepackage{hyperref}
% commands generated by html2latex


%%\begin{document}
%%\begin{center}
 %%EUROPEAN ORGANIZATION FOR NUCLEAR RESEARCH 
%%\includegraphics{http://cern.ch/madx/icons/mx7_25.gif}

\subsection{Element Input Format}
%%\end{center}

 All physical elements are defined by statements of the form 
\begin{verbatim}

label: keyword {,attribute};
\end{verbatim} Example: 
\begin{verbatim}

QF: QUADRUPOLE,L=1.8,K1=0.015832;
\end{verbatim} where 
\begin{itemize}
	\item \href{label.html}{label} is a name to be given to the element (in the example QF), 
	\item \href{keyword.html}{keyword} is an element type keyword (in the example QUADRUPOLE). 
	\item \href{attribute.html}{attribute} normally has the form "attribute-name=attribute-value" or "attribute-name:=attribute-value" (except for multipoles). 
	\item \href{label.html}{attribute-name} selects the attribute, as defined for the element type keyword (in the example L and K1). 
	\item \href{attribute.html}{attribute-value} gives it a value (in the example 1.8 and 0.015832). The value may be specified by an expression. The "=" assigns the value on the right to the attribute at the time of definition, regardless of any further change of the right hand side; the ":=" re-evaluates the expression at the right every time the attribute is being used. For constant right hand sides, "=" and ":=" are of course equivalent. 
\end{itemize} Omitted attributes are assigned a default value, normally zero. 

 A special format is used for a \href{multipole.html}{multipole}: 
\begin{verbatim}

m:multipole, kn= {kn0, kn1, kn2, ..., knmax},
             ks= {ks0, ks1, ks2, ..., ksmax};
\end{verbatim} where kn and ks give the integrated normal and skew strengths, respectively. The commas are mandatory, each strength can be an expression; their position defines the order. example: 
\begin{verbatim}

m:multipole, kn={0,0,0,myoct*lrad}, ks={0,0,0,0,-1.e-5};
\end{verbatim} defines a multipole with a normal octupole, and a skew decapole component. 

 To know the current maximum order, enter the command 
\begin{verbatim}

help,multipole;
\end{verbatim} and count. 

\href{http://www.cern.ch/hansg/hansg_sign.html}{hansg}, January 24, 1997 

%%\end{document}
