%%\title{BV flag in the Beam command}
%  Changed by: Thys Risselada 29-Mar-2009 

\subsection{Effect of the bv flag in MAD-X}

When reversing the direction ("V") of a particle in a magnetic field
("B") while keeping its charge constant, the resulting force V * B
changes sign. This is equivalent to flipping the field, but not the
direction.  

For practical reasons the properties of all elements of the LHC are
defined in the MADX input as if they apply to a clockwise proton beam
("LHC beam 1"). This allows a single definition for elements traversed
by both beams. Their effects on a beam with identical particle charge
but running in the opposite direction ("LHC beam 2") must then be
reversed inside the program.  

In MADX this may be taken into account by setting the value of the BV
attribute in the Beam commands. In the case of LHC beam 1 (clockwise)
and beam 2 (counter-clockwise), treated in MADX both as clockwise proton
beams, the Beam commands must look as follows: 

\begin{verbatim}
   beam, sequence = lhcb1, particle = proton, pc = 450, bv = +1;
   beam, sequence = lhcb2, particle = proton, pc = 450, bv = -1;
\end{verbatim}

%\href{http://www.cern.ch/Frank.Schmidt/frs_sign.html}{frs}, March 29, 2009 
