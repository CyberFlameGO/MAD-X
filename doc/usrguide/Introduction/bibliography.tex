%%\title{References}
%  Changed by: Chris ISELIN, 18-Dec-1997 

%  Changed by: Hans Grote, 10-Jun-2002 

%%\usepackage{hyperref}
% commands generated by html2latex


%%\begin{document}
%%\begin{center}
 %%EUROPEAN ORGANIZATION FOR NUCLEAR RESEARCH 
%%\includegraphics{http://cern.ch/madx/icons/mx7_25.gif}

\subsection{References}
%%\end{center}
\begin{itemize}
	\item \href{gks}{\textit{1}}\textit{The Graphical Kernel System (GKS)}. ISO, Geneva, July 1985. International Standard ISO 7942. 


	\item \href{autin}{\textit{2}} B. Autin and Y. Marti. \textit{Closed Orbit Correction of Alternating Gradient Machines   using a small Number of Magnets}. CERN/ISR-MA/73-17, CERN, 1973. 


	\item \href{barber}{\textit{3}} D.P. Barber, K. Heinemann, H. Mais and G. Ripken, \textit{A Fokker--Planck Treatment of Stochastic Particle Motion within   the Framework of a Fully Coupled 6-dimensional Formalism for   Electron-Positron Storage Rings including Classical Spin Motion in   Linear Approximation}, DESY report 91-146, 1991. 


	\item \href{bartolini}{\textit{4}} R. Bartolini, A. Bazzani, M. Giovannozzi, W. Scandale and E. Todesco, \textit{Tune evaluation in simulations and experiments}, CERN SL/95-84 (AP) (1995). 


	\item \href{bjorken}{\textit{5}} J. D. Bjorken and S. K. Mtingwa. \textit{Particle Accelerators}\textbf{13}, pg. 115. 


	\item \href{moon}{\textit{6}} E. M. Bollt and J. D. Meiss, \textit{Targeting chaotic orbits to the Moon through recurrence}, Phys. Lett. A 204,373 (1995). 


	\item \href{bramham}{\textit{7}} P. Bramham and H. Henke. private communication and LEP Note LEP-70/107, CERN. 


	\item \href{slac75}{\textit{8}} Karl L. Brown. \textit{A First-and Second-Order Matrix Theory for the Design   of Beam Transport Systems and Charged Particle Spectrometers}. SLAC 75, Revision 3, SLAC, 1972. 


	\item \href{transport}{\textit{9}} Karl L. Brown, D. C. Carey, Ch. Iselin, and F. Rothacker. \textit{TRANSPORT - A Computer Program for Designing Charged   Particle Beam Transport Systems}. CERN 73-16, revised as CERN 80-4, CERN, 1980. 


	\item \href{chao}{\textit{10}} A. Chao. \textit{Evaluation of beam distribution parameters in an electron   storage ring}. Journal of Applied Physics, 50:595-598, 1979. 


	\item \href{chao1}{\textit{11}} A. W. Chao and M. J. Lee. \textit{SPEAR II Touschek lifetime}. SPEAR-181, SLAC, October 1974. 


	\item \href{conte}{\textit{12}} M. Conte and M. Martini. \textit{Particle Accelerators}\textbf{17}, 1 (1985). 


	\item \href{courant}{\textit{13}} E. D. Courant and H. S. Snyder. \textit{Theory of the alternating gradient synchrotron}. Annals of Physics, 3:1-48, 1958. 


	\item \href{tfs}{\textit{14}} Ph. Defert, Ph. Hofmann, and R. Keyser. \textit{The Table File System, the C Interfaces}. LAW Note 9, CERN, 1989. 


	\item \href{donald}{\textit{15}} M. Donald and D. Schofield. \textit{A User's Guide to the \texttt{HARMON} Program}. LEP Note 420, CERN, 1982. 


	\item \href{dragt}{\textit{16}} A. Dragt. \textit{Lectures on Nonlinear Orbit Dynamics}, 1981 Summer School on High   Energy Particle Accelerators, Fermi National Accelerator Laboratory, July   1981. American Institute of Physics, 1982. 


	\item \href{edwards}{\textit{17}} D. A. Edwards and L. C. Teng. \textit{Parametrisation of linear coupled motion in periodic systems}. IEEE Trans. on Nucl. Sc., 20:885, 1973. 


	\item \href{giovanozzi}{\textit{18}} M. Giovannozzi, \textit{Analysis of the stability domain of planar symplectic maps using invariant manifolds}, CERN/PS 96-05 (PA) (1996). 


	\item \href{gxplot}{\textit{19}} H. Grote. \textit{GXPLOT User's Guide and Reference Manual}. LEP TH Note 57, CERN, 1988. 


	\item \href{lep}{\textit{20}} LEP Design Group. \textit{Design Study of a 22 to 130 GeV electron-positron Colliding Beam Machine   (LEP)}. CERN/ISR-LEP/79-33, CERN, 1979. 


	\item \href{beamparam<i}{21} M. Hanney, J. M. Jowett, and E. Keil. \textit{BEAMPARAM - A program for computing beam dynamics and   performance of electron-positron storage rings}. CERN/LEP-TH/88-2, CERN, 1988. 


	\item \href{helm}{\textit{22}} R. H. Helm, M. J. Lee, P. L. Morton, and M. Sands. \textit{Evaluation of synchrotron radiation integrals}. IEEE Trans. Nucl. Sc., NS-20, 1973. 


	\item \href{minuit}{\textit{23}} F. James. \textit{MINUIT, A package of programs to minimise a function of n   variables, compute the covariance matrix, and find the true errors}. program library code D507, CERN, 1978. 


	\item \href{keil}{\textit{24}} E. Keil. \textit{Synchrotron radiation from a large electron-positron storage ring}. CERN/ISR-LTD/76-23, CERN, 1976. 


	\item \href{knuth}{\textit{25}} D. E. Knuth. \textit{The Art of Computer Programming}. Volume 2, Addison-Wesley, second edition, 1981. Semi-numerical Algorithms. 


	\item \href{laskar}{\textit{26}} J. Laskar, C. Froeschle and A. Celletti, \textit{The measure of chaos by the numerical analysis of the fundamental   frequencies. Application to the standard mapping}, Physica D 56, 253 (1992). 


	\item \href{mais}{\textit{27}} H. Mais and G. Ripken, \textit{Theory of Coupled Synchro-Betatron Oscillations}. DESY internal Report, DESY M-82-05, 1982. 


	\item \href{malika}{\textit{28}} M. Meddahi, \textit{Chromaticity correction for the 108/60 degree lattice}, CERN SL/Note 96-19 (AP) (1996). 


	\item \href{ruggiero}{\textit{29}} J. Milutinovic and S. Ruggiero. \textit{Comparison of Accelerator Codes for a RHIC Lattice}. AD/AP/TN-9, BNL, 1988. 


	\item \href{montague}{\textit{30}} B. W. Montague. \textit{Linear Optics for Improved Chromaticity Correction}. LEP Note 165, CERN, 1979. 


	\item \href{ripken}{\textit{31}} Gerhard Ripken, \textit{Untersuchungen zur Strahlf\"uhrung und Stabilit\"at der Teilchenbewegung in Beschleunigern und Storage-Ringen unter strenger Ber\"ucksichtigung einer Kopplung der Betatronschwingungen}. DESY internal Report R1-70/4, 1970. 


	\item \href{chamonix96}{\textit{32}} F. Ruggiero, \textit{Dynamic Aperture for LEP 2 with various optics and tunes}, Proc. Sixth Workshop on LEP Performance, Chamonix, 1996, ed. J. Poole (CERN SL/96-05 (DI),1996), pp. 132--136. 


	\item \href{teng}{\textit{33}} L. C. Teng. \textit{Concerning n-Dimensional Coupled Motion}. FN 229, FNAL, 1971. 


	\item \href{voelkel}{\textit{34}} U. V\"olkel. \textit{Particle loss by Touschek effect in a storage ring}. DESY 67-5, DESY, 1967. 


	\item \href{walker}{\textit{35}} R. P. Walker. \textit{Calculation of the Touschek lifetime in electron storage rings}. 1987. Also SERC Daresbury Laboratory preprint, DL/SCI/P542A. 


	\item \href{wilson}{\textit{36}} P. B. Wilson. \textit{Proc. 8th Int. Conf. on High-Energy Accelerators}. Stanford, 1974. 


	\item \href{wrulich}{\textit{37}} A. Wrulich and H. Meyer. \textit{Life time due to the beam-beam bremsstrahlung effect}. PET-75-2, DESY, 1975. 


	\item \href{SXF}{\textit{38}}  H. Grote, J. Holt, N. Malitsky, F. Pilat, R. Talman, C.G. Trahern. \textit{SXF (Standard eXchange Format): definition, syntax, examples}. RHIC/AP/155, August, 1998.  




	\item \href{SixTrack}{\textit{39}} F. Schmidt. \textit{SixTrack, User's Reference Manual}. CERN SL/94-56 (AP). 




	\item \href{SixTrack_Run_Environment}{\textit{40}} M. Hayes and F. Schmidt. \textit{Run Environment for SixTrack}. Physics Note 53 (unpublished) \& LHC Project Note 300. 




	\item \href{SODD}{\textit{41}} F. Schmidt. \textit{SODD: A computer code to calculate detuning and distortion function terms in first and second order}. CERN SL/Note 99-009 (AP). 




	\item \href{TEAPOT}{\textit{42}} R.Talman and L.Schachinger. \textit{TEAPOT. A Thin Element Accelerator Program for Optics and Tracking}. SSC-52. 




	\item \href{bm1}{\textit{43}} J.D. Bjorken and S.K. Mtingwa,  \textit{Intrabeam Scattering},  FERMILAB-Pub-82/47-THY, July 1982.  
\end{itemize}

\line(1,0){300}
\\
\href{http://www.cern.ch/Frank.Schmidt/frs_sign.html}{frs}, May 01, 2003 %] ] ] ] ] ] ] ] ] ] ] ] ] ] ] ] ] ] ] ] ] ] ] ] ] ] ] ] ] ] ] ] ] ] ] ] ] ] ] ] ] ] ] 

%%\end{document}
