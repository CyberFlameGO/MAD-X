%%\title{LINE}
%  Changed by: Chris ISELIN, 27-Jan-1997 
%  Changed by: Hans Grote, 11-Jun-2002 


\chapter{Beam Lines}
\label{chap:lines}

The accelerator to be studied is known to MAD-X either as a sequence of
physical elements called \href{sequence.html}{sequence}, or as a
hierarchically structured list of elements called a \emph{beam line}. A
beam line is built from simpler beam lines whose definitions can be
nested to any level. A powerful syntax allows to repeat, to reflect, or
to replace pieces of beam lines. However, internally MAD-X knows only
sequences, and lines as shown below are converted into flat sequences
with the same name when they are expanded. Consequently, only sequences
can be SAVEd onto a file (see
\href{../control/general.html#save}{save}).  

Formally a beam line is defined by a LINE command: 
\begin{verbatim}
label(arg{,arg}): LINE=(member{,member});
\end{verbatim}
\href{label.html}{Label} gives a name to the beam line for later reference. 

The formal argument list (arg\{,arg\}) is optional (see below). Each
"member" may be one of the following:  
\begin{itemize}
	\item  Element label, 
	\item  Beam line label, 
	\item  Sub-line, enclosed in parentheses, 
	\item  Formal argument name, 
	\item  Replacement list label. 
\end{itemize} 
Beam lines may be nested to any level.  

\section{\href{simple}{Simple Beam Lines}} 
The simplest beam line consists of single elements: 
\begin{verbatim}
label: LINE=(member{,member});
\end{verbatim} 
Example: 
\begin{verbatim}
l:      line=(a,b,c,d,a,d);
        use,period=l;
\end{verbatim} 
The \href{../control/general.html#use}{USE} command tells MAD to perform
all subsequent calculations on the sequence  
\begin{verbatim}
a,b,c,d,a,d
\end{verbatim}

\section{\href{subline}{Sub-lines}} 
Instead of referring to an element, a beam line member can refer to
another beam line defined in a separate command. This provides a
shorthand notation for sub-lines which occur several times in a beam
line. Lines and sub-lines can be entered in any order, but when a line
is expanded, all its sub-lines must be known.  

Example: 
\begin{verbatim}
l:      line=(a,b,s,b,a,s,a,b);
s:      line=(c,d,e);
        use,period=l;
\end{verbatim} 
this example produces the following expansion steps: 

1. replace sub-line s: 
\begin{verbatim}
(a,b,(c,d,e),b,a,(c,d,e),a,b)
\end{verbatim}

2. omit parentheses: 
\begin{verbatim}
a,b,c,d,e,b,a,c,d,e,a,b
\end{verbatim}

\section{\href{reflect}{Reflection and Repetition}} 
An unsigned repetition count and an asterisk indicate repetition of a
beam line member. A minus prefix causes reflection, i.e. all elements in
the subsequence are taken in reverse order. Sub-lines of reflected lines
are also reflected, but physical elements are not. If both reflection
and repetition are desired, the minus sign must precede the repetition
count.  

Example: 
\begin{verbatim}
r:      line=(g,h);
s:      line=(c,r,d);
t:      line=(2*s,2*(e,f),-s,-(a,b));
        use,period=t;
\end{verbatim}
Attention: the repetition "2*s" will only work if
"s" is itself a line. In case "s" is an element replace by
"2*(s)".  Proceeding step by step, this example produces 

1. Replace sub-line S: 
\begin{verbatim}
((c,r,d),(c,r,d),(e,f),(e,f),(d,-r,c),(b,a))
\end{verbatim}

2. replace sub-line r: 
\begin{verbatim}
((c,(g,h),d),(c,(g,h),d),(e,f),(e,f),(d,(h,g),c),(b,a))
\end{verbatim}

3. omit parentheses: 
\begin{verbatim}
c,g,h,d,c,g,h,d,e,f,e,f,d,h,g,c,b,a
\end{verbatim}  
Note that the inner sub-line R is reflected together with the outer
sub-line S.   

\section{\href{argument}{Replaceable Arguments}} 
A beam line definition may contain a formal argument list, consisting of
labels separated by commas and enclosed in parentheses. Such a line can
be expanded for different values of its arguments. When this line is
referred to, its label must be followed by a list of actual arguments
separated by commas and enclosed in parentheses. These arguments must be
beam line, or element names. The number of actual arguments must agree
with the number of formal arguments. All occurrences of a formal
argument on the right-hand side of the line definition are replaced by
the corresponding actual argument.  

Example: 
\begin{verbatim}
s:        line=(a,b,c);
l(x,y):   line=(d,x,e,3*y);
l4f:      line=(4*f);
lm2s:     line=(-2*s);
res:      line=l(l4f,lm2s);
\end{verbatim} 

Proceeding step by step, this example generates the expansion 
\begin{verbatim}
d,f,f,f,f,e,c,b,a,c,b,a,c,b,a,c,b,a,c,b,a,c,b,a
\end{verbatim} 

Second example: 
\begin{verbatim}
cel(sf,sd):     line=(qf,d,sf,d,b,d,qd,d,sd,d,b,d);
arc:            line=(cel(sf1,sd1),cel(sf2,sd2),cel(sf1,sd1));
                use,period=arc;
\end{verbatim} 
This example generates the expansion 

 1. Replace the line CEL and its formal arguments: 
\begin{verbatim}
((qf,d,(sf1),d,b,d,qd,d,(sd1),d,b,d)
 (qf,d,(sf2),d,b,d,qd,d,(sd2),d,b,d)
 (qf,d,(sf1),d,b,d,qd,d,(sd1),d,b,d))
\end{verbatim}

 2. Omit parentheses: 
\begin{verbatim}
qf,d,sf1,d,b,d,qd,d,sd1,d,b,d
qf,d,sf2,d,b,d,qd,d,sd2,d,b,d
qf,d,sf1,d,b,d,qd,d,sd1,d,b,d
\end{verbatim}

\section{\href{Line_Depreciation}{Warning: Line Depreciation}} 
MADX has been devolopped using sequences, in fact internally the code
works with sequences only. Consequently, there may exist some
inconveniences when only lines are used. It is recommended to convert as
soon as possible lines into sequences (by means of the save command) in
a design phase and to use only sequences for a finalised machine.    

\section{\href{Line_Limits}{ Limits of Construction of Lines}}  
Since Lines are in fact depreciated there are some limits of how they
can be constructed. Please find below a running MADX run which shows an
example of OK (valid) and WRONG (invalid) cases.   

\begin{verbatim}
!----------------------------------------------------------------------
beam, PARTICLE=electron, energy=1;

qf: QUADRUPOLE, L:=1,K1:=1;
qd: QUADRUPOLE, L:=1,K1:=-1;
d: DRIFT, l=1;
m: MARKER;

rpl(a,b): LINE=(a,b);
sl: LINE=(qf,d,qd);
test0: LINE=(rpl(sl,sl));          !OK 
test1: LINE=(rpl((sl),(sl)));      !OK
test2: LINE=(rpl((sl),(-sl)));     !OK
test3: LINE=(sl,-sl);              !OK
test4: LINE=(rpl((3*sl),(3*sl)));  ! WRONG
test5: LINE=(3*sl,3*sl);           !OK
test6: LINE=(rpl((3*sl),(-3*sl))); ! WRONG
test7: LINE=(3*sl,-3*sl);          !OK

use, period=test0;
twiss,BETX=1,bety=1;

use, period=test1;
twiss,BETX=1,bety=1;

use, period=test2;
twiss,BETX=1,bety=1;

use, period=test3;
twiss,BETX=1,bety=1;

use, period=test4;
twiss,BETX=1,bety=1;

use, period=test5;
twiss,BETX=1,bety=1;

use, period=test6;
twiss,BETX=1,bety=1;

use, period=test7;
twiss,BETX=1,bety=1;
!----------------------------------------------------------------------
\end{verbatim}

%\href{http://www.cern.ch/Hans.Grote/hansg_sign.html}{hansg}, June 17, 2002 

