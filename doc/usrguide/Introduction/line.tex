%%\title{LINE}
%  Changed by: Chris ISELIN, 27-Jan-1997 
%  Changed by: Hans Grote, 11-Jun-2002 


\chapter{Beam Lines}
\label{chap:beamlines}

The accelerator to be studied is known to \madx either as a sequence of
physical elements called \href{sequence.html}{sequence}, or as a
hierarchically structured list of elements called a \emph{beam line}. 
A beam line is built from simpler beam lines whose definitions can be
nested to any level. A powerful syntax allows to repeat, to reflect, or
to replace pieces of beam lines. However, internally \madx knows only
sequences, and lines as shown below are converted into flat sequences
with the same name when they are expanded. Consequently, only sequences
can be SAVEd onto a file.

Formally a beam line is defined by a LINE command: 
\madbox{
label( arg {,arg} ): LINE=( member {,member} );
}
\href{label.html}{Label} gives a name to the beam line for later reference. 

The formal argument list (arg\{,arg\}) is optional (see below). Each
"member" may be one of the following:  
\begin{itemize}
\item  Element label, 
\item  Beam line label, 
\item  Sub-line, enclosed in parentheses, 
\item  Formal argument name, 
\item  Replacement list label. 
\end{itemize} 

Beam lines may be nested to any level (see \ref{sec:sublines})  


{\bf Warning:}\\
\madx has been developped using sequences and machine description using
beamlines was kept to a minimal extent to keep backward compatibility
but also because it is often easier to start from beamlines when
designing an accelerator from scratch.   
But since \madx works only with sequences internally, there may exist
some inconveniences when only beamlines are defined by the user. It is
therefore recommended to convert beamlines into sequences (e.g. by means
of the \texttt{SAVE} command) as soon as possible lines in the design
phase and to use only sequences for a finalised machine.     

Attempting to do sequence edition on a sequence that has
been expanded in memory from a beam line specified by the user is
strongly discouraged.

\section{Simple Beam Lines} 
\label{sec:beamline}
The simplest beam line consists of single elements: 
\madbox{
label: LINE= ( member \{,member\} );
}
Example: 
\madxmp{
exmp1: \=LINE=(a,b,c,d,a,d); \\
USE,   \>PERIOD=exmp1;
}

The \texttt{USE} command causes \madx to load and expand the specified
beamline. This in particular loads the beamline as a SEQUENCE
representation in memory. All subsequent calculations are done on this
sequence in memory.  

\section{Nested Beam Lines}
\label{sec:sublines}

Instead of referring to an element, a beam line member can refer to
another beam line defined in a separate command. This provides a
shorthand notation for sub-lines or nested beam lines which can occur
several times in a beam line. 
Lines and sub-lines can be entered in any order, but when a line is
expanded, all its sub-lines must be known.   

Example: 
\madxmp{
top: \=LINE=(a,b,S,b,a,S,a,b); \\
S:   \>LINE=(c,d,e); \\
USE, \>PERIOD=top;
}
produces the following expansion steps: 
\begin{enumerate}
  \item Replace sub-line S: 
    \madxmp{(a, b, (c,d,e), b, a, (c,d,e), a, b)}
  \item Omit parentheses: 
    \madxmp{a, b, c, d, e, b, a, c, d, e, a, b}
\end{enumerate}

\section{Reflection and Repetition} 
\label{sec:reflect_repeat_lines}
An unsigned repetition count and an asterisk (multiplication sign)
indicate repetition of a beam line member. 

A minus prefix causes reflection, i.e. all elements in the sub-line are
taken in reverse order. 
Sub-lines of reflected lines are also reflected, but physical elements
are not. 

If both reflection and repetition are desired, the minus sign must
precede the repetition count.  

Example: 
\madxmp{
xxxxx\= \kill
R:   \>LINE= (g, h); \\
S:   \>LINE= (c, R, d); \\
top: \>LINE= (2*S, 2*(e,f), -S, -(a,b)); \\
USE, \>PERIOD=top;
}
produces the following expansion steps:
\begin{enumerate}
  \item Replace sub-line S: 
    \madxmp{( (c,R,d), (c,R,d), (e,f), (e,f), (d,-R,c), (b,a))}
  \item Replace sub-line R: 
    \madxmp{( (c, (g,h), d), (c, (g,h), d), (e,f), (e,f), (d, (h,g), c), (b,a))}
  \item Omit parentheses: 
    \madxmp{c, g, h, d, c, g, h, d, e, f, e, f, d, h, g, c, b, a}
\end{enumerate}

Note that the inner sub-line R is reflected together with the outer
sub-line S.   

Attention: the repetition "2*S" only works if "S" is itself a
line. If "S" is a single element, the repetition syntax should  "2*(S)".
 


\section{Replaceable Arguments}
\label{sec:repl_args}

A beam line definition may contain a formal argument list, consisting of
labels separated by commas and enclosed in parentheses. 

\madbox{
label( arg \{,arg\} ): LINE= ( member \{,member\} );
}
where \texttt{member} can be one of the \texttt{arg} arguments.

A beam line defined with arguments can be expanded for different values
of its arguments.  
The arguments can be beam lines or element names. The number of actual
arguments given at time of use must agree with the number of formal
arguments defined at declaration time. All occurrences of a formal
argument on the right-hand side of the line definition are replaced by
the corresponding actual argument.  

Example:
\madxmp{
fodo(Q1,Q2): \=LINE=(Q1, d, Q2, d);\\
top: \> LINE = ( fodo(qf12, qd23), fodo(qd34,qf45) );\\
USE, PERIOD=top;
}
expands to 
\madxmp{qf12, d, qd23, d, qd34, d, qf45, d}

Example showing that arguments can also be beam lines:
\madxmp{
xxxxxxxx\= \kill
s:        \>LINE = (a,b,c); \\
l(x,y):   \>LINE = (d,x,e,3*y); \\
l4f:      \>LINE = (4*f); \\
lm2s:     \>LINE = (-2*s); \\
res:      \>LINE = l(l4f,lm2s); \\
}

Proceeding step by step, this example generates the expansion 
\madxmp{d, f, f, f, f, e, c, b, a, c, b, a, c, b, a, c, b, a, c, b, a, c, b, a}


\section{Limits of Construction of Beam Lines}  

Since Lines are in fact depreciated there are some limits of how they
can be constructed. Please find below a running \madx run which shows an
example of OK (valid) and WRONG (invalid) cases.   

\begin{verbatim}
!----------------------------------------------------------------------
beam, PARTICLE=electron, energy=1;

qf: QUADRUPOLE, L:=1,K1:=1;
qd: QUADRUPOLE, L:=1,K1:=-1;
d: DRIFT, l=1;
m: MARKER;

rpl(a,b): LINE=(a,b);
sl: LINE=(qf,d,qd);
test0: LINE=(rpl(sl,sl));          !OK 
test1: LINE=(rpl((sl),(sl)));      !OK
test2: LINE=(rpl((sl),(-sl)));     !OK
test3: LINE=(sl,-sl);              !OK

test4: LINE=(rpl((3*sl),(3*sl)));  ! WRONG

test5: LINE=(3*sl,3*sl);           !OK

test6: LINE=(rpl((3*sl),(-3*sl))); ! WRONG

test7: LINE=(3*sl,-3*sl);          !OK

use, period=test0;
twiss,BETX=1,bety=1;

use, period=test1;
twiss,BETX=1,bety=1;

use, period=test2;
twiss,BETX=1,bety=1;

use, period=test3;
twiss,BETX=1,bety=1;

use, period=test4;
twiss,BETX=1,bety=1;

use, period=test5;
twiss,BETX=1,bety=1;

use, period=test6;
twiss,BETX=1,bety=1;

use, period=test7;
twiss,BETX=1,bety=1;
!----------------------------------------------------------------------
\end{verbatim}

%\href{http://www.cern.ch/Hans.Grote/hansg_sign.html}{hansg}, June 17, 2002 

