%%\title{KICK, HKICK, VKICK}
%  Changed by: Chris ISELIN, 27-Jan-1997 
%  Changed by: Hans Grote, 30-Sep-2002 
%  Changed by: Frank Schmidt, 28-Aug-2003 
%  Changed by: Werner Herr, 22-May-2007 

\section{Closed Orbit Correctors}
  
Three types of closed orbit correctors are available: 
\begin{itemize}
   \item \href{hkick}{HKICKER}, a corrector for the horizontal plane, 
   \item \href{vkick}{VKICKER}, a corrector for the vertical plane, 
   \item \href{kick}{KICKER}, a corrector for both planes. 
\end{itemize}

\begin{verbatim}
label: HKICKER, L = real, KICK = real, TILT = real;
label: VKICKER, L = real, KICK = real, TILT = real;
label:  KICKER, L = real, HKICK = real, VKICK = real, TILT = real;
\end{verbatim} 

{\bf The type KICKER should not be used when an orbit corrector kicks
  only in one plane.}  

The attributes have the following meaning: 
\begin{itemize}
   \item L: The length of the closed orbit corrector (default: 0 m). 
   \item KICK: The kick angle for either horizontal or vertical correctors. (default: 0 rad). 
   \item HKICK: The horizontal kick angle for a corrector in both planes (default: 0 rad). 
   \item VKICK: The vertical kick angle for a corrector in both planes (default: 0 rad). 
   \item TILT: The roll angle about the longitudinal axis (default: 0
     rad). A positive angle represents a clockwise rotation of the
     kicker.  
\end{itemize} 

A positive kick increases \textit{p$_x$} or \textit{p$_y$}
respectively. This means that a positive horizontal kick bends to the
left,  i.e. to positive x which is opposite of what is true for bends.   

It should be noted that the kick values assigned to an orbit corrector
like above are not overwritten by an orbit correction using the CORRECT
command. Instead the kicks computed by an orbit correction and the
assigned values are added when the correctors are used.  

 Examples: 
\begin{verbatim}
HK1:   HKICKER, KICK = 0.001;
VK3:   VKICKER, KICK = 0.0005;
VK4:   VKICKER, KICK := AVK4;
KHV1:  KICKER,  HKICK = 0.001, VKICK = 0.0005;
KHV2:  KICKER,  HKICK := AKHV2H, VKICK := AKHV2V;
\end{verbatim} 

The assignment in the form of a deferred expression has the advantage
that the values can be assigned and/or modified at any time (and matched
!).  

The \href{local_system.html#straight}{straight reference system} for an
orbit corrector is a Cartesian coordinate s ystem.  

Please note that there is a new feature introduced by Stefan Sorge from
GSI. Here his decription:

The elements KICKER, HKICKER, and VKICKER can also be used as  an
exciter providing a sinusoidal momentum kick. The usage in this case is   

\begin{verbatim}
xykick: KICKER,  SINKICK = integer, SINPEAK = real, SINTUNE = real, SINPHASE = real;  
xkick : HKICKER, SINKICK = integer, SINPEAK = real, SINTUNE = real, SINPHASE = real;  
ykick : VKICKER, SINKICK = integer, SINPEAK = real, SINTUNE = real, SINPHASE = real;  
\end{verbatim}
where a sinusoidal momentum kick dpz as a function of the  revolution
number n given by\\   
dpz(n)=SINPEAK * sin(2*PI*SINTUNE*n + SINPHASE), pz=px,py \\ 
is provided. 

The variables are 

\begin{itemize}
   \item SINKICK - integer, must be set to 1 to switch on the sinusoidal
     signal, default: 0.  
   \item SINPEAK - amplitude of the bending angle (rad), default: 0 rad.  
   \item SINTUNE - frequency of the signal times the revolution
     frequency.  Hence, the phase per revolution is 2*PI*SINTUNE,
     default: 0.   
   \item SINPHASE - initial phase, default: 0 rad.  
   \item KICKER generates a kick in horizontal and a kick vertical
     direction,  where both are synchron, HKICKER generates a horizontal
     kick,  and VKICKER generates a vertical kick.   
\end{itemize}

The momentum kick of a kicker has only a single frequency. An element
having a finite bandwidth can approximately created by defining  thin
kickers with all amplitudes SINPEAK, frequencies SINTUNE, and  initial
phases SINPHASE desired and putting them at the same position s in  the
accelerator.   

From S.Sorge@gsi.de  

%\href{http://www.cern.ch/Hans.Grote/hansg_sign.html}{hansg}, 
%\href{http://www.cern.ch/Frank.Schmidt/frs_sign.html}{frs}, August 28, 2003  
