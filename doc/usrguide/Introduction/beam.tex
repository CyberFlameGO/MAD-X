%%\title{BEAM}
%  Changed by: Hans Grote, 30-Sep-2002 

\section{BEAM: Set Beam Parameters}

Many commands in MAD-X require the setting of various quantities related
to the beam in the machine. Therefore, MAD-X will stop with a fatal
error if an attempt is made to expand (USE) a sequence for which no BEAM
command has been issued before.  

The quantities are entered by a BEAM command: 
\begin{verbatim}
BEAM, PARTICLE=name,MASS=real,CHARGE=real,
      ENERGY=real,PC=real,GAMMA=real,
      EX=real,EXN=real,EY=real,EYN=real,
      ET=real,SIGT=real,SIGE=real,
      KBUNCH=integer,NPART=real,BCURRENT=real,
      BUNCHED=logical,RADIATE=logical,BV=integer,SEQUENCE=name;
\end{verbatim} 

{\bf Warning:} BEAM updates, i.e. it replaces attributes explicitely
mentioned, but does not return to default values for others! To reset to
\href{resbeam.html#defaults}{beam value defaults},  use
\href{resbeam.html}{RESBEAM}.

The particle restmass and \href{charge}{charge} are defined by:
\begin{itemize}
   \item \href{particle}{PARTICLE}: The name of particles in the
     machine. MAD knows the restmass and the charge for the
     following particles:  
     \begin{itemize}
	  \item POSITRON: The particles are positrons (MASS=\textit{m$_e$}, CHARGE=1), 
	  \item ELECTRON: The particles are electrons (MASS=\textit{m$_e$}, CHARGE=-1), 
	  \item PROTON: The particles are protons (MASS=\textit{m$_p$}, CHARGE=1), 
	  \item ANTIPROTON: The particles are anti-protons (MASS=\textit{m$_p$}, CHARGE=-1). 
	  \item POSMUON: The particles are positive muons (MASS=\textit{m$_mu$}, CHARGE=1), 
	  \item NEGMUON: The particles are negative muons (MASS=\textit{m$_mu$}, CHARGE=-1). 
     \end{itemize}
\end{itemize} 

Therefore neither restmass nor charge can be modified for these
predefined particles. On the other hand, for ions and all other user
defined particles the name, restmass, and charge can be entered
independently.  

By default the total particle energy is 1 GeV. A different value can be 
defined by one of the following: 
 
\begin{itemize}
   \item \href{energy}{ENERGY}: The total energy per particle in
     GeV. If given, it must be greater then the particle restmass.  
   \item \href{pc}{PC}: The momentum per particle in GeV/c. If
     given, it must be greater than zero.  
   \item \href{gamma}{GAMMA}: The ratio between total energy and
     rest energy of the particles: GAMMA = \textit{E / m$_0$}. If
     given, it must be greater than one. If the restmass is changed
     a new value for the energy should be entered. Otherwise the
     energy remains unchanged, and the momentum PC and the quantity
     GAMMA are recalculated . 
\end{itemize}  

The emittances are defined by: 
\begin{itemize}
   \item \href{ex}{EX}: The horizontal emittance \textit{E$_x$} (default: 1 m). 
   \item \href{ey}{EY}: The vertical emittance \textit{E$_y$} (default: 1 m). 
   \item \href{et}{ET}: The longitudinal emittance \textit{E$_t$} (default: 1 m). 
\end{itemize}  

The emittances can be replaced by the normalised emittances and the
energy spread:  
\begin{itemize}
   \item \href{exn}{EXN}: The normalised horizontal emittance [m]:
     \textit{E$_xn$} = 4 (GAMMA$^2$ - 1)$^{1/2}$\textit{E$_x$} (ignored
     if \textit{E$_x$} is given).  
   \item \href{eyn}{EYN}: The normalised vertical emittance [m]:
     \textit{E$_yn$} = 4 (GAMMA$^2$ - 1)$^{1/2}$\textit{E$_y$} (ignored
     if \textit{E$_x$} is given).  
   \item \href{sigt}{SIGT}: The bunch length \textit{c}
     sigma(\textit{t}) in [m].  
   \item \href{sige}{SIGE}: The \emph{relative} energy spread
     sigma(\textit{E})/\textit{E} in [1].  
\end{itemize} 

Certain commands compute the synchrotron tune \textit{Q$_s$} from the RF
cavities. If \textit{Q$_s$} is non-zero, the relative energy spread and
the bunch length are  \\
sigma(\textit{E}) / \textit{p$_0$ c =  (2 pi Q$_s$ E$_t$ / ETA C)$^{1/2}$}, 

\textit{c} sigma(\textit{t}) = (ETA C E$_t$ / 2 pi Q$_s$)$^{1/2}$. 

\textit{C} is the machine circumference, and 

\textit{ETA} = GAMMA$^{-2}$ - GAMMA(transition)$^{-2}$. 

The order of precedence in the parameter evaluation is given below: 
\begin{verbatim}
    particle->(mass+charge)
    energy->pc->gamma->beta
    ex->exn
    ey->eyn
    current->npart
    et->sigt->sige
\end{verbatim} 

where any item to the left takes precendence over the others. 

Finally, the BEAM command accepts 
\begin{itemize}
   \item \href{kbunch}{KBUNCH}: The number of particle bunches in the
     machine (default: 1).  
   \item \href{npart}{NPART}: The number of particles per bunch (default: 0). 
   \item \href{bcurrent}{BCURRENT}: The bunch current (default: 0 A). 
   \item \href{bunched}{BUNCHED}: A logical flag. If set, the beam is
     treated as bunched whenever this makes sense.  
   \item \href{radiate}{RADIATE}: A logical flag. If set, synchrotron
     radiation is considered in all bipolar magnets.  
   \item \href{bv}{BV}: an integer specifying the direction of the
     particle movement in a beam line; either +1 (default), or -1. For a
     detailed explanation see under \href{bv_flag.html}{bv flag}.  
   \item \href{sequence}{SEQUENCE}: this attaches the beam command to a
     specific sequence; if the name is omitted, the BEAM command refers
     to the default beam always present. Sequences without attached beam
     use this default beam. When updating a beam, the corresponding
     sequence name, if any, must always be mentioned.  
\end{itemize} 

The BEAM command changes only the parameters entered. The command
\href{resbeam.html}{RESBEAM} resets all beam data to their
\href{resbeam.html#defaults}{beam value defaults}.  

Examples: 
\begin{verbatim}
BEAM, PARTICLE = ELECTRON, ENERGY = 50, EX = 1.E-6, EY = 1.E-8, SIGE = 1.E-3;
 ...
BEAM, RADIATE;
 ...
RESBEAM;
BEAM, EX = 2.E-5, EY = 3.E-7, SIGE = 4.E-3;
\end{verbatim} 

The first command selects electrons, and sets energy and emittances. The
second one turns on synchrotron radiation. The last two select positrons
(by default), set the energy to 1 GeV (default), clear the synchrotron
radiation flag, and set the emittances to the values entered.  

Some program modules of MAD-X may also store data into a beam data
block. Expressions may refer to data in this beam data block using the
notation  
\begin{verbatim}
BEAM->attribute-name
\end{verbatim} 
or 
\begin{verbatim}
BEAM%sequence->attribute-name.
\end{verbatim} 

This notation refers to the value of attribute-name found in the default
BEAM resp. the beam belonging to the sequence given. This can be used
for receiving or using values, e.g. 

\begin{verbatim}
value, beam%lhcb2->bv;
\end{verbatim} 
or for storing values in the beam (this does NOT trigger an update of dependent variables !), e.g. 
\begin{verbatim}
beam->charge=-1;
\end{verbatim} 

The current values in the BEAM bank can be obtained by the command
\begin{verbatim}
show,beam;
\end{verbatim}
resp.
\begin{verbatim}
show,beam%sequence;
\end{verbatim}


%\href{http://www.cern.ch/Hans.Grote/hansg_sign.html}{hansg} 11.9.2000 
