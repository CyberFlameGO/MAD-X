%%\title{Field Errors}
%  Changed by: Hans Grote, 19-Jun-2002 

\section{Field Errors}
Field errors can be entered as relative or absolute errors. Different
multipole components can be specified with different kinds of errors
(relative or absolute). Relations between absolute and relative field
errors are listed below.  

In MAD8 two commands were used for that purpose: EFIELD and EFCOMP. Only
EFCOMP was implemented in MAD-X since it provides the full functionality
of EFIELD and there was no need for duplication.  

All field errors are specified as the integrated value
int(\textit{K*ds}) of the \href{../Introduction/sign_convent.html}{field
  components} along the magnet axis in m$^{-i}$. There is no provision
to specify a global relative excitation error affecting all field
components in a combined function magnet. Such an error may only be
entered by defining the same relative error for all field components.  

Field errors can be specified for all magnetic elements by the statement  

\begin{verbatim}
SELECT, FLAG = ERROR, RANGE = range, CLASS = name, PATTERN = string;
EFCOMP, ORDER := integer, RADIUS := real,
        DKN  := {dkn(0), dkn(1), dkn(2),...},
        DKS  := {dks(0), dks(1), dks(2),...},
        DKNR := {dknr(0),dknr(1),dknr(2),...},
        DKSR := {dksr(0),dksr(1),dksr(2),...};
\end{verbatim}
and elements are now selected by the
\href{../Introduction/select.html}{SELECT} command. Each new
\href{efcomp}{EFCOMP} statement replaces the field errors for all
elements in its range (s). Any old field errors present in the range are
discarded or incremented depending on the setting of
\href{error_option.html}{EOPTION,ADD}. EFCOMP defines them in terms of
relative or absolute components.  

The attributes are: 
\begin{itemize}
\item ORDER: If relative errors are entered for multipoles, this defines
  the order of the base component to which the relative  errors
  refer. This reference strength \textit{k$_ref$} always refers to the
  normal component. To use a skew component as the reference the
  reference radius should be specified as a negative number. The default
  is zero.  \\
  Please note that this implies to specify \textit{k$_0$} to assign
  relative field errors to a bending magnet since \textit{k$_0$} is used
  for the normalization and NOT the ANGLE.  

\item RADIUS: Radius \textit{R} were dknr(i) or dksr(i) are specified
  for 0...i...20 (default 1 m). This attribute is required if dknr(i) or
  dksr(i) are specified. If \textit{R} is negativ, the skew component is
  used for the reference strength.  

\item DKN(i): Absolute error for the normal multipole strength with
  (2i+2) poles (default: 0  m$^{-i}$).  

\item DKS(i): Absolute error for the skewed multipole strength with
  (2i+2) poles (default: 0  m$^{-i}$).  

\item DKNR(i): Relative error for the normal multipole strength with
  (2i+2) poles (default: 0  m$^{-i}$).  

\item DKSR(i): Relative error for the skewed multipole strength with
  (2i+2) poles (default: 0  m$^{-i}$).  
\end{itemize}


\subsection{Time memory effects:}

The relative errors can be corrected for possible time memory effects. A
correction term is computed and added to the relative error. 

The correction term is parametrized as a 3rd order polynomial in the
reference strength \textit{k$_ref$} according to:  

%\begin{verbatim}
\[ \Delta = \sum (c_i * \textit{k}^{i}_{ref})            i = 0...3\]
%\end{verbatim}
The coefficients c$_i$ for the polynominal must be supplied in the
command.  

Two additional parameters and options are required: 

HYSTER: if it is set to 1 applies the correction term derived from the
reference strength and the coefficients.  

HCOEFFN and HCOEFFS: coefficients (normal and skew) for the computation
of the correction term. The 4 coefficients are specified in increasing
order, starting with the 0th order. Each group of four coefficients is
valid for one order of the field errors. Trailing zeros can be omitted,
preceding zeros must be given.  

\subsection{Examples} 

Example 1 (assign relative errors to quadrupoles); 
\begin{verbatim}
select, flag = error, pattern = "q.*";
efcomp, order := 1, radius := 0.010,
dknr := {0, 4e-1, 1e-1, 2e-3, 0, 0, 0, 0, 0, 0, 0, 0, 0, 0, 0, 0, 0, 0, 0, 0},
dksr := {0, 4e-1, 1e-1, 2e-3, 0, 0, 0, 0, 0, 0, 0, 0, 0, 0, 0, 0, 0, 0, 0, 0};
\end{verbatim}

Example 2 (add time memory effect to relative errors): 
\begin{verbatim}
select, flag = error, pattern = "^q.*";
efcomp, order = 1, radius = 0.020, hyster = 1,
hcoeffn := {0.000, 0.000, 0.000, 0.000,   // coefficients multipole order 0
            0.001, 0.000, 0.000, 0.000,   // coefficients multipole order 1
            0.000, 0.000, 0.002, 0.000},  // coefficients multipole order 2
dknr := {0, 1e-2, 2e-4, 4e-5, 1e-5, 0, 0, 0, 0, 0, 0, 0, 0, 0, 0, 0, 0, 0, 0, 0},
dksr := {0, 1e-2, 2e-4, 4e-5, 1e-5, 0, 0, 0, 0, 0, 0, 0, 0, 0, 0, 0, 0, 0, 0, 0};
\end{verbatim}

See also: \href{../Introduction/expression.html#random}{Random values}
and \href{../Introduction/expression.html#defer}{deferred expressions}.  

%\href{http://consult.cern.ch/xwho/people/1808}{Werner Herr}  6.12.2004 
