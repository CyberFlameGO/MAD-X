
\chapter{Program Flow Statements}

%% \madx consists of a core program, and modules for specific tasks such as
%% twiss parameter calculation, matching, thin lens tracking, and so on.  
 
%% The statements listed here are those executed by the program core. They
%% deal with the I/O, element and sequence declaration, sequence
%% manipulation, statement flow control (e.g. IF, WHILE), MACRO
%% declaration, saving sequences onto files in \madx or \madeight format,
%% and so on.  




\section{IF...ELSEIF...ELSE}
\label{sec:if}

\madbox{
IF (logical\_expression) \{ statements; \} \\ \\
ELSEIF (logical\_expression) \{ statements; \} \\ \\
ELSE \{ statements; \}
}

where "logical\_expression" is one of 
\madxmp{
xxxxxxx\=xxxxxxxxxxxxxxxx\=xxxxx\= \kill
expr1  \>oper expr2 \\
expr11 \>oper1 expr12   \>\&\&   \>expr21 oper2 expr22 \\
expr11 \>oper1 expr12   \>||    \>expr21 oper2 expr22 
}
and \texttt{oper} is one of 
\begin{madlist}
\ttitem{==}  equal
\ttitem{<>}  not equal
\ttitem{<}  less than
\ttitem{>}  greater than
\ttitem{<=}  less than or equal
\ttitem{>=}  greater than or equal
\end{madlist}

The expressions are arithmetic expressions of type real. The statements
in the curly brackets are executed if the logical expression is true.  

\texttt{ELSEIF} statements are only possible (in any number) behind an
\texttt{IF}, or another \texttt{ELSEIF}; the branch is executed if
"logical\_expression" is true, and if none of the preceding \texttt{IF}
or \texttt{ELSEIF} logical conditions was true.   


\texttt{ELSE} statement is only possible once behind an \texttt{IF}, or
an \texttt{ELSEIF}; the branch is executed if "logical\_expression" is
true, and if none of the preceding \texttt{IF} or \texttt{ELSEIF}
logical conditions was true.   

Because \texttt{IF ... ELSEIF ... ELSE} statements are a \madx
construct and not part of a full language, \madx  allows only one level
of inclusion of other \texttt{IF ... ELSEIF ... ELSE}, \texttt{WHILE}
or \texttt{MACRO} statements.  

For a real life example, see \href{foot.html}{ELSE example}. 

\section{WHILE}
\label{sec:while}
\madbox{
WHILE (logical\_condition) \{ statements; \}
}
executes the statements in curly brackets while the logical\_expression
is true. 

A simple example giving the value of the first ten factorials:
\madxmp{
n = 1; m = 1; \\
while (n <= 10) \{ \\
\qquad  m = m * n;  value, m; \\
\qquad  n = n + 1; \\
\};
}

Because \texttt{WHILE} statements are a \madx
construct and not part of a full language, \madx  allows only one level
of inclusion of another \texttt{IF ... ELSEIF ... ELSE}, \texttt{WHILE}
or \texttt{MACRO} statements.  


For a real life example, see \href{foot.html}{WHILE example}.

\section{MACRO}
\label{sec:macro}

The MACRO construct allows the execution of a group of statements via a
single command. Optionally the MACRO construct takes arguments.

\madbox{
label: MACRO = \{ statements; \}; \\
\\
label(arg1, \ldots ,argn): MACRO = \{ statements; \};
}

The first form allows the execution of the defined group of statements via a
single command,  
\madxmp{EXEC, label;}
that executes the statements defined between curly brackets exactly
once. The \texttt{EXEC} command can then be issued any number of times.  

The second form allows to replace strings anywhere inside the statements
in curly brackets by other strings, or integer numbers prior to
execution. This is a powerful construct and should be handled with care.  

Simple example: 
\madxmp{
simple(xx,yy): MACRO = \{ xx = yy*yy + xx; VALUE, xx;\}; \\
a = 3; \\
b = 5; \\
\\
EXEC, simple(a,b);
}


{\bf Passing arguments}\\
In the following example we use the fact that a "\$" in front of an
argument means that the truncated integer value of this argument is used
for replacement, rather than the argument string itself.  
\madxmp{
tricky(xx,yy,zz): MACRO = \{mzz.yy: xx, l = 1.yy, kzz = k.yy;\}; \\
n=0; \\
WHILE \=(n < 3) \{ \\
  \>n = n+1; \\
  \>EXEC, tricky(quadrupole, \$n, 1); \\
  \>EXEC, tricky(sextupole, \$n, 2);  \\
\};
}


Whereas the actual use of the preceding example is NOT recommended,
a real life example, showing the full power (!) of macros is to be
found under \href{foot.html}{macro usage} for the usage, and
under \href{foot.html#macro}{macro definition} for the
definition.


Because \texttt{MACRO} statements are a \madx
construct and not part of a full language, \madx  allows only one level
of inclusion of another \texttt{IF ... ELSEIF ... ELSE}, \texttt{WHILE}
or \texttt{MACRO} statements.  


Macros cannot be called with number arguments but always with string
arguments. In case numerical values should be passed to a \texttt{MACRO} in an
\texttt{EXEC} statement, one can conveniently use variables names: 
\madxmp{
n1=99; n2=219;\\
EXEC, thismacro(\$n1, \$n2);
}
instead of 
\madxmp{
EXEC, thismacro(\$99, \$129); ! fails...
}


%% EOF


