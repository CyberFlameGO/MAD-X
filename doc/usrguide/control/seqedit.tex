%%%\title{Sequence Editor}
%  Changed by: Chris ISELIN, 24-Feb-1998 
%  Changed by: Hans Grote, 17-Jun-2002 

\chapter{Sequence Editor}
\label{chap:seqedit}
With the help of the commands explained below, a sequence may be
modified in many ways: the starting point can be moved to another place;
the order of elements can be inverted; elements can be inserted one by
one, or as a whole group with one single command; single elements, or
classes thereof can be removed; elements can be replaced by others;
finally, the sequence can be "flattened", i.e. all inserted sequences
are replaced by their actual elements, such that a flattened sequence
contains only elements. It is good practice to add a \textit{ flatten; }
statement at the end of a \textit{ seqedit operation } to ensure a fully
operational sequence. And this is particularly useful for the \textit{
  save } command to properly save \textit{ shared sequences } and to
write out in \madeight format.  

%% 2013-Jul-11  17:35:46  ghislain: need a synopsis paragraph similar to
%% what PTC offers and a complement of a few examples using all commands
%% shown here...

\section{SEQEDIT}
\label{sec:seqedit}
\madbox{SEQEDIT, SEQUENCE=string;}
selects the sequence named for editing. The editing is performed on the
non-expanded sequence; after having finished the editing, one has to
re-expand the sequence if necessary.  

\section{FLATTEN}
\label{sec:flatten}
\madbox{
FLATTEN;
}
This command recursively includes all sub-sequences within the sequence
being edited, if any. The resulting sequence contains only elements.  

\section{CYCLE}
\label{sec:cycle}
\madbox{
CYCLE, START=string;
}
This makes the sequence start at the location given with the START
argument, which must be a marker. \\ 
In the case there is a shared sequence in the used sequence, the
command FLATTEN has to be used before the command CYCLE. Example:  
\madxmp{
flatten ; cycle, start=place; 
}

\section{REFLECT}
\label{sec:reflect}
\madbox{
REFLECT;
}
This inverts the order of element in the sequence, starting from the
last element. \\ 
If there are shared sequences inside the USEd sequence, the command
FLATTEN must be used before the command REFLECT.  Alternatively each
shared sequence must first be reflected. Example:   
\madxmp{
flatten ; reflect; 
}


\section{INSTALL}
\label{sec:install}
\madbox{
INSTALL, ELEMENT=string, CLASS=string, AT=real, FROM=string|SELECTED;
}
where the parameters have the following meaning: 
\begin{madlist}
   \ttitem{ELEMENT} name of the (new) element to be inserted (mandatory) 
   \ttitem{CLASS} class of the new element to be inserted (mandatory) 
   \ttitem{AT} position where the element is to be inserted; if no "from"
     is given,this is relative to the start of the sequence. If "from"
     is given, it is relative to the position specified there. 
   \ttitem{FROM} either a place (i.e. the name(+occurrence count) of an
     element already existing in the sequence, e.g. mb[15], or
     mq.a..i1..4 etc.; or the string "SELECTED"; in the latter case an
     element of the type specified will be inserted behind all elements
     in the sequence that are currently selected by one or several
     \href{../Introduction/select.html}{SELECT} commands of the type 
\begin{verbatim}
select, flag=seqedit, class=.., pattern=.., range=..;
\end{verbatim} 
   \item \textit{Attention: No element definition can occur inside SEQEDIT. }
\end{madlist}

\section{MOVE}
\label{sec:move}
\madbox{
MOVE, ELEMENT=string|SELECTED, BY=real, TO=real, FROM=string;
}
\begin{madlist}
   \ttitem{ELEMENT} name of the existing element to be moved, or
     "SELECTED", in which case all elements from existing
     \href{../Introduction/select.html}{SELECT} commands will be moved;
     in the latter case, "BY" must be given.  
   \ttitem{BY} distance by which the element(s) is/are to be moved; no "TO"
     nor "FROM" attributes should be given in this case.  
   \ttitem{TO} position to which the element has to be moved; if no FROM
   attribute is given, the position is relative to the start of the sequence; otherwise, it
     is relative to the location given in the "FROM" argument  
   \ttitem{FROM} place in the sequence with respect to which the element
     is to be positioned.  
\end{madlist}

\section{REMOVE}
\label{sec:remove}
\madbox{
REMOVE, ELEMENT=string|SELECTED;
}
\begin{madlist}
   \ttitem{ELEMENT} name of the existing element to be removed, or
     "SELECTED", in which case all elements from existing
     \href{../Introduction/select.html}{SELECT} commands will be
     removed. 
\end{madlist}

\textit{Attention: It is a bad idea to remove all markers from
  a sequence! In particular the "start=" marker and the new markers
  added by "cycle" must never be removed from a sequence.} 


\section{REPLACE}
\label{sec:replace}
\madbox{
REPLACE, ELEMENT=string|SELECTED, BY=string;
}
Element with name1 is replaced by element with name2. 
If name1 is "selected", then all elements selected by
\href{../Introduction/select.html}{SELECT} commands will be replaced by
the element name2.  

\section{EXTRACT}
\label{sec:extract}
\madbox{
EXTRACT, SEQUENCE=string, FROM=string, TO=string, NEWNAME=string;
}
A new sequence with name given by the NEWNAME argument is extracted from
the existing sequence given by the SEQUENCE argument, 
starting from the position given by the FROM argument and ending at
position given by the TO argument. The resulting new sequence
can be USEd as any other sequence. It is declared as "shared" and
can therefore be combined e.g. into the cycled original sequence. \\ 
Note that the position given by the FROM argument must be located before
the position given by the TO argument in the original sequence given by
the SEQUENCE argument, or \mad will fail with a fatal error. 
In the case of circular sequences, this can be ensured by performing a CYCLE 
of the orginal sequence with {\tt START} pointing to the position given
in the FROM argument. 


\section{ENDEDIT}
\label{sec:endedit}
\madbox{
ENDEDIT;
}
terminates the sequence editing process. The nodes in the sequence are
renumbered according to their occurrence which might have changed during
editing.  



\section{SAVE}
\label{sec:save}
\madbox{
SAVE, SEQUENCE= string{,string}, FILE=string, BEAM, BARE, MAD8;
}
saves the sequence or sequences specified, with all variables and elements needed
for their expansion, onto the file given with the FILE argument.

{\bf Warning:} If double quotes are used to specify
the "file\_name", upper and lower case characters are kept as specified;
if single quotes are used or quotes are simply omitted the actual
"file\_name" consists of only lower case characters. 

Example:
\madxmp{save, sequence = lhc, file = "Test\_One";}
saves the lhc sequence to a file named Test\_One on disk, while
\madxmp{save, sequence = lhc, file = Test\_One;}
saves the lhc sequence to a file named test\_one on disk.

The optional flag can have the value "mad8" (without the quotes), in
which case the sequence(s) is/are saved in \madeight input format.  

The flag "beam" is optional; when given, all beams belonging to the
sequences specified are saved at the top of the save file.  

The parameter "sequence" is optional; when omitted, all sequences are
saved.  

However, it is not advisable to use "save" without the "sequence" option
unless you know what you are doing. This practice will avoid spurious
saved entries.    Any number of "select,flag=save" commands may precede
the SAVE command. In that case, the names of elements, variables, and
sequences must match the pattern(s) if given, and in addition the
elements must be of the class(es) specified. See here for a
\href{../Introduction/select.html#save_select}{SAVE with SELECT}
example.  

It is important to note that the precision of the output of the save
command depends on the output precision. Details about default
precisions and how to adjust those precisions can be found at the
\href{../Introduction/set.html#Format}{SET Format} instruction page.   
 
The attribute 'bare' allows to save just the sequence without the
element definitions nor beam information. This allows to re-read in a
sequence with might otherwise create a stop of the program. This is
particularly useful to turn a line into a sequence to seqedit
it. 

Example:  
\begin{verbatim}
tl3:line=(ldl6,qtl301,mqn,qtl301,ldl7,qtl302,mqn,qtl302,ldl8,ison);
DLTL3 : LINE=(delay, tl3);
use, period=dltl3;

save,sequence=dltl3,file=t1,bare; // new parameter "bare": only sequ. saved
call,file=t1; // sequence is read in and is now a "real" sequence
// if the two preceding lines are suppressed, seqedit will print a warning
// and else do nothing
use, period=dltl3;
twiss, save, betx=bxa, alfx=alfxa, bety=bya, alfy=alfya;
plot, vaxis=betx, bety, haxis=s, colour:=100;
SEQEDIT, SEQUENCE=dltl3;
  remove,element=cx.bhe0330;
  remove,element=cd.bhe0330;
ENDEDIT;

use, period=dltl3;
twiss, save, betx=bxa, alfx=alfxa, bety=bya, alfy=alfya;
\end{verbatim}

%% 2013-Jul-11  17:24:21  ghislain: I propose to move DUMPSEQU to the
%% sequence edition and manipulation part of the manual
\section{DUMPSEQU}
\label{sec:dumpsequ}
\madbox{
DUMPSEQU, SEQUENCE=string, LEVEL=integer;
}
This command is actually more of a debug statement, but it may come handy at certain
occasions. The argument of the SEQUENCE attribute is the name of an
already expanded (i.e. USEd) sequence. The amount of detail in the
output is controlled by the LEVEL argument:
\begin{itemize}
\item[$=0$ : ]    print only the cumulative node length = sequence length
\item[$>0$ : ]    print all node (element) names except drifts
\item[$>2$ : ]    print all nodes with their attached parameters
\item[$>3$ : ]    print all nodes, and their elements with all parameters
\end{itemize}



%% EOF
