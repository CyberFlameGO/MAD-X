%%%\title{Range Selection}
%  Changed by: Chris ISELIN, 27-Jan-1997 
%  Changed by: Hans Grote, 10-Jun-2002 

\paragraph{Real life example for IF statements, and MACRO usage}


\begin{verbatim}

! Creates a footprint for head-on + parasitic collisions at IP1+5 
! of lhc.6.5; both lhcb1 (for tracking) and lhcb2 (to define the
! beam-beam elements, i.e. weak-strong) are used; there are flags to
! select head-on, left, and right parasitic separately at all IPs.
! The bunch spacing can be given in nanosec and automatically creates
! the beam-beam interaction points at the correct positions.
! It is important to set the correct BEAM parameters, i.e. number
! of particles, emittances, bunch length, energy.

!--- For completeness, all files needed by this job are copied
!    to the local directory ldb. The links to the the originals
!    in offdb (official database) are commented out.

Option,  warn,info,echo;
!System,
"ln -fns /afs/cern.ch/eng/sl/MAD-X/dev/test_suite/foot/V3.01.01 ldb";
!system,"ln -fns /afs/cern.ch/eng/lhc/optics/V6.4 offdb";
Option, -echo,-info,warn;
SU=1.0;
call, file = "ldb/V6.5.seq";
call,file="ldb/slice_new.madx";
Option, echo,info,warn;

!+++++++++++++++++++++++++ Step 1 +++++++++++++++++++++++
! 	define beam constants
!++++++++++++++++++++++++++++++++++++++++++++++++++++++++

b_t_dist = 25.e-9;                  !--- bunch distance in [sec]
b_h_dist = clight * b_t_dist / 2 ;  !--- bunch half-distance in [m]
ip1_range = 58.;                     ! range for parasitic collisions
ip5_range = ip1_range;
ip2_range = 60.;
ip8_range = ip2_range;

npara_1 = ip1_range / b_h_dist;     ! # parasitic either side
npara_2 = ip2_range / b_h_dist;
npara_5 = ip5_range / b_h_dist;
npara_8 = ip8_range / b_h_dist;

value,npara_1,npara_2,npara_5,npara_8;

 eg   =  7000;
 bg   =  eg/pmass;
 en   = 3.75e-06;
 epsx = en/bg;
 epsy = en/bg;

Beam, particle = proton, sequence=lhcb1, energy = eg,
          sigt=      0.077     , 
          bv = +1, NPART=1.1E11, sige=      1.1e-4, 
          ex=epsx,   ey=epsy;

Beam, particle = proton, sequence=lhcb2, energy = eg,
          sigt=      0.077     , 
          bv = -1, NPART=1.1E11, sige=      1.1e-4, 
          ex=epsx,   ey=epsy;

beamx = beam%lhcb1->ex;   beamy%lhcb1 = beam->ey;
sigz  = beam%lhcb1->sigt; sige = beam%lhcb1->sige;

!--- split5, 4d
long_a= 0.53 * sigz/2;
long_b= 1.40 * sigz/2;
value,long_a,long_b;

ho_charge = 0.2;

!+++++++++++++++++++++++++ Step 2 +++++++++++++++++++++++
! 	slice, flatten sequence, and cycle start to ip3
!++++++++++++++++++++++++++++++++++++++++++++++++++++++++

use,sequence=lhcb1;
makethin,sequence=lhcb1;
!save,sequence=lhcb1,file=lhcb1_thin_new_seq;
use,sequence=lhcb2;
makethin,sequence=lhcb2;
!save,sequence=lhcb2,file=lhcb2_thin_new_seq;
!stop;

option,-warn,-echo,-info;
call,file="ldb/V6.5.thin.coll.str";
option,warn,echo,info;

! keep sextupoles
ksf0=ksf; ksd0=ksd;
use,period=lhcb1;
select,flag=twiss.1,column=name,x,y,betx,bety;
twiss,file;
plot,haxis=s,vaxis=x,y,colour=100,noline;

use,period=lhcb2;
select,flag=twiss.2,column=name,x,y,betx,bety;
twiss,file;
plot,haxis=s,vaxis=x,y,colour=100,noline;
seqedit,sequence=lhcb1;
flatten;
endedit;

seqedit,sequence=lhcb1;
cycle,start=ip3.b1;
endedit;

seqedit,sequence=lhcb2;
flatten;
endedit;

seqedit,sequence=lhcb2;
cycle,start=ip3.b2;
endedit;

bbmarker: marker;  /* for subsequent remove */


!+++++++++++++++++++++++++ Step 3 +++++++++++++++++++++++
! 	define the beam-beam elements
!++++++++++++++++++++++++++++++++++++++++++++++++++++++++
!
!===========================================================
! read macro definitions
option,-echo;
call,file="ldb/bb.macros";
option,echo;

!
!===========================================================
!   this sets CHARGE in the head-on beam-beam elements. 
!   set +1 * ho_charge   for parasitic on, 0 for off

 on_ho1  = +1 * ho_charge; ! ho_charge depends on split
 on_ho2  = +0 * ho_charge; ! because of the "by hand" splitting
 on_ho5  = +1 * ho_charge;
 on_ho8  = +0 * ho_charge;

!
!===========================================================
!   set CHARGE in the parasitic beam-beam elements. 
!   set +1 for parasitic on, 0 for off
 on_lr1l = +1;
 on_lr1r = +1;
 on_lr2l = +0;
 on_lr2r = +0;
 on_lr5l = +1;
 on_lr5r = +1;
 on_lr8l = +0;
 on_lr8r = +0;

!
!===========================================================
!   define markers and savebetas
assign,echo=temp.bb.install;
!--- ip1
if (on_ho1  0)
{
  exec, mkho(1);
  exec, sbhomk(1);
}
if (on_lr1l  0 || on_lr1r  0)
{
  n=1; ! counter
  while (n  0 || on_lr1l  0)
{
  n=1; ! counter
  while (n  0)
{
  exec, mkho(5);
  exec, sbhomk(5);
}
if (on_lr5l  0 || on_lr5r  0)
{
  n=1; ! counter
  while (n  0 || on_lr5l  0)
{
  n=1; ! counter
  while (n  0)
{
  exec, mkho(2);
  exec, sbhomk(2);
}
if (on_lr2l  0 || on_lr2r  0)
{
  n=1; ! counter
  while (n  0 || on_lr2l  0)
{
  n=1; ! counter
  while (n  0)
{
  exec, mkho(8);
  exec, sbhomk(8);
}
if (on_lr8l  0 || on_lr8r  0)
{
  n=1; ! counter
  while (n  0 || on_lr8l  0)
{
  n=1; ! counter
  while (n  0)
{
exec, inho(mk,1);
}
if (on_lr1l  0 || on_lr1r  0)
{
  n=1; ! counter
  while (n  0 || on_lr1l  0)
{
  n=1; ! counter
  while (n  0)
{
exec, inho(mk,5);
}
if (on_lr5l  0 || on_lr5r  0)
{
  n=1; ! counter
  while (n  0 || on_lr5l  0)
{
  n=1; ! counter
  while (n  0)
{
exec, inho(mk,2);
}
if (on_lr2l  0 || on_lr2r  0)
{
  n=1; ! counter
  while (n  0 || on_lr2l  0)
{
  n=1; ! counter
  while (n  0)
{
exec, inho(mk,8);
}
if (on_lr8l  0 || on_lr8r  0)
{
  n=1; ! counter
  while (n  0 || on_lr8l  0)
{
  n=1; ! counter
  while (n betx) / 0.0007999979093;
value,on_sep2;
!===========================================================
!   define bb elements
assign,echo=temp.bb.install;
!--- ip1
if (on_ho1  0)
{
exec, bbho(1);
}
if (on_lr1l  0)
{
  n=1; ! counter
  while (n  0)
{
  n=1; ! counter
  while (n  0)
{
exec, bbho(5);
}
if (on_lr5l  0)
{
  n=1; ! counter
  while (n  0)
{
  n=1; ! counter
  while (n  0)
{
exec, bbho(2);
}
if (on_lr2l  0)
{
  n=1; ! counter
  while (n  0)
{
  n=1; ! counter
  while (n  0)
{
exec, bbho(8);
}
if (on_lr8l  0)
{
  n=1; ! counter
  while (n  0)
{
  n=1; ! counter
  while (n  0)
{
exec, inho(bb,1);
}
if (on_lr1l  0)
{
  n=1; ! counter
  while (n  0)
{
  n=1; ! counter
  while (n  0)
{
exec, inho(bb,5);
}
if (on_lr5l  0)
{
  n=1; ! counter
  while (n  0)
{
  n=1; ! counter
  while (n  0)
{
exec, inho(bb,2);
}
if (on_lr2l  0)
{
  n=1; ! counter
  while (n  0)
{
  n=1; ! counter
  while (n  0)
{
exec, inho(bb,8);
}
if (on_lr8l  0)
{
  n=1; ! counter
  while (n  0)
{
  n=1; ! counter
  while (n  footprint";
stop;
\end{verbatim}

\paragraph{\href{macro}{Real life example of MACRO definitions}}

\begin{verbatim}

bbho(nn): macro = {
!--- macro defining head-on beam-beam elements; nn = IP number
print, text="bbipnnl2: beambeam, sigx=sqrt(rnnipnnl2->betx*epsx),";
print, text="          sigy=sqrt(rnnipnnl2->bety*epsy),";
print, text="          xma=rnnipnnl2->x,yma=rnnipnnl2->y,";
print, text="          charge:=on_honn;";
print, text="bbipnnl1: beambeam, sigx=sqrt(rnnipnnl1->betx*epsx),";
print, text="          sigy=sqrt(rnnipnnl1->bety*epsy),";
print, text="          xma=rnnipnnl1->x,yma=rnnipnnl1->y,";
print, text="          charge:=on_honn;";
print, text="bbipnn:   beambeam, sigx=sqrt(rnnipnn->betx*epsx),";
print, text="          sigy=sqrt(rnnipnn->bety*epsy),";
print, text="          xma=rnnipnn->x,yma=rnnipnn->y,";
print, text="          charge:=on_honn;";
print, text="bbipnnr1: beambeam, sigx=sqrt(rnnipnnr1->betx*epsx),";
print, text="          sigy=sqrt(rnnipnnr1->bety*epsy),";
print, text="          xma=rnnipnnr1->x,yma=rnnipnnr1->y,";
print, text="          charge:=on_honn;";
print, text="bbipnnr2: beambeam, sigx=sqrt(rnnipnnr2->betx*epsx),";
print, text="          sigy=sqrt(rnnipnnr2->bety*epsy),";
print, text="          xma=rnnipnnr2->x,yma=rnnipnnr2->y,";
print, text="          charge:=on_honn;";
};

mkho(nn): macro = {
!--- macro defining head-on markers; nn = IP number
print, text="mkipnnl2: bbmarker;";
print, text="mkipnnl1: bbmarker;";
print, text="mkipnn:   bbmarker;";
print, text="mkipnnr1: bbmarker;";
print, text="mkipnnr2: bbmarker;";
};

inho(xx,nn): macro = {
!--- macro installing bb or markers for head-on beam-beam (split into 5)
print, text="install, element= xxipnnl2, at=-long_b, from=ipnn;";
print, text="install, element= xxipnnl1, at=-long_a, from=ipnn;";
print, text="install, element= xxipnn,   at=1.e-9,   from=ipnn;";
print, text="install, element= xxipnnr1, at=+long_a, from=ipnn;"; 
print, text="install, element= xxipnnr2, at=+long_b, from=ipnn;"; 
};

sbhomk(nn): macro = {
!--- macro to create savebetas for ho markers
print, text="savebeta, label=rnnipnnl2, place=mkipnnl2;";
print, text="savebeta, label=rnnipnnl1, place=mkipnnl1;";
print, text="savebeta, label=rnnipnn,   place=mkipnn;";
print, text="savebeta, label=rnnipnnr1, place=mkipnnr1;";
print, text="savebeta, label=rnnipnnr2, place=mkipnnr2;";    
};

mkl(nn,cc): macro = {
!--- macro to create parasitic bb marker on left side of ip nn; cc = count
print, text="mkipnnplcc: bbmarker;";
};

mkr(nn,cc): macro = {
!--- macro to create parasitic bb marker on right side of ip nn; cc = count
print, text="mkipnnprcc: bbmarker;";
};

sbl(nn,cc): macro = {
!--- macro to create savebetas for left parasitic
print, text="savebeta, label=rnnipnnplcc, place=mkipnnplcc;";
};

sbr(nn,cc): macro = {
!--- macro to create savebetas for right parasitic
print, text="savebeta, label=rnnipnnprcc, place=mkipnnprcc;";
};

inl(xx,nn,cc): macro = {
!--- macro installing bb and markers for left side parasitic beam-beam
print, text="install, element= xxipnnplcc, at=-cc*b_h_dist, from=ipnn;";
};

inr(xx,nn,cc): macro = {
!--- macro installing bb and markers for right side parasitic beam-beam
print, text="install, element= xxipnnprcc, at=cc*b_h_dist, from=ipnn;";
};

bbl(nn,cc): macro = {
!--- macro defining parasitic beam-beam elements; nn = IP number
print, text="bbipnnplcc: beambeam, sigx=sqrt(rnnipnnplcc->betx*epsx),";
print, text="          sigy=sqrt(rnnipnnplcc->bety*epsy),";
print, text="          xma=rnnipnnplcc->x,yma=rnnipnnplcc->y,";
print, text="          charge:=on_lrnnl;";
};

bbr(nn,cc): macro = {
!--- macro defining parasitic beam-beam elements; nn = IP number
print, text="bbipnnprcc: beambeam, sigx=sqrt(rnnipnnprcc->betx*epsx),";
print, text="          sigy=sqrt(rnnipnnprcc->bety*epsy),";
print, text="          xma=rnnipnnprcc->x,yma=rnnipnnprcc->y,";
print, text="          charge:=on_lrnnr;";
};
\end{verbatim}

%\href{http://www.cern.ch/Hans.Grote/hansg_sign.html}{hansg}, June 17, 2002 

