%%%\title{Range Selection}
%  Changed by: Chris ISELIN, 27-Jan-1997 
%  Changed by: Hans Grote, 16-Jan-2003 

\section{General Control Statements}

\subsection{ASSIGN}
\begin{verbatim}
assign, echo="file_name", truncate;
\end{verbatim} 
where "file\_name" is the name of an output file, or "terminal" and
truncate specifies if the file must be truncated when opened (ignored
for terminal). This allows switching the echo stream to a file or back,
but only for the commands value, show, and print. Allows easy
composition of future MAD-X input files. A real life example (always the
same) is to be found under \href{foot.html}{footprint example}.  

\subsection{CALL}
\begin{verbatim}
call, file = "file_name";
\end{verbatim} 
where "file\_name"  is the name of an input file. This file will be read
until a "return;" statement, or until end\_of\_file; it may contain any
number of calls itself, and so on to any depth.  

%% 2013-Jul-11  17:23:00  ghislain: I propose to move COGUESS to the
%% orbit correction part of the manual
\subsection{COGUESS}
\label{subsec:general_coguess}
\begin{verbatim}
coguess, tolerance = double, 
         x = double, px = double, 
         y = double, py = double, 
         t = double, pt = double;
\end{verbatim} 
sets the required convergence precision in the closed orbit search
("tolerance", see as well Twiss command
\href{../twiss/twiss.html#tolerance}{tolerance}).  

The other parameters define a first guess for all future closed orbit
searches in case they are different from zero.  

\subsection{CREATE}
\begin{verbatim}
create, table = table, column = var1, var2,_name,...;
\end{verbatim} 
creates a table with the specified variables as columns. This table can
then be \hyperlink{fill}{fill}ed, and finally one can
\hyperlink{write}{write} it in TFS format. The attribute "\_name" adds
the element name to the table at the specified column, this replaces the
undocumented "withname" attribute that was not always working properly.  

See the \href{../Introduction/select.html#ucreate}{user table I}
example; 
or an example of joining 2 tables of different length in one table
including the element name:
\href{../Introduction/select.html#screate}{user table II} 

\subsection{DELETE}
\begin{verbatim}
delete, sequence = s_name, table = t_name;
\end{verbatim} 
deletes a sequence with name "s\_name" or a table with name "t\_name"
from memory. The sequence deletion is done without influence on other
sequences that may have elements that were in the deleted sequence.  

%% 2013-Jul-11  17:24:21  ghislain: I propose to move DUMPSEQU to the
%% sequence edition and manipulation part of the manual
\subsection{DUMPSEQU}
\begin{verbatim}
dumpsequ, sequence = s_name, level = integer;
\end{verbatim} 
Actually a debug statement, but it may come handy at certain
occasions. Here "s\_name" is the name of an expanded (i.e. USEd)
sequence. The amount of detail is controlled by "level":  
\begin{verbatim}
level = 0:    print only the cumulative node length = sequence length
      > 0:    print all node (element) names except drifts
      > 2:    print all nodes with their attached parameters
      > 3:    print all nodes, and their elements with all parameters
\end{verbatim}


\subsection{EXEC}
\begin{verbatim}
exec, label;
\end{verbatim} 
Each statement may be preceded by a label; it is then stored and can be
executed again with "exec, label;" any number of times; the executed
statement may be another "exec", etc.; however, the major usage of this
statement is the execution of a \href{special.html#macro}{macro}.  

\subsection{EXIT}
\begin{verbatim}
exit;
\end{verbatim} 
ends the program execution. 

\subsection{FILL} 
Every command 
\begin{verbatim}
fill, table = table;
\end{verbatim} 
adds a new line with the current values of all column variables into the
user table \hyperlink{create}{create}d beforehand. This table one can
then \hyperlink{write}{write} in TFS format.  See as well the
\href{../Introduction/select.html#ucreate}{user table} example.  

\subsection{OPTION}
\label{subsec:general_option}
\begin{verbatim}
option, flag { = true | = false };
option, flag | -flag;
\end{verbatim} 
sets an option as given in "flag"; the part in curly brackets is
optional: if only the name of the option is given, then the option will
be set true (see second line); a "-" sign preceding the name sets it to
"false".  

Example: 
\begin{verbatim}
option, echo = true;
option, echo;
\end{verbatim} 
are identical, ditto 
\begin{verbatim}
option, echo = false;
option, -echo;
\end{verbatim} 

The available options are: 
\begin{verbatim}
  name           default meaning if true
  ====           ======= ===============
  echo            true   echoes the input on the standard output file
  warn            true   issues warnings
  info            true   issues informations
  debug           false  issues debugging information
  trace           false  prints the system time after each command
  verify          false  issues a warning if an undefined variable is used
  tell            false  prints the current value of all options
  reset           false  resets all options to their defaults
  no_fatal_stop   false  Prevents madx from stopping in case of a fatal error. 
                         Use at your own risk.

  rbarc           true   converts the RBEND straight length into the arc 
                         length
  thin_foc        true   if false suppresses the 1(rho**2) focusing of thin 
                         dipoles
  bborbit         false  the closed orbit is modified by beam-beam kicks
  sympl           false  all element matrices are symplectified in Twiss
  twiss_print     true   controls whether the twiss command produces output.
\end{verbatim} 

The option "rbarc" is implemented for backwards compatibility with MAD-8
up to version 8.23.06 included; in this version, the RBEND length was
just taken as the arc length of an SBEND with inclined pole faces,
contrary to the MAD-8 manual.  


\subsection{PRINT}
\begin{verbatim}
print, text = "...";
\end{verbatim} 
prints the text to the current output file (see ASSIGN above). The text
can be edited with the help of a  \href{special.html#macro}{macro
  statement}. For more details, see there.  


\subsection{QUIT}
\begin{verbatim}
quit;
\end{verbatim} 
ends the program execution. 


\subsection{READTABLE}
\begin{verbatim}
readtable, file = "file_name";
\end{verbatim} 
reads a TFS file containing a MAD-X table back into memory. This table
can then be manipulated as any other table, i.e. its values can be
accessed, it can be plotted, written out again etc.  


\subsection{READMYTABLE}
\label{subsec:general_readmy}
\begin{verbatim}
readmytable, file = "file_name", table = internalname;
\end{verbatim} 
reads a TFS file containing a MAD-X table back into memory. This table
can then be manipulated as any other table, i.e. its values can be
accessed, it can be plotted, written out again etc. 

An internal name for
the table can be freely assigned while for the command READTABLE it is
taken from the information section of the table itself. This feature
allows to store multiple tables of the same type in memory without
overwriting existing ones.   


\subsection{REMOVEFILE}
\begin{verbatim}
removefile, file = "file_name";
\end{verbatim} 
remove the file "file\_name" from disk. It is more portable than  
\begin{verbatim}
system("rm filename"); // Unix specific
\end{verbatim}


\subsection{RENAMEFILE}
\begin{verbatim}
renamefile, file = "file_name", name = "new_file_name";
\end{verbatim} 
rename the file "file\_name" to "new\_file\_name" on the disk. It is more
portable than  
\begin{verbatim}
system("mv file_name new_file_name"); // Unix specific
\end{verbatim}

%% 2013-Jul-11  17:24:21  ghislain: I propose to move RESBEAM to the
%% beam declaration part of the manual
\subsection{RESBEAM}
\begin{verbatim}
resbeam, sequence = sequence_name;
\end{verbatim} 
resets the default values of the beam belonging to sequence sequence\_name, or
of the default beam if no sequence is given.  


\subsection{RETURN}
\begin{verbatim}
return;
\end{verbatim} 
ends reading from a "called" file; if encountered in the standard input
file, it ends the program execution.  


\subsection{SAVE}
\label{subsec:general_save}
\begin{verbatim}
save, sequence = sequ1, sequ2, ..., file = "file_name", beam, bare;
\end{verbatim} 
saves the sequence(s) specified with all variables and elements needed
for their expansion, onto the file "file\_name". 

{\bf Warning:} If quotes are used for
the "file\_name", capital and low characters are kept as specified, if they
are omitted the "filename" will have lower characters only. 

Example:
\begin{verbatim}
save, sequence = lhc, file = "Test_One";
\end{verbatim}
saves the lhc sequence to a file name Test\_One on disk, while
\begin{verbatim}
save, sequence = lhc, file = Test_One;
\end{verbatim}
saves the lhc sequence to a file name test\_one on disk.

The optional
flag can have the value "mad8" (without the quotes), in which case the
sequence(s) is/are saved in MAD-8 input format.  

The flag "beam" is optional; when given, all beams belonging to the
sequences specified are saved at the top of the save file.  

The parameter "sequence" is optional; when omitted, all sequences are
saved.  

However, it is not advisable to use "save" without the "sequence" option
unless you know what you are doing. This practice will avoid spurious
saved entries.    Any number of "select,flag=save" commands may precede
the SAVE command. In that case, the names of elements, variables, and
sequences must match the pattern(s) if given, and in addition the
elements must be of the class(es) specified. See here for a
\href{../Introduction/select.html#save_select}{SAVE with SELECT}
example.  

It is important to note that the precision of the output of the save
command depends on the output precision. Details about default
precisions and how to adjust those precisions can be found at the
\href{../Introduction/set.html#Format}{SET Format} instruction page.   
 
The attribute 'bare' allows to save just the sequence without the
element definitions nor beam information. This allows to re-read in a
sequence with might otherwise create a stop of the program. This is
particularly useful to turn a line into a sequence to seqedit
it. 

Example:  
\begin{verbatim}
tl3:line=(ldl6,qtl301,mqn,qtl301,ldl7,qtl302,mqn,qtl302,ldl8,ison);
DLTL3 : LINE=(delay, tl3);
use, period=dltl3;

save,sequence=dltl3,file=t1,bare; // new parameter "bare": only sequ. saved
call,file=t1; // sequence is read in and is now a "real" sequence
// if the two preceding lines are suppressed, seqedit will print a warning
// and else do nothing
use, period=dltl3;
twiss, save, betx=bxa, alfx=alfxa, bety=bya, alfy=alfya;
plot, vaxis=betx, bety, haxis=s, colour:=100;
SEQEDIT, SEQUENCE=dltl3;
  remove,element=cx.bhe0330;
  remove,element=cd.bhe0330;
ENDEDIT;

use, period=dltl3;
twiss, save, betx=bxa, alfx=alfxa, bety=bya, alfy=alfya;
\end{verbatim}


\subsection{SAVEBETA}
\label{subsec:general_savebeta}
\begin{verbatim}
savebeta, label = label, place = place, sequence = sequence_name;
\end{verbatim} 
marks a place named "place" in an expanded sequence "sequence\_name"; 
at the next TWISS command execution, a
\href{../twiss/twiss.html#beta0}{beta0} 
block will be saved at that place with the label "label". This is done
only once; in order to get a new beta0 block there, one has to re-issue
the command. The contents of the beta0 block can then be used in other
commands, e.g. TWISS and MATCH.  

Example (after sequence expansion): 
\begin{verbatim}
savebeta, label = sb1, place = mb[5], sequence = fivecell;
twiss;
show, sb1;
\end{verbatim} 
will save and show the beta0 block parameters at the end (!) of the
fifth element of type mb in the sequence.  


\subsection{SELECT} %select</a}{SELECT}
\begin{verbatim}
select, flag = flag, range = range, class = class, pattern = pattern,
        slice = integer, column =s1, s2, s3,..,sn, sequence=sequence_name,
        full, clear;
\end{verbatim} 
selects one or several elements for special treatment in a subsequent
command. All selections for a given command remain valid until "clear"
is specified; the selection criteria on the same command are logically
ANDed; the selection criteria on different SELECT statements logically
ORed.   

 Example: 
\begin{verbatim}
select, flag = error, class = quadrupole, range = mb[1]/mb[5];
select, flag = error, pattern = "^mqw.*";
\end{verbatim} 
selects all quadrupoles in the range mb[1] to mb[5], and all elements
(in the whole sequence) the name of which starts with "mqw", for
treatment by the error module.  

"flag" can be one of the following: 
\begin{itemize}
   \item seqedit: selection of elements for the
     \href{seqedit.html}{seqedit} module.  
   \item error: selection of elements for the
     \href{../error/error.html}{error} assignment module.  
   \item makethin: selection of elements for the
     \href{../makethin/makethin.html}{makethin} module that
     converts the sequence into one with thin elements only.  
   \item sectormap: selection of elements for the
     \href{../Introduction/sectormap.html}{sectormap} output file
     from the Twiss module.  
   \item table: here "table" is a table name such as twiss, track
     etc., and the rows and columns to be written are selected.  
\end{itemize} 

For the RANGE, CLASS, PATTERN, FULL, and CLEAR parameters
see \href{../Introduction/select.html}{SELECT}.  

"slice" is only used by \href{../makethin/makethin.html}{makethin} and
prescribes the number of slices into which the selected elements have to
be cut (default = 1).  

"column" is only valid for tables and decides the selection of columns
to be written into the TFS file. The "name" argument is special in that
it refers to the actual name of the selected element. For an example,
see \href{../Introduction/select.html}{SELECT}.  


\subsection{SHOW}
\begin{verbatim}
show, command;
\end{verbatim} 
prints the "command" (typically "beam", "beam\%sequ", or an element
name), with the actual value of all its parameters.  


\subsection{STOP}
\begin{verbatim}
stop;
\end{verbatim} 
ends the program execution. 


\subsection{SYSTEM}
\begin{verbatim}
system, "string";
\end{verbatim} 
transfers the string in quotes to the system for execution.  

Example: 
\begin{verbatim}
system,"ln -s /afs/cern.ch/user/u/user/public/some/directory short";
\end{verbatim}


\subsection{TABSTRING}
Note: this is not a command and should appear in the variables section
\begin{verbatim}
tabstring(arg1,arg2,arg3)
\end{verbatim}  
The "string function" tabstring(arg1,arg2,arg3) with exactly  three
arguments; arg1 is a table name (string), arg2 is a column name
(string), arg3 is a row number (integer), count starts at 0. The
function can be used in any context where a string appears; in case
there is no match, it returns \_void\_.  


\subsection{TITLE}
\begin{verbatim}
title, "title";
\end{verbatim} 
inserts the string in quotes as title in various tables and plots.  


\subsection{USE}
\label{subsec:general_use}
\begin{verbatim}
use, period = sequence_name, range = range, survey;
\end{verbatim} 
expands the sequence with name "sequence\_name", or a part of it as specified
in the \href{../Introduction/ranges.html#range}{range}. The
\texttt{survey} option plugs the survey data into the sequence elements
nodes on the first pass (see \href{../survey/survey.html}{survey}).  


\subsection{VALUE}
\begin{verbatim}
value, exp1, exp2,...;
\end{verbatim} 
prints the actual values of the expressions given. 

Example: 
\begin{verbatim}
a = clight/1000.;
value, a, pmass, exp(sqrt(2));
\end{verbatim} 
results in 
\begin{verbatim}
a = 299792.458         ;
pmass = 0.938271998        ;
exp(sqrt(2)) = 4.113250379        ;
\end{verbatim}


\subsection{WRITE}
\label{subsec:general_write}
\begin{verbatim}
write, table = table, file = "file_name";
\end{verbatim} 
writes the table "table" onto the file "file\_name"; only the rows and
columns of a preceding \verb+select, flag = table,...;+ are written. If no select
has been issued for this table, the file will only contain the
header. If the FILE argument is omitted, the table is written to
standard output.  


%\href{http://www.cern.ch/Hans.Grote/hansg_sign.html}{hansg}, June 17, 2002 
