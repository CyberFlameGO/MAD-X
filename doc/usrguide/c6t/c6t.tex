%%%\title{SIXTRACK Convertor}
%  Changed by: Mark HAYES, 19-Sep-2002 

\chapter{SixTrack: Produce input files for tracking in SixTrack}
\label{chap:sixtrack}

In dynamic aperture studies
\href{../Introduction/bibliography.html#SixTrack}{[SixTrack]} is often
used because of its  speed and controllability. However the input files
are notoriously difficult  to produce by hand. This command may be used
to produce SixTrack files from  a sequence in MAD-X's memory.  
 
 N.B.: The files contain all information concerning optics, field errors
 and misalignments. Hence these should all be set and a   
\begin{verbatim}
TWISS, SAVE;
\end{verbatim} 
command should always be issued beforehand.

 The generation of the SixTrack input files is then done by the command: 
\begin{verbatim}
SIXTRACK, CAVALL,
          MULT_AUTO_OFF,
          MAX_MULT_ORD,
          SPLIT,
	  APERTURE,
          RADIUS = ref. radius of magnets;
\end{verbatim} 

The parameters are defined as: 
\begin{itemize}
   \item  CAVALL - (optional flag) This puts a cavity element (SixTrack
     identifier 12) with Volt, Harmonic Number and Lag attributes at
     each location in the machine. Since for large hadron machines the
     cavities are typically all located at one particular spot in the
     machine and since many cavities slow down the tracking simulations
     considerably all cavities are lumped into one and located at the
     first appearance of a cavity. This default is enforced by omitting
     this flag.  
   \item  MULT\_AUTO\_OFF - (optional flag, default = .FALSE.) If
     .TRUE. this module does not process zero value
     multipoles. Moreover, multipoles are prepared in SixTrack (file
     fc.3) to be treated up to the order as specified with
     \textit{MAX\_MULT\_ORD}.  
   \item  MAX\_MULT\_ORD - (optional parameter, default = 11) Process up
     to this order for \textit{ mult\_auto\_off = .TRUE.}.  
   \item  SPLIT  - (optional flag) OBSOLETE. This splits all the
     elements in  two. This is for backwards compatibilty only. The user
     should now use the  \href{../makethin/makethin.html}{MAKETHIN}
     command instead.  
   \item  APERTURE - (optional flag) Set this to convert the apertures
     from MAD-X to SixTrack, so SixTrack will track with aperture.  
   \item  RADIUS - (optional, default value is 1m). This sets the
     reference  radius for the magnets. This argument is optional but
     should normally  be set.  
   \item \textit{ Note: the bv flag is presently ignored} 
   \item \textit{ WARNING: SixTrack and c6t are presently set up for
     names of a maximum of 16 characters!!!!! Therefore, it is mandatory
     to respect this limit for MAD-X names.} 
\end{itemize}  

The command will then always produce the following file: 
\begin{itemize}
   \item  fc.2 - contains the basic structure of the lattice. 
\end{itemize} 
and may produce any or all of the following files, depending on  the sequence: 
\begin{itemize}
   \item  fc.3 - contains the multipole mask(s). 
   \item  fc.3.aux - contains various beam parameters. 
   \item  fc.8 - contains the misalignments and tilts. 
   \item  fc.16 - contains the field errors and/or combined multipole kicks. 
   \item  fc.34 - file with various optics parameters at various
     locations (not needed by SixTrack but may be used as input to
     \href{../Introduction/bibliography.html#SODD}{[SODD]}.)  
\end{itemize}  

For a full description of these files see
\href{../Introduction/bibliography.html#SixTrack}{[SixTrack]} and for
information on running SixTrack see
\href{../Introduction/bibliography.html#SixTrack_Run_Environment}{[Run
    Environment]}.  

% Mark Hayes 20.06.02 
 
