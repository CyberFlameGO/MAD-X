%%\title{SODD}
%  Changed by: E. T. d'Amico, 8-Sep-2004 

\chapter{Second Order Detuning and Distortion} % SODD}

\begin{verbatim}
sodd, detune = logical, distort1 = logical, distort2 = logical,
      start_stop = start,stop,
      multipole_order_range = first,last,
      noprint = logical, print_all = logical, print_at_end = logical,
      nosixtrack  = logical
\end{verbatim} 

where the parameters have the following meaning: 
\begin{itemize}
   \item detune : logical, default=false. If true, the detune subroutine
     is executed.  
   \item distort1 : logical, default=false. If true, the distort1
     subroutine is executed.  
   \item distort2 : logical, default=false. If true, the distort2
     subroutine is executed.  
   \item start\_stop : longitudinal interval of the beam line (in
     m). start and stop should be given as real numbers.  
   \item multipole\_order\_range : the lowest and the largest multipole
     order which will be taken in account. first and last should be
     given as integers.  
   \item noprint : logical, default=false. If true, no file or internal
     table will be created to keep the results. In this case the
     attributes print\_all or print\_at\_end have no effect.  
   \item print\_all : logical, default=false. If true, the files and
     internal tables containing results at each multipole will be
     generated.  
   \item print\_at\_end : logical, default=false. If true, the files and
     internal tables containing results at the end of the position range
     will be generated.  
   \item nosixtrack  : logical, default=false. If true, the input file
     fc.34 will not be generated internally by invoking the conversion
     routine of sixtrack and the user should provide it before the
     execution of the sodd command.  
\end{itemize}

This command will execute the Second Order Detuning and Distortion as
described in the paper of J. Bengtsson and J. Irwin  "Analytical
Calculation of Smear and Tune Shift " (SSC-232, February 1990), on the
beam line defined by the last USE command followed by a TWISS
command. It is based on the stand-alone program written by Frank Schmidt
in November 1998 - January 1999 who also extended the analytical
computation to the second order distortion (cfr. Beam Physics Note 60
F. Schmidt "SODD: A physics Guide").  It consists of three parts:
Detune, Distort1 and Distort2

Note that the first row of every file generated is a header containing
the names of the columns. This row is absent in the internal tables.   

A more detailed description can be found in
\href{http://cern.ch/madx/doc/ab-note-2004-069}{AB-note-2004-069} 
 
\section{Detune} 
The subroutine ``detune'' is launched by the attribute detune and
calculates the detuning function terms in first and second order in the
strength of the multipoles. If the attribute print\_at\_end has been
set, the following two files  (and the corresponding madx tables) are
created :  

\begin{itemize}
   \item \textit{detune\_1\_end} containing five columns :
     \begin{enumerate}
        \item 'multipole order', 
        \item '(hor., ver. plane =\textgreater (1/2)',
        \item 'hor. or ver. detuning', 
        \item 'order of horizontal invariant', 
        \item 'order of vertical invariant'. 
     \end{enumerate}
   \item \textit{detune\_2\_end}  containing five columns : 
     \begin{enumerate}
        \item 'first multipole order', 
        \item 'second multipole order',  
        \item 'horizontal detuning', 
        \item 'order of horizontal invariant', 
        \item 'order of vertical invariant'.  
     \end{enumerate}
\end{itemize}

If the attribute print\_all has been set, the following two files, and
the corresponding madx tables, are created :  

\begin{itemize}
   \item \textit{detune\_1\_all}  containing  five columns :  
     \begin{enumerate}
        \item 'multipole order', 
        \item '(hor., ver. plane =\textgreater (1/2)',  
        \item 'hor. or ver. detuning', 
        \item 'order of horizontal invariant', 
        \item 'order of vertical invariant'. 
     \end{enumerate}
   \item \textit{detune\_2\_all}  containing five columns :  
     \begin{enumerate}
        \item 'first multipole order', 
        \item 'second multipole order',  
        \item 'horizontal detuning', 
        \item 'order of horizontal invariant', 
        \item 'order of vertical invariant'. 
     \end{enumerate}
\end{itemize}
 
\section{Distort 1}
The subroutine distort1 is launched by the attribute distort1 and
calculates the distortion function and the Hamiltonian terms in first
order in the strength of the multipoles. 

If the attribute print\_at\_end has been set, the two files and the
corresponding madx tables are created:  

\begin{itemize} 
    \item \textit{distort\_1\_F\_end} containing eight columns : 
      \begin{enumerate}
         \item 'multipole order', 
         \item 'cosine part of distortion', 
         \item 'sine part of distortion',
         \item 'amplitude of distortion', 
         \item to 8: 'j', 'k', 'l', 'm'. 
      \end{enumerate}
    \item \textit{distort\_1\_H\_end}  containing eight columns : 
      \begin{enumerate}
         \item 'multipole order', 
         \item 'cosine part of Hamiltonian', 
         \item 'sine part of Hamiltonian', 
         \item 'amplitude of Hamiltonian', 
         \item to 8: 'j', 'k', 'l', 'm'.  
      \end{enumerate}
\end{itemize}

If the attribute print\_all has been set, the following two files  (and
the corresponding madx tables) are created :  

\begin{itemize} 
   \item \textit{distort\_1\_F\_all} containing eleven columns :  
     \begin{enumerate}
        \item 'multipole order', 
        \item 'appearance number in position range', 
        \item 'number of resonance', 
        \item 'position', 
        \item 'cosine part of distortion', 
        \item 'sine part of distortion', 
        \item 'amplitude of distortion', 
        \item to 11: 'j', 'k', 'l', 'm'. 
     \end{enumerate}
   \item \textit{distort\_1\_H\_all}  containing eleven columns : 
     \begin{enumerate}
        \item 'multipole order', 
        \item 'appearance number in position range', 
        \item 'number of resonance', 
        \item 'position', 
        \item 'cosine part of Hamiltonian', 
        \item 'sine part of Hamiltonian', 
        \item 'amplitude of Hamiltonian', 
        \item to 11: 'j', 'k', 'l', 'm'. 
     \end{enumerate}
\end{itemize}


\section{Distort 2}
The subroutine distort2 is launched by the attribute distort3 and
calculates the distortion function and Hamiltonian terms in second order
in the strength of the multipoles. 

If the attribute print\_at\_end has been set, the following two files
and the corresponding madx tables are created:  

\begin{itemize}
   \item \textit{distort\_2\_F\_end} containing nine columns : 
     \begin{enumerate}
        \item 'first multipole order',
        \item 'second multipole order',  
        \item 'cosine part of distortion', 
        \item 'sine part of distortion', 
        \item 'amplitude of distortion', 
        \item to 9: 'j', 'k', 'l', 'm'. 
     \end{enumerate}
   \item \textit{distort\_2\_H\_end}  containing nine columns :  
     \begin{enumerate}
        \item 'first multipole order', 
        \item 'second multipole order', 
        \item 'cosine part of Hamiltonian', 
        \item 'sine part of Hamiltonian', 
        \item 'amplitude of Hamiltonian', 
        \item to 9: 'j', 'k', 'l', 'm'.  
     \end{enumerate}
\end{itemize}



%\href{http://xwho.web.cern.ch/xwho/people/show/6175}{damico}, September 10, 2004 
