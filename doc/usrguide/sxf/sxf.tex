%%\title{SXF}
%  Changed by: Chris ISELIN, 27-Jan-1997 

%  Changed by: Hans Grote,  4-Dec-2002 

%%\usepackage{hyperref}
% commands generated by html2latex


%%\begin{document}

\subsection{\href{sxf}{SXF file input and output}} The command 
\begin{verbatim}

SXFWRITE,FILE=filename;
\end{verbatim} writes the currently (i.e. last) USEd sequence with all alignment and field errors in \href{../Introduction/bibliography.html#SXF}{[SXF]} format onto the file specified. This then represents one "instance" of the sequence, where all parameters are given by numbers rather than expressions; the file can be read by other programs to get a complete picture of the sequence. 

 The command 
\begin{verbatim}

SXFREAD,FILE=filename;
\end{verbatim} reads a file in SXF format, stores the sequence away and USEs it(!) in order to keep the existing errors. The following does therefore work: 

 Example: 
\begin{verbatim}

job 1:

! define sequence MYSEQU

use,mysequ;

! add alignment errors and field errors

sxfwrite,file=file;
stop;

job 2:

sxfread,file=file;
twiss;
stop;

\end{verbatim}\href{http://www.cern.ch/Hans.Grote/hansg_sign.html}{hansg}, January 24, 1997 

%%\end{document}
