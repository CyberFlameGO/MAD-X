%%\title{SXF}
%  Changed by: Chris ISELIN, 27-Jan-1997 
%  Changed by: Hans Grote,  4-Dec-2002 

\chapter{SXF file format}

An SXF lattice description is an ascii listing that contains one named,
``flat'', ordered list of elements, delimited as \{\ldots\}, with one
entry for each element. The list resembles a \madx ``sequence'' describing
the entire machine. The syntax is supposed to be adapted for ease of
reading by human beings and for ease of parsing by LEX and YACC. 

\section{SXFWRITE}

The command 
\begin{verbatim}
SXFWRITE, FILE = file_name;
\end{verbatim} 
writes the current  sequence with all alignment and
field errors in \href{../Introduction/bibliography.html#SXF}{[SXF]}
format onto the file specified. This then represents one "instance" of
the sequence, where all parameters are given by numbers rather than
expressions; the file can be read by other programs to get a complete
picture of the sequence.  



\section{SXFREAD}

The command 
\begin{verbatim}
SXFREAD, FILE = file_name;
\end{verbatim} 
reads a file in SXF format, stores the sequence away and USEs it(!) in
order to keep the existing errors. The following does therefore work:

A writing an SXF lattice with errors:
\begin{verbatim}
...
! define sequence MYSEQU
use,mysequ;

! add alignment errors and field errors

sxfwrite, file = file;

stop;
\end{verbatim}

Subsequently the lattice, with errors, can be reloaded in another \madx job:

\begin{verbatim}
sxfread, file = file;
! sequence mysequ is now reloaded and active, complete with errors.

twiss;
...
\end{verbatim}

%\href{http://www.cern.ch/Hans.Grote/hansg_sign.html}{hansg}, January 24, 1997 

