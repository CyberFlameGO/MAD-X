%%\title{MAKETHIN}
%  Changed by: Helmut Burkhardt,  7-April-2005 
%  Changed by: Helmut Burkhardt, Ghislain Roy -  6-Mai-2014 
 
\chapter{MAKETHIN: Slice a sequence into thin lenses}
\label{chap:makethin}

This module converts a sequence with thick elements into one composed of
thin (zero length) element slices or simplified thick slices as required
by MAD-X tracking or conversion to SIXTRACK input format.

\section{Slicing to thin elements}

The slicing is performed by the command: 
\begin{verbatim}
MAKETHIN, SEQUENCE=seq_name, STYLE=slicing_style,
          MAKEDIPEDGE= logical, 
          MINIMIZEPARENTS= logical, 
          MAKECONSISTENT= logical; 
\end{verbatim} 

The parameters are defined as: 
\begin{madlist}
   \ttitem{SEQUENCE} seq\_name is the name of the sequence to be
   processed to thin slices. The sequence must be active, i.e. it should
   have been previously loaded with a USE command.  The sequence must
   use the default positioning of elements ({\tt REFER=centre}). 

   \ttitem{STYLE} the slicing\_style provided as argument is optional
   and selects the slicing style. Available slicing styles are:     

     \begin{madlist}
     \ttitem{SIMPLE}  produces equal strength slices at equidistant
     positions with the kick in the middle of each slice. \\ 
     It is the default if the number of slices is larger than 4. 

     \ttitem{TEAPOT} improves the slice positioning using the algorithm
     described in
     \href{http://accelconf.web.cern.ch/AccelConf/IPAC2013/papers/mopwo027.pdf}{IPAC'13 MOPWO027}.\\
     It is the default if the number of slice is less or equal to 4.

     \ttitem{COLLIM} is the default slicing for collimators. If only
     one slice is chosen it is placed in the middle of the old
     element. If two slices are chosen they are placed at either
     end. Three slices or more are treated as one slice.      

     \ttitem{(option not given)} The standard default slicing for all
     elements (except collimators). It uses {\tt TEAPOT} if the number
     of slices is less or equal to 4, and {\tt SIMPLE} otherwise.
     \end{madlist}
 
   \textbf{Note:} It is strongly recommended to always specify {\tt STYLE=teapot}  
   to use the improved slicing for any number of slices.
      
   \ttitem{MAKEDIPEDGE} is a flag that controls the generation of
   {\tt DIPEDGE} elements at the start and/or end of bending magnets,
   to conserve edge focusing from pole face angles {\tt e1}, {\tt e2}
   or extra fields described by {\tt fint}, {\tt fintx}, in the
   process of slicing bending magnets to thin multipole slices.   
   Selection with {\tt THICK=true} will translate a complex thick 
   {\tt RBEND} or {\tt SBEND}, including edge effects, to a simple
   thick {\tt SBEND} with edge focusing transferred to extra 
   {\tt DIPEDGE} elements. \\ 
   (Default: {\tt MAKEDIPEDGE= false}) \\
   Example:
\begin{verbatim}
!  keep translated rbend as thick sbend
select, flag=makethin, rbend, thick=true;
\end{verbatim}

   \ttitem{MINIMIZEPARENTS} is a flag that controls the removal of
   inconsistent numbers of slices for parent elements. \\
   (Default: {\tt MINIMIZEPARENTS= true})

   \ttitem{MAKECONSISTENT} is a flag to ensure an equal number of slices
   for parent and children elements, using the larger value. \\
   (Default: {\tt MAKECONSISTENT= false})

\end{madlist}

\section{Number of slices}

The number of slices can be set individually for elements or groups of
elements using
\href{http://mad.web.cern.ch/mad/madx.old/Introduction/select.html}{\tt SELECT} 
commands
\begin{verbatim}
SELECT, FLAG=makethin, 
        RANGE=range, CLASS=class, PATTERN=pattern[,FULL][,CLEAR],
        THICK=logical, SLICE=integer;
\end{verbatim}
where the argument to the parameter {\tt SLICE} stands for the number of
slices for the selected elements. The default is one slice and
{\tt THICK=false} for all elements, i.e. conversion of all thick
elements to a single thin slice positioned at the centre of the original
thick element.

Note that {\tt THICK=true} only applies to dipole or quadrupole magnet
elements and is ignored otherwise.  

{\tt MAKETHIN} allows for thick quadrupole slicing with insertion of
markers between thick slices. Positioning is done with markers between
slices, here however with thick slice quadrupole piece filling the whole
length.  \\  
Examples:
\begin{verbatim}
! slice quadrupoles thick, insert 2 markers per quadrupole
select, flag=makethin, class=quadrupole, thick=true, slice=2; 

! thick slicing for quadrupoles named mqxa, insert one marker in the middle
select, flag=makethin, pattern=mqxa\., slice=1,  thick=true; 
\end{verbatim}


Slicing can be turned off for certain elements or classes by specifying
a number of slices $< 1$. Examples: 
\begin{verbatim}
! turn off slicing for sextupoles
select, flag=makethin, class=sextupole, slice=0; 

! keep elements unchanged with names starting by mbxw
select, flag=makethin, pattern=mbxw\., slice=0; 
\end{verbatim}
This option allows to introduce slicing step by step and monitor the
resulting changes in optics parameters.\\
Keep in mind however that subsequent tracking generally requires full slicing, with
possible exception of quadrupoles and bending magents. 


\section{Additional information}

The generated thin lens sequence has the following properties: 
\begin{itemize}
\item The new sequence has the same name as the original. The original sequence
  is replaced by the new one in memory. If the original sequence is
  needed for further processing in MAD-X, it should be reloaded.
\item The algorithm also processes any sub-sequence inserted in the main
  sequence. These sub-sequences are also given the same names as the
  original ones. 
\item Any element transformed into a single thin lens element has the
  same name as the original. 
\item If an element is sliced into more than one slice, the individual
  slices have the same basename as the original element plus a suffix 
  {\tt ..1}, {\tt ..2}, etc. and a marker, with the name of the original
  element, is placed at the location of the center of the original element.
\end{itemize}


{\bf Hints}
\begin{enumerate}
\item Compare the main optics parameters like tunes before and after slicing
  with {\tt MAKETHIN}. Rematch tunes and chromaticity as necessary after
  {\tt MAKETHIN}. 

\item In tests, turn off slicing for some of the main element classes to
  identify the main sources of changes. 

\item For sextuples and octupoles, a single slice should always be sufficient.

\item Increase the number of slices for critical elements like mini-beta
  quadrupoles. Even there, more than four slices should rarely be
  required. 

\item In case of problems or doubts, consider to
  \href{http://madx.web.cern.ch/madx/madX/doc/usrguide/control/seqedit.html#flatten}{flatten}
  the sequence before slicing.  

\item See the
  \href{http://madx.web.cern.ch/madx/madX/examples/makethin/}{examples}
  for makethin. \\ 
  See also the presentations on the upgrade of the makethin module:\\
  \href{http://ab-dep-abp.web.cern.ch/ab-dep-abp/LCU/LCU_meetings/2012/20120918/LCU_makethin_2012_09_18.pdf}{LCU\_makethin\_2012\_09\_18.pdf}, and \\
  \href{http://ab-dep-abp.web.cern.ch/ab-dep-abp/LCU/LCU_meetings/2013/20130419/LCU_makethin_2013_04_19.pdf}{LCU\_makethin\_2013\_04\_19.pdf}. \\ 
  TEAPOT is documented in \href{http://accelconf.web.cern.ch/AccelConf/IPAC2013/papers/mopwo027.pdf}{IPAC'13 MOPWO027}

\end{enumerate}



%%%%%%%%










