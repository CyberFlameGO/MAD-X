%%\title{MAKETHIN}
%  Changed by: Helmut Burkhardt,  7-April-2005 
 
\chapter{MAKETHIN: Slice a sequence into thin lenses}
\label{chap:makethin}

This module converts a sequence with thick elements into one composed
entirely of thin elemtens as required by the default MAD-X tracking.  
 
Slicing is done by the MAKETHIN command: 
\begin{verbatim}
MAKETHIN, SEQUENCE=sequence name,
          STYLE=slicing style;
\end{verbatim} 

The parameters are defined as: 
\begin{itemize}
   \item SEQUENCE chooses the sequence you wish to slice. 
   \item STYLE (optional) chooses the slicing style. The options are:   
   \item SIMPLE : this is a simplified slicing algorithm which produces
     any number of equal strength slices at equidistant positions with
     the kick in the middle of each slice.      
   \item COLLIM : this is the default slicing for collimators. If only
     one slice is  chosen it is placed in the middle of the old
     element. If two slices are chosen they  are placed at either
     end. Three slices or more are treated as one slice.      
   \item TEAPOT : this is the slicing algorithm described in
     \href{../Introduction/bibliography.html#TEAPOT}{[TEAPOT]} but
     generalized to any number of slices.      
   \item      (default): this is the standard default slicing for all
     elements (except collimators). It uses TEAPOT if the number of
     slices $\leq$ 4, and SIMPLE otherwise.     
\end{itemize}
\textbf{By default} all elements are converted to one thin element
positioned at the center of  the thick element. To get a greater slicing
for certain elements use a standard SELECT command with FLAG=MAKETHIN
and  CLASS, RANGE or PATTERN:  

\begin{verbatim}
SELECT,FLAG=MAKETHIN,
       CLASS=class,RANGE=range,
       SLICE=no of slices;
\end{verbatim}

The created thin lens sequence has the following properties: 
\begin{itemize}
   \item The created sequence has the same name as the original. The
     original is therefore no longer  available and has to be reloaded
     if it is needed again.   
   \item The slicer also slices any inserted sequence used in the main
     sequence. These are also given the same names as the originals.  
   \item Any component changed into a single thin lens has the same name
     as the original.  
   \item If a component is sliced into more than one slice, the
     individual slices have the same name as the original component and
     a suffix \texttt{..1}, \texttt{..2}, etc... and a marker  will be
     placed at the center with the original name of the component.  
\end{itemize} 
Hints: 
\begin{itemize}
   \item See the
     \href{http://cern.ch/frs/mad-X_examples/makethin/}{examples} for
     makethin.  
   \item Compare the optics before and after slicing with
     makethin. Consider to increase the number of slices and rematch
     after makethin to \textbf{reach the required accuracy}.  
   \item Consider to replace rbend by sbend + thin quads taking into
     account the edge focusing before slicing with makethin.  
   \item The selection works on the current sequence. Consider to insert
     a "USE,SEQUENCE=.." before SELECT  
\end{itemize}


%Helmut Burkhardt, September 2005 
