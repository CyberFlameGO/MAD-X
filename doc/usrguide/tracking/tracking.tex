%\documentclass[a4paper,11pt]{article}
%%\usepackage{ulem}
%%\usepackage{a4wide}
%%\usepackage[dvipsnames,svgnames]{xcolor}
%%\usepackage[pdftex]{graphicx}
%%\title{Overview of MAD-X Tracking Modules}
% 
% div.Section1
% 	{page:Section1;}
% span.SpellE
% 	{}
% 

%%\usepackage{hyperref}
% commands generated by html2latex


%%\begin{document}
%%\begin{center}

   %%EUROPEAN ORGANIZATION FOR NUCLEAR RESEARCH
   
%%\includegraphics{http://cern.ch/madx/icons/mx7_25.gif}

\section{Overview of MAD-X Tracking Modules}
%%\end{center}


 A number of particles with given initial conditions can be tracked through a 
 beam-line or a ring. The 
 particles can be tracked either for a single passage or for many turns. 


 While MAD-X [\hyperlink{F._Schmidt}{a}] is keeping most of the functionality of 
 its predecessor \href{http://cern.ch/mad8}{MAD-8}, the trajectory tracking in MAD-X is 
 considerably modified 
 comparing to MAD-8 . The reason is that in MAD8 the thick lens tracking is 
 inherently not symplectic, which implies that the phase space volume is not 
 preserved during the tracking, i.e. contrary to the real particle the tracked 
 particle amplitude is either growing or decreasing.


 The non-symplectic tracking as in MAD-8 has been completely excluded from MAD-X 
 by taking out the thick lens part from the tracking modules. Instead two 
 types of tracking modules (both symplectic) are implemented into MAD-X.


 The first part of this design decision is the thin-lens tracking module (\texttt{\href{../thintrack/thintrack.html}{thintrack}}) 
 which tracks symplecticly through drifts and kicks and by replacing 
 the end effects by their symplectic part in form of an additional kick on either 
 end of the element. This method demands a preliminary conversion of a sequence 
 with thick elements into one composed entirely of thin elements (see the
 \texttt{\href{../makethin/makethin.html}{MAKETHIN}} 
 command). The details of its usage are given on the page "\href{../thintrack/thintrack.html}{thintrack}".


 The second part of this design decision is to produce a thick lens tracking 
 module based on the PTC code [\hyperlink{E._Forest}{b}] that allows a 
 symplectic treatment of all accelerator elements giving the user full control 
 over the precision (number of steps and integration type) and exactness (full or 
 extended Hamiltonian) of the results.


 The first PTC thick-lens tracking module is named
 \href{../ptc_track/ptc_track.html}{ptc\_track}. It 
 has the same features as the thin-lens tracking code (\texttt{\href{../thintrack/thintrack.html}{thintrack}}) 
 except it treats thick-lenses in a symplectic manner.


 There is a second PTC tracking module called the line tracking module (\href{../ptc_track_line/ptc_track_line.html}{ptc\_track\_line}). 
 It is meant for tracking particles in
 \href{http://clic-study.web.cern.ch/CLIC-Study/}{CLIC}, in fact it treats beam-lines 
 containing traveling-wave cavities and includes a beam acceleration.

\line(1,0){300}

%\begin{description}
\begin{description}
	\item[References] 
	
%\end{description}
\begin{enumerate}
	\item \href{F._Schmidt}{F. Schmidt}, "`\href{http://cern.ch/madx/doc/MPPE012.pdf}{MAD-X PTC Integration}'', 
   Proc. of the 2005 PAC Conference in Knoxville, USA, pp.1272.
	\item \href{E._Forest}{E.Forest}, F.Schmidt 
   and E.McIntosh, �Introduction to the Polymorphic Tracking Code�, 
   \href{http://cern.ch/madx/doc/sl-2002-044.pdf}{
   CERN-SL-2002-044-AP},
   \href{http://ccdb4fs.kek.jp/cgi-bin/img/allpdf?200302020}{KEK 
   report 2002-3}, July 2002.
\end{enumerate}
%\begin{description}
	\item[See Also] \href{../ptc_general/ptc_general.html}{PTC 
   Set-up Parameters}
%\end{description}
\end{description}

\line(1,0){300}

\href{mailto:kapin@itep.ru}{
 V.Kapin}(ITEP) and \href{mailto:Frank.Schmidt@cern.ch}{
 F.Schmidt}, March  2006

%%\end{document}
