%%\title{PLOT}
%  Changed by: Chris ISELIN, 27-Jan-1997 
%  Changed by: Hans Grote, 25-Sep-2002 
%  Changed by: E. T. d'Amico, 20-Oct-2004 

\chapter{PLOT}


Values contained in MAD-X tables can be plotted in the form column
versus column, with up to four differently scaled vertical axes;
furthermore, if the horizontal axis is the position "s" of the elements
in a sequence, then the symbolic machine can be plotted above the curves
as well. In certain conditions True interpolation inside the element is
available (through calls to the Twiss module for each slice) .  The
"environment" (interpolation, line thickness, annotation size,
PostScript format) can be set with the \hyperlink{setplot}{setplot}
command.   


\section{PLOT}	
\begin{verbatim}
plot, vaxis=vname1,vname2,..,vnamen,
      vaxis1=vname1,vname2,..,vnamen, vaxis2=vname1,vname2,..,vnamen,
      vaxis3=vname1,vname2,..,vnamen, vaxis4=vname1,vname2,..,vnamen,
      haxis=vname, hmin=real, hmax=real, vmin=reals, vmax=reals, 
      bars=integer, style=integer, colour=integer, symbol=integer, 
      noversion=logical, interpolate=logical, noline=logical, notitle=logical, marker_plot=logical, 
      range_plot=logical, 
      table=table_name, particle=particle1,particle2,...,particlen,
      multiple=logical, title=string, range=range, file=file_name_start, 
      ptc=logical, ptc_table=table_name, trackfile=table_name; 
\end{verbatim} 
where the parameters have the following meaning: 

\begin{itemize}
   \item vaxis: one or several variables from the table to be plotted
     against the (only) vertical axis.   
   \item vaxis1: one or several variables from the table to be plotted
     against the vertical axis number 1 (out of 4 possible ones).  
   \item vaxis2: one or several variables from the table to be plotted
     against the vertical axis number 2 (out of 4 possible ones).  
   \item vaxis3: one or several variables from the table to be plotted
     against the vertical axis number 3 (out of 4 possible ones).  
   \item vaxis4: one or several variables from the table to be plotted
     against the vertical axis number 4 (out of 4 possible ones).  
   \item \textit{Important: vaxis and vaxisI are exclusive in their
     application!} 
   \item haxis: name of the horizontal variable 
   \item hmin: lower horizontal edge 
   \item hmax: upper horizontal edge; to be used, both hmin and hmax
     must be given.   
   \item vmin:lower edges of vertical axes, up to four numbers 
   \item vmax:upper edges of vertical axes, up to four numbers; both
     vmin and vmax must be given for an axis to be effective.   
   \item bars: 0 (default) or 1 - in the latter case, all curve points
     coming from the table are connected with the horizontal axis by
     vertical bars.   
   \item style: 1 (default), 2, 3, or 4: curve style, being solid,
     dashed, dotted, and dot-dashed; a value of 100 makes MAD-X use
     these four styles in turn for successive curves in the same
     plot. If style is 0 no curve is printed between points.  N.B. If
     symbol and style are null at the same time, style is forced to its
     default value (= 1).   
   \item colour: 1 (default), 2, 3, , or 5: colour, being black, red,
     green, blue, and magenta; a value of 100 makes MAD-X use these five
     colours in turn for successive curves.  
   \item symbol: 0 (default), 1, 2, 3, 4, or 5: none, dot, "+", "*",
     circle, and "x". These symbols are potted at all curve points;
     there size may have to be adapted (see below).   
   \item noversion: logical, default=false. If set true, the information
     concerning the madx version and the date are suppressed from the
     title. This option frees more space for the user's title.   
   \item interpolate: logical, default=false. Normally the curve points
     from the table are connected by straight lines; if "interpolate" is
     requested, then on-momentum Twiss parameters such as beta, alfa,
     and dispersion are interpolated with calls to the Twiss module
     inside each element, for all other variables splines are used to
     smooth the curves.   
   \item noline: logical, default=false. If s is the horizontal
     variable, then the machine will be plotted in symbolic form above
     the curve plot (except for tables having been read back into
     MAD-X). This may result in a thick black block if the horizontal
     scale is too large. "noline" allows the user to suppress the
     machine plotting.    
   \item notitle: logical, default=false. If true, suppresses the title
     line, including the information on the version and date.  
   \item marker\_plot: logical, default=false. If true, plotting is done
     also at the location of marker elements. This is only useful for
     the plotting of non-continuous functions like the "N1" from the
     aperture module. Beware that the PS file might became very large if
     this flag is invoked.  
   \item range\_plot: logical, default=false. Needed to allow to specify
     a plotting range also for user defined horizontal axis.   
   \item table: name of the table to be plotted from (default:
     twiss). If it is \textit{track}, the data to be plotted are taken
     from the tracking files generated for each required particle as
     defined by the attribute \textit{particle}. The name of this file
     has the following format: file name as defined by the attribute
     \textit{trackfile}, the observation point fixed to 1 and the
     particle number, e.g. \textit{testtrack.obs0001.p0003}.  If the
     required file has not been generated by the previous MAD-X command
     track, no plot is done for that particle.  The plot is obtained
     through the \textit{gnuplot} package.  N.B. the previous track
     command should contain the attribute \textit{dump}. The tracking
     plots appends the plots to an existing file specified via
     \textit{filename} appended by \textit{.ps}. The user should make
     sure that this file does not exist before starting a MAD-X run! 
   \item particle: one or several numbers associated to the tracked
     particles for which the specified plot has to be displayed.  
   \item multiple: logical, default=false. If true all the curves
     generated for each tracked particle are put on one plot. Otherwise
     there will be one plot for each particle.   
   \item title: plot title string; if absent, the last overall title is
     used; if no such overall title as well, the sequence name is used.   
   \item range: horizontal plot
     \href{../Introduction/ranges.html}{range} given by elements.  
   \item file\_name: start of the file name for the Postscript
     file(s). Only the first occurrence of such a name will be
     used. Default is "madx" or "madx\_track" if the \textit{table}
     attribute is track.  Depending on the format (.ps or .eps, see
     below) the plots will either all be written into one file
     file\_name.ps, or one per plot into file\_name01.eps,
     file\_name02.eps, etc.   
   \item ptc: logical, default=false. If set true, the data to be
     plotted are taken from the table defined by the attribute
     \textit{ptc\_table} which is expected to be generated previously by
     the ptc package. The data belong to the column identified by one of
     the names set in the definition of the ptc twiss
     table. Interpolation is not available and the attribute
     \textit{interpolate} has no effect.   
   \item ptc\_table: name of the ptc twiss table to be plotted from
     (default: ptc\_twiss)  
   \item trackfile: first part of the name of the files containing
     tracking data for each particle (default: track)  
\end{itemize}


\section{SETPLOT}
\begin{verbatim}
setplot, post=integer, font=integer, lwidth=real, xsize=real, ysize=real,
         ascale=real, lscale=real, sscale=real, rscale=real;
\end{verbatim} 

where the parameters have the following meaning: 
\begin{itemize}
   \item post: default = 1. If =1, makes one PostScript file (.ps) with
     all plots; if =2, makes one Encapsulated PostSscript file (.eps)
     per plot.   
   \item font: there are two defaults: 1 for screen plotting: this uses
     characters made from polygons; -1 for PostScript files; this is
     Times-Italic. There are various fonts available for positive and
     negative integers, best to be tried out, since they will look
     different on different systems anyway. GhostView will show strange
     vertical axis annotations, but the printed versions are normally
     OK.   
   \item lwidth: default = 1. Allows the user to set the curve line
     width.  Depends on the system as well, so to be tried out.   
   \item xsize: bounding box size for PostScript, default=27 cm.   
   \item ysize: bounding box size for PostScript, default=19 cm.   
   \item ascale: annotation character height scale factor, default=1.   
   \item lscale: axis label character height scale factor, default=1.  
   \item sscale: curve symbol (see above) scale factor, default=1.  
   \item rscale: axis text character height scale factor, default=1.  
\end{itemize}


\section{RESPLOT}
\begin{verbatim}
 resplot; 
\end{verbatim} 
resets all defaults for the setplot command.  


%\href{http://www.cern.ch/Hans.Grote/hansg_sign.html}{hansg}, June 17, 2002, 
%rdemaria \href{http://cern.ch/rdemaria}{rdemaria}, September 2007. 


