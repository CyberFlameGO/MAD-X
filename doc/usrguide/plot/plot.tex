%%\title{PLOT}
%  Changed by: Chris ISELIN, 27-Jan-1997 
%  Changed by: Hans Grote, 25-Sep-2002 
%  Changed by: E. T. d'Amico, 20-Oct-2004 

\chapter{PLOT}

Values contained in MAD-X tables can be plotted in the form column
versus column, with up to four differently scaled vertical axes;
furthermore, if the horizontal axis is the position "s" of the elements
in a sequence, then the symbolic machine can be plotted above the curves
as well. In certain conditions True interpolation inside the element is
available (through calls to the Twiss module for each slice) .  

The "environment" (interpolation, line thickness, annotation size,
PostScript format) can be set with the \hyperlink{setplot}{setplot}
command.   


\section{PLOT}	
\begin{verbatim}
plot, vaxis=  vname1,vname2,..,vnamen,
      vaxis1= vname1,vname2,..,vnamen, 
      vaxis2= vname1,vname2,..,vnamen,
      vaxis3= vname1,vname2,..,vnamen, 
      vaxis4= vname1,vname2,..,vnamen,
      haxis=  vname, hmin=real, hmax=real, 
      vmin= reals, vmax= reals, 
      bars= integer, style= integer, colour= integer, symbol= integer, 
      noversion= logical, interpolate= logical, noline= logical, notitle= logical, 
      marker_plot= logical, range_plot= logical, 
      table= table_name, particle= particle1,particle2,...,particlen,
      multiple= logical, title= string, range= range, 
      file= file_name_start, trackfile= table_name,
      ptc= logical, ptc_table= table_name; 
\end{verbatim} 

where the parameters have the following meaning: 

\begin{itemize}
   \item vaxis: one or several variables from the table to be plotted
     against a single vertical axis.   
   \item vaxis1: one or several variables from the table to be plotted
     against vertical axis number 1 (out of 4 possible vertical axes).  
   \item vaxis2: one or several variables from the table to be plotted
     against vertical axis number 2 (out of 4 possible vertical axes).  
   \item vaxis3: one or several variables from the table to be plotted
     against vertical axis number 3 (out of 4 possible vertical axes).  
   \item vaxis4: one or several variables from the table to be plotted
     against vertical axis number 4 (out of 4 possible vertical axes). \\ 
   \textit{Important: vaxis and vaxisI are exclusive in their
     application!} 
   \item haxis: name of the horizontal variable 
   \item hmin: lower edge for horizontal axis 
   \item hmax: upper edge for horizontal axis; to be used, both hmin and hmax
     must be given.   
   \item vmin: lower edge(s) for vertical axis or axes, up to four
     numbers separated by commas.
   \item vmax: upper edge(s) of vertical axis or axes, up to four
     numbers separated by commas; 
     Note that both vmin and vmax must be given for an axis to be effective.   
   \item bars: 0 (default) or 1 - with bars=1, all data points
     are connected to the horizontal axis with vertical bars.   
   \item style: 1 (default), 2, 3, or 4: line style for connecting the
     successive data points, respectively solid, dashed, dotted, and dot-dashed; 
     the special value style=100 uses the four styles in turn for
     successive curves in the same plot. 
     With style=0 successive data points are not connected. 
     N.B. If symbol and style are both zero at the same time, style is
     forced to its default value (style=1).
   \item colour: 1 (default), 2, 3, 4 or 5: colour for the symbols and lines, 
     respectively black, red, green, blue, and magenta; 
     The special value colour=100  uses the five colours in turn for
     successive symbols and lines.
   \item symbol: 0 (default), 1, 2, 3, 4, or 5: The symbols to be
     plotted at data points, respectively none, dot ("."), plus ("+"),
     star ("*"), circle ("o"), and cross ("x").  
     The symbol size may have to be adapted through the SETPLOT command
     (see below).   
   \item noversion: logical, default=false. If noversion=true, the
     information concerning the version of MAD-X and the date of the run
     are suppressed from the title.  
     This option frees additional space available for the user specified
     title of the plot.  
   \item interpolate: logical, default=false. The data points are
     normally connected by straight lines; if "interpolate" is
     specified, the on-momentum Twiss parameters such as beta, alfa, and
     dispersion are interpolated with calls to the Twiss module inside
     each element; for all other variables splines are used to smooth
     the curves.  
   \item noline: logical, default=false. If s is the horizontal
     variable, then a symbolic representation of the beamline is plotted
     above the plot, except for tables that have been read back into MAD-X. 
     In case the horizontal scale is too large and the density of
     beamline elements is too high, this may result in hardly legible
     representation and a thick black block in teh worst case. 
     The "noline" option suppresses this symbolic representation of the
     beamline. 
   \item notitle: logical, default=false. If true, suppresses the title
     line, including the information on the version and date.  
   \item marker\_plot: logical, default=false. If true, plotting is done
     also at the location of marker elements. This is only useful for
     the plotting of non-continuous functions like the "N1" from the
     aperture module. Beware that the PS file might became very large if
     this flag is invoked.  
   \item range\_plot: logical, default=false. Allows the specification
     of a plotting range for the user defined horizontal axis.   


   \item table: name of the table from which data is plotted (default:
     twiss).  

     If the first part of the name of the table is "track", the
     data to be plotted are taken from the tracking file(s) generated by
     a previous TRACK command for each requested particle as defined by
     the \textit{particle} attribute. If the required file has not been
     generated by a preceding TRACK command, no plot is done for that
     particle.   

     The plot is generated through the GNUPLOT program and is available
     in the format specified by the SETPLOT command. 

     The preceding TRACK command should contain the attribute \textit{DUMP}
     and may contain the attribute \textit{ONETABLE}. The tracking plots
     are appended to the file file\_name.ps where file\_name can be
     specified via the attribute \textit{filename=file\_name}. Note that
     the plots are appended to this file and the file is not
     overwritten. 

     The PLOT command uses the following tracking output files depending on
     the name of the table.  
     With the attribute \textit{table=trackone}, the data file is assumed
     to have been generated with the \textit{ONETABLE=true} attribute of
     the TRACK command, and the file name has the following format: 
     \textit{basisone} where the basis for the file name is defined by the
     attribute \textit{trackfile=basis} (default=track).

     With the attribute \textit{table=trackxxx} where xxx is any string
     other than "one", the data files are assumed to have been generated
     with the \textit{ONETABLE=false} attribute of the TRACK command, and
     the file names have the following format: \textit{basis.obs0001.p00i}
     where the basis for the file name is defined by the attribute
     \textit{trackfile=basis} (default=track), the observation point fixed
     is to 1 and the particle number \textit{i} is given by the attribute
     \textit{particle=i}. 
 
   \item particle: one or several numbers associated to the tracked
     particles for which the specified plot has to be displayed.  

   \item multiple: logical, default=false. If true all the curves
     generated for each tracked particle are put on one plot. Otherwise
     there will be one plot for each particle.   

   \item title: plot title string; if absent, the last overall title is
     used; if no such overall title as well, the sequence name is used.   

   \item range: horizontal plot
     \href{../Introduction/ranges.html}{range} given by elements.  

   \item file\_name: start of the file name for the Postscript
     file(s). Only the first occurrence of such a name will be
     used. Default is "madx" or "madx\_track" if the \textit{table}
     attribute is track.  Depending on the format (.ps or .eps, see
     below) the plots will either all be written into one file
     file\_name.ps, or one per plot into file\_name01.eps,
     file\_name02.eps, etc.   

   \item ptc: logical, default=false. If set true, the data to be
     plotted are taken from the table defined by the attribute
     \textit{ptc\_table} which is expected to be generated previously by
     the ptc package. The data belong to the column identified by one of
     the names set in the definition of the ptc twiss
     table. Interpolation is not available and the attribute
     \textit{interpolate} has no effect.   

   \item ptc\_table: name of the ptc twiss table to be plotted from
     (default: ptc\_twiss)  

   \item trackfile: first part of the name of the files containing
     tracking data for each particle (default: track)  
\end{itemize}


\section{SETPLOT}
\begin{verbatim}
setplot, post= integer, font= integer, lwidth= real, xsize= real, ysize= real,
         ascale= real, lscale= real, sscale= real, rscale= real;
\end{verbatim} 

where the parameters have the following meaning: 
\begin{itemize}
   \item post: default = 1. If =1, makes one PostScript file (.ps) with
     all plots; if =2, makes one Encapsulated PostSscript file (.eps)
     per plot.   
   \item font: there are two defaults: 1 for screen plotting: this uses
     characters made from polygons; -1 for PostScript files; this is
     Times-Italic. There are various fonts available for positive and
     negative integers, best to be tried out, since they will look
     different on different systems anyway. GhostView will show strange
     vertical axis annotations, but the printed versions are normally
     OK.   
   \item lwidth: default = 1. Allows the user to set the curve line
     width.  Depends on the system as well, so to be tried out.   
   \item xsize: bounding box size for PostScript, default=27 cm.   
   \item ysize: bounding box size for PostScript, default=19 cm.   
   \item ascale: annotation character height scale factor, default=1.   
   \item lscale: axis label character height scale factor, default=1.  
   \item sscale: curve symbol (see above) scale factor, default=1.  
   \item rscale: axis text character height scale factor, default=1.  
\end{itemize}


\section{RESPLOT}
\begin{verbatim}
resplot; 
\end{verbatim} 
resets all defaults for the setplot command.  


\section{First example for plots of tracking data}

The following MAD-X code sample defines the tracking of four particles 
with the generation of a single file with name \textit{basisone} 
holding the tracking data for all four particles.  

\begin{verbatim}
// track particles
track, file=basis, dump, onetable;
 start, x= 2e-3, px=0, y= 2e-3, py=0;
 start, x= 4e-3, px=0, y= 4e-3, py=0;
 start, x= 6e-3, px=0, y= 6e-3, py=0;
 start, x= 8e-3, px=0, y= 8e-3, py=0;
 run,turns=1024;
endtrack;
\end{verbatim}

The following sample code defines the plotting of the x-px and y-py
phase space coordinates for all four particles. 
It takes into account the fact that all coordinates are in a single file 
with \textit{table=trackone} and defines the filename where tracking data 
is to be found (\textit{basisone}) with \textit{trackfile=basis}. 

\begin{verbatim}
// plot trajectories
setplot, post=1; 
title, "FODO phase-space test";
plot, file=plot, table=trackone, trackfile=basis, noversion, multiple, 
      haxis=x, vaxis=px, particle=1,2,3,4; 
plot, file=plot, table=trackone, trackfile=basis, noversion, multiple, 
      haxis=y, vaxis=py, particle=1,2,3,4;
\end{verbatim}

With each plot command a temporary file \textit{gnu\_plot.gp} containing
GNUPLOT instructions is generated.  
The file generated by the first plot command reads: 

{\footnotesize \begin{verbatim}  
set terminal postscript color
set pointsize 0.48
set output 'tmpplot.ps'
set title "FODO phase-space test"
set xlabel 'x'
set ylabel 'px'
plot 'basisone' using 3:($1==1 ? $4 : NaN) title 'particle 1' with points pointtype 1 , \
     'basisone' using 3:($1==2 ? $4 : NaN) title 'particle 2' with points pointtype 2 , \
     'basisone' using 3:($1==3 ? $4 : NaN) title 'particle 3' with points pointtype 3 , \
     'basisone' using 3:($1==4 ? $4 : NaN) title 'particle 4' with points pointtype 4 
\end{verbatim}}

MAD-X then calls GNUPLOT as a subprocess to execute this file, which
generates the file \textit{tmpplot.ps}.  
The file \textit{tmpplot.ps} is then {\bf appended} to the file 
{\textit plot.ps} determined by the attribute \textit{file=plot}.  
The files \textit{gnu\_plot.gp} and \textit{tmpplot.ps} are then
discarded. 

The same process is repeated for the second plot command, resulting in a
growing file \textit{plot.ps}.


\section{Second example for plots of tracking data}

The following MAD-X code sample defines the tracking of four particles 
with the generation of individual files with name
\textit{basis.obs0001.p000i} with \textit{i=1..4}  
holding the tracking data for each of the four particles.  

\begin{verbatim}
// track particles
track, file=basis, dump;
 start, x= 2e-3, px=0, y= 2e-3, py=0;
 start, x= 4e-3, px=0, y= 4e-3, py=0;
 start, x= 6e-3, px=0, y= 6e-3, py=0;
 start, x= 8e-3, px=0, y= 8e-3, py=0;
 run,turns=1024;
endtrack;
\end{verbatim}

The following sample code defines the plotting of the x-px and y-py
phase space coordinates for all four particles with the data of all four
particles on a single plot.  
It takes into account the fact that coordinates for all four particles
are in separate files with 
\textit{table={\underline track}fodo} and defines the filename where tracking
data is to be found (\textit{basis.obs0001.p000i}) with
\textit{trackfile=basis}.  

\begin{verbatim}
// plot trajectories
setplot, post=1; 
title, "FODO phase-space test";
plot, file=plot, table=trackfodo, trackfile=basis, noversion, multiple, 
      haxis=x, vaxis=px, particle=1,2,3,4; 
plot, file=plot, table=trackfodo, trackfile=basis, noversion, multiple, 
      haxis=y, vaxis=py, particle=1,2,3,4;  
\end{verbatim}

With each plot command a temporary file gnu\_plot.gp containing GNUPLOT instruction is generated. 
The file generated by the first plot command reads: 

{\footnotesize \begin{verbatim}
set terminal postscript color
set pointsize 0.48
set output 'tmpplot.ps'
set title "FODO phase-space test"
set xlabel 'x'
set ylabel 'px'
plot 'basis.obs0001.p0001' using 3:4 title 'particle 1' with points pointtype 1 , \
     'basis.obs0001.p0002' using 3:4 title 'particle 2' with points pointtype 2 , \
     'basis.obs0001.p0003' using 3:4 title 'particle 3' with points pointtype 3 , \
     'basis.obs0001.p0004' using 5:4 title 'particle 4' with points pointtype 4 
\end{verbatim}}

MAD-X then calls GNUPLOT as a subprocess to execute this file, which
generates the file \textit{tmpplot.ps}.  
The file \textit{tmpplot.ps} is then {\bf appended} to the file
\textit{plot.ps} determined by the attribute \textit{file=plot}.  
The files \textit{gnu\_plot.gp} and \textit{tmpplot.ps} are then
discarded. 

The same process is repeated for the second plot command, resulting in a
growing file \textit{plot.ps}.


%\href{http://www.cern.ch/Hans.Grote/hansg_sign.html}{hansg}, June 17, 2002, 
%rdemaria \href{http://cern.ch/rdemaria}{rdemaria}, September 2007. 


