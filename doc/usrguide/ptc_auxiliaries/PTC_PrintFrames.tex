%%\title{PTC\_PRINTFRAMES}

\section{PTC\_PRINTFRAMES}

\begin{verbatim}
PTC_PRINTFRAMES, 
    file = [s, none] ,
    format = [s, text] ; 
\end{verbatim}

Print the PTC geometry of a layout to a specified file.   \\

{\bf Command parameters and switches}
\begin{itemize}
   \item {\bf file}=string (Default: NULL)\\
     Specifies the name of the file.   
   \item {\bf format}=string (Default: text)\\
     Format of geometry.\\
     Currently two formats are accepted:       
     \begin{itemize}
	\item text: Prints a simple text file.           
	\item rootmacro: Creates \href{http://root.cern.ch}{root} macro
          that produces 3D display of the geometry.            
     \end{itemize}
\end{itemize}


{\bf Example }\\
\href{http://cern.ch/frs/mad-X_examples/ptc_madx_interface/eplacement/eplacement.madx}{Dog
  leg chicane} with some elements displaced with help of
ptc\_eplacement.

 
% <h3> PROGRAMMERS MANUAL </h3>
% 
% <p> 
% The command is implemented pro_ptc_knob function in madxn.c and 
% by subroutine xxxx in madx_ptc_xxx.f90.
% <p>
% Sopecified range is resolved with help of get_range command. Number of the element in the current sequence
% is resolved and passed as the parameter to the fortran routine. It allows to resolve uniquely the corresponding
% element in the PTC layout.
% <p>
