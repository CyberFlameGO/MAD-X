%\documentclass[a4paper,11pt]{article}
%%\usepackage{ulem}
%%\usepackage{a4wide}
%%\usepackage[dvipsnames,svgnames]{xcolor}
%%\usepackage[pdftex]{graphicx}
%%\title{PTC\_SELECT\_MOMENT}
%%\usepackage{hyperref}
% commands generated by html2latex


%%\begin{document}
%%\begin{center}
   %%EUROPEAN ORGANIZATION FOR NUCLEAR RESEARCH   
%%\includegraphics{http://cern.ch/madx/icons/mx7_25.gif}

\section{PTC\_SELECT\_MOMENT}
%%\end{center}
%   ##########################################################              

%   ##########################################################              

%   ##########################################################              

%   ##########################################################              


\subsubsection{   USER MANUAL   }
%   ##########################################################              


\paragraph{SYNOPSIS}
\begin{verbatim}

PTC_SELECT_MOMENT, 
table      = [s, none, none], 
column     = [s, none, none], 
moments   = [s, none] , 
moments   = [i, {0}] , 
parametric = [l, false, true], 

\end{verbatim}
%   ##########################################################              


\textbf{ Description }\\

 Selects a moment to be: 

    a) \textbf{Stored in a user specified table and column. }

    b) \textbf{Stored as a function (taylor series) of        \href{PTC_Knob.html}{knobs},  if any is defined.}        Than, \textit{parametric} switch should be set to true.  

 Both a) and b) can be joined in one command.    \\

\textbf{ Examples}\\

\href{http://cern.ch/frs/mad-X_examples/ptc_madx_interface/moments/moments.madx}{ ATF2 }
%   ##########################################################              


\paragraph{ Command parameters and switches }
\begin{description}
	\item[\textbf{ moment\_s }] \textit{ list of coma separated strings composed of up to 6 digits }

 Defines moment of the polynomial in PTC nomenclature.        String 'ijklmn' (i,j,k,l,m,n are digits ) defines        $<$x$^i$p$_x$$^j$y$^k$p$_y$$^l$       $\Delta$T$^m$($\Delta$p/p)$^n$$>$.       For example, moments=1000000 defines $<$x$^1$$>$   

       Note that for input we always use MAD-X notation where dp/p is always the 6th coordinate.       (Internaly PTC dp/p is the 5th coordinate. We perform automatic conversion that is transparent for the user.)       As the consequence RMS in dp/p is always defined as 000002, even in 5D case.   

          This notations allows to define more then one moment with one command. In this case,       the corresponding column names are as the passed strings with "mu" prefix.       However, they are always extended to 6 digits, i.e. the trailing 0 are automatically added.       For example, if specified moments=2, the column name is mu200000.    

           This method does not allow to pass bigger numbers then 9. If you need to define       such a moment, use the command switch below.   
	\item[\textbf{ moment }] \textit{ list of up to 6 coma separated integers, }

 Defines a moment. For example: moment=2 defines $<$x$^2$$>$ ,        moment=0,0,2 : $<$y$^2$$>$,        moment=0,14,0,2 : $<$px$^{14}$py$^2$$>$, etc.   
	\item[\textbf{ table }] \textit{ string, default "moments" }

 Specifies name of the table where the calculated moments should be stored.   
	\item[\textbf{ column }] \textit{ string }

 Ignored if \textit{ moments } is specified.        Defines name of the column where values should be stored.        If not specified then it is automatically generated from moment the definition       $<$x$^i$p$_x$$^j$y$^k$p$_y$$^l$       $\Delta$T$^m$($\Delta$p/p)$^n$$>$ =$>$       mu\_i\_j\_k\_l\_m\_n (numbers separated with underscores).                 
	\item[\textbf{ parametric }] \textit{ logical, default false, if value explicitly not specified then true}

 If it is true, and any       \href{PTC_Knob.html}{knobs}        are defined the map element is stored as the parametric result.            
\end{description}

\paragraph{}
%  ############################################################ 

%  ############################################################ 

%  ############################################################ 

% 
% <h3> PROGRAMMERS MANUAL </h3>
% 
% <p> 
% The command is implemented pro_ptc_SELECT function in madxn.c and 
% by subroutine xxxx in madx_ptc_xxx.f90.
% <p>
% Sopecified range is resolved with help of get_range command. Number of the element in the current sequence
% is resolved and passed as the parameter to the fortran routine. It allows to resolve uniquely the corresponding
% element in the PTC layout.
% <p>
% 
% 


%%\end{document}
