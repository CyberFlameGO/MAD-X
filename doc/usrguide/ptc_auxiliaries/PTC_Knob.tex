%%\title{PTC\_KNOB}

\section{PTC\_KNOB}

\begin{verbatim}
PTC_KNOB, 
    elementname = [s, none] , 
    kn    = [i, {-1}], 
    ks    = [i, {-1}], 
    exactmatch = [l, true, true] ; 
\end{verbatim}

 Sets knobs in PTC calculations (currently ony in PTC\_TWISS,
 PTC\_NORMAL will follow). Knobs appear as the additional parameters of
 the phase space. Twiss functions are then obtained  as functions of
 these parameters (Taylor series).  Also map elements might be stored as
 functions of knobs, see  \href{PTC_Select.html}{ ptc\_select} command
 description to lear how to request given element to be stored as a
 Taylor series.  \\

Further, the parametric results can be: 
\begin{enumerate}
   \item  written to a file with
     \href{PTC_PrintParametric.html}{ptc\_printparametric}. 
   \item  plotted and studied using rviewer command (rplot plugin). 
   \item  used to obtain very quickly approximate values of lattice
     functions for given values of knobs
     (\href{PTC_SetKnobValue.html}{ptc\_setknobvalue}). This
     feature is the foundation of a fast matching algorithm with
     PTC.      
\end{enumerate}


{\bf Command parameters and switches}
\begin{itemize}
   \item {\bf elementname}=string in range format (Default: NULL)\\
     Specifies name of the element containing the knob(s) to be set.   
   \item {\bf kn,ks}=list of integers (Default: ???)\\
     Defines which order    
   \item {\bf exactmatch}=logical (Default: .true.)\\
     Normally a knob is a property of a single element in a layout.
     The specified name must match 1:1 to an element name. This is the
     case when exactmatch is true.\\  
     Knobs might be also set to all family of elements. In such case
     the exactmatch switch must be false. A given order field
     component of all the elements that name starts with the
     name specified by the user become a single knob.
   \item {\bf initial}
\end{itemize}


{\bf Example}

\href{http://cern.ch/frs/mad-X_examples/ptc_madx_interface/knobs/knobs.madx}{dog
  leg chicane}: Dipolar components of both rbends and dipolar and
quadrupolar components of the focusing quads set as knobs. Some first
and second order map coefficients set to be stored as parametric
results. ptc\_twiss command is performed and the parametric results are
written to files in two formats. 

\href{http://cern.ch/frs/mad-X_examples/ptc_madx_interface/matchknobs/matchknobs.madx}{dog
  leg chicane}: Knob values are matched to get requested lattice
functions.  


 
% <h3> PROGRAMMERS MANUAL </h3>
% 
% <p> 
% The command is implemented pro_ptc_knob function in madxn.c and 
% by subroutine xxxx in madx_ptc_xxx.f90.
% <p>
% Sopecified range is resolved with help of get_range command. Number of the element in the current sequence
% is resolved and passed as the parameter to the fortran routine. It allows to resolve uniquely the corresponding
% element in the PTC layout.
% <p>
