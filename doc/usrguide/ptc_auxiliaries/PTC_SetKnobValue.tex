%\documentclass[a4paper,11pt]{article}
%%\usepackage{ulem}
%%\usepackage{a4wide}
%%\usepackage[dvipsnames,svgnames]{xcolor}
%%\usepackage[pdftex]{graphicx}
%%\title{PTC\_SETKNOBVALUE}
%%\usepackage{hyperref}
% commands generated by html2latex


%%\begin{document}
%%\begin{center}
   %%EUROPEAN ORGANIZATION FOR NUCLEAR RESEARCH   
%%\includegraphics{http://cern.ch/madx/icons/mx7_25.gif}

\section{PTC\_SETKNOBVALUE}
%%\end{center}
%   ##########################################################              

%   ##########################################################              

%   ##########################################################              

%   ##########################################################              


\subsubsection{   USER MANUAL   }
%   ##########################################################              


\paragraph{SYNOPSIS}
\begin{verbatim}

PTC_SETKNOBVALUE, 
elementname = [s, none] , 
kn    = [i, {-1}], 
ks    = [i, {-1}], 
value = [r] ; 

\end{verbatim}
%   ##########################################################              


\textbf{ Description }\\
  With this command the user set a given knob value. In its effect all the values in 
\begin{itemize}
	\item  the twiss table used by the last ptc\_twiss command 
	\item  the columns specified with \href{PTC_Select.html}{ptc\_select}, parametric=true;
\end{itemize} are reevaluated using the buffered parametric results.  

 The parameters of the command basically contains the fields that allow to identify uniquely the knob and the value to be set.    \\

\textbf{ Example }\\

\href{http://cern.ch/frs/mad-X_examples/ptc_madx_interface/matchknobs/matchknobs.madx}{  dog leg chicane } : strength of dipol field component in quads is matched to obtain required R56 value.  
%   ##########################################################              


\paragraph{ Command parameters and switches }
\begin{description}
	\item[\textbf{ elementname }] \textit{ string in range format, }

 Specifies name of the element containing the knob to be set.   
	\item[\textbf{ kn,ks }] \textit{ list of integers,   }

 Defines the knob   
	\item[\textbf{ value }] \textit{ real, default 0, if value explicitly not specified then 0}

 Specifies the value the knob is set to.             
\end{description}
%  ############################################################ 

%  ############################################################ 

%  ############################################################ 

% 
% <h3> PROGRAMMERS MANUAL </h3>
% 
% <p> 
% The command is implemented pro_PTC_SETKNOBVALUE function in madxn.c and 
% 
% 


%%\end{document}
