%\documentclass[a4paper,11pt]{article}
%%\usepackage{ulem}
%%\usepackage{a4wide}
%%\usepackage[dvipsnames,svgnames]{xcolor}
%%\usepackage[pdftex]{graphicx}
%%\title{PTC\_MOMENTS}
%%\usepackage{hyperref}
% commands generated by html2latex


%%\begin{document}
%%\begin{center}
   %%EUROPEAN ORGANIZATION FOR NUCLEAR RESEARCH   
%%\includegraphics{http://cern.ch/madx/icons/mx7_25.gif}

\section{PTC\_MOMENTS}
%%\end{center}
%   ##########################################################              

%   ##########################################################              

%   ##########################################################              

%   ##########################################################              


\subsubsection{   USER MANUAL   }
%   ##########################################################              


\paragraph{SYNOPSIS}
\begin{verbatim}

PTC_MOMENTS, 
no = [i, 1], 
xdistr   = [s, gauss, gauss], 
ydistr   = [s, gauss, gauss], 
zdistr   = [s, gauss, gauss], 

\end{verbatim}
%   ##########################################################              


\textbf{ Description }\\

 Calculates moments previously selected with the  \href{PTC_SelectMoment.html}{ptc\_select\_moment} command.  It uses maps saved by the ptc\_twiss command, hence, the savemaps switch of ptc\_twiss must be set to true (default) to be able to calculate moments.  \\

\textbf{ Examples}\\

\href{http://cern.ch/frs/mad-X_examples/ptc_madx_interface/moments/moments.madx}{ ATF2 }
%   ##########################################################              

\paragraph{ Command parameters and switches }
\begin{description}
	\item[\textbf{ no }] \textit{ integer }

 order of the calculation, maximally twise the order of the last twiss   
	\item[\textbf{ xdistr, ydistr, zdistr }] \textit{ string defining type of distribution for x, y, z dimension, respectively,     
\begin{enumerate}
	\item \textbf{gauss} - Gaussian
	\item \textbf{flat5} - flat distribution in the first of variables (dp over p) of a given dimension                           and Delta Dirac in the second one (T) 
	\item \textbf{flat56} - flat rectangular distribution 
\end{enumerate}}


\end{description}

%\paragraph{}
%  ############################################################ 

%  ############################################################ 

%  ############################################################ 

% 
% <h3> PROGRAMMERS MANUAL </h3>
% 
% <p> 
% The command is implemented pro_ptc_SELECT function in madxn.c and 
% by subroutine xxxx in madx_ptc_xxx.f90.
% <p>
% Sopecified range is resolved with help of get_range command. Number of the element in the current sequence
% is resolved and passed as the parameter to the fortran routine. It allows to resolve uniquely the corresponding
% element in the PTC layout.
% <p>
% 
% 


%%\end{document}
