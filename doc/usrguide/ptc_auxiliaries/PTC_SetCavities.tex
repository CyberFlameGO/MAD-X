%%\title{PTC\_SetCavities}

\section{PTC\_SetCavities}

\begin{verbatim}
PTC_SetCavities;
\end{verbatim}

This command adjusts cavities and sets appropriate reference momenta for
a layout containing traveling wave cavities.

This command sets up the properties of a layout and traveling wave
cavities.  The main goal is to update  reference beam energy for the
elements  that follow a traveling wave cavity.  It traces the
synchronous particle, i.e. one that has  all its parameters set to zero
at the beginning of the layout under study. At the point it arrives to a
cavity,  the parameters of the latter one are adjusted according to the
switches  defined by the user. 

There are 2 cases   
\begin{enumerate}
   \item \textbf{Leaves all parameters untouched}
   \item \textbf{Phase of cavity is adjusted so it gives the maximum
     acceleration} Afterwards to the calculated phase the lag
     is added. This setting is acquired using set\_switch
     command, setting maxaccel parameter to true.   
\end{enumerate} 

Afterwards,  the synchronous particle is tracked through traveling wave
cavity  and its energy gain is known.  This energy becomes the reference
one for all the elements downstream of the cavity.  The particle is
tracked further to the next cavity,  for which the procedure described
above is repeated.    

Parameters of the cavities are dumped to the file named
twcavsettings.txt.    

At the end patches at the ends of the cavities are set,  so the
parameters after them are  calculated taking to the account reference
energy increase.   

The exact program behavior depends on the  \href{PTC_SetSwitch.html}{
  PTC switches settings}.   

Please note that in PTC phase velocity of a cavities wave is always
equal to speed of light.  Hence, if PTC internal state TIME is TRUE,
what is the most correct setting,  then voltage seen by a particle is
varying along the structure. If TIME is FALSE,  track is assumed to fly
with speed of light and in such case a particle moves together with the
wave front.    


{\bf PROGRAMMERS MANUAL}

{\bf CAUTION} For the time being cavities MUST not be placed one after
another, and at least a marker must be inserted between two neighboring
accelerating structures. Otherwise, program will stop with the error
message.    
