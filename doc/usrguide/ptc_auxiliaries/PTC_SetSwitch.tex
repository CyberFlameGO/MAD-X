%\documentclass[a4paper,11pt]{article}
%%\usepackage{ulem}
%%\usepackage{a4wide}
%%\usepackage[dvipsnames,svgnames]{xcolor}
%%\usepackage[pdftex]{graphicx}
%%\title{PTC\_SETSWITCH}
%%\usepackage{hyperref}
% commands generated by html2latex


%%\begin{document}
%%\begin{center}
   %%EUROPEAN ORGANIZATION FOR NUCLEAR RESEARCH   
%%\includegraphics{http://cern.ch/madx/icons/mx7_25.gif}

\section{PTC\_SETSWITCH}

\subsection{ routine that sets the internal PTC switches }
%%\end{center}
%   ##########################################################              

%   ##########################################################              

%   ##########################################################              


\subsubsection{   USER MANUAL   }
%   ##########################################################              


\paragraph{SYNOPSIS}
\begin{verbatim}

PTC_SETSWITCH,
debuglevel = [i,0], 
maxacceleration = [l, true, true],
exact_mis = [l, false, true],
totalpath = [l, false, true],
radiation = [l false, true],
fringe = [l, false, true],
time = [l, true, true];
\end{verbatim}
%   ##########################################################              


\textbf{ Description } \\

Using this command the user can set switches of PTC and the MAD-X-PTC interface, adapting  this way the program behavior to his needs.  

\paragraph{ Command parameters and switches }
\begin{description}
	\item[\textbf{debuglevel}] \textit{ integer, default value 1, no default value if value explicitly not specified}

 Sets the level of debugging printout 0 none, 4 everything   
	\item[\textbf{maxacceleration}] \textit{ logical, default true, if value explicitly not specified then true }

 Switch saying to set cavities phases so the reference orbit is always on the crest, i.e. gains max energy   
	\item[\textbf{ exact\_mis }] \textit{ logical, default false, true if value explicitly specified }

 Switch ensures exact misalignment treatment.   
	\item[\textbf{ totalpath }] \textit{ logical, default false, true if value explicitly specified }

 If true, the 6th variable of PTC, i.e. 5th of MAD-X is the total path.       If false it is deviation from the reference particle, which is normally the closed orbit for closed       layouts.   
	\item[\textbf{ radiation }] \textit{ logical, default false, if value explicitly not specified then true}:    

 Sets the radiation switch/internal state of PTC.   
	\item[\textbf{ fringe }] \textit{ logical, default false, true if value explicitly specified}:    

 Sets the fringe switch/internal state of PTC. If true the influence of the fringe fields is evaluated for       all the elements.    

 Please note that currently fringe fields are always taken to the account for some elements       (f.g. traveling wave cavities) even if this flag is set to false.        The more detailed list of the elements will be provided later,        when the situation in this matter will be definitely settled.   
	\item[\textbf{ time }] \textit{ logical, default true, if value explicitly not specified then true }:    

 If true, Selects time of flight rather than path length. (\textit{cT} to be precise)       as the 6th variable of PTC, i.e. 5th of MAD-X.   
\end{description}
%  ############################################################ 

%  ############################################################ 

%  ############################################################ 


\subsubsection{ PROGRAMMERS MANUAL }   Values of the switches are stored in Fortran 90 module mad\_ptc\_intstate (mad\_ptc\_intstate.f90). The command is processed by pro\_ptc\_setswitch C function in file madxn.c It call an appropriate routines of the Fortran module to set each of the switches: 
\begin{itemize}
	\item  ptc\_setdebuglevel 
	\item  ptc\_setaccel\_method 
	\item  ptc\_setexactmis 
	\item  ptc\_setradiation 
	\item  ptc\_settotalpath 
	\item  ptc\_settime 
	\item  ptc\_setfringe  
\end{itemize}

%%\end{document}
