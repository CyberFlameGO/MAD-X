%%\title{Geometric Layout}
%  Created by: Andre VERDIER, 21-Jun-2002 
%  Changed by: Andre Verdier, 25-Jun-2002 
%  Changed by: Hans Grote, 30-Sep-2002 
%  Changed by: Andre Verdier, 14-Jan-2003 
%  Directory :  /afs/cern.ch/eng/sl/MAD-X/pro/docu/survey/ 
 
\chapter{GEOMETRIC LAYOUT}

The SURVEY command computes the coordinates of all machine elements in a \href{../Introduction/conventions.html}{global reference system}. These coordinates can be used for installation. In order to produce coordinates in a particular system, the initial coordinates and angles can be specified. The computation results are written on an internal table (survey) and can be written on an external file. Each line contains the coordinates at the end of the element.
\\ The last "USEd" sequence is used except if another one is specified.
\\
-----------------------------------------------------------------------------------------
\\ WARNING : in the case a machine geometry is constructed with thick lenses, the circumference will change if the structure is converted into thin lenses (via the \href{../makethin/makethin.html}{makethin} command). This is an unavoidable feature. ONLY the structure with thick lenses must be used for practical purposes.
\\ INFORMATION : The skew dipole component of a MULTIPOLE element (MULTIPOLE, KSL=\{FLOAT\}) is NOT taken into account in the survey calculation. You should use a tilted normal MULTIPOLE or BEND instead.
\\
\\
-----------------------------------------------------------------------------------------
\\ The survey calculation is launched by a single command line with the following syntax : 
\\
\\

\begin{verbatim}
SURVEY, x0=double, y0=double, z0=double, 
        theta0=double, phi0=double, psi0=double,
        file=string, table=string, sequence=string;

parameter   meaning                                     default value
x0 	    initial horizontal transverse coordinate    0.0
y0          initial vertical   transverse coordinate    0.0
z0          initial longitudinal coordinate             0.0
theta0      initial horizontal angle                    0.0
phi0        initial vertical angle                      0.0
psi0        initial transverse tilt                     0.0
file        name of external file                       null (default name survey)
table       name of internal table                      null (ignored, always table='survey')
sequence    name of sequence to be surveyed             last used sequence
\end{verbatim}

\section{ Example : average LHC ring with CERN coordinates.}

\begin{verbatim}
real const R0 = 1.0;          ! to obtain the average ring
OPTION, -echo, -info;
CALL, file="V6.4.seq.070602"; ! follow this \href{V6.4.seq.070602}{link} for the file
OPTION, echo;
BEAM, particle=proton, energy=450, sequence=lhcb1;
USE, period=lhcb1;
! SELECT, flag=survey,clear;  ! uncomment if the optional select below is used
! optional SELECT to specify a class and the output columns
! SELECT, flag=survey, class=marker, column=name,s,psi;
SURVEY, x0=-2202.21027, z0=2710.63882, y0=2359.00656,
        theta0=-4.315508007, phi0=0.0124279564, psi0=-0.0065309236,
        file=survey.lhcb1;
\href{../control/general.html#write}{WRITE}, table=survey; ! to display the results immediately
STOP;
!*********** The external file "survey.lhcb1" can now be read **************
\end{verbatim}
% There is a problem here. the "verbatim" command does not interpret "\href" as latex command. It is output as it is.

%\textit{F.Tecker, March 2006}