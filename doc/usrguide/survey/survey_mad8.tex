%%\title{SURVEY}
%  Changed by: Chris ISELIN,  4-Aug-1997 
%  Changed by: Hans Grote, 23-Sep-2002 


\section{SURVEY: Compute Geometric Layout}  
The SURVEY command computes the \href{global.html}{geometry} of the machine: 
\begin{verbatim}
SURVEY, X0=real,Y0=real,Z0=real,&
        THETA0=real,PHI0=real,PSI0=real,TAPE=file-name
\end{verbatim} 

It operates on the working beam line entered in the latest
\href{use.html}{USE} command. Its parameter list specifies the initial
position and orientation of the reference orbit in the
\href{global.html}{global coordinate system} (X,Y,Z). Omitted attributes
assume zero values. Valid attributes are:  
\begin{itemize}
   \item X0: The initial X coordinate [m]. 
   \item Y0: The initial Y coordinate [m]. 
   \item Z0: The initial Z coordinate [m]. 
   \item THETA0: The initial angle theta [rad]. 
   \item PHI0: The initial angle phi. 
   \item PSI0: The initial angle psi. 
   \item TAPE: If TAPE=file-name appears MAD writes a full
     \href{tape3.html#survey}{survey table} on a disk file
     \href{files.html}{file-name}. TAPE alone is equivalent to
     TAPE="survey":  
\end{itemize} 

SURVEY prints one line for either end of the computation range and a
summary. It also prints one line for each element and for the entrance
and exit of each beam line, if this position has been selected by
\href{print.html}{PRINT} or by \href{print.html}{SELECT,FLAG=TWISS}.  

Example: 
\begin{verbatim}
SURVEY, TAPE=LAYOUT
\end{verbatim} 
This example computes the machine layout with zero initial conditions
and writes the results on a file called LAYOUT.

% \href{http://wwwslap.cern.ch/fci/fci_sign.html}{fci}, January 27, 1997 
