%%\title{Touschek}
%  Changed by: Frank Zimmermann, 18-Jun-2002  

\chapter{TOUSCHEK: Touschek Lifetime and Scattering Rates}

The TOUSCHEK module computes the Touschek lifetime and the scattering rates 
around a lepton or hadron storage ring, based on the formalism of Piwinski [A. 
Piwinski, "The Touschek Effect in Strong Focusing Storage Rings," DESY-98-179; 
see also Piwinski's article on Touschek lifetime in the Handbook of Accelerator 
Physics and Engineering (A. Chao, M. Tigner, eds.), World Scientific, 1999] .

The syntax of the TOUSCHEK command is: 
\begin{verbatim}
TOUSCHEK, FILE=file_name;
\end{verbatim}

TOUSCHEK should be called after a TWISS command. One or several cavities 
with rf voltages should be defined prior to calling TWISS and TOUSCHEK. 

{\bf Warning:} Calling EMIT between the TWISS and TOUSCHEK commands
leads to TOUSCHEK using wrong beam parameters, even if the BEAM command
is reiterated.] 
 

The momentum acceptance is taken from the bucket size taking into
account the  energy loss per turn \textit{U0 }from synchrotron
radiation. The value of \textit{U0} is computed from the second
synchrotron radiation integral \textit{synch\_2} in the TWISS summ table
(\textit{synch\_2} is calculated only when the TWISS option 'chrom' is
invoked), using Eq. (3.61) in Matt Sands' report SLAC-121, which was
generalized to the case of several harmonic rf systems. If
\textit{synch\_2=0}, not defined, or not calculated, zero energy loss is
assumed. In the case of several rf systems with nonzero voltages, it is
assumed that the lowest frequency system defines the phase of the outer
point on the separatrix when calculating the momentum acceptance, and
that all higher-harmonic systems are either in phase or in anti-phase to
the lowest frequency system. (Note: if a storage rings really uses a
different rf scheme, one would need to change the acceptance function in
the routine \textit{cavtousch0} for that ring.) 

The arguments have the following meaning: 
\begin{itemize}
   \item FILE=file\_name: The name of the output file (default: 'touschek') 
\end{itemize}

Example: 
\begin{verbatim}
BEAM, PARTICLE = PROTON, ENERGY = 450, NPART = 1.15e11, 
      EX = 7.82E-9, EY = 7.82E-9, ET = 5.302e-5,
      SIGE = 7.164e-4, SIGT = 0.1124, RADIATE = TRUE;
...
USE, PERIOD = FODO;	
... 	
VRF = 400;
...
SELECT,FLAG = TWISS,CLEAR;
TWISS, CHROM, TABLE, FILE;
...
TOUSCHEK, FILE;
...
\end{verbatim}

The first command defines the beam parameters. It is essential that the
longitudinal emittances and bunch length are set. The command
\textit{use} selects the beam line or sequence. The next command assign
a value to the cavity rf voltage vrf  (example name). The
\textit{select} clear previous assignments to the \textit{twiss} module,
\textit{twiss}calculates and saves the values of all twiss parameters
for all elements in the ring; the \textit{touschek} command computes the
Touschek lifetime and writes it to the file 'touschek' (default name).   

The results are stored in the \textit{TOUSCHEK} tables, and can be written to a 
file (with the default name 'touschek' in the example above), or values can be 
extracted from the table using the value command as follows 

\begin{verbatim}
value, table(touschek,name), table(touschek,s), table(touschek,tli),
       table(touschek,tliw), table(touschek,tlitot); 
\end{verbatim}

where 'name' denotes the name of a beamline element, \textit{s} the
position of the center of the element,\textit{ tli} the instanteneous
Touschek loss rate within the element, and \textit{tliw} the
instantaneous rate weighted by the length of the element divided by the
circumference (its contribution to the total loss rate), and\textit{
tlitot }the accumulated loss rate adding the rates over all beamline
elements through the present position. The value of \textit{tlitot} at
the end of the beamline is the inverse of the Touschek lifetime in units
of 1/s. 

All results can also be printed to a file using the command 
\begin{verbatim}
write, table = touschek, file;
\end{verbatim}

%The MADX Touschek module was developed by
% \href{mailto:catia.milardi@lnf.infn.it}{Catia Milardi} and
% \href{mailto:frank.zimmermann@cern.ch}{Frank Zimmermann . }
%\\\href{http://consult.cern.ch/xwho/people/62690}{frankz}   11.03.2008 
