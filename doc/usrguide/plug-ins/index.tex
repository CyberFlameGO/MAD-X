%\documentclass[a4paper,11pt]{article}
%%\usepackage{ulem}
%%\usepackage{a4wide}
%%\usepackage[dvipsnames,svgnames]{xcolor}
%%\usepackage[pdftex]{graphicx}
%%\title{MAD-X PLUG-INS}
%%\usepackage{hyperref}
% commands generated by html2latex


%%\begin{document}
%%\begin{center}
   %%EUROPEAN ORGANIZATION FOR NUCLEAR RESEARCH   
%%\includegraphics{http://cern.ch/mad/madx/icons/mx7_25.gif}

\section{MAD-X PLUGINS}
%%\end{center}
%   ##########################################################              

%   ##########################################################              

%   ##########################################################              

%   ##########################################################              
  MAD-X provides a plug-in mechanism for functionality extensions. Plug-in technique does not require linking at build time. The job is done at the time plug-in is loaded by the dynamic linker.  In order to use any plug-in, the plugin support must be compiled in MADX. At the top of every MADX makefile there is variable PLUGIN\_SUPPORT that must be set to "YES".  Then, the appropriate C interface is compiled in and MADX is  linked dynamically.   Plug-ins must be accessible to the dynamic linker.  The default location where plug-ins are searched is \$HOME/.madx/plugins. Otherwise, the directory containing the library must be either listed in  
\begin{enumerate}
	\item  the configuration file of the dynamic linker (on SLC3 it is /etc/ld.so.conf) 
	\item  LD\_LIBRARY\_PATH environment variable 
\end{enumerate}

 Existing plug-ins  
\begin{enumerate}
	\item \href{rplot/index.html}{RPLOT}.
\end{enumerate}

\textbf{ Example }\\
%  ############################################################ 

%  ############################################################ 

%  ############################################################ 

\textbf{ PROGRAMMERS MANUAL  }\\


The interface is not yet fully defined. The documentation apears at the moment it happens.   

%%\end{document}
