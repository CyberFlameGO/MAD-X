%\documentclass[a4paper,11pt]{article}
%%\usepackage{ulem}
%%\usepackage{a4wide}
%%\usepackage[dvipsnames,svgnames]{xcolor}
%%\usepackage[pdftex]{graphicx}
%%\title{Threader}
%%\usepackage{hyperref}
% commands generated by html2latex


%%\begin{document}  %%EUROPEAN ORGANIZATION FOR NUCLEAR RESEARCH 
%%\includegraphics{../icons/mx7_25.gif align=right}

\subsection{Beam Threader}  The \textbf{threader} simulates beam steering through a machine with field and alignment errors in situations where the beam does not circulate and the closed orbit cannot be measured. 

 If enabled, threading is executed whenever a trajectory or closed orbit search is carried out by the \href{../twiss/twiss.html}{TWISS} module. 

 The following MAD-X commands control the action of the threader :  


\begin{verbatim}
option, threader ;
\end{verbatim}  enables the threader 

 when set, the threader checks at all monitors the difference with respect to the stored orbit there (from \href{../twiss/twiss.html}{keeporbit}) if \href{../twiss/twiss.html}{useorbit} is present. The threader then provides kicks (if possible) to reduce the orbit difference below the maxima specified on the threader command. This procedure allows to thread with orbit bumps present  

\begin{verbatim}
threader, vector = {xmax, ymax, att} ;
\end{verbatim}  sets the parameters for the threader 
\begin{verbatim}
xmax, ymax : orbit excursion (at a monitor) at which threader acts
att        : attenuation factor for the kicks applied by the threader
defaults   : {0.005, 0.005, 1.000}
\end{verbatim}


\line(1,0){300}

 Hans Grote, 31.10.2008 

%%\end{document}
