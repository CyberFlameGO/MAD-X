%%\title{PTC Set-up Parameters}
%  Created by: Valery KAPIN, 21-Mar-2006 
%  Changed by: ____________, ___________ 

\chapter{PTC Set-up Parameters}

The Polymorphic Tracking Code (\hyperlink{E.Forest}{PTC}) of Etienne
Forest is a kick code, allowing a symplectic integration through all
accelerator elements giving the user full control over the precision
(number of   steps and integration type) and exactness (full or extended
Hamiltonian) of the   results. 
The degree of exactness is determined by the user and the speed of his
computer.  
The main advantage is that the code is inherently based on the map
formalism and provides users with all associated tools. 

The PTC code is actually a library that can be used in many different
ways to create an actual module that calculates some property of
interest. 
Several modules using the PTC code have been presently implemented in
MAD-X. These MADX-PTC modules [\hyperlink{F.Schmidt}{b}] are executed by
the following commands: 
\href{../ptc_twiss/ptc_twiss.html}{  ptc\_twiss},
\href{../ptc_normal/ptc_normal.html}{  ptc\_normal},
\href{../ptc_track/ptc_track.html}{  ptc\_track},
\href{../ptc_track_line/ptc_track_line.html}{  ptc\_track\_line}. 
To perform calculations with these MADX-PTC commands, the PTC
environment must be initialized, handled and turned off by the special
commands within the MAD-X input script. 

\section{Command Synopsis}

A typical set of commands to invoke PTC is given below: 

%% \begin{description}
%%    \item[Synopsis] 
%%    \textbf{\\}
%%    \texttt{PTC\_CREATE\_UNIVERSE, sector\_nmul\_max=integer, \\
%%      sector\_nmul=integer, ntpsa=logical, symprint=logical;}\\ \\
%%    \texttt{PTC\_CREATE\_LAYOUT, time=logical, model=integer,\\
%%      method=integer, nst=integer, exact=logical, \\
%%      offset\_deltap=double, errors\_out=logical, \\
%%      magnet\_name=string, resplit=logical, thin=double, xbend=double, \\ 
%%      even=logical;}\\ \\
%% %   \texttt{.........................}\\
%%    \textit{\texttt{PTC\_READ\_ERRORS, overwrite=logical;}}\\ \\
%% %   \texttt{.........................}\\
%%    \textit{\texttt{PTC\_MOVE\_TO\_LAYOUT, index=integer;}}\\ \\
%% %   \texttt{.........................}\\
%%    \textit{\texttt{PTC\_ALIGN;}}\\ \\
%% %   \texttt{.........................}\\
%%    \textit{\texttt{ PTC\_END;}}
%% \end{description}


\begin{verbatim}
PTC_CREATE_UNIVERSE, sector_nmul_max = integer,
                     sector_nmul = integer, 
                     ntpsa = logical, symprint = logical;

PTC_CREATE_LAYOUT, time = logical, model = integer,
                   method = integer, nst = integer, exact = logical,
                   offset_deltap = double, errors_out = logical,
                   magnet_name = string, resplit = logical, 
                   thin = double, xbend = double, 
                   even = logical;
...

PTC_READ_ERRORS, overwrite = logical;
...

PTC_MOVE_TO_LAYOUT, index = integer;
...

PTC_ALIGN;
...

PTC_END;
\end{verbatim}

\section{PTC\_CREATE\_UNIVERSE}

\begin{verbatim}
PTC_CREATE_UNIVERSE, sector_nmul_max=integer, sector_nmul=integer,
                     ntpsa=logical, symprint=logical;
\end{verbatim}

% <strong>TRACK, DELTAP= double, ONEPASS, DUMP, ONETABLE, FILE= string;</strong> (MADX version 1)<br />  
% <strong>TRACK, DELTAP= double, ONEPASS, DAMP, QUANTUM, DUMP, ONETABLE, FILE= string;</strong> (MADX version 2)<br />  

The \texttt{"PTC\_CREATE\_UNIVERSE"} command is required to set-up the
PTC environment.  
     
   %% \item[Options]
   %%   %\resizebox{15cm}{!} {
   %%   \begin{tabular}{l p{6cm} p{1.5cm} p{1.5cm}}
   %%     \hline 
   %%     \textbf{Option} & \textbf{Meaning} & \textbf{Default Value} &
   %%     \textbf{Value Type} \\  
   %%     \hline
   %%     SECTOR\_NMUL\_MAX & Global variable in PTC needed for       exact
   %%     sector bends defining up to which order Maxwell's       equation
   %%     are solved [\hyperlink{E.Forest}{a}, page       76-77]. The value
   %%     of SECTOR\_NMUL\_MAX must not be       smaller than SECTOR\_NMUL
   %%     otherwise MAD-X stops with an       error. & 10 & integer \\  
   %%     \hline
   %%     SECTOR\_NMUL & Global variable in PTC needed for exact sector
   %%     bends defining up to which order the multipole are included in
   %%     solving Maxwell's equation up to order
   %%     SECTOR\_NMUL\_MAX. Multipoles of order N with N $>$ SECTOR\_NMUL
   %%     and N $\leq$ SECTOR\_NMUL\_MAX are treated a la SixTrack. & 10 &
   %%     integer \\  
   %%     \hline
   %%     NTPSA & This attribute invokes the second DA package written in C++ and
   %%     kindly provided by Lingyun Yang  lyyang@lbl.gov. Etienne Forest has
   %%     written the wrapper to allow  both the use of the legendary DA packages
   %%     written in Fortran by  Martin Berz (default) or this new DA package. It
   %%     is expected that this DA package will allow for the efficient
   %%     calculation of a large number of DA parameters. & .FALSE. & logical
   %%     \\   
   %%     \hline
   %%     SYMPRINT & This flag allows the supression of the printing of the
   %%     check of symplecticity. It is recommended to leave this flag set
   %%     to TRUE. & .TRUE. & logical \\  
   %%     \hline
   %%   \end{tabular}
   %%   %}

{\bf Options} \\
\begin{itemize}

   \item {\bf SECTOR\_NMUL\_MAX}=integer (Default: 10) \\
     Global variable in PTC needed for exact sector bends defining up to
     which order Maxwell's equation are solved [\hyperlink{E.Forest}{a}, page
       76-77]. The value of SECTOR\_NMUL\_MAX must not be smaller than
     SECTOR\_NMUL otherwise MAD-X stops with an error. 

   \item {\bf SECTOR\_NMUL}=integer (Default: 10) \\
     Global variable in PTC needed for exact sector bends defining up to
     which order the multipole are included in solving Maxwell's
     equation up to order SECTOR\_NMUL\_MAX. Multipoles of order N with
     N $>$ SECTOR\_NMUL and N $\leq$ SECTOR\_NMUL\_MAX are treated a la
     SixTrack. 

   \item {\bf NTPSA}=logical (Default: .false.) \\
     This attribute invokes the second DA package written in C++ and
     kindly provided by Lingyun Yang (lyyang@lbl.gov). \\
     Etienne Forest has written the wrapper to allow  both the use of
     the legendary DA packages written in Fortran by  Martin Berz
     (default) or this new DA package. It is expected that this DA
     package will allow for the efficient calculation of a large number
     of DA parameters. 

   \item {\bf SYMPRINT}=logical (Default: .true.) \\
     This flag allows the supression of the printing of the check of
     symplecticity. It is recommended to leave this flag set to TRUE.  

\end{itemize}


\section{PTC\_CREATE\_LAYOUT}

\begin{verbatim}
PTC_CREATE_LAYOUT, time=logical, model=integer, method=integer, 
                   nst=integer, exact=logical, offset_deltap=double, 
                   errors_out=logical, magnet_name=string,
                   resplit=logical, thin=double, xbend=double, 
                   even=logical; 	
\end{verbatim}

% <strong>TRACK, DELTAP= double, ONEPASS, DUMP, ONETABLE, FILE= string;</strong> (MADX version 1)<br />  
% <strong>TRACK, DELTAP= double, ONEPASS, DAMP, QUANTUM, DUMP, ONETABLE, FILE= string;</strong> (MADX version 2)<br />  
The \texttt{"PTC\_CREATE\_LAYOUT"} command creates the PTC-layout
according to  the specified integration method and fills it with the
current  MAD-X sequence defined in the latest
\href{../control/general.html#use}{ USE} command.  
\\ 
The logical input variable time  controls the coordinate system that
is being used. 

%% \item[Options] 
%%   %http://stackoverflow.com/questions/2896833/how-to-stretch-a-table-over-multiple-pages 
%%   %http://nepsweb.co.uk/docs/tableTricks.pdf
%%   \begin{longtable}{l p{8cm} p{2cm} p{2cm}}
%%     %\begin{tabular}{l p{8cm} ll}
%%     \hline 
%%     \textbf{Option} & \textbf{Meaning} &  \textbf{Default Value} &
%%     \textbf{Value Type} \\  
%%     \hline
%%     TIME
%%     & 
%%     \begin{tabular}{p{0.2cm} p{7.0cm}}
%%       \multirow{2}{*}{5D} & 
%%       "time=true": fifth coordinate is
%%       \href{../Introduction/tables.html#canon}{PT},
%%       \textit{p$_t$}=$\Delta$\textit{E}/\textit{p}$_0$\textit{c;} \\ 
%%       & "time=false": fifth coordinate is
%%       \href{../Introduction/tables.html#canon}{DELTAP},
%%       \textit{$\delta$}$_\textit{p}$=$\Delta$\textit{p}/\textit{p}$_0$
%%       \\ 
%%       \hline
%%       \multirow{2}{*}{6D} & 
%%       "time=true": \href{../Introduction/tables.html#canon}{   MAD-X
%%         coordinate system} \{-\textit{ct}, \textit{p$_t$}\} \\ 
%%       & "time=false": second PTC coordinate system \{-pathlength,
%%       \textit{$\delta$}$_\textit{p}$\} \\       
%%     \end{tabular}
%%     & 
%%     .TRUE. & 
%%     logical\\
    
%%     \hline
%%     MODEL &  Type of element: 1, 2,or 3. & 1 & integer \\ 
%%     \hline
%%     METHOD & Integration order (2, 4, 6) [\hyperlink{E.Forest}{a},
%%       Chapter K] & 2 & integer \\  
%%     \hline
%%     NST & Number of integration steps: 1, 2, 3,  . & 1 & integer \\ 
%%     \hline
%%     EXACT & Switch to turn on calculations with an exact Hamiltonian,
%%     otherwise the expanded Hamiltonian is used. & .FALSE. & logical \\  
%%     \hline
%%     OFFSET\_DELTAP & Beware: Expert flag! The relative momentum
%%     deviation of the reference particle (6D case ONLY). This option
%%     implies "totalpath=true". & 0.0 & double \\  
%%     \hline
%%     ERRORS\_OUT & Flag to write-out multipolar errors in Efcomp table
%%     format. Two     tables are filled "errors\_field"     and
%%     "errors\_total". In the first     case only field errors are written
%%     out and in the second one also     desired field components are
%%     added. The latter is useful e.g. to     include the strength of
%%     correctors. The choice of magnets is     defined by the
%%     "magnet\_name"     attribute (see below). As usual the tables can be
%%     written to files for later use for read-in via the "ERRORS\_IN"
%%     flag:\newline
%%     \newline
%%     write, table=errors\_field,file=Your\_Errors\_Field\_File;\newline
%%     write, table=errors\_total,file=Your\_Errors\_Total\_File;\newline
%%     \newline
%%     The "ERRORS\_IN" flag has precedence over     this "ERRORS\_OUT" flag.   & 
%%     .FALSE. & logical \\
%%     \hline
%%     MAGNET\_NAME & 
%%     Simple selection for the names of magnet to be used for an error
%%     write-out   using the "ERRORS\_OUT" flag (see above).   In fact, the
%%     errors are recorded for all magnets with their name starting with
%%     the exact string of "MAGNET\_NAME".   &  
%%     NULL & string \\
%%     \hline
%%     RESPLIT & 
%%     Flag to apply the PTC resplit procedure. This is meant to
%%     create an "adaptive" setting of the "METHOD" and  "NST"  attributes
%%     according to the        strength of the quadrupoles (using the
%%     "THIN"  attribute) and separatedly the dipoles (using the "XBEND"
%%     attribute). Additionally, there        is the  "EVEN" attribute for
%%     even and odd number        of splits. &      
%%     .FALSE. & logical \\
%%     \hline
%%     THIN & 
%%     This is the main "RESPLIT" attribute. It is meant        for
%%     splitting quadrupoles according to their strength. The default
%%     of "THIN=0.001" has shown in practice to        work well without
%%     costing too much with respect of performance.  &  
%%     0.001 & double \\
%%     \hline
%%     XBEND & 
%%     This attribute is meant for splitting dipoles,
%%     e.g. "XBEND=0.001". It is an optional "RESPLIT" attribute and
%%     therefore has the        default set to -1, which means no
%%     splitting. A splitting by "XBEND=0.001" maybe advisable for dipoles
%%     as well.  &  
%%     -1 (off) & double \\
%%     \hline
%%     EVEN & 
%%     Switch to ensure even number of splits when using the PTC "RESPLIT"
%%     procedure. The default is "EVEN=TRUE". This is particularly useful
%%     when one attempts to calculate "PTC\_TWISS" with then
%%     "CENTER\_MAGNETS" option, i.e. if one        would like to calculate
%%     the TWISS parameters in the center of        the element. Uneven
%%     number of splits will be achieved with "EVEN=FALSE". &  
%%     .TRUE. & logical \\
%%     \hline
%%     %\end{tabular}
%%   \end{longtable}


{\bf Options} \\ 
\begin{itemize}

   \item {\bf TIME}=logical (Default= .true.) \\
     This option changes the canonical coordinate system depending
     whether the calculation is done in 5D or 6D: 
     \begin{itemize}
       \item {\bf 5D} \\ 
       if TIME=.true. , the fifth coordinate is
       \href{../Introduction/tables.html#canon}{PT},
       \textit{p$_t$}=$\Delta$\textit{E}/\textit{p}$_0$\textit{c;} \\ 
       if TIME=.false. , the fifth coordinate is
       \href{../Introduction/tables.html#canon}{DELTAP},
       \textit{$\delta$}$_\textit{p}$=$\Delta$\textit{p}/\textit{p}$_0$

       \item {\bf 6D} \\ 
       if TIME=.true. , the \href{../Introduction/tables.html#canon}{   MAD-X
         coordinate system} \{-\textit{ct}, \textit{p$_t$}\} is used. \\ 
       if TIME=.false. , the second PTC coordinate system \{-pathlength,
       \textit{$\delta$}$_\textit{p}$\} is used \\       
     \end{itemize}
     
   
   \item {\bf MODEL}=integer (Default: 1) \\
     1 for "Drift-Kick-Drift"; \\ 
     2 for "Matrix-Kick-Matrix" and \\ 
     3 for "Delta-Matrix-Kick-Matrix" (SixTrack-code model).

   \item {\bf METHOD}=integer (Default: 2) \\
     Integration order: 2, 4, or 6 [\hyperlink{E.Forest}{a}, Chapter K] 

   \item {\bf NST}=integer (Default: 1) \\
     Number of integration steps: 1, 2 or 3. \\
     This option sets the same value  for all "thick" elements
     (\textit{l} $>$ 0) of a beam-line, however please note, that each
     individual element may have its own NST value
     (\hyperlink{individual}{see below}). 

   \item {\bf EXACT}=logical (Default: .false.) \\ 
     Switch to turn on calculations with an exact Hamiltonian,
     otherwise the expanded Hamiltonian is used. 

   \item {\bf OFFSET\_DELTAP}=double (Default: 0.0) \\ 
     {\bf [ Beware: Expert flag! ]}\\
     The relative momentum deviation of the reference particle (6D case
     ONLY). This option implies "totalpath=true". 

   \item {\bf ERRORS\_OUT}=logical (Default: .false.) \\
     Flag to write-out multipolar errors in Efcomp table format. Two
     tables are filled "errors\_field" and "errors\_total". In the first
     case only field errors are written out and in the second one also
     desired field components are added. The latter is useful e.g. to
     include the strength of correctors. The choice of magnets is
     defined by the "magnet\_name" attribute (see below). As usual the
     tables can be written to files for later use for read-in via the
     "ERRORS\_IN" flag: \\ 
     write, table=errors\_field,file=Your\_Errors\_Field\_File;\\
     write, table=errors\_total,file=Your\_Errors\_Total\_File;\\
     \\
     The "ERRORS\_IN" flag has precedence over this "ERRORS\_OUT" flag.   

   \item {\bf MAGNET\_NAME}=string (Default: NULL) \\ 
     Simple selection for the names of magnet to be used for an error
     write-out using the "ERRORS\_OUT" flag (see above). In fact, the
     errors are recorded for all magnets with their name starting with
     the exact string of "MAGNET\_NAME".

   \item {\bf RESPLIT}=logical (Default: .false.) \\ 
     Flag to apply the PTC resplit procedure. This is meant to create an
     "adaptive" setting of the "METHOD" and "NST" attributes according
     to the strength of the quadrupoles (using the "THIN"  attribute)
     and separatedly the dipoles (using the "XBEND"
     attribute). Additionally, there is the  "EVEN" attribute for even
     and odd number of splits. 

   \item {\bf THIN}=double (Default: 0.001) \\ 
     This is the main "RESPLIT" attribute. It is meant for splitting
     quadrupoles according to their strength. The default of
     "THIN=0.001" has shown in practice to work well without costing too
     much with respect of performance.   

   \item {\bf XBEND}=double (Default: -1.0) \\ 
     This attribute is meant for splitting dipoles,
     e.g. "XBEND=0.001". It is an optional "RESPLIT" attribute and
     therefore has the default set to -1, which means no splitting. A
     splitting by "XBEND=0.001" maybe advisable for dipoles as well. 

   \item {\bf EVEN}=logical (Default: .true.) \\
     Switch to ensure even number of splits when using the PTC "RESPLIT"
     procedure. The default is "EVEN=TRUE". This is particularly useful
     when one attempts to calculate "PTC\_TWISS" with then
     "CENTER\_MAGNETS" option, i.e. if one would like to calculate the
     TWISS parameters in the center of the element. Uneven number of
     splits will be achieved with "EVEN=FALSE".  

\end{itemize}
  

{\bf Remarks} \\ 
\begin{itemize}  
  \item {\bf TIME:} at small energy ($\beta$$_0$  $<$$<$1),
  momentum-dependent variables like dispersion will depend  strongly on
  the choice of  the logical input variable "time".  In fact, the
  derivative ($\partial$/$\partial$\textit{$\delta$}$_\textit{p}$)  and
  ($\partial$/$\partial$\textit{p}$_\textit{t}$)  are different by the
  factor \textit{$\beta$}$_0$. One would  therefore typically  choose
  the option "time=false",  which sets the fifth variable to
  the relative momentum deviation \textit{$ \delta_p$ }.\\   

\end{itemize}

\section{PTC\_READ\_ERRORS}

\begin{verbatim}
PTC_READ_ERRORS, overwrite=logical; 
\end{verbatim}

The \texttt{"PTC\_READ\_ERRORS" }command lets you read any numbers of
\textbf{"errors\_read"} table
\href{../control/general.html#readmytable}{READMYTABLE}  

{\bf Options} \\ 
\begin{madlist}
   \ttitem{OVERWRITE}=logical (Default: .false.) \\ 
     Flag to either OVERWRITE the read-in errors (on request by
     using this flag) or by DEFAULT just add them to multipole
     components already present.
\end{madlist}

{\bf Remarks}\\
Because of the way the table is read in memory, a warning will always be
issued by default in the form:
\begin{verbatim}
warning: string_from_table_row: row out of range: errors_read->name[1>=n+1<=n]
\end{verbatim}
where n is  the number of records read from the table. 
This warning has no consequence on the errors read and the following
calculation. \\
The warning is purely the result of the way that the reading loop is
programmed with a break based on the return value of
string\_from\_table\_row.  
But if string\_from\_table\_row tries to read in a row (n+1) past the
last row (n) of the table, it prints a warning before returning a value
that will effectively break the loop. Of course this will only happen if
the WARN option is true and this can be turned off with  
\begin{verbatim}
Option, -warn;
\end{verbatim}

       
\section{PTC\_MOVE\_TO\_LAYOUT}

\begin{verbatim}
PTC_MOVE_TO_LAYOUT, index=integer;
\end{verbatim}


Several PTC layouts can be created within a one PTC-"universe". The
layouts are automatically numbered with sequential integers  by the
MAD-X code. The \texttt{"PTC\_MOVE\_TO\_LAYOUT" }is used for an
activation of a requested layout and the next PTC commands will be
applied to this active PTC layout until a new PTC layout is created
or activated. 

{\bf Option} \\
\begin{itemize}
   \item {\bf INDEX}=integer (Default: 1) \\
     Number of the PTC layout to be activated.
\end{itemize}

\section{PTC\_ALIGN}

\begin{verbatim}
PTC_ALIGN;
\end{verbatim}

The \texttt{"PTC\_ALIGN" }command is used to apply the MAD-X alignment
errors to the current PTC layout.  

\section{PTC\_END}

\begin{verbatim}
PTC_END;
\end{verbatim}

The \texttt{"PTC\_END" }command  is turning off the PTC environment,
which releases all memory back to the MAD-X world proper.


\section{Additional Options for Physical Elements}
\label{sec:add_option_PTC}

\begin{verbatim}
[SBEND | RBEND | QUADRUPOLE | SEXTUPOLE | OCTUPOLE | SOLENOID ],
         l=double , ... , tilt=double, ... , nst=integer, ... ,
         knl:={0, double, double,...}, ksl:={0, double, double,..}; 
\end{verbatim}


{\bf Description} \\
\begin{itemize}
   \item The full range of  normal and skew multipole components on the
     bench can be  specified for the following physical elements:
     \texttt{sbend, rbend, quadrupole, sextupole,  octupole and
       solenoid. }Multipole coefficients are specified as the integrated
     value "$\int$\textit{Kds"}  of the field components along the
     magnet axis (see \hyperlink{Multipoles_on_Bench_(PTC_only)}{the
       table} below). These multipole components in PTC are spread over
     a whole element, if \textit{l }$>$ 0. This is a considerable
     advantage of PTC input compare  to MAD-X  which allows only
     \href{../Introduction/multipole.html}{ thin multipoles}. 

   \item  To preserve the reference orbit of straight elements, dipole
     components  for those elements are ignored, knl(0)=0, ksl(0)=0.
      
   \item  \href{individual}{Individual}\hyperlink{NST}{NST} values for
     a  particular "thick" element (\textit{l }$>$ 0) can be
     specified. For example,  in MAD-X any RF cavity is represented
     by a single kick, while  PTC splits the RF cavity into (global)
     NST segments. In this  way, PTC considers properly transit-time
     effects of the cavity.  In case, one wants to reproduce the
     approximate results of  MAD-X, one can use NST=1 for RF cavity
     in PTC. 
\end{itemize}

{\bf Multipoles on Bench  (PTC only)} \\ 
\begin{itemize}
  \item {\bf KNL}=double array (Default: 0   [m$^{-1}$]) \\
    The normal multipole coefficient 
  \item {\bf KSL}=double array (Default: 0   [m$^{-1}$]) \\
    The skew multipole coefficient 
\end{itemize}
        
{\bf Remarks} \\

\begin{itemize}
   \item {\bf Length l:} if their length \textit{l} is equal to zero,
     bending magnets (\texttt{sbend, rbend}) are treated as "markers". 
     
   \item {\bf Additional Field Errors:} A full range of multipole
     \href{../error/error_field.html}{ field errors} can be
     additionally specified with
     \href{../error/error_field.html#efcomp}{ EFCOMP} command. Errors
     are added to the above multipole fields on the bench. 
\end{itemize}

\section{Caution}

A user has to understand that PTC exists inside of MAD-X as a
library. MAD-X offers the interface to PTC, i.e. the MAD-X input file is
used as input for PTC. Internally, both PTC and MAD-X have their own
independent databases which are linked via the interface. With the
\texttt{\hyperlink{PTC_CREATE_LAYOUT}{"PTC\_CREATE\_LAYOUT"}} command,
only numerical numbers are transferred from the MAD-X database to the
PTC database. Any modification to the MAD-X database is ignored in PTC
until the next call to
\texttt{\hyperlink{PTC_CREATE_LAYOUT}{"PTC\_CREATE\_LAYOUT"}}.For
example, a \href{../Introduction/expression.html#defer}{deferred
  expression} of MAD-X after a
\texttt{\hyperlink{PTC_CREATE_LAYOUT}{"PTC\_CREATE\_LAYOUT"}} command is
ignored within PTC.  


\section{Examples}

 Examples for any MADX-PTC module contain the above PTC set-up commands. 


\section{References}

\begin{enumerate}
   \item \href{E.Forest}{E.Forest}, F.Schmidt and E.McIntosh,
     Introduction to the Polymorphic Tracking Code ,
     \href{http://cern.ch/madx/doc/sl-2002-044.pdf}{CERN-SL-2002-044},
     \href{http://ccdb4fs.kek.jp/cgi-bin/img/allpdf?200302020}{KEK
       report 2002-3}, July 2002. 
   \item \href{F.Schmidt}{F. Schmidt},
     "`\href{http://cern.ch/madx/doc/MPPE012.pdf}{MAD-X PTC
     Integration}'', Proc. of the 2005 PAC Conference in
     Knoxville, USA, pp.1272. 
\end{enumerate}


{\bf See Also:}
\begin{itemize} 
   \item \href{../ptc_twiss/ptc_twiss.html}{ptc\_twiss},
   \item \href{../ptc_normal/ptc_normal.html}{ptc\_normal},
   \item \href{../ptc_track/ptc_track.html}{ ptc\_track},
   \item \href{../ptc_track_line/ptc_track_line.html}{ptc\_track\_line}. 
\end{itemize}

%\href{mailto:kapin@itep.ru}{  V.Kapin}(ITEP) and 
%\href{mailto:Frank.Schmidt@cern.ch}{  F.Schmidt}, March  2006

