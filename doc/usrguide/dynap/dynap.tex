%%\title{DYNAP}
%  Changed by: Hans Grote, 17-Jun-2002 
%  Changed by: Frank Zimmermann, 18-Jun-2002 

\chapter{DYNAP: Tunes, Tune Footprints, Smear and Lyapunov Exponent}

DYNAP can be called instead of RUN inside a TRACK command: 
\begin{verbatim}
DYNAP, TURNS=real, FASTUNE=logical, LYAPUNOV=real,
       MAXAPER:={..,..,..,..,..,..}, ORBIT=logical;
\end{verbatim}
 
For each previously entered start command, DYNAP tracks two close-by
particles over a selected number of turns, from which it obtains the
betatron tunes with error, the action smear, and an estimate of  the
lyapunov exponent. Many such companion particle-pairs can be tracked at
the same time, which speeds up the calculation. The \textit{ smear } is
defined as  \textit{2.0 (wxy$_{max}$ - wxy$_{min}$) / (wxy$_{max}$ + wxy
  $_{min}$)}, where the \textit{wxy$_{min,max}$} refer to the  miminum and
maximum values of the sum of the transverse betatron invariants
\textit{wx+wy} during the tracking. The tunes are computed by  using an
FFT and either formula (18) or formula (25) of CERN SL/95-84 (AP),
depending on whether the number of turns is less-equal or larger than
64. 
 
The arguments have the following meaning:
 
\begin{itemize}
   \item \href{particle}{TURNS}:
     The number of turns to be tracked (default: 64, present maximum: 1024).
     
   \item \href{particle}{FASTUNE}:
     A logical flag. If set, the tunes are computed (default: false).
 
   \item \href{particle}{MAXAPER}:
     If the particle phase-space coordinates exceed certain 
     \textit{ maximum }
     values, the particle is considered lost. The maximum aperture
     is a vector of 6 real numbers 
     (default: (0.1, 0.01, 0.1,0.01, 1.0, 0.1) ).
     
   \item \href{particle}{LYAPUNOV}:
     The launch distance 
     between two companion particles 
     added to the \textit{x} coordinate (default: 1.e-7 m).
     
   \item \href{particle}{ORBIT}:
     A logical flag. If set, the flag \textit{orbit} 
     is true during the tracking and its initialization
     (default: true).
     \textbf{ This flag should be set to be true, if 
       normalized coordinates are to be entered.}
\end{itemize}

Example:
\begin{verbatim}
BEAM,PARTICLE=ELECTRON,ENERGY=50,EX=1.E-6,EY=1.E-8,ET=0.002,SIGT=1.E-2;
...
USE,PERIOD=FODO;
...
TRACK;
START,X=0.0010,Y=0.0017,PT=0.0003;
DYNAP,FASTUNE,TURNS=1024,LYAPUNOV=1.e-7;
ENDTRACK;
...
\end{verbatim}

The first command defines the beam parameters. It is  essential that the
longitudinal emittance \textit{ET} is set. The command \textit{use}
selects the beam line or sequence. The \textit{track} activates the
tracking module, \textit{start } enters the starting coordinates (more
than one particle can be defined),  \textit{dynap} finally tracks two
nearby particles  with an initial distance \textit{lyapunov}  for each
\textit{start} definition over \textit{ turns } revolutions, and
\textit{endtrack} terminates the execution of the tracking module. 

The results are stored in the \textit{DYNAP} and \textit{DUNAPTUNE}
tables, and can be obtained by the commands  
 
\begin{verbatim}
value, table(dynap,smear);
\end{verbatim} 
resp. 
\begin{verbatim}
value, (dynaptune,tunx), (dynaptune,tuny), (dynaptune,dtune);
\end{verbatim}

More generally, all results can be printed to a file, using the commands 
 
\begin{verbatim}
write, table=dynap, file;
write, table=dynaptune, file;
\end{verbatim}

The output file 'lyapunov.data' lists  the turn number and phase
distance between the  two Lyapunov partners, respectively, allowing for
visual inspection of chaoticity. 
 
%\href{http://consult.cern.ch/xwho/people/62690}{frankz} 20.03.2006
