%%%\title{(De)activate Correctors and Monitors}
%  Changed by: Hans Grote, 13-Sep-2000 
%  Changed by: Werner Herr, 19-Jun-2002 
%  Changed by: Hans Grote, 25-Sep-2002 

\section{Activate/Deactivate Correctors or Monitors}
\label{sec:activate}
To provide more flexibility with orbit correction two commands are
provided:  

\begin{verbatim}
USEMONITOR, STATUS=flag, SEQUENCE=sequence, RANGE=range, 
            CLASS=class, PATTERN=regex;

USEKICK,    STATUS=flag, SEQUENCE=sequence, RANGE=range,
            CLASS=class, PATTERN=regex;
\end{verbatim}

The command \href{monitor}{USEMONITOR} activates or deactivates a
selection of \href{../Introduction/monitors.html}{beam position
 monitor}s. This command affects elements of types MONITOR, HMONITOR,
or VMONITOR.    

The command  \href{kick}{USEKICK} activates or deactivates a selection
of \href{../Introduction/kickers.html}{orbit correctors}. This command
affects elements of types KICKER, HKICKER, or VKICKER.   


%% The purpose of the two commands is: 
%% \begin{itemize}
%% 	\item \href{monitor}{USEMONITOR}: Activates or deactivates a
%%           selection of \href{../Introduction/monitors.html}{beam
%%             position monitor}s. This command affects elements of types
%%           MONITOR, HMONITOR, or VMONITOR.  
%% 	\item \href{kick}{USEKICK}: Activates or deactivates a selection
%%           of \href{../Introduction/kickers.html}{orbit correctors}. This
%%           command affects elements of types KICKER, HKICKER, or VKICKER.  
%% \end{itemize} 

Both commands have the same attributes: 
\begin{itemize}
   \item STATUS: If this flag is true (on), the selected elements
     are activated. Active orbit monitor readings will be
     considered, and active correctors can change their strengths
     in subsequent correction commands. Inactive elements will be
     ignored subsequently.  
   \item SEQUENCE: The sequence can be specified, otherwise the
     currect sequence is used for this operation.  
   \item RANGE, CLASS, PATTERN: The usual selection commands are
     used to identify the elements for this operation.  
\end{itemize} 

Example:
\begin{verbatim}
USE,...                                ! set working beam line 
...                                    ! define imperfections 
USEKICK, STATUS = OFF, RANGE = ...;    ! deactivate selected correctors 
USEMONITOR, STATUS = OFF, RANGE = ...; ! deactivate selected monitors   
CORRECT, NCORR = 32;                   ! uses different set of correctors
USEKICK, STATUS = OFF, RANGE = ...;    ! deactivate different set of correctors 
CORRECT, NCORR = 32;                   ! uses different set of correctors
\end{verbatim}

%\href{http://consult.cern.ch/xwho/people/1808}{Werner Herr} 18.6.2002 
