%%\title{EMIT}
%  Changed by: Chris ISELIN, 27-Jan-1997 
%  Changed by: Hans Grote, 15-Oct-2002 
%  Changed by: Ralph Assmann, 02-Sep-2003 

\chapter{Fully Coupled Motion and Radiation}

\subsection{\href{emit}{EMIT: Equilibrium Emittances}} 

The command 
\begin{verbatim}
EMIT,DELTAP=real,TOL=tolerance;
\end{verbatim}
adjusts the RF frequencies such as to obtain the specified average energy error. More precisely, the revolution frequency \textit{f$_0$} is determined for a fictitious particle with constant momentum error 

DELTAP = delta$_\textit{s}$ = delta(\textit{E}) / \textit{p$_s$ c}

which travels along the design orbit. The RF frequencies are then set to 

\textit{f$_{RF}$ = h f$_0$}. 

If the machine contains at least one RF cavity, and if synchrotron radiation is \href{../Introduction/beam.html#radiate}{on}, the EMIT command computes the equilibrium emittances and other electron beam parameters using the method of \href{../Introduction/bibliography.html#chao}{[Chao]}. In this calculation the effects of quadrupoles, sextupoles, and octupoles along the closed orbit is also considered. Thin multipoles are used only if they have a fictitious length LRAD  different from zero. 

If the machine contains no RF cavity, if synchrotron radiation is  \href{../Introduction/beam.html#radiate}{off}, or if the longitudinal motion is not stable, it only computes the parameters which are not related to radiation. 

The tolerance is for the distinction static/dynamic: if for the eigenvalues of the one-turn matrix, $|$e\_val\_5$|$ \textless tol and $|$e-val\_6$|$ \textless tol, then the longitudinal motion is not considered, otherwise it is. The default for TOL is 1.000001. 
%  |e\_val\_5| \textless tol and |e-val\_6| \textless tol, then the longitudinal motion is not considered, otherwise it is. The default for TOL is 1.000001. 
 
In the current implementation, the BEAM values of the emittances and beam sizes are only updated for deltap = zero. 
Example: 
\begin{verbatim}
RFC: RFCAVITY,HARMON...,VOLT=...;
     BEAM,ENERGY=100.0,RADIATE;
     EMIT,DELTAP=0.01;
\end{verbatim}
\textbf{Remark:}  This module assumes nearly constant lattice functions inside elements. This assumption works for many machines,  like LEP (\href{http://cern.ch/frs/mad-X_examples/emit/LEP/}{see example}), but it fails when the lattice funcionts largely vary inside single elements. In the later case it is advised to slice the elements as shown in     \href{http://cern.ch/frs/mad-X_examples/emit/ALBA/}{ALBA}.   

%\line(1,0){300}

% \begin{tabular}{lr}
% \href{Rogelio HREF=http://consult.cern.ch/xwho/people/69118}{R. Tom\'as} & \textbf{Last updated:} 03/13/2013 13:47:20
% \end{tabular}