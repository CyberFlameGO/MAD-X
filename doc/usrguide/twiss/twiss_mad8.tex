%%\title{PRINT / SELECT}
%  Changed by: Chris ISELIN, 27-Jan-1997 
%  Changed by: Frank Schmidt, 28-Jun-2002 
%  Changed by: Hans Grote, 23-Sep-2002 

\section{Output Selection Statements}

\line(1,0){300}
%\href{beam.html}{
%%\includegraphics{/icons/left.gif} BEAM: Set Beam Parameters}\href{action.html}{
%%\includegraphics{/icons/up.gif} Action Commands}\href{split.html}{
%%\includegraphics{/icons/right.gif} SPLIT: Request Interpolation for OPTICS}

\line(1,0){300}

Each position in the beam line carries several associated selection
flags. They are initially cleared by the USE command when the beam line
is expanded. Output is \href{ranges.html}{selected} by setting some of
these flags  by one of the commands  
\begin{verbatim}
PRINT,RANGE=range{,range},TYPE=type{,type},FULL,CLEAR
SELECT,FLAG=name,RANGE=range{,range},TYPE=type{,type},FULL,CLEAR
\end{verbatim} 
(See \href{ranges.html}{obsolete element selection}). As from Version
8.18 the \href{new_select.html}{new element selection} is used:  
\begin{verbatim}
PRINT,RANGE=range,CLASS=class,PATTERN=pattern[,FULL][,CLEAR]
SELECT,FLAG=name,RANGE=range,CLASS=class,PATTERN=pattern[,FULL][,CLEAR]
\end{verbatim} 
The PRINT command always affects the print flag for
\href{survey.html}{SURVEY}, or \href{survey.html}{TWISS}. In SELECT the
flag type is chosen by the attribute FLAG: Three of its possible values
affect action commands:  

TWISS: A SELECT,TWISS statement is equivalent to PRINT: The two commands 
\begin{verbatim}
PRINT,  FULL
SELECT, FLAG=TWISS,FULL
\end{verbatim} 
have identical effect. 

OPTICS: Selects output positions for \href{optics.html}{OPTICS}. 

TRACK: Selects print positions for \href{track.html}{TRACK}. Care must
be taken in using this option, as it may generate a lot of output.  

Four more values are intended for debugging only: 
\begin{itemize}
   \item FIRST: Selects dumping of first-order transfer matrices for
     selected elements during closed orbit search in TWISS:  
   \item SECOND: Selects dumping of second-order TRANSPORT maps for
     selected elements during their accumulation in TWISS:  
   \item REFER: Selects dumping of first-order transfer matrices for
     selected elements during accumulation for adjusting RF cavities.  
   \item LIE: Selects dumping of Lie-algebraic maps during their accumulation. 
   \item ERROR: Selects error definition and/or printing for the
     commands \href{error_align.html}{EALIGN},
     \href{error_field.html}{EFIELD},  \href{error_field.html}{EFCOMP},
     and  \href{error_print.html}{EPRINT}.  
\end{itemize} 

PRINT and/or SELECT command(s) must be placed after the USE command, and
before any action command (e.g. TWISS) to be affected. Regardless of the
setting of print flags, start and end points of the computation range
are always printed by NORMAL, SURVEY, and TWISS:  

 Examples: 
\begin{verbatim}
USE,OCT                ! print at beginning and end only
PRINT,#35/37           ! print at positions number 35 to 37
SELECT,TWISS,FULL      ! set all print flags
PRINT,CLEAR            ! clear all print flags
PRINT,OCT              ! set all print flags
PRINT,CELL[3],CLEAR    ! clear all flags,
                       ! then set flags for all of third CELL
\end{verbatim}

\subsection{\href{example}{Examples for Element Selections}} 

First example: 
\begin{verbatim}
! Define element classes for a simple cell:
B:     SBEND,L=35.09, ANGLE = 0.011306116
QF:    QUADRUPOLE,L=1.6,K1=-0.02268553
QD:    QUADRUPOLE,L=1.6,K1=0.022683642
SF:    SEXTUPOLE,L=0.4,K2=-0.13129
SD:    SEXTUPOLE,L=0.76,K2=0.26328
! Define the cell as a sequence:
CELL:  SEQUENCE
   B1:    B,      AT=19.115
   SF1:   SF,     AT=37.42
   QF1:   QF,     AT=38.70
   B2:    B,      AT=58.255,ANGLE=B1[ANGLE]
   SD1:   SD,     AT=76.74
   QD1:   QD,     AT=78.20
   ENDM:  MARKER, AT=79.0
ENDSEQUENCE
USE,CELL
SELECT,OPTICS,SBEND,QUAD,SEXT
OPTICS,FILENAME="cell.optics.f",EXIT,COLUMN=NAME,S,BETX,BETY
\end{verbatim} 

The resulting table file is: 
\begin{verbatim}
@ GAMTR            %f    64.3336
@ ALFA             %f   0.241615E-03
@ XIY              %f   -.455678
@ XIX              %f    2.05279
@ QY               %f   0.250049
@ QX               %f   0.249961
@ CIRCUM           %f    79.0000
@ DELTA            %f   0.000000E+00
@ COMMENT          %20s "DATA FOR TEST CELL"
@ ORIGIN           %24s "MAD 8.01    IBM - VM/CMS"
@ DATE             %08s "19/06/89"
@ TIME             %08s "09.47.40"
* NAME             S              BETX           BETY
$ %16s             %f             %f             %f
  B1                  36.6600        24.8427        126.380
  SF1                 37.6200        23.8830        130.925
  QF1                 39.5000        23.6209        132.268
  B2                  75.8000        124.709        25.2153
  SD1                 77.1200        130.933        23.8718
  QD1                 79.0000        132.277        23.6098
\end{verbatim} 

Second Example. The following is an excerpt of the LEP description: 
\begin{verbatim}
! Bending magnet pairs:
! The definitions take into account the different magnetic length
! for the inner and outer pairs of a group of six.
B2:    RBEND,     L=11.55,ANGLE=KMB2,K1=KQB,K2=KSB, \&
                  E1=-.25*B2[ANGLE],E2=-.25*B2[ANGLE]
B2OUT: B2,        ANGLE=1.00055745184472*KMB2, \&
                  E1=-.25*B2OUT[ANGLE],E2=-.25*B2OUT[ANGLE]
B2MID: B2,        ANGLE=1.00111490368947*KMB2, \&
                  E1=-.25*B2MID[ANGLE],E2=-.25*B2MID[ANGLE]
 
! Quadrupoles:
MQ:    QUADRUPOLE,L=1.6       ! standard quadrupoles =
QD:    MQ,        K1=KQD      ! cell quadrupoles, defocussing
QF:    MQ,        K1=KQF      !cell quadrupoles, focussing
 
! Sextupoles:
MSF:   SEXTUPOLE, L=0.40      ! F sextupoles
MSD:   SEXTUPOLE, L=0.76      ! D sextupoles
SF1.2: MSF,       K2=KSF1.2   ! F family 1, circuit 2
SF2.2: MSF,       K2=KSF2.2   ! F family 2, circuit 2
SF3.2: MSF,       K2=KSF3.2   ! F family 3, circuit 2
SD1.2: MSD,       K2=KSD1.2   ! D family 1, circuit 2
SD2.2: MSD,       K2=KSD2.2   ! D family 2, circuit 2
SD3.2: MSD,       K2=KSD3.2   ! D family 3, circuit 2
 
! Orbit correctors and monitors:
CH:    HKICK,     L=0.4       ! Horizontal orbit correctors
CV:    VKICK,     L=0.4       ! Vertical orbit correctors
MHV:   MONITOR,   L=0         ! Orbit position monitors
 
LEP:SEQUENCE
    ... 
QF23.R1:        QF,    AT=639.180037
   SF2.QF23.R1: SF2.2, AT=640.460037
   B2L.QF23.R1: B2OUT, AT=647.257037
   B2M.QD24.R1: B2MID, AT=659.147037
   B2R.QD24.R1: B2OUT, AT=671.037037
   CV.QD24.R1:  CV,    AT=677.392037, KICK=KCV24.R1
   PU.QD24.R1:  MHV,   AT=677.712037
QD24.R1:        QD,    AT=678.680037
   SD2.QD24.R1: SD2.2, AT=680.140037
   B2L.QD24.R1: B2OUT, AT=686.757037
   B2M.QF25.R1: B2MID, AT=698.647037
   B2R.QF25.R1: B2OUT, AT=710.537037
   CH.QF25.R1:  CH,    AT=716.942037, KICK=KCH25.R1
QF25.R1:        QF,    AT=718.180037
   SF1.QF25.R1: SF1.2, AT=719.460037
   B2L.QF25.R1: B2OUT, AT=726.257037
   B2M.QD26.R1: B2MID, AT=738.147037
   B2R.QD26.R1: B2OUT, AT=750.037037
   CV.QD26.R1:  CV,    AT=756.392037, KICK=KCV26.R1
   PU.QD26.R1:  MHV,   AT=756.712037
QD26.R1:        QD,    AT=757.680037
   SD1.QD26.R1: SD1.2, AT=759.140037
   B2L.QD26.R1: B2OUT, AT=765.757037
   B2M.QF27.R1: B2MID, AT=777.647037
   B2R.QF27.R1: B2OUT, AT=789.537037
QF27.R1:        QF,    AT=797.180037
   SF3.QF27.R1: SF3.2, AT=798.460037
   B2L.QF27.R1: B2OUT, AT=805.257037
   B2M.QD28.R1: B2MID, AT=817.147037
   B2R.QD28.R1: B2OUT, AT=829.037037
   PU.QD28.R1:  MHV,   AT=835.712037
QD28.R1:        QD,    AT=836.680037
   SD3.QD28.R1: SD3.2, AT=838.140037
   B2L.QD28.R1: B2OUT, AT=844.757037
   B2M.QF29.R1: B2MID, AT=856.647037
   B2R.QF29.R1: B2OUT, AT=868.537037
   CH.QF29.R1:  CH,    AT=874.942037, KICK=KCH29.R1
    ... 
ENDSEQUENCE
\end{verbatim} 
In the above structure it is easy to select many sets of observation points: 

Print at all F sextupoles: 
\begin{verbatim}
PRINT, MSF
\end{verbatim}

Split all quadrupoles at 1/3 of their length for OPTICS command: 
\begin{verbatim}
SPLIT,QUADRUPOLE,FACTOR=1/3
\end{verbatim}

Misalign two quadrupole QF25.R1 and QD26.R1: 
\begin{verbatim}
EALIGN, QF25.R1, QD26.R1, DX=0.001*GAUSS(), DY=0.0005*GAUSS()
\end{verbatim}

Print first-order matrices for elements B2L.QD24.R1 through CV.QD26.R1:  
\begin{verbatim}
SELECT, FIRST, B2L.QD24.R1[1]/CV.QD24.R1[1]
\end{verbatim}

Print lattice functions at all F-sextupoles of the first family, if connected to the second circuit: 
\begin{verbatim}
PRINT,SF1.2
\end{verbatim}

\line(1,0){300}
%\href{beam.html}{
%%\includegraphics{/icons/left.gif} BEAM: Set Beam Parameters}\href{action.html}{
%%\includegraphics{/icons/up.gif} Action Commands}\href{split.html}{
%%\includegraphics{/icons/right.gif} SPLIT: Request Interpolation for OPTICS}

\line(1,0){300}

%\href{http://www.cern.ch/Hans.Grote/hansg_sign.html}{hansg}, January 27, 1997 
