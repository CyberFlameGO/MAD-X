%%\title{Twiss Module}
%  Changed by: Chris ISELIN, 27-Jan-1997 
%  Changed by: Frank Schmidt, 11-Jul-2002 
%  Changed by: Hans Grote, 15-Jan-2003 
%  Changed by: Frank Schmidt, 06-APR-2003 

%  IMG ISMAP SRC="http://cern.ch/Frank.Schmidt/dynap/icons/at_work.gif"height=90 Under construction and not yet reliable!!!!

\chapter{Twiss Module}

%% should add the synopsis of the command with all options here

The TWISS command calculates the
\href{../Introduction/bibliography.html#courant}{[Courant and
    Snyder]}\href{../Introduction/tables.html#linear}{linear lattice
  functions}, and optionally the
\href{../Introduction/tables.html#chrom}{chromatic functions}. The
coupled functions are calculated in the sense of
\href{../Introduction/bibliography.html#edwards}{[Edwards and
    Teng]}. For the uncoupled cases they reduce to the C and S
functions. TWISS operates on the working beam line defined in the latest
\href{../control/general.html#use}{USE} command. One can also specify
either a \href{../Introduction/sequence.html}{SEQUENCE}="sequence\_name"
or a \href{../Introduction/line.html}{LINE}="line\_name" on the TWISS
command. Moreover, one can restrict the TWISS calculation to a desired
\href{../Introduction/ranges.html#range}{RANGE}.  

The relative energy error DELTAP may be entered in one of the two forms
\begin{verbatim}
DELTAP=real{,real}
DELTAP=initial:final:step
\end{verbatim} 

The first form lists several numbers, which may be general expressions,
separated by commas. The second form specifies an initial value, a final
value, and a step, which must be constant expressions, separated by
colons.  

Examples: 
\begin{verbatim}
DELTAP=0.001                    ! a single value
DELTAP=0.001,0.005              ! two values
DELTAP=0.001:0.007:0.002        ! four values
\end{verbatim}

If DELTAP is missing, MAD-X uses the value 0.0. 

Further attributes of the TWISS statements are: 
\begin{itemize}

   \item CHROM: A logical flag. If set, MAD-X also computes the
     \href{../Introduction/tables.html#chrom}{chromatic
       functions}.  \\
     \textit{Please note that this option is needed for a proper
       calculation of the chromaticities in the presence of coupling!}\\
     \textit{Please note that this option also changes the way that the
       chromaticities are calculated: The chromaticities are normally
       calculated from the analysis of the first and second order
       matrices. With CHROM, the chromaticities are recalculated by
       explicitely calculating the tunes for the case of the specified momentum
       deviation DELTAP and for the case of a momentum deviation equal
       to DELTAP+1.e-6. The tune differences divided by 1.e-6 yield the
       chromaticities.}

   \item FILE: If FILE="file\_name" appears MAD-X writes a full TFS
     Twiss table \href{../Introduction/select.html#tfs}{Example TFS
       Twiss table} on the disk file "file\_name". FILE alone is
     equivalent to FILE="twiss".\\
     The columns of the table can be selected using the \tt{SELECT}
     command with the \tt{flag=''twiss''} attribute. 

   \item TABLE (overrides SAVE): MAD-X creates a full
     \href{../Introduction/tables.html#linear}{Twiss table} in
     memory and gives it the name TWISS, unless TABLE="table\_name"
     appears on the command, then it is called
     \href{../Introduction/label.html}{table\_name}. This table
     includes linear lattice functions as well as the chromatic
     functions for all positions. An important new feature of MAD-X
     is the possibility to access entries of tables and in
     particular the twiss table (see
     \href{../Introduction/expression.html#table}{table access}).\\
     If the TABLE option is given the selection of column names via the
     \tt{SELECT} command is ignored. Hence if both \tt{TABLE} and
     \tt{FILE} options are given, the table written to file is the full
     twiss table, containing all elements as rows and all known twiss 
     parameters as columns. 

   \item NOTABLE: This flag prevents the creation of the internal Twiss
     table. Consequently no output file can be created either.  \\ 
     (Default: NOTABLE=false)

   \item CENTRE: This flag enforces the calculation of the
     \href{../Introduction/tables.html#linear}{linear lattice
       functions} at the center of the element instead of the end
     of it. \textit{ Mind you that since this is inside the element
       the closed orbit includes the misalignment of the element.} 

   \item RMATRIX: If this flag is used the the one-turn map at the
     location of every element is calculated and prepared for
     storage in the TWISS table. Using the
     \href{../Introduction/select.html}{SELECT} command and using
     the column RE, RE11...RE16...RE61...RE66 these components will
     be added to the TWISS table, i.e. with "column, RE" and
     "column, REij" one gets all or the component "ij"
     respectively.    

   \item SECTORMAP: This flag initiates the calculation of a sector
     map as described at:
     \href{../Introduction/sectormap.html}{SECTORMAP}.    

   \item SECTORFILE: Used to write SECTORMAPs to the file
     SECTORFILE="file\_name", if missing the output of SECTORMAP
     will go to the file "sectormap" with the format as found in
     \href{../Introduction/sectormap.html}{SECTORMAP}.    

   \item KEEPORBIT: The keeporbit attribute (with an optional name,
     keeporbit="name") stores the orbit under this name at the
     start, and at all monitors.    

   \item USEORBIT: The useorbit attribute (with an optional name,
     useorbit="name") uses the start value provided for the closed
     orbit search; the values at the monitors are used by the
     threader.    

   \item \textit{COUPLE (obsolete)} : This MAD8 option can no
     longer be set since TWISS in MAD-X is always calculated in
     coupled mode. MAD-X computes the coupled functions in the
     sense of
     \href{../Introduction/bibliography.html#edwards}{[Edwards and
         Teng]}. For the uncoupled cases they reduce to the C and S
     functions.    

   \item \textit{ Twiss calculation is 4D only!} : The Twiss
     command will calculate an approximate 6D closed orbit when the
     accelerator structure includes an active
     \href{../Introduction/cavity.html}{cavity}. However, the
     calcuation of the Twiss parameters are 4D only. This may
     result in apparently non-closure of the beta values in the
     plane with non-zero dispersion. The full 6D Twiss parameters
     can be calculated with the
     \href{../ptc_twiss/ptc_twiss.html}{ptc\_twiss} command.    

   \item RIPKEN: This flag will calculate the Ripken-Mais TWISS
     parameters (beta11, beta12, beta21, beta22, alfa11, alfa12,
     alfa21, alfa22, gama11, gamma12, gamm21 and gamm22) using
     betx, bety, alfx, alfy, gamax, gamay, R11, R12, R21 and R22 as
     input.  

\end{itemize}

The tables are suitable for \href{../plot/plot.html}{plot}.   After a
successful TWISS run MAD-X creates an implicit
\href{../Introduction/tables.html#summ}{table of summary parameters}
named "summ" which includes tunes, chromaticities etc (Please note that
the \href{../Introduction/tables.html#chrom}{chrom} option is needed
for a proper calculation of the chromaticities in the presence of
coupling!) versus the selected values of DELTAP. Notice that in MAD-X
DELTAP is converted in PT, which is used as longitudinal
variable. Dispersive and chromatic functions are hence derivatives with
respects to PT( see
\href{../Introduction/tables.html#summ}{table}). These summary
parameters can later be accessed via the
\href{../Introduction/expression.html#table}{table access} function
using the aforementionned implicit table named "summ". There is no way
to change the name of this summary table.  

\section{Twiss Parameters for a Period}

The simplest form of the TWISS command is 
\begin{verbatim}
TWISS, DELTAP=real{,value},CHROM,
       TABLE=table_name;
\end{verbatim}

It computes the periodic solution for the specified beam line for all
values of DELTAP entered (or for DELTAP = 0, if none is entered).  

Example: 
\begin{verbatim}
USE,period=OCT;
TWISS,DELTAP=0.001,CHROM;
\end{verbatim}

This example computes the periodic solution for the linear lattice and
chromatic functions for the beam line OCT. The DELTAP value used is
0.001. Apart from saving computing time, it is equivalent to the command
sequence  
\begin{verbatim}
RING: LINE=(4*(OCT,-OCT));
      USE,period=RING;
      TWISS,DELTAP=0.001,CHROM;
\end{verbatim}

\section{Initial Values from a Periodic Line}

It is possible to track the lattice functions starting with the periodic
solution for another beam line. If this is desired the TWISS command
takes the form  
\begin{verbatim}
TWISS, DELTAP=real{,value},LINE=beam-line,
       MUX=real,MUY=real,
       TABLE=table_name;
\end{verbatim}

No other attributes should appear in the command. For each value of
DELTAP MAD-X first searches for the periodic solution for the beam line
mentioned in LINE=beam-line: The result is used as an initial condition
for the lattice function tracking. 

Example: 
\begin{verbatim}
CELL:   LINE=(...);
INSERT: LINE=(...);
        USE,period=INSERT;
        TWISS,LINE=CELL,DELTAP=0.0:0.003:0.001,CHROM,FILE;
\end{verbatim}

For four values of DELTAP the following happens: First MAD-X finds the
periodic solution for the beam line CELL: Then it uses this solution as
initial conditions for tracking the lattice functions of the beam line
CELL: Output is also written on the file TWISS:  

If any of the beam lines was defined with formal arguments, actual
arguments must be provided:  
\begin{verbatim}
CELL(SF,SD): LINE=(...);
INSERT(X):   LINE=(...);
             USE,period=INSERT;
             TWISS,LINE=CELL(SF1,SD1);
\end{verbatim}

\section{Given Initial Values}

Initial values for \href{../Introduction/tables.html#linear}{linear
  lattice functions} and
\href{../Introduction/tables.html#chrom}{chromatic functions} may also
be numerical. Initial values can be specified on the TWISS command:  
\begin{verbatim}
TWISS,   BETX=real,ALFX=real,MUX=real,
         BETY=real,ALFY=real,MUY=real,
         DX=real,DPX=real,DY=real,DPY=real,
         X=real,PX=real,Y=real,PY=real,
         T=real,PT=real,
         WX=real,PHIX=real,DMUX=real,
         WY=real,PHIY=real,DMUY=real,
         DDX=real,DDY=real,DDPX=real,DDPY=real,
         R11=real,R12=real,R21=real,R22=real,  !coupling matrix
         TABLE=table_name,
         TOLERANCE=real,
         DELTAP=real:real:real;
\end{verbatim}

All initial values for \href{../Introduction/tables.html#linear}{linear
  lattice functions} and
\href{../Introduction/tables.html#chrom}{chromatic functions} are
permitted, but BETX and BETY are required. Moreover, a
\href{beta0}{beta0} block can be added as filled by the
\href{../control/general.html#savebeta}{savebeta} command or see
below. The lattice parameters are taken from this block, but will be
overwritten by explicitly stated lattice parameters. As entered, the
initial conditions cannot depend on DELTAP, and can thus be correct only
for one such value.  

\section{Tolerance}

This value defines the maximum closed orbit error of all six orbit
components during the closed orbit search. The default value is
1.e-6. The value is only valid for the current twiss command; a
permanent value can be entered via the
\href{../control/general.html#coguess}{COGUESS} command.  

\section{SAVEBETA: Save Lattice Parameters}

Initial lattice parameters can be transfered for later commands, in
particular for twiss or the \href{../match/match.html}{match module}, by
using the \href{../control/general.html#savebeta}{savebeta} command
sequence.  

%\textit{ It should be mentioned that parameters can be also accessed
%from tables using the
%\href{../Introduction/expression.html#table}{table access} function.} 
It should be mentioned that parameters can be also accessed from tables
using the \href{../Introduction/expression.html#table}{table access}
function. 
\begin{verbatim}
USE,period=...;
SAVEBETA,LABEL=name,PLACE=place,SEQUENCE=s_name;
TWISS,...;
\end{verbatim}

When reaching the \href{../control/general.html#place}{place} in the
sequence "s\_name" during execution of TWISS, MAD-X will save a
\hyperlink{beta0}{beta0} block with the
\href{../Introduction/label.html}{label} name: This block is filled with
the values of all lattice parameters in place.  

Example 1: 
\begin{verbatim}
USE,period=CELL;
SAVEBETA,LABEL=END,PLACE=#E,SEQUENCE=CELL;
TWISS;
USE,period=INSERT;
TWISS,BETA0=END;
\end{verbatim}

This first example calculates the \hyperlink{periodic}{periodic
  solution} of the line CELL, and then track lattice parameters through
INSERT, using all end conditions (including orbit) in CELL to start.  

Example 2: 
\begin{verbatim}
USE,period=CELL;
SAVEBETA,LABEL=END,PLACE=#E,SEQUENCE=CELL;
TWISS;
USE,period=INSERT;
TWISS,BETX=END-$>$BETY,BETY=END-$>$BETX;
\end{verbatim}

This is similar to the first example,but the beta functions are interchanged (overwritten).  

\section{BETA0: Set Initial Lattice Parameters}

Initial lattice parameters can be set 'manually' for later commands, in
particular for twiss or the \href{../match/match.html}{match module}, by
using the BETA0 command attached to a label.  

Example 3: 
\begin{verbatim}
INITIAL: BETA0, BETX=BX0,ALFX=0.0,MUX=0.0,BETY=BY0,ALFY=0.0,DX=DX0,DPX=0.0;
TWISS,BETA0=INITIAL;
\end{verbatim}

Example 4: 
\begin{verbatim}
INITIAL: BETA0, BETX=BX0,ALFX=0.0,MUX=0.0,BETY=BY0,ALFY=0.0,DX=DX0,DPX=0.0;
TWISS,BETX=INITIAL-$>$BETY,BETY=INITIAL-$>$BETX;
\end{verbatim}



%%%\title{SELECT}
%  Changed by: Hans Grote, 11-Sep-2002 

%%\usepackage{hyperref}
% commands generated by html2latex


%%\begin{document}
%%\begin{center}
 %%EUROPEAN ORGANIZATION FOR NUCLEAR RESEARCH 
%%\includegraphics{http://cern.ch/madx/icons/mx7_25.gif}

\subsection{Sectormap output}
%%\end{center}
% Begin New version: Jean-Luc Nougaret, 18-Dec-2008 


 The flag "sectormap" on the Twiss command (together with an element selection via select,flag=sectormap,...) causes a file "sectormap" to be written.  

For each user-selected element, it contains the user-selected coefficients of the kick vector 
\texttt{K} (6 values), of the first-order map 
\texttt{R} (6 x 6 values) and of the second-order map 
\texttt{T} (6 x 6 x 6 values)

 The sector file is the output of a standard TFS table, which means that the set of columns of interest may be selected through a MAD-X command such as the following: 


%\begin{tabular}
\begin{verbatim}
select,flag=my_sect_table,column=name,pos,k1,r11,r66,t111;
\end{verbatim}
%\end{tabular}  
Each line of the sectormap file contains for each selected element, the set of chosen K,R,T matrix coefficients:
\\
\\
\begin{tabular}{l|l|l}
@ NAME &              \%13s &  "MY\_SECT\_TABLE" \\ 
@ TYPE &              \%09s &  "SECTORMAP" \\ 
@ TITLE &             \%08s &  "no-title" \\ 
@ ORIGIN &           \%19s &  "MAD-X 3.04.62 Linux" \\ 
@ DATE &              \%08s &  "18/12/08" \\ 
@ TIME &              \%08s &  "10.33.58"
\end{tabular}
\\
\\
\begin{tabular}{l | l | l | l | l | l }
* NAME & POS & K1 & R11 & R66 & T111 \\ 
\$ \%s & \%le & \%le & \%le  & \%le & \%le \\ 
 "FIVECELL\$START"  & 0 & 0 & 1 & 1 & 0 \\ 
 "SEQSTART"  & 0 &  0  &  1 &  1  &  0 \\ 
 "QF.1"  & 3.1 & -1.305314637e-05 & 1.042224745 & 1 & 0 \\ 
 "DRIFT\_0" & 3.265 & 7.451656548e-21 & 1 & 1 & 0 \\ 
 "MSCBH" & 4.365 & -1.686090613e-15 & 0.9999972755 & 1 & 0.006004411526 \\ 
 "CBH.1" & 4.365 & 0 & 1 & 1 & 0 \\ 
 "DRIFT\_1" & 5.519992305 & -6.675347543e-21 & 1 & 1 & 0 \\ 
 "MB" & 19.72000769 & 2.566889547e-18 & 1.000000091 & 1 & -4.135903063e-25 \\ 
 "DRIFT\_2" & 21.17999231 & -1.757758802e-20 & 1 & 1 & 0 \\ 
 "MB" & 35.38000769 & 2.822705549e-18 & 1.000000091 & 1 & -4.135903063e-25 \\ 
 "DRIFT\_2" & 36.83999231 & 2.480880093e-20 & 1 & 1 & 0 \\ 
 "MB" & 51.04000769 & 3.006954115e-18 & 1.000000091 & 1 & -4.135903063e-25 \\ 
 "DRIFT\_3" & 52.21 & -4.886652187e-20 & 1 & 1 & 0 \\ 
... & ... & ... & ... & ... & ... \\ 
... & ... & ... & ... & ... & ... \\ 
... & ... & ... & ... & ... & ...
\end{tabular}
\\
\\ Of course, the 
\texttt{select} statement can be combined with additional options to filter-out the list of elements, such as in the following statement, which for instance only retains drift-type elements: 


%\begin{tabular}
\begin{verbatim}
select,flag=my_sect_table,class=drift,column=name,pos,k1,r11,r66,t111;
\end{verbatim}
%\end{tabular}


\texttt{K} coefficients range: 
\texttt{K1}... 
\texttt{K6}


\texttt{R} coefficients range: 
\\
\begin{tabular}{ccc}
\texttt{R11} & ... & \texttt{R61} \\ 
\texttt{R12} & ... & \texttt{R62} \\ 
... & ... & ... \\ 
\texttt{R61} & ... & \texttt{R66}
\end{tabular}


\texttt{T} coefficients range: 
\\
\begin{tabular}{ccc}
\texttt{T111} & ... &\texttt{T611} \\ 
\texttt{T121} & ... & \texttt{T621} \\ 
... & ... & ... \\ 
\texttt{T161} & ... & \texttt{T661} \\ 
\texttt{T112} & ... & \texttt{T612} \\ 
... & ... & ... \\ 
\texttt{T166} & ... & \texttt{T666}
\end{tabular}

 In the above notation, 
\texttt{Rij} stands for "effect on the 
\texttt{i}-th coordinate of the 
\texttt{j}-th coordinate in phase-space", whereas 
\texttt{Tijk} stands for "combined effect on the 
\texttt{i}-th coordinate of both the 
\texttt{j}-th and 
\texttt{k}-th coordinates in phase-space" along the lattice. 
% End New Version 

%  Commented by jluc, on 18 December 2008
% The flag "sectormap" on the Twiss command (together with an element
% selection via select,flag=sectormap,...) causes a file "sectormap" to
% be written. This is a fixed format file; per selected element it
% contains:
% 
% <pre>
% end_position   element_name
% kick vector (6 values)
% first order map (6 lines with 6 values each), column-wise
% second order map (36 lines with 6 columns each, column-column-wise)
% </pre>
% 
% Or more explicitly:
% 
% <pre>
% The first line is:
% K[1] ... K[6]
% 
% Then: 
% R[1,1] ... R[6,1]
% R[1,2] ... R[6,2]
% ...
% R[1,6] ... R[6,6]
% 
% 
% Then:
% T[1,1,1] ... T[6,1,1]
% T[1,2,1] ... T[6,2,1]
% ...
% T[1,6,1] ... T[6,6,1]
% T[1,1,2] ... T[6,1,2]
% ...
% T[1,6,6] ... T[6,6,6]
% </pre>
% 
   The maps are the accumulated maps between the selected elements. They contain both the alignment, and field errors present. Together with the starting value of the closed orbit (which can be obtained from the standard twiss file) this allows the user to track particles over larger sectors, rather than element per element. A typical usage therefore lies in the interface to other programs, such as TRAIN. 
\\
\href{http://www.cern.ch/Hans.Grote/hansg_sign.html}{hansg}, May 8, 2001 

%%\end{document}

\subsection{Sectormap output}
\label{subsec:sectormap}
% Begin New version: Jean-Luc Nougaret, 18-Dec-2008 

The flag "sectormap" on the Twiss command (together with an element
selection via select,flag=sectormap,...) causes a file "sectormap" to be
written.   

For each user-selected element, it contains the user-selected coefficients of the kick vector 
\texttt{K} (6 values), of the first-order map 
\texttt{R} (6 x 6 values) and of the second-order map 
\texttt{T} (6 x 6 x 6 values)

The sector file is the output of a standard TFS table, which means that
the set of columns of interest may be selected through a SELECT command
as in the following example:  


\begin{verbatim}
select, flag=my_sect_table, column=name, pos, k1, r11, r66, t111;
\end{verbatim}


The sectormap file contains for each selected element, one element per line, the
set of chosen K, R, and T matrix coefficients: 
\\
\\
\begin{tabular}{l|l|l}
@ NAME &              \%13s &  "MY\_SECT\_TABLE" \\ 
@ TYPE &              \%09s &  "SECTORMAP" \\ 
@ TITLE &             \%08s &  "no-title" \\ 
@ ORIGIN &           \%19s &  "MAD-X 3.04.62 Linux" \\ 
@ DATE &              \%08s &  "18/12/08" \\ 
@ TIME &              \%08s &  "10.33.58"
\end{tabular}
\\
\\
\begin{tabular}{l | l | l | l | l | l }
* NAME & POS & K1 & R11 & R66 & T111 \\ 
\$ \%s & \%le & \%le & \%le  & \%le & \%le \\ 
 "FIVECELL\$START"  & 0 & 0 & 1 & 1 & 0 \\ 
 "SEQSTART"  & 0 &  0  &  1 &  1  &  0 \\ 
 "QF.1"  & 3.1 & -1.305314637e-05 & 1.042224745 & 1 & 0 \\ 
 "DRIFT\_0" & 3.265 & 7.451656548e-21 & 1 & 1 & 0 \\ 
 "MSCBH" & 4.365 & -1.686090613e-15 & 0.9999972755 & 1 & 0.006004411526 \\ 
 "CBH.1" & 4.365 & 0 & 1 & 1 & 0 \\ 
 "DRIFT\_1" & 5.519992305 & -6.675347543e-21 & 1 & 1 & 0 \\ 
 "MB" & 19.72000769 & 2.566889547e-18 & 1.000000091 & 1 & -4.135903063e-25 \\ 
 "DRIFT\_2" & 21.17999231 & -1.757758802e-20 & 1 & 1 & 0 \\ 
 "MB" & 35.38000769 & 2.822705549e-18 & 1.000000091 & 1 & -4.135903063e-25 \\ 
 "DRIFT\_2" & 36.83999231 & 2.480880093e-20 & 1 & 1 & 0 \\ 
 "MB" & 51.04000769 & 3.006954115e-18 & 1.000000091 & 1 & -4.135903063e-25 \\ 
 "DRIFT\_3" & 52.21 & -4.886652187e-20 & 1 & 1 & 0 \\ 
... & ... & ... & ... & ... & ... \\ 
... & ... & ... & ... & ... & ... \\ 
... & ... & ... & ... & ... & ...
\end{tabular}
\\
\\ 
Of course, the \texttt{select} statement can be combined with additional
options to filter-out the list of elements, such as in the following
statement, which for instance only retains drift-type elements:  

\begin{verbatim}
select, flag=my_sect_table, class=drift, column=name, pos, k1, r11, r66, t111;
\end{verbatim}



\texttt{K} coefficients range: 
\texttt{K1}... 
\texttt{K6}


\texttt{R} coefficients range: 
\\
\begin{tabular}{ccc}
\texttt{R11} & ... & \texttt{R61} \\ 
\texttt{R12} & ... & \texttt{R62} \\ 
... & ... & ... \\ 
\texttt{R61} & ... & \texttt{R66}
\end{tabular}


\texttt{T} coefficients range: 
\\
\begin{tabular}{ccc}
\texttt{T111} & ... &\texttt{T611} \\ 
\texttt{T121} & ... & \texttt{T621} \\ 
... & ... & ... \\ 
\texttt{T161} & ... & \texttt{T661} \\ 
\texttt{T112} & ... & \texttt{T612} \\ 
... & ... & ... \\ 
\texttt{T166} & ... & \texttt{T666}
\end{tabular}

 In the above notation, 
\texttt{Rij} stands for "effect on the 
\texttt{i}-th coordinate of the 
\texttt{j}-th coordinate in phase-space", whereas 
\texttt{Tijk} stands for "combined effect on the 
\texttt{i}-th coordinate of both the 
\texttt{j}-th and 
\texttt{k}-th coordinates in phase-space" along the lattice. 
% End New Version 

%  Commented by jluc, on 18 December 2008
% The flag "sectormap" on the Twiss command (together with an element
% selection via select,flag=sectormap,...) causes a file "sectormap" to
% be written. This is a fixed format file; per selected element it
% contains:
% 
% <pre>
% end_position   element_name
% kick vector (6 values)
% first order map (6 lines with 6 values each), column-wise
% second order map (36 lines with 6 columns each, column-column-wise)
% </pre>
% 
% Or more explicitly:
% 
% <pre>
% The first line is:
% K[1] ... K[6]
% 
% Then: 
% R[1,1] ... R[6,1]
% R[1,2] ... R[6,2]
% ...
% R[1,6] ... R[6,6]
% 
% 
% Then:
% T[1,1,1] ... T[6,1,1]
% T[1,2,1] ... T[6,2,1]
% ...
% T[1,6,1] ... T[6,6,1]
% T[1,1,2] ... T[6,1,2]
% ...
% T[1,6,6] ... T[6,6,6]
% </pre>
% 
   
The maps are the accumulated maps between the selected elements. They
contain both the alignment, and field errors present. Together with the
starting value of the closed orbit (which can be obtained from the
standard twiss file) this allows the user to track particles over larger
sectors, rather than element per element. A typical usage therefore lies
in the interface to other programs, such as TRAIN.  


%%\documentclass[a4paper,11pt]{article}
%%\usepackage{ulem}
%%\usepackage{a4wide}
%%\usepackage[dvipsnames,svgnames]{xcolor}
%%\usepackage[pdftex]{graphicx}
%%\title{Threader}
%%\usepackage{hyperref}
% commands generated by html2latex


%%\begin{document}  %%EUROPEAN ORGANIZATION FOR NUCLEAR RESEARCH 
%%\includegraphics{../icons/mx7_25.gif align=right}

\subsection{Beam Threader}  The \textbf{threader} simulates beam steering through a machine with field and alignment errors in situations where the beam does not circulate and the closed orbit cannot be measured. 

 If enabled, threading is executed whenever a trajectory or closed orbit search is carried out by the \href{../twiss/twiss.html}{TWISS} module. 

 The following MAD-X commands control the action of the threader :  


\begin{verbatim}
option, threader ;
\end{verbatim}  enables the threader 

 when set, the threader checks at all monitors the difference with respect to the stored orbit there (from \href{../twiss/twiss.html}{keeporbit}) if \href{../twiss/twiss.html}{useorbit} is present. The threader then provides kicks (if possible) to reduce the orbit difference below the maxima specified on the threader command. This procedure allows to thread with orbit bumps present  

\begin{verbatim}
threader, vector = {xmax, ymax, att} ;
\end{verbatim}  sets the parameters for the threader 
\begin{verbatim}
xmax, ymax : orbit excursion (at a monitor) at which threader acts
att        : attenuation factor for the kicks applied by the threader
defaults   : {0.005, 0.005, 1.000}
\end{verbatim}


\line(1,0){300}

 Hans Grote, 31.10.2008 

%%\end{document}

%%\title{Threader}

\section{Beam Threader} 
The \textbf{threader} simulates beam steering through a machine with
field and alignment errors in situations where the beam does not
circulate and the closed orbit cannot be measured.  

If enabled, threading is executed whenever a trajectory or closed orbit
search is carried out by the \href{../twiss/twiss.html}{TWISS} module.  

The following MAD-X commands control the action of the threader :   
\begin{verbatim}
option, threader ;
\end{verbatim}  
enables the threader 

When set, the threader checks at all monitors the difference with
respect to the stored orbit there (from
\href{../twiss/twiss.html}{keeporbit}) if
\href{../twiss/twiss.html}{useorbit} is present. The threader then
provides kicks (if possible) to reduce the orbit difference below the
maxima specified on the threader command. This procedure allows to
thread with orbit bumps present.   

\begin{verbatim}
threader, vector = {xmax, ymax, att} ;
\end{verbatim}  
sets the parameters for the threader 
\begin{verbatim}
xmax, ymax : orbit excursion (at a monitor) at which threader acts
att        : attenuation factor for the kicks applied by the threader
defaults   : {0.005, 0.005, 1.000}
\end{verbatim}


