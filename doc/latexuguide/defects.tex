%%\title{Known Defects of MAD8 and MAD-X}
%  Changed by: Chris ISELIN, 24-Jan-1997 
%  Changed by: Hans Grote, 25-Sep-2002 

\chapter{Known Differences to Other Programs}

\section{Definitions in \madeight}

\madeight uses full 6-by-6 matrices to allow coupling effects to be treated,
and the canonical variable set ($x$ , $p_x / p_0$),
($y$ , $p_y / p_0$), ($-ct$ , $\delta E / p_0 c$), as opposed to other
programs most of which use the set ($x$, $x'$), ($y$, $y'$),
($-\delta s$), $\delta p / p_0$). 

Like Dragt \cite{dragt1982}, \madeight uses the relative
energy error $p / p_0$, which is equal the relative momentum error
${\texttt delta} = \delta p / p_0$ multiplied by $\beta = v/c$.  

As from Version 8.13, \madeight used an additional \textbf{constant}
momentum error $\texttt{delta}_s$ in all optical calculations. 
The transfer maps contained the \textbf{exact} dependence upon this
value; therefore the tunes for large deviations could be computed with
high accuracy as opposed to previous versions.  
 
The choice of canonical variables in \madx still leads to slightly
different definitions of the lattice functions. In \madx the
Courant-Snyder invariants mentioned in \cite{Courant_Snyder1958} take
the form  

$W_x = \gamma_x x^2 - 2 \alpha_x x p_x + \beta_x  p_x^2$

Comparison to the original form 

$W_x = \gamma_x x^2 - 2 \alpha_x x x' + \beta_x  {x'}^2$

shows that the orbit functions cannot be the same. 

A more detailed analysis, using 
$x' = p_x / (1 + \delta)$
shows that all formulas can be made consistent by defining the MAD orbit
functions (index M) wrt. the Courant and Snyder functions (index C) as follows:

$\beta_{xM} = \beta_{xC} \times (1 + \delta)$, \\
$\alpha_{xM} = \alpha_{xC}$, \\
$\gamma_{xM} = \gamma_{xC} / (1 + \delta)$

For constant $\delta_s$ along the beam line and $\delta = 0$, the
lattice functions are the same. In a machine where $\delta$ varies along
the circumference, e.g. in a linear accelerator or in an
electron-positron storage ring, the definition of the Courant-Snyder
invariants must be generalised. The \madeight invariants have the advantage
that they remain invariants along the beam line even for variable $\delta$.  

With the new method this problem occurs in Twiss module (see
\ref{chap:twiss}) only for non-constant $\delta$.    

\section{Treatment of Energy Error in TWISS} 
It has been noted by Milutinovic and Ruggiero \cite{milutinovic1988}
that \madeight returned tunes which are too low for non-zero
$\delta$. The difference was found to be quadratic in delta with a negative
coefficient. This problem has been eliminated thanks to the new
treatment  of momentum errors from \madeight Version 8.13 onwards.   

\section{Definition of emittance in \madx}
This section should explain how natural and normalised emittances are
defined in \madx and why there is a factor 4 in the definition of the
natural emittance, which is a marked difference with other programs such
as sixtrack (see the conversion to sixtrack format)

%% EOF

