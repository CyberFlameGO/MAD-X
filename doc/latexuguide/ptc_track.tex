%%\title{Thick-Lens Tracking Module (PTC-TRACK Module)}
%  Created by: Valery KAPIN, 06-Apr-2006 
%  Changed by: ____________, ___________ 


\chapter{Thick-Lens Tracking Module}
\label{chap:ptc_track}

The {\tt PTC-TRACK} module \cite{kapin2006,schmidt2005} is the
symplectic thick-lens tracking facility in \madx.  It is based on \ptc
library \cite{forest2002} written by E.~Forest.  The commands of this
module are described below, optional parameters are denoted by square
brackets ([]).

Prior to using this module the active beam line must be selected by
means of a \hyperref[sec:use]{USE} command.  
The general \hyperref[chap:ptc_setup]{PTC environment} must
also be initialized.  

{\bf Examples}\\
Several examples can be found on the web at
\href{http://cern.ch/frs/mad-X_examples/ptc_track}{http:\/\/cern.ch\/frs\/mad-X\_examples\/ptc\_track}. 


\section{Synopsis}
\label{sec:ptc_track_synopsis}

A typical tracking job in \ptc requires a number of commands to be issued:

\madxmp{
xxxx\= \kill
PTC\_CREATE\_UNIVERSE; \\
PTC\_CREATE\_LAYOUT, MODEL=integer, METHOD=integer, NST=integer, [EXACT];\\
\>\ldots\\
\>PTC\_START, X=real, PX=real, Y=real, PY=real, T=real, PT=real;\\
\>PTC\_START, FX=real, PHIX=real, FY=real, PHIY=real, FT=real, PHIT=real;\\
\>\ldots\\
\>PTC\_OBSERVE, PLACE=string; \\
\>\ldots\\
\>PTC\_TRACK, \ldots ;\\
\>\ldots\\
\>PTC\_TRACKLINE, \ldots ;\\
\>\ldots \\
\>PTC\_TRACK\_END;\\
\>\ldots\\
PTC\_END;
}


\section{PTC\_START}
\label{sec:ptc_start}

To start particle tracking, a series of initial trajectory coordinates
must be given with the {\tt PTC\_START} command;  and as many
commands as initial trajectories can be given. 

{\tt PTC\_START} commands must appear before the
\hyperref[sec:ptc_track]{\tt PTC\_TRACK} command. 

\madbox{  
PTC\_START, \=X=real, PX=real, Y=real, PY=real, T=real, PT=real,\\
            \>FX=real, PHIX=real, FY=real, PHIY=real, FT=real, PHIT=real; 
}

The coordinates can be 

\begin{madlist}
   \ttitem{X, PX, Y, PY, T, PT} i.e. the standard canonical
   coordinates. \\ (Default: 0.0)  

   \ttitem{FX, PHIX, FY, PHIY, FT, PHIT} i.e. the action-angle
   coordinates which are expressed by the normalized amplitude, $F_z$ and
   the phase, $\Phi_z$ for the \textit{z}-th mode plane 
   (\textit{z = x, y, t}).
   The actions are computed with the values of the emittances,
   $F_z$, which must be specified in a preceding 
   \hyperref[sec:beam]{\tt BEAM} command. $F_z$ are expressed in number
   of r.m.s. beam sizes and $\Phi_z$ are expressed in
   radians.\\ (Default: 0.0) 
\end{madlist}


{\bf Remarks} \\
In the uncoupled case, the canonical and the action-angle
variables are related with equations 
\begin{equation} 
z = F_z (E_z)^{1/2} cos(\Phi_z) \qquad  p_z = F_z (E_z)^{1/2} sin(\Phi_z)
\end{equation}

If both the canonical and the action-angle coordinates are 
given in the {\tt PTC\_START} command, they are summed after conversion
of the action-angle coordinates to canonical coordinates.

The use of the action-angle coordinates requires the option 
\hyperref[opt:closed_orbit]{\tt CLOSED\_ORBIT} in the 
\hyperref[sec:ptc_track]{\tt PTC\_TRACK} command. 

If the option
\hyperref[opt:closed_orbit]{\tt CLOSED\_ORBIT} in the
\hyperref[sec:ptc_track]{\tt PTC\_TRACK} command is active
(see above) all coordinates are specified with respect to the actual
closed orbit (possibly off-momentum with magnet errors) and NOT with
respect to the reference orbit. 
If the option \hyperref[opt:closed_orbit]{\tt CLOSED\_ORBIT} is absent,
then coordinates are specified with respect to the reference orbit.  



\section{PTC\_OBSERVE} 
\label{sec:ptc_observe}

Besides the beginning of the beam-line, one can define an additional
observation points along the machine. Subsequent \texttt{PTC\_TRACK}
command will then record the tracking data on all these observation
points.  

\madbox{
PTC\_OBSERVE, PLACE=string; 
}

The only attribute is 
\begin{madlist}
  \ttitem{PLACE} the name of observation point.  \\ (Default: NULL)
\end{madlist}


{\bf Remarks}\\
The first observation point at the beginning of the beam-line is marked
as {\bf "start"}.  
       
It is {\em strongly} recommended to specify markers as observation points.

The data at observation points other than {\bf "start"} can be produced
in two different ways:
\begin{enumerate}
  \item traditional element-by-element tracking. (See
    \hyperref[chap:thintrack]{\madx thin tracking}) which 
    requires the option \hyperref[opt:element_by_element]{\tt
      ELEMENT\_BY\_ELEMENT} of \hyperref[sec:ptc_track]{\tt PTC\_TRACK} to
    be active.

  \item coordinate transformation from {\bf "start"} to the
  respective observation points using high-order \ptc transfer
  maps, which requires the option \hyperref[opt:closed_orbit]{\tt
    CLOSED\_ORBIT} of \hyperref[sec:ptc_track]{\tt PTC\_TRACK} to be
  active, and the options \hyperref[opt:radiation]{\tt RADIATION}
  and \hyperref[opt:element_by_element]{\tt ELEMENT\_BY\_ELEMENT} of
  \hyperref[sec:ptc_track]{\tt PTC\_TRACK} to be inactive.
\end{enumerate} 


\section{PTC\_TRACK}
\label{sec:ptc_track}

The {\tt PTC\_TRACK} command initiates trajectory tracking by
entering the thick-lens tracking module.  

The tracking can be done element-by-element or "turn-by-turn" with
coordinate transformations over the whole turn.  

Tracking is done in parallel, i.e. the coordinates of all particles are
transformed through each beam element, or over full turns. 

A particle is lost if its trajectory is outside specified boundaries. 
In \ptc, there is a continuous check that the particle trajectories stay
within the aperture limits.  

The Normal Form calculation is controlled through options of the 
{\tt PTC\_TRACK} command.

\madbox{
PTC\_TRACK, \=ICASE=integer, DELTAP=real, CLOSED\_ORBIT=logical,\\ 
            \>ELEMENT\_BY\_ELEMENT=logical, TURNS=integer, \\
            \>DUMP=logical, ONETABLE=logical, MAXAPER=real\_array, \\
            \>NORM=integer, NORM\_OUT=logical, \\
            \>FILE[=string], EXTENSION=string, FFILE=integer, \\
            \>RADIATION=logical, RADIATION\_MODEL1=logical, \\
            \>RADIATION\_ENERGY\_LOSS=logical, \\
            \>RADIATION\_QUADR=logical, \\
            \>BEAM\_ENVELOPE=logical, SPACE\_CHARGE=logical;
}


The attributes are: 
\begin{madlist}
  
  \ttitem{ICASE} the user-defined dimensionality of the phase-space 
  (4, 5 or 6). \\ 
  {\tt ICASE} has higher priority over other options. In particular:
  \begin{enumerate}
    \item RF cavities with non-zero voltage are ignored for
      {\tt ICASE=4} or {\tt ICASE=5}.
    \item A non-zero {\tt DELTAP} is ignored for {\tt ICASE=4} or {\tt ICASE=6}.
  \end{enumerate}
  However, if an RF cavity has voltage set to zero and {\tt ICASE=6} is
  specified, \ptc sets {\tt ICASE=4}.
  \\(Default: 4)

   \ttitem{DELTAP} the relative momentum offset for reference closed
   orbit (used for 5D case ONLY). \\
   {\tt DELTAP} is ignored for {\tt ICASE=6}, but the option 
   {\tt OFFSET\_DELTAP} of command \hyperref[sec:ptc_create_layout]
   {\tt PTC\_CREATE\_LAYOUT} may be used, if the reference particle should
   have a momentum offset. \\ 
   (Default: 0.0) 

   \ttitem{CLOSED\_ORBIT}\label{opt:closed_orbit}
   a logical switch to activate the closed orbit calculation.
   This option must be used for closed rings only. This
   option allows to activate the Normal Form analysis, if
   required. With {\tt CLOSED\_ORBIT=false}, the sequence is treated as
   a transfer line. \\ (Default: false)
     
   \ttitem{ELEMENT\_BY\_ELEMENT}\label{opt:element_by_element}
   a logical switch to activate the element-by-element tracking, from
   the default turn-by-turn tracking. \\ (Default: false)

   \ttitem{TURNS} number of turns to be tracked. \\ (Default: 1)

   \ttitem{DUMP} a logical flag to enforce writing particle coordinates
   to formatted text files. \\
   (Default: false)

   \ttitem{ONETABLE}\label{opt:onetable} a logical switch to write all 
	particle coordinates to a single file instead of separate files. \\
   (Default: false)

   \ttitem{MAXAPER} an array defining upper limits for particle
   coordinates, essentially defining the aperture to trigger particle
   loss. \\ (Default: \{0.1, 0.01, 0.1, 0.01, 1.0, 0.1\})

   \ttitem{NORM\_NO} order of the Normal Form.\\
   {\tt NORM\_NO=1} makes the Normal Form linear (always true for
   \madx). \\ (Default: 1)

   \ttitem{NORM\_OUT} a logical switch to transform canonical variables
   to action-angle variables. \\ (Default: false) 

   \ttitem{FILE} if FILE is omitted, no output is written to file.\\
   if FILE is present, track tables are printed, optionally to 
   files with name constructed from the base filename specified. \\
   The actual name of the output file is constructed from
   tyhe baseline given with {\tt FILE} to which are appended the
   strings ".obsnnnn" (where nnnn is the observation point index) and
   ".pnnnn" (where nnnn is now the particle number), unless the
   \hyperref[opt:onetable]{\tt ONETABLE} option is activated.  \\
   (Default: "track") 

   \ttitem{EXTENSION} a string providing the filename extension for the
   track table files, e.g., txt, doc...  \\ (Default: nil)

   \ttitem{FFILE} defines the periodicity {\tt n} of the printout:
   coordinates are printed every n turns. \\ (Default: 1)

   \ttitem{RADIATION}\label{opt:radiation} a logical flag to turn on
     the synchrotron radiation calculated by an internal procedure of
     \ptc. \\
     The option {\tt RADIATION} has precedence over {\tt RADIATION\_MODEL1}
     when both are activated. 
     \\ (Default: false)

   \ttitem{RADIATION\_MODEL1} a logical flag to turn on the synchrotron
   radiation according to the method given in \cite{roy1990}. This model
   simulates quantum excitation via a random number generator and tables
   for photon emission. It can be used only with the option
   \hyperref[opt:element_by_element]{\tt ELEMENT\_BY\_ELEMENT}.\\ 
   (Default: false) 

   \ttitem{RADIATION\_ENERGY\_LOSS} a logical flag to add back the
   average energy loss for {\tt RADIATION\_MODEL1}, thereby taking only
   the quantum excitation into effect.\\ (Default: false)

   \ttitem{RADIATION\_QUADR} a logical flag to add the effect of
   synchrotron radiation in quadrupoles. It supplements either model
   {\tt RADIATION} or {\tt RADIATION\_MODEL1}. \\
   (Default: false)

   \ttitem{BEAM\_ENVELOPE} a logical switch to activate the calculation
   of the beam envelopes with \ptc. It requires the options
   {\tt RADIATION} and {\tt ICASE=6}.\\
   (Default: false)

   \ttitem{SPACE\_CHARGE} {\bf [under construction]}\\
     a logical flag to activate the simulation of space charge forces
     between particles. \\ (Default: false)  
\end{madlist}


\section{PTC\_TRACKLINE}
\label{sec:ptc_trackline}

The {\tt PTC\_TRACKLINE} command performs particle tracking that takes
into account acceleration in travelling wave cavities. 
It must be invoked in the scope of correctly initialized
\hyperref[chap:ptc_setup]{PTC environment}, 
i.e. after \hyperref[sec:ptc_create_universe]{\tt PTC\_CREATE\_UNIVERSE}
and \hyperref[sec:ptc_create_layout]{\tt PTC\_CREATE\_LAYOUT} commands, and before
corresponding \hyperref[sec:ptc_end]{\tt PTC\_END}. 

All tracks created with \hyperref[sec:ptc_start]{\tt PTC\_START}
commands before {\tt PTC\_TRACKLINE} command is issued are 
tracked. Track parameters are dumped at every defined observation point
(see \hyperref[sec:ptc_observe]{\tt PTC\_OBSERVE} command). 

Please note that \madx always creates an observation point at the end of a
sequence.

\madbox{
PTC\_TRACK\_LINE, \=TURNS=integer, \\ 
                  \>ONETABLE=logical, FILE=string, EXTENSION=string, \\
                  \>ROOTNTUPLE=logical, \\
                  \>EVERYSTEP=logical, TABLEALLSTEPS=logical, \\
                  \>GCS=logical;
}

The attributes are:
\begin{madlist}
   \ttitem{TURNS} number of turns to be tracked. If the layout of the
   machine is not closed, this value is forced to {\tt TURNS=1} by \ptc 
   \\ (Default: 1)
     
   \ttitem{ONETABLE}\label{opt:onetable2} a logical switch to write all
   particle coordinates to a single file instead of separate files. \\
   %% If false one file per track per observation point is written.\\ 
   %% File format is filename.obsNNNN.pMMMM, where NNNN and MMMM are
   %% respectively the number of the observation point and of the track\\
   %% Filename is defined by the switch described below. \\
   %% If false, tracking data are written to a single table for each
   %% track for each observation point. Table names follow the naming
   %% \textit{filename}.obsMMMM.pNNNN, where \textit{filename} is
   %% settable prefix with \textbf{file} parameter (see below), MMMM is
   %% observation point number and NNNN is track number. \\  
   %% If true, all data are written to single table called onetable. \\
   (Default: false) [to be clarified]

   \ttitem{FILE} if FILE is omitted, no output is written to file.\\
   if FILE is present, track tables are printed, optionally to 
   files with name constructed from the base filename specified. \\
   The actual name of the output file is constructed from
   tyhe baseline given with {\tt FILE} to which are appended the
   strings ".obsnnnn" (where nnnn is the observation point index) and
   ".pnnnn" (where nnnn is now the particle number), unless the
   \hyperref[opt:onetable2]{\tt ONETABLE} option is activated.  \\
   (Default: "track") 

   \ttitem{EXTENSION} a string providing the filename extension for the
   track table files, e.g., txt, doc...  \\ (Default: nil)

   \ttitem{ROOTNTUPLE} a logical switch to store data to ROOT file as
   ntuple. Accessible only if \hyperref[sec:rplot]{\tt RPLOT} plugin is
   available. i.e. only if \madx is dynamically linked and
   \hyperref[sec:rplot]{\tt RPLOT} plugin is present. \\ 
   (Default: false)

   \ttitem{EVERYSTEP} a logical switch to activate the recording of
   track parameters at every integration step. Normally tracking data
   are stored internally only at the end of each element. {\tt EVERYSTEP}
   provides the user with finer data points. It implies usage of the so
   called node (thin) layout. \\ 
   Track parameters are stored for each step in file
   "thintracking\_ptc.txt". Storage of parameters in a table for each
   step might be very memory consuming. To switch it off use option
   {\tt TABLEALLSTEPS}.\\  
   Collective effects can be taken into account only using this mode
   (this feature of PTC is not interfaced into MAD-X). \\
   (Default: false)

   \ttitem{TABLEALLSTEPS} (to be completed) \\ (Default: false)
   
   \ttitem{GCS} a logical switch to store track parameters in Global
   Coordinate System - normally it starts at the entrance face of the
   first element. \\ (Default: false)
\end{madlist}


%% Depending on value of {\tt ONETABLE} switch, all output information
%% is stored in one table (and also file), or in one table per track per
%% observation point is written if the switch is false. 

Plotting of track parameters (see \hyperref[sec:plot]{\tt PLOT}
command) is only possible if {\tt ONETABLE} switch is set to false (status 
as for Feb. 2006). This unfortunate solution is the legacy of the
regular \madx \hyperref[chap:thintrack]{\tt TRACK} command, 
designed for circular machines where the user usually tracks a few
particles for many turns rather then many particles for one turn each.

Tracks that do not fit in the defined aperture for elements are
immediately stopped.  

Behavior of PTC calculations can be adapted with
\hyperref[sec:ptc_setswitch]{\tt PTC\_SETSWITCH} command 
and with appropriate switches of
\hyperref[sec:ptc_create_layout]{\tt PTC\_CREATE\_LAYOUT} command.  


%% {\bf PROGRAMMER'S MANUAL}\\
%% The routine PTC\_TRACKLINE is implemented in file
%% madx\_ptc\_trackcavs.f90\\
%% Its single parameter is the number of observation points.  

%% The call sequence from MAD-X interpreter is the following 
%% \\ exec\_command in madxp.c; 
%% \\ pro\_ptc\_trackline in madxn.c; This routine creates appropriates
%% tables where the track parameters are stored, and after execution of the
%% Fortran routine dumps filled table(s) to files. 
%% \\ w\_ptc\_trackline\_ in wrap.f90; Just interface to the appropriate
%% Fortran module  
%% \\ ptc\_trackline in madx\_ptc\_trackline.f90 

%% The key routine that enables appropriate calculation of beam and track
%% parameters in the presence of traveling wave cavities is setcavities.  

%% Firstly, the ptc\_trackline routine finds out which are the observation
%% points. For this purpose array of integers observedelements is
%% allocated. Its length is equal to the number of elements in the
%% sequence. All elements are zero by default. If an element with an index
%% n is an observation point then observedelements[n] is equal to 1. This
%% solution enables fast checking if track parameters should be sent to a
%% table after a given element.  

%% Further \href{../ptc_auxiliaries/PTC_SetCavities.html}{setcavities
%%   subroutine} is called if it was not executed yet before.   

%% PTC\_TRACKLINE reads the track initial parameters from the table with
%% the help of gettrack function (implemented in C in file madxn.c). For
%% the performance reasons gettrack creates a two dimensional array and
%% buffers there all the initial track parameters upon first call. The
%% array is destroyed with a call of deletetrackstrarpositions function
%% that is performed at the very end of ptc\_trackline subroutine.  

%% Tracking itself is implemented in a doubly nested loop.  The external
%% one goes over all initiated tracks,  and the internal one performs
%% tracking of a given track element by element. The key PTC routine is
%% called TRACK. It propagates a track described by an array of 6 real
%% numbers, denoted as X in equations below. The important issue is that
%% they are the canonical variables. In order to follow the standard MAD-X
%% representation the values that are written to tables and files are
%% scaled appropriately to the reference momentum for a given element. In
%% the general case it changes along the line if traveling wave cavities
%% are present. Hence, momenta xp, yp and zp are  

%% \begin{verbatim}
%% zp=sqrt((1+x(5))**2 - x(2)**2 - x(4)**2) 
     
%% xp = x(2)/zp
     
%% yp = x(4)/zp
%% \end{verbatim}  

%% where array x containing 6 elements is the track position in the PTC
%% representation, i.e. x(1) is horizontal spatial coordinate, x(2) -
%% horizontal momentum, x(3) - vertical spatial coordinate, x(4) - vertical
%% momentum, x(5) - $\delta$p/p$_0$c, x(6) - longitudinal coordinate
%% (caution, the exact meaning depends on the PTC settings, see
%% \href{../ptc_auxiliaries/PTC_SetSwitch.html}{ PTC\_SETSWITCH command }).   





\section{PTC\_TRACK\_END}
\label{sec:ptc_track_end}

The {\tt PTC\_TRACK\_END} command terminates the command lines related
to the {\tt PTC\_TRACK} module. 

\madbox{
PTC\_TRACK\_END;
}

The initial and final canonical coordinates are collected in the
internal table "tracksumm", which can be \hyperref[sec:write]{written}
to file.


\section{Choice of options}
\label{sec:choice_options}

The following table facilitates the choice of the correct options for a
number of typical tasks: 

\begin{enumerate} 
   \item The tracking of a beam-line with default parameters.
   \item Smilar to "1." but with element-by-element tracking and an output 
     at observation points. 
   \item Tracking in a closed ring with closed orbit search and the 
     Normal Forms calculations.\\ 
     Both canonical and action-angle input/output coordinates are
     possible.\\
     Output at observation points is produced via PTC maps. 
   \item Similar to "3." except that output at observation points is 
     created by element-by-element tracking.
   \item The ??? with PTC radiation.
\end{enumerate}
    

\begin{tabular}{|lccccc|}
\hline 
{\bf Option} & {\bf case 1} & {\bf case 2} & {\bf case 3} & {\bf case 4} & {\bf case 5 } \\ 
\hline
{\tt CLOSED\_ORBIT} & - & - & + & + & + \\ 
\hline
{\tt ELEMENT\_BY\_ELEMENT} & - & + & - & + & - \\ 
\hline
{\tt PTC\_START, X, PX, ...} & + & + & + & + & + \\ 
\hline
{\tt PTC\_START, FX, PHIX, ...} & -  & - & + & + & + \\ 
\hline
{\tt NORM\_NO} & - & - & $>$1 & $>$1 & $>$1 \\ 
\hline
{\tt NORM\_OUT} & - & - & + & - & + \\ 
\hline
{\tt PTC\_OBSERVE} & - & + & + & + & - \\ 
\hline
{\tt RADIATION} & - & - & - & - & + \\ 
\hline
{\tt RADIATION\_MODEL1} & - & - & - & - & - \\ 
\hline
{\tt RADIATION\_ENERGY\_LOSS} & - & - & - & - & - \\ 
\hline
{\tt RADIATION\_QUAD} & - & - & - & - & +/- \\ 
\hline
{\tt BEAM\_ENVELOPE} & - & - & - & - & - \\ 
\hline
{\tt SPACE\_CHARGE} & - & - & - & - & - \\ 
\hline
\end{tabular}



%\href{mailto:kapin@itep.ru}{V.Kapin}(ITEP) and 
%\href{mailto:Frank.Schmidt@cern.ch}{F.Schmidt}, July 2005; revised in April, 2006

