%%\title{Thick-Lens Tracking Module (PTC-TRACK Module)}
%  Created by: Valery KAPIN, 06-Apr-2006 
%  Changed by: ____________, ___________ 


\chapter{Thick-Lens Tracking Module (PTC-TRACK)}

The \textbf{PTC-TRACK module }[\hyperlink{V._Kapin}{a}] is the
symplectic thick-lens tracking facility in MAD-X
[\hyperlink{F._Schmidt}{b}]. It is based on PTC library written by
E.Forest [\hyperlink{E._Forest}{c}]. The commands of this module are
described below, optional parameters are denoted by square brackets ([
]). Prior to using this module the active beam line must be selected by
means of a \href{../control/general.html#use}{USE} command.  
The general \href{../ptc_general/ptc_general.html}{PTC environment} must
also be initialized.  

\section{Synopsis}

A typical tracking job in PTC requires a number of commands to be issued:

\begin{verbatim}
PTC_CREATE_UNIVERSE;
PTC_CREATE_LAYOUT, model=integer,method=integer, 
                   nst=integer, [exact];
...
PTC_START, ... ;
...
PTC_OBSERVE, ... ;
...
PTC_TRACK, ... ;
...
PTC_TRACK_END;
...
PTC_END;
\end{verbatim}

\section{PTC\_START}

\begin{verbatim}
PTC_START,  
  x=double, px=double, y=double, py=double, t=double, pt=double,
  fx=double, phix=double, fy=double, phiy=double, ft=double, phit=double; 
\end{verbatim}

To start particle tracking, a series of initial trajectory coordinates
has to be given by means of \texttt{PTC\_START} command (as many
commands as trajectories). It must be done before the
\texttt{\hyperlink{PTC_TRACK}{PTC\_TRACK}}command. 

The coordinates can be either
\href{../Introduction/tables.html#canon}{canonical  coordinates}
(\textbf{x}, \textbf{px}, \textbf{y}, \textbf{py},
\textbf{t}, \textbf{pt}) 
or action-angle coordinates (\textbf{fx}, \textbf{phix}, 
\textbf{fy}, \textbf{phiy}, \textbf{ft}, \textbf{phit}), which 
are expressed by the normalized amplitude, \textit{F}$_\textit{z }$ and
the phase,  \textit{$\Phi$}$_\textit{z}$ for the \textit{z}-th mode
plane  (\textit{z}=\{\textit{x},\textit{y},\textit{t}\}). 
The actions are computed with the values of the emittances,
\textit{F}$_\textit{z}$, which must be specified in the preceding 
\href{../Introduction/beam.html}{BEAM} command. \textit{F}$_\textit{z}$
are expressed in number of r.m.s. beam sizes and
\textit{$\Phi$}$_\textit{z}$  are expressed in radians.


{\bf Options} \\ 
\begin{itemize}
   \item {\bf X, PX, Y, PY, T, PT}=doubles (Default: 0.0) \\
     canonical coordinates 
   \item {\bf FX, PHIX, FY, PHIY, FT, PHIT}=doubles (Default: 0.0) \\
     action-angle coordinates
\end{itemize}

{\bf Remarks} \\
\begin{enumerate}

   \item If the option
     \texttt{\hyperlink{CLOSED_ORBIT}{closed\_orbit}}in the
     \texttt{\hyperlink{PTC_TRACK}{PTC\_TRACK}} command is active (see
     below), all coordinates are specified with respect to the actual
     closed orbit (possibly off-momentum with magnet errors) and NOT
     with respect to the reference orbit. If the option
     \texttt{\hyperlink{CLOSED_ORBIT}{closed\_orbit}} is absent, then
     coordinates are specified with respect to the reference orbit. 

   \item In the uncoupled case, the canonical and the action-angle
     variables are related with equations 
     \( z = F_z (E_z)^{1/2} cos(\Phi_z),  p_z = F_z (E_z)^{1/2} sin(\Phi_z)\).

   \item The use of the action-angle coordinates requires the option
     \texttt{\hyperlink{CLOSED_ORBIT}{closed\_orbit}}in 
     the \texttt{\hyperlink{PTC_TRACK}{PTC\_TRACK}} command. 

   \item If both the canonical and the action-angle coordinates are 
     given in the \texttt{PTC\_START }command, they 
     are summed after conversion of the action-angle coordinates to 
     the canonical ones.
\end{enumerate}


\section{PTC\_OBSERVE} 

\begin{verbatim}
PTC_OBSERVE,  place=string; 
\end{verbatim}


Besides the beginning of the beam-line, one can define an additional
observation points along the machine. Subsequent \texttt{PTC\_TRACK}
command will then record the tracking data on all these observation
points.  

{\bf Options}\\
\begin{itemize}
   \item {\bf PLACE}=string (Default: NULL) \\
     name of observation point. Specifying markers as observation points
     is very much preferred.
\end{itemize}

{\bf Remarks}\\
\begin{enumerate}
   \item The first observation point at the beginning of the beam-line
     is marked as \textbf{"start"}.  
       
   \item It is recommended to use
     \href{../Introduction/label.html}{labels} of
     \href{../Introduction/marker.html}{markers} in order to avoid usage
     observations at the ends of thick elements. 

   \item The data at the observation points other than at
     \textbf{"start"} can be produced by two different means: 
     \begin{enumerate}
        \item traditional (\href{../thintrack/thintrack.html}{MADX})
          element-by-element tracking.\\
          (use option \hyperlink{ELEMENT_BY_ELEMENT}{element\_by\_element}); 
       
        \item coordinate transformation from \textbf{"start"} to the
          respective observation point using high-order PTC transfer
          maps. \\
          (required option \texttt{\hyperlink{CLOSED_ORBIT}{closed\_orbit}}; 
          turned off options \hyperlink{RADIATION}{radiation} 
          and \hyperlink{ELEMENT_BY_ELEMENT}{element\_by\_element}).
     \end{enumerate} 
\end{enumerate}

\section{PTC\_TRACK}

\begin{verbatim}
PTC_TRACK,  deltap=double, icase=integer, closed_orbit, 
            element_by_element, turns=integer, 
            dump, onetable, maxaper=double array, norm=integer, norm_out, 
            file[=string], extension=string, ffile=integer, 
            radiation, radiation_model1, radiation_energy_loss, 
            radiation_quadr, beam_envelope, space_charge;
\end{verbatim}

The \texttt{PTC\_TRACK} command initiates trajectory tracking by
entering the thick-lens tracking module.  
Several options can be specified, the most important are presented under
"Basic Options". 
There are also switches to use special modules for particular
tasks. They are presented under "Special Switches" below.
The tracking can be done element-by-element using the option
\hyperlink{ELEMENT_BY_ELEMENT}{element-by-element}, or "turn-by-turn"
(default) with coordinate transformations over the whole turn. 
Tracking is done in parallel, i.e. the coordinates of all particles are
transformed through each beam element (option
\hyperlink{ELEMENT_BY_ELEMENT}{element-by-element}) or over full turns. 
The particle is lost if its trajectory is outside the boundaries as
specified by \hyperlink{MAXAPER}{maxaper} option. 
In PTC, there is a continuous check, if the particle trajectories stays
within the aperture limits.  

The Normal Form calculations (required option
\hyperlink{CLOSED_ORBIT}{closed\_orbit}) is controlled by
\hyperlink{NORM_NO}{norm\_no} and \hyperlink{NORM_OUT}{norm\_out} are
used.

{\bf Basic Options}\\
\begin{itemize}
   \item {\bf ICASE}=integer (Default: 4)\\ 
     user-defined dimensionality of the phase-space (4, 5 or 6). 

   \item {\bf DELTAP}=double (Default: 0.0)\\
     relative momentum offset for reference closed orbit (used for 5D
     case ONLY). 

   \item {\bf CLOSED\_ORBIT}=logical (Default: .false.)\\
     switch to turn on the closed orbit calculation
     
   \item {\bf ELEMENT\_BY\_ELEMENT}=logical (Default: .false.)\\
     switch from the default turn-by-turn tracking to the
     element-by-element tracking. 

   \item {\bf TURNS}=integer (Default: 1)\\
     number of turns to be tracked.

   \item {\bf DUMP}=logical (Default: .false.)\\ 
     enforces writing of particle coordinates to formatted text files.

   \item {\bf ONETABLE}=logical (Default: .false.)\\ 
     writing all particle coordinates to a single file.

   \item {\bf MAXAPER}=array(1:6) (Default: \{0.1, 0.01, 0.1, 0.01, 1.0, 0.1\})\\
     upper limits for the particle coordinates, essentially defining the
     aperture to define particle loss. 
     
   \item {\bf NORM\_NO}=integer (Default: 1)\\ 
     order of the Normal Form. 

   \item {\bf NORM\_OUT}=logical (Default: .false.) \\
     switch to transform canonical variables to action-angle variables.

   \item {\bf FILE}=string (Default: track)\\ 
     if FILE is omitted, no output is written to file.\\
     if FILE is present, track tables are printed, optionally to the file specified.

   \item {\bf EXTENSION}=string (Default: NULL)\\
     the filename extension for the track table files, e.g., txt, doc etc 

   \item {\bf FFILE}=integer (Default: 1)\\
     prints coordinates every n turns, where n is the integer specified.
\end{itemize}


{\bf Remarks}\\
\begin{enumerate}
   \item {\bf ICASE}:has a highest priority over other options: 
     \begin{enumerate}
        \item RF cavity with non-zero voltage will be ignored for
          \hyperlink{ICASE}{icase}=4, 5;
        \item A non-zero \hyperlink{DELTAP}{deltap} will be ignored 
          for \hyperlink{ICASE}{icase}=4, 6.
     \end{enumerate}
     However, if RF cavity has the voltage set to zero and 
     for \hyperlink{ICASE}{icase}=6, the code sets
     \hyperlink{ICASE}{icase}=4.

   \item {\bf DELTAP}: is ignored for \hyperlink{ICASE}{icase}=6, but
     the option \href{../ptc_general/ptc_general.html}{offset\_deltap}
     of the command \texttt{PTC\_CREATE\_LAYOUT} may be used, if the
     reference particle should have an momentum off-set as specified by
     \href{../ptc_general/ptc_general.html}{offset\_deltap}. 
 
   \item {\bf CLOSED\_ORBIT}: must be used for closed rings only. This
     option allows to switch ON the Normal Form analysis, if
     required. If CLOSED\_ORBIT is off, the sequence is treated as a
     transfer line. 

   \item {\bf NORM\_NO=1}: makes the Normal Form linear (always true for
     MAD8 or MAD-X). 

   \item {\bf FILE}: The output file endings are: .obsnnnn(observation
     point), followed by .pnnnn (particle number), if the
     \hyperlink{ONETABLE}{onetable} option is not used.  
\end{enumerate}

{\bf Special Switches}\\
\begin{itemize}
   \item {\bf RADIATION}=logical (Default: .false)\\ 
     turn on the synchrotron radiation calculated by an internal procedure of PTC.

   \item {\bf RADIATION\_MODEL1}=logical (Default: .false.)\\
     switch to turn on the radiation according to the method given in
     Ref.~[\hyperlink{G.J._Roy}{d}] 

   \item {\bf RADIATION\_ENERGY\_LOSS}=logical (Default: .false.)\\
     adds the energy loss for \hyperlink{RADIATION_MODEL1}{radiation\_model1}.

   \item {\bf RADIATION\_QUADR}=logical (Default: .false.)\\
     adds the radiation in quadrupoles. It supplements
     either\hyperlink{RADIATION}{radiation},
     \hyperlink{RADIATION_MODEL1}{radiation\_model1}.

   \item {\bf BEAM\_ENVELOPE}=logical (Default: .false.)\\
     turn on the calculations of the beam envelope with PTC.

   \item {\bf SPACE\_CHARGE}=logical (Default: .false.)  {\bf [under
       construction]}\\
     turn on the simulations of the space charge forces between particles. 
\end{itemize}

{\bf Remarks}\\
\begin{enumerate}
   \item{\bf RADIATION}: Has precedence over 
     \hyperlink{RADIATION_MODEL1}{radiation model1.} 

   \item{\bf RADIATION\_MODEL1}: Additional 
     module by F. Zimmermann. The model simulates 
     quantum excitation via a random number generator and tables for 
     photon emission. It can be used only with the element-by-element 
     tracking (option \hyperlink{ELEMENT_BY_ELEMENT}{element-by-element}).
     
   \item {\bf RADIATION\_ENERGY\_LOSS}: Of use for
     \hyperlink{RADIATION_MODEL1}{radiation\_model1}.

   \item {\bf BEAM\_ENVELOPE:} It requires the options
     \hyperlink{RADIATION}{radiation} and \hyperlink{ICASE}{icase}=6.
     
   \item {\bf SPACE\_CHARGE:} This option 
     is under construction and is reserved for future use.
\end{enumerate}



\section{PTC\_TRACK\_END}

\begin{verbatim}
PTC_TRACK_END;
\end{verbatim}

The \texttt{PTC\_TRACK\_END }command terminate the command lines related
to the PTC\_TRACK module. 

The initial and final canonical coordinates are collected in the
internal table "tracksumm" (printed to the file with
\href{../control/general.html#write}{WRITE} command). 


{\bf Examples}\\
 Several examples can be found on the web at
 \href{http://cern.ch/frs/mad-X_examples/ptc_track}{http:\/\/cern.ch\/frs\/mad-X\_examples\/ptc\_track}. 

{\bf Typical tasks}\\ 
The following table facilitates the choice of the correct options for a
number of typical tasks: 
\begin{enumerate} %[1]
   \item The tracking of a beam-line with default parameters.
   \item Smilar to "1." but with element-by-element tracking and an output 
     at observation points. 
   \item Tracking in a closed ring with closed orbit search and the 
     Normal Forms calculations.\\ 
     Both canonical and action-angle input/output coordinates are
     possible.\\
     Output at observation points is produced via PTC maps. 
   \item Similar to "3." except that output at observation points is 
     created by element-by-element tracking.
   \item The ??? with PTC radiation.
\end{enumerate}
    

\begin{tabular}{cccccc}
\hline 
\textbf{Option} & \textbf{1} & \textbf{2} & \textbf{3} & \textbf{4} & \textbf{5 } \\ 
\hline
CLOSED\_ORBIT & - & - & + & + & + \\ 
\hline
ELEMENT\_BY\_ELEMENT & - & + & - & + & - \\ 
\hline
PTC\_START, X, PX, ... & + & + & + & + & + \\ 
\hline
PTC\_START, FX, PHIX, $\backslash$85 & -  & - & + & + & + \\ 
\hline
NORM\_NO & - & - & $>$1 & $>$1 & $>$1 \\ 
\hline
NORM\_OUT & - & - & + & - & + \\ 
\hline
PTC\_OBSERVE & - & + & + & + & - \\ 
\hline
RADIATION & - & - & - & - & + \\ 
\hline
RADIATION\_MODEL1 & - & - & - & - & - \\ 
\hline
RADIATION\_ENERGY\_LOSS & - & - & - & - & - \\ 
\hline
RADIATION\_QUAD & - & - & - & - & +/- \\ 
\hline
BEAM\_ENVELOPE & - & - & - & - & - \\ 
\hline
SPACE\_CHARGE & - & - & - & - & - \\ 
\hline
\end{tabular}



\section{References}
\begin{enumerate} %[a] for style change
   \item \href{V._Kapin}{V. Kapin} and F. Schmidt, $\backslash$93PTC
     modules for MAD-X code$\backslash$94, to be published as CERN
     internal note by the end of 2006 
   \item \href{F._Schmidt}{F. Schmidt},
     "`\href{http://cern.ch/madx/doc/MPPE012.pdf}{MAD-X PTC
     Integration}'', Proc. of the 2005 PAC Conference in Knoxville, USA,
     pp.1272.  
   \item \href{E._Forest}{E. Forest}, F. Schmidt and E. McIntosh,
     ``Introduction to the Polymorphic Tracking Code'', KEK report
     2002-3, July 2002 
   \item \href{G.J._Roy}{G.J. Roy}, ``A new method for the simulation of 
   synchrotron radiation in particle tracking codes'', Nuclear
   Instruments \& Methods in Phys. Res., Vol. A298, 1990, pp. 128-133. 
\end{enumerate}


{\bf See Also}: 
\href{../tracking/tracking.html}{Overview of MAD-X Tracking Modules}, 
\href{../ptc_general/ptc_general.html}{PTC Set-up Parameters}, 
\href{../thintrack/thintrack.html}{thintrack},   
\href{http://cern.ch/frs/mad-X_examples/ptc_track}{PTC-TRACK Examples}. 


%\href{mailto:kapin@itep.ru}{V.Kapin}(ITEP) and 
%\href{mailto:Frank.Schmidt@cern.ch}{F.Schmidt}, July 2005; revised in April, 2006

