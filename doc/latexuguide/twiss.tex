%%\title{Twiss Module}
%  Changed by: Chris ISELIN, 27-Jan-1997 
%  Changed by: Frank Schmidt, 11-Jul-2002 
%  Changed by: Hans Grote, 15-Jan-2003 
%  Changed by: Frank Schmidt, 06-APR-2003 
%  Changed by: ghislain 18-JUN-2014   

%  Last changed: Irina Tecker, 12-OCT-2016 

\chapter{Twiss Module}
\label{chap:twiss}

The \texttt{TWISS} command calculates the
\hyperref[subsec:tables-linear]{linear lattice
  functions} \cite{Courant-Snyder1958}, and optionally the
\hyperref[subsec:tables-chrom]{chromatic functions}. 
The coupled functions are calculated in the sense of
Edwards and Teng\cite{edwards1973}.
For the uncoupled cases they reduce to the C and S functions. 

\madbox{
TWISS, \=SEQUENCE=seqname, LINE=linename, RANGE=range,  \\
       \>DELTAP=real\{,real\}||initial:final:step,\\
       \>CHROM=logical,\\
       \>CENTRE=logical, TOLERANCE=real, \\
       \>FILE=filename, \\
       \>TABLE=tabname, NOTABLE=logical,\\
       \>RMATRIX=logical, SECTORMAP=logical, \\
       \>SECTORTABLE=tabname, SECTORFILE=filename,\\
       \>KEEPORBIT=name, USEORBIT=name,\\
       \>COUPLE=logical,\\
       \>RIPKEN=logical;
}

The  attributes of the \texttt{TWISS} command are: 
\begin{madlist}
  \ttitem{SEQUENCE} the name of a valid sequence for
  which the calculation of optical functions should be performed. \\ 
  \texttt{SEQUENCE} and \texttt{LINE} are mutually exclusive.\\
  (Default: sequence or beam line defined in the latest \texttt{USE}
  command) 

  \ttitem{LINE} the name of a valid beamline for which
  the calculation of optical functions should be performed. \\
  \texttt{SEQUENCE} and \texttt{LINE} are mutually exclusive.\\
  (Default: sequence or beam line defined in the latest \texttt{USE}
  command) 
  
  \ttitem{RANGE} (Default: \texttt{\#S/\#E})\\
  The \texttt{TWISS} calculation is restricted to the specified range.
  See \hyperref[sec:range]{\texttt{RANGE}}.  

  \ttitem{DELTAP}=\texttt{real\{,real\}} or \texttt{initial:final:step}
  (Default: 0.0)\\ 
  The relative energy error \texttt{DELTAP} may be entered in one of the
  two forms above. \\ 
  The first form lists several numbers, which may be general expressions,
  separated by commas. The second form specifies an initial value, a final
  value, and a step, which must be constant expressions, separated by
  colons. \\
  For example, \texttt{DELTAP=0.001} defines a single value, 
  \texttt{DELTAP=0.001,0.005} defines two values and 
  \texttt{DELTAP=0.001:0.007:0.002} defines four values. 


  \ttitem{CHROM} a logical flag to trigger computation of the
  \hyperref[subsec:tables-chrom]{chromatic functions} as well as the radiation 
  synchrotron integrals. \\
   %irina
  \textit{Please note that the calculation is done without taking coupling into 
  account. In case of coupled lattice the warning isissued that the calculation 
  of chromatic functions can be inaccurate!}\\
  \textit{Please note that this option also changes the way that the
    chromaticities are calculated: The chromaticities are normally
    calculated from the analysis of the first and second order
    matrices. With \texttt{CHROM}, the chromaticities are recalculated by
    explicitely calculating the tunes for the case of the specified momentum
    deviation \texttt{DELTAP} and for the case of a momentum deviation equal
    to \texttt{DELTAP}+1.e-6. The tune differences divided by 1.e-6 yield the
    chromaticities.}

  \ttitem{CENTRE} a logical flag to enforce the calculation of the
  \hyperref[subsec:tables-linear]{linear lattice functions} at the
  center of the element instead of the end of the element. The values in
  the tables and in the output files are affected by this
  flag. \\ (Default: false) \\ 
  \textit{ Since the lattice functions are calculated inside the element
    the closed orbit coordinates in the output also include the misalignment
    of the element.}  

  \ttitem{TOLERANCE} the maximum closed orbit error, for all six orbit 
  components, that can be tolerated during the closed orbit search. 
  The value given in the \texttt{TWISS} command is only valid for the current 
  calculation; 
  the \hyperref[sec:coguess]{\texttt{COGUESS}} command allows to 
  change the default value for all subsequent closed orbit search calculations. 
  \\ (Default: 1.e-6)

  \ttitem{FILE} causes \madx to write a \hyperref[chap:tfs]{TFS Twiss table} 
  to the file specified. (Default: ``twiss'') \\
  The columns of the table can be selected using the \texttt{SELECT}
  command with the \texttt{FLAG=twiss} attribute. 

  \ttitem{TABLE} the name of the table where 
  \hyperref[subsec:tables-linear]{linear lattice functions} as well 
  as \hyperref[subsec:tables-chrom]{chromatic functions} are stored. 
  (Default: ``twiss'') \\
%  \madx creates a full
%  \hyperref[subsec:tables-linear]{Twiss table} in
%  memory and gives it the name \texttt{TWISS}, unless \texttt{TABLE=tabname}
%  is given, then it is called "tabname". This table
%  includes linear lattice functions as well as the chromatic
%  functions for all positions.\\
  \textbf{Note:} If the \texttt{TABLE} option is given the selection of column 
  names via the \texttt{SELECT} command is ignored. Hence if both \texttt{TABLE} and
  \texttt{FILE} options are given, the table written to file is the full
  twiss table, containing all elements as rows and all known Twiss 
  parameters as columns. 

  \ttitem{NOTABLE} logical flag to prevent the creation of the internal twiss
  table. Consequently, no output file is created either.  \\ 
  (Default: false)
  

  \ttitem{RMATRIX} If this flag is used the the one-turn map at the
  location of every element is calculated and prepared for
  storage in the twiss table. Using the
  \hyperref[sec:select]{\texttt{SELECT}} command and using
  the column \texttt{RE, RE11 \ldots RE16 \ldots RE61 \ldots RE66} these
  components will be added to the twiss table, i.e. with \texttt{"COLUMN, RE"} and
  \texttt{"COLUMN, REij"} one gets all or the component "ij" respectively.    

  \ttitem{SECTORMAP} a logical flag to initiate the calculation of a 
  \hyperref[sec:sectormap]{sector map}. Default the Rij contains feed-down from higher order maps. In order to turn it off use the flag SECTORPURE. 
  
  \ttitem{SECTORACC} a logical flag to save composition of maps instead of individual maps of a
  \hyperref[sec:sectormap]{sector map}.

  \item{SECTORPURE} a logical flag to save the transfer map Rij without effects from higher order map (Tijk). This option should be used to get it in the correct format for TRAIN.

  \ttitem{SECTORTABLE} the name of the table containing the \texttt{SECTORMAP}
  values. The elements (lines) and parameters (columns) 
  of the table can be tailored using the \hyperref[sec:select]{\texttt{SELECT}} 
  command as specified in \hyperref[sec:sectormap]{\texttt{SECTORMAP}} \\
  (Default: sectortable)

  \ttitem{SECTORFILE} the name of the file to which the \texttt{SECTORMAP} is
  written. 
  The format of the output can be adjusted as specified in 
  \hyperref[sec:sectormap]{\texttt{SECTORMAP}} \\
  (Default: "sectormap")
  %% Used to write SECTORMAPs to the file
  %% SECTORFILE="filename", if missing the output of SECTORMAP
  %% will go to the file "sectormap" with the format as found in
  %% \href{../Introduction/sectormap.html}{SECTORMAP}.    
  
  \ttitem{KEEPORBIT} The keeporbit attribute (with an optional name,
  keeporbit="name") stores the orbit under this name at the
  start, and at all monitors.    

  \ttitem{USEORBIT} The useorbit attribute (with an optional name,
  useorbit="name") uses the start value provided for the closed
  orbit search; the values at the monitors are used by the
  threader.    

  \ttitem{COUPLE} (obsolete) This \madeight option can no
  longer be set since TWISS in \madx is always calculated in
  coupled mode. \madx computes the coupled functions in the
  sense of Edwards and Teng \cite{edwards1973}. 
  For the uncoupled cases they reduce to the C and S functions.    
  
  \textit{ Twiss calculation is 4D only!} : The \texttt{TWISS}
  command calculates an approximate 6D closed orbit when the
  accelerator structure includes an active
  \hyperref[sec:rfcavity]{cavity}. However, the
  calculation of the Twiss parameters are 4D only. This may
  result in apparently non-closure of the beta values in the
  plane with non-zero dispersion. The full 6D Twiss parameters
  can be calculated with the
  \hyperref[chap:ptc-twiss]{PTC\_TWISS} command.    

  \ttitem{RIPKEN} This flag calculates the Ripken-Mais Twiss
  parameters (\texttt{beta11, beta12, beta21, beta22, alfa11, alfa12,
  alfa21, alfa22, gamma11, gamma12, gamma21} and \texttt{gamma22}) using
  the parameters \texttt{betx, bety, alfx, alfy, \\ gamax, gamay, R11, R12,
    R21} and \texttt{R22} as input.  

\end{madlist}

The tables are suitable for \hyperref[chap:plot]{\texttt{PLOT}}.

After a successful \texttt{TWISS} run \madx creates a 
table of summary parameters named "SUMM" which includes tunes,
chromaticities, etc. versus the selected values of \texttt{DELTAP}.
Please note that with the \texttt{CHROM} attribute the calculation is done without
taking coupling into account. In case of coupled lattice the warning isissued that
the calculation of chromatic functions can be inaccurate!

Notice also that in \madx, \texttt{DELTAP} is converted to \texttt{PT},
which is used as longitudinal variable. 
Dispersive and chromatic functions are hence derivatives with
respect to \texttt{PT}. (see \hyperref[subsec:tables-summ]{summ table}). 

These summary parameters can later be accessed via the 
\hyperref[chap:tables]{table access functions} using the "SUMM" table.  

\section{Twiss Parameters for a Period}
\label{sec:twissperiod}

The simplest form of the \texttt{TWISS} command is
\madbox{
TWISS;
}
which calculates the periodic solution for the last beamline or sequence
declared in a \texttt{USE} statement, and with zero \texttt{DELTAP}.
Chromatic functions are not calculated. 
Standard tables ("TWISS" and "SUMM") are created in memory but no file
is written to disk. 

The slightly more elaborate version 
\madbox{
TWISS, DELTAP=real\{,real\}, CHROM, TABLE=tabname;
}
computes the periodic solution, including chromatic functions, for the last beam
line or sequence declared in a \texttt{USE} statement, for all values of
DELTAP entered (or for \texttt{DELTAP=0}, if none is entered). 
The tables "tabname" and "SUMM" are created in memory and no file is
written to disk. 

\textbf{Example:} 
\madxmp{
USE, period=OCT; \\
TWISS, DELTAP=0.001, CHROM;
}
computes the periodic solution for the linear lattice and
chromatic functions for the beam line OCT and for DELTAP=0.001. 

%% Apart from saving computing time, it is equivalent to the command
%% sequence  
%% \begin{verbatim}
%% RING: LINE=(4*(OCT,-OCT));
%%       USE,period=RING;
%%       TWISS,DELTAP=0.001,CHROM;
%% \end{verbatim}

\section{Initial Values from a Periodic Line}
\label{sec:twissinitial}

It is possible to track the lattice functions starting with the periodic
solution for another beam line. If this is desired the TWISS command
takes the form  
\madbox{
TWISS, \=DELTAP=real\{,real\}, LINE=beamline, \\
       \>MUX=real, MUY=real, \\
       \>TABLE=tabname;
}
No other attributes should appear in the command. For each value of
DELTAP, \madx first searches for the periodic solution for the beam line
mentioned in LINE=beamline: The result is used as an initial condition
for the lattice function tracking. 

\textbf{Example:} 
\madxmp{
xxxxxxxx\= \kill
CELL:   \>LINE=(...); \\
INSERT: \>LINE=(...); \\
USE, period=INSERT; \\
TWISS, LINE=CELL, DELTAP=0.0:0.003:0.001, CHROM, FILE;
}

For four values of DELTAP the following happens: First \madx finds the
periodic solution for the beam line CELL: Then it uses this solution as
initial conditions for tracking the lattice functions of the beamline
CELL: Output is also written on the file TWISS:  

If any of the beam lines was defined with formal arguments, actual
arguments must be provided:  
\madxmp{
CELL(SF,SD): \=LINE=(...); \\
INSERT(X):   \>LINE=(...); \\
USE, period=INSERT; \\
TWISS, LINE=CELL(SF1,SD1);
}

\section{Given Initial Values}

Initial values for \hyperref[subsec:tables-linear]{linear
lattice functions} and \hyperref[subsec:tables-chrom]{chromatic
  functions} may also be numerical.
Initial values can be specified on the \texttt{TWISS} command:  
\madbox{
TWISS,   \=BETX=real, ALFX=real, MUX=real,\\
         \>BETY=real, ALFY=real, MUY=real,\\
         \>DX=real, DPX=real, DY=real, DPY=real,\\
         \>X=real, PX=real, Y=real, PY=real,\\
         \>T=real, PT=real,\\
         \>WX=real, PHIX=real, DMUX=real,\\
         \>WY=real, PHIY=real, DMUY=real,\\
         \>DDX=real, DDY=real, DDPX=real, DDPY=real,\\
         \>R11=real, R12=real, R21=real, R22=real,  !coupling matrix\\
         \>TABLE=tabname,\\
         \>TOLERANCE=real,\\
         \>DELTAP=real:real:real;
}

All initial values for
\hyperref[subsec:tables-linear]{linear lattice functions} and
\hyperref[subsec:tables-chrom]{chromatic functions} are
permitted, but \texttt{BETX} and \texttt{BETY} are required.
Moreover, a \hyperref[sec:beta0]{BETA0} block can be added as filled by the
\hyperref[sec:savebeta]{\texttt{SAVEBETA}} command (see below).
The lattice parameters are taken from this block, but are also
overwritten by lattice parameters explicitly decalred on the command 
line. As entered, the initial conditions cannot depend on
\texttt{DELTAP},
and can thus be correct only for one such value.  

\section{SAVEBETA}
\label{sec:savebeta}
Initial lattice parameters can be saved and transfered for later commands, in
particular for \hyperref[chap:twiss]{\texttt{TWISS}} or the
\hyperref[chap:match]{match module}, with the \texttt{SAVEBETA} command
sequence.   

\madbox{
SAVEBETA, LABEL=string, PLACE=string, SEQUENCE=sequencename;
}
marks a location given by attribute \texttt{PLACE} in an expanded
sequence \texttt{sequence\_name}; at the next \texttt{TWISS} command
execution, a \hyperref[sec:beta0]{\texttt{BETA0}} block is saved at that
location with the label given by the attribute \texttt{LABEL}. This is
done only once; in order to get a new \texttt{BETA0} block at the same
location in a subsequent \texttt{TWISS} command, the \texttt{SAVEBETA}
command  must be repeated. The content of the \texttt{BETA0} block can
then be used in other commands, \textsl{e.g.} \texttt{TWISS} and
\texttt{MATCH}. 

\textbf{Example} (after sequence expansion): 
\madxmp{
SAVEBETA, LABEL=sb1, PLACE=mb[5], SEQUENCE=fivecell; \\
TWISS; \\
SHOW, sb1;
}
saves and then shows the \texttt{BETA0} block parameters at the end (!) of the
fifth element of type \texttt{mb} in the sequence.  


Parameters in tables can also be accessed 
using the \hyperref[chap:tables]{table access} functions. 
\begin{verbatim}
USE, period=...;
SAVEBETA, LABEL=name, PLACE=place, SEQUENCE=s_name;
TWISS,...;
\end{verbatim}

When reaching the \texttt{PLACE} in the
sequence \texttt{s\_name} during execution of
\hyperref[chap:twiss]{\texttt{TWISS}}, \madx saves a
\hyperref[sec:beta0]{BETA0} block with the \texttt{LABEL} name: This
block is filled with the values of all lattice parameters in place.   

\textbf{Example 1:} 
\begin{verbatim}
USE, period=CELL;
SAVEBETA, LABEL=END, PLACE=#E, SEQUENCE=CELL;
TWISS;
USE, period=INSERT;
TWISS, BETA0=END;
\end{verbatim}
This first example calculates the \hyperref[sec:twissperiod]{periodic
  solution} of the line \texttt{CELL}, and then tracks lattice parameters through
\texttt{INSERT}, using all end conditions (including orbit) in
\texttt{CELL} at the start of \texttt{INSERT}.  

\textbf{Example 2:} 
\begin{verbatim}
USE, period=CELL;
SAVEBETA, LABEL=END, PLACE=#E, SEQUENCE=CELL;
TWISS;
USE, period=INSERT;
TWISS, BETX=END->BETY, BETY=END->BETX;
\end{verbatim}
This is similar to the first example,but the beta functions are interchanged (overwritten).  

\section{BETA0: Set Initial Lattice Parameters}
\label{sec:beta0}
Initial lattice parameters can be set 'manually' for later commands, in
particular for \hyperref[chap:twiss]{\texttt{TWISS}} or the 
\hyperref[chap:match]{\texttt{MATCH}} module, by
using the \texttt{BETA0} command attached to a label.  

\madbox{
label: BETA0, \=BETX=real, ALFX=real, MUX=real,\\
              \>BETY=real, ALFY=real, MUY=real,\\
              \>\{etc for linear and chromatic lattice functions\};
}

A \texttt{BETA0} block can be used as a whole with all values declared,
as a block with overriden values explicitly, or by extracting single
values as shown in the three examples below:

Example of \texttt{BETA0} block used as a whole in \texttt{TWISS}: 
\madxmp{
initial: BETA0, \=BETX=10., ALFX=0.0, MUX=0.0, \\
                \>BETY=10., ALFY=0.0, MUY=0.0, \\
                \>DX=1., DPX=0.0;\\
TWISS, BETA0=initial;}

Example of \texttt{BETA0} block used as a whole but with overriden
values in the \texttt{TWISS} command:
\madxmp{
initial: BETA0, \=BETX=10., ALFX=0.0, MUX=0.0, \\
                \>BETY=10., ALFY=0.0, MUY=0.0, \\
                \>DX=1., DPX=0.0;\\
TWISS, BETA0=initial, ALFX=-0.1, ALFY=0.1;}

Example of using \texttt{BETA0} block by extracting single values in the
\texttt{TWISS} command:
\madxmp{
initial: BETA0, \=BETX=10., ALFX=0.0, MUX=0.0, \\
                \>BETY=10., ALFY=0.0, MUY=0.0, \\
                \>DX=1., DPX=0.0;\\
TWISS, BETX=initial-$>$BETX, BETY=initial-$>$BETY;}


%%%\title{SELECT}
%  Changed by: Hans Grote, 11-Sep-2002 

%%\usepackage{hyperref}
% commands generated by html2latex


%%\begin{document}
%%\begin{center}
 %%EUROPEAN ORGANIZATION FOR NUCLEAR RESEARCH 
%%\includegraphics{http://cern.ch/madx/icons/mx7_25.gif}

\subsection{Sectormap output}
%%\end{center}
% Begin New version: Jean-Luc Nougaret, 18-Dec-2008 


 The flag "sectormap" on the Twiss command (together with an element selection via select,flag=sectormap,...) causes a file "sectormap" to be written.  

For each user-selected element, it contains the user-selected coefficients of the kick vector 
\texttt{K} (6 values), of the first-order map 
\texttt{R} (6 x 6 values) and of the second-order map 
\texttt{T} (6 x 6 x 6 values)

 The sector file is the output of a standard TFS table, which means that the set of columns of interest may be selected through a MAD-X command such as the following: 


%\begin{tabular}
\begin{verbatim}
select,flag=my_sect_table,column=name,pos,k1,r11,r66,t111;
\end{verbatim}
%\end{tabular}  
Each line of the sectormap file contains for each selected element, the set of chosen K,R,T matrix coefficients:
\\
\\
\begin{tabular}{l|l|l}
@ NAME &              \%13s &  "MY\_SECT\_TABLE" \\ 
@ TYPE &              \%09s &  "SECTORMAP" \\ 
@ TITLE &             \%08s &  "no-title" \\ 
@ ORIGIN &           \%19s &  "MAD-X 3.04.62 Linux" \\ 
@ DATE &              \%08s &  "18/12/08" \\ 
@ TIME &              \%08s &  "10.33.58"
\end{tabular}
\\
\\
\begin{tabular}{l | l | l | l | l | l }
* NAME & POS & K1 & R11 & R66 & T111 \\ 
\$ \%s & \%le & \%le & \%le  & \%le & \%le \\ 
 "FIVECELL\$START"  & 0 & 0 & 1 & 1 & 0 \\ 
 "SEQSTART"  & 0 &  0  &  1 &  1  &  0 \\ 
 "QF.1"  & 3.1 & -1.305314637e-05 & 1.042224745 & 1 & 0 \\ 
 "DRIFT\_0" & 3.265 & 7.451656548e-21 & 1 & 1 & 0 \\ 
 "MSCBH" & 4.365 & -1.686090613e-15 & 0.9999972755 & 1 & 0.006004411526 \\ 
 "CBH.1" & 4.365 & 0 & 1 & 1 & 0 \\ 
 "DRIFT\_1" & 5.519992305 & -6.675347543e-21 & 1 & 1 & 0 \\ 
 "MB" & 19.72000769 & 2.566889547e-18 & 1.000000091 & 1 & -4.135903063e-25 \\ 
 "DRIFT\_2" & 21.17999231 & -1.757758802e-20 & 1 & 1 & 0 \\ 
 "MB" & 35.38000769 & 2.822705549e-18 & 1.000000091 & 1 & -4.135903063e-25 \\ 
 "DRIFT\_2" & 36.83999231 & 2.480880093e-20 & 1 & 1 & 0 \\ 
 "MB" & 51.04000769 & 3.006954115e-18 & 1.000000091 & 1 & -4.135903063e-25 \\ 
 "DRIFT\_3" & 52.21 & -4.886652187e-20 & 1 & 1 & 0 \\ 
... & ... & ... & ... & ... & ... \\ 
... & ... & ... & ... & ... & ... \\ 
... & ... & ... & ... & ... & ...
\end{tabular}
\\
\\ Of course, the 
\texttt{select} statement can be combined with additional options to filter-out the list of elements, such as in the following statement, which for instance only retains drift-type elements: 


%\begin{tabular}
\begin{verbatim}
select,flag=my_sect_table,class=drift,column=name,pos,k1,r11,r66,t111;
\end{verbatim}
%\end{tabular}


\texttt{K} coefficients range: 
\texttt{K1}... 
\texttt{K6}


\texttt{R} coefficients range: 
\\
\begin{tabular}{ccc}
\texttt{R11} & ... & \texttt{R61} \\ 
\texttt{R12} & ... & \texttt{R62} \\ 
... & ... & ... \\ 
\texttt{R61} & ... & \texttt{R66}
\end{tabular}


\texttt{T} coefficients range: 
\\
\begin{tabular}{ccc}
\texttt{T111} & ... &\texttt{T611} \\ 
\texttt{T121} & ... & \texttt{T621} \\ 
... & ... & ... \\ 
\texttt{T161} & ... & \texttt{T661} \\ 
\texttt{T112} & ... & \texttt{T612} \\ 
... & ... & ... \\ 
\texttt{T166} & ... & \texttt{T666}
\end{tabular}

 In the above notation, 
\texttt{Rij} stands for "effect on the 
\texttt{i}-th coordinate of the 
\texttt{j}-th coordinate in phase-space", whereas 
\texttt{Tijk} stands for "combined effect on the 
\texttt{i}-th coordinate of both the 
\texttt{j}-th and 
\texttt{k}-th coordinates in phase-space" along the lattice. 
% End New Version 

%  Commented by jluc, on 18 December 2008
% The flag "sectormap" on the Twiss command (together with an element
% selection via select,flag=sectormap,...) causes a file "sectormap" to
% be written. This is a fixed format file; per selected element it
% contains:
% 
% <pre>
% end_position   element_name
% kick vector (6 values)
% first order map (6 lines with 6 values each), column-wise
% second order map (36 lines with 6 columns each, column-column-wise)
% </pre>
% 
% Or more explicitly:
% 
% <pre>
% The first line is:
% K[1] ... K[6]
% 
% Then: 
% R[1,1] ... R[6,1]
% R[1,2] ... R[6,2]
% ...
% R[1,6] ... R[6,6]
% 
% 
% Then:
% T[1,1,1] ... T[6,1,1]
% T[1,2,1] ... T[6,2,1]
% ...
% T[1,6,1] ... T[6,6,1]
% T[1,1,2] ... T[6,1,2]
% ...
% T[1,6,6] ... T[6,6,6]
% </pre>
% 
   The maps are the accumulated maps between the selected elements. They contain both the alignment, and field errors present. Together with the starting value of the closed orbit (which can be obtained from the standard twiss file) this allows the user to track particles over larger sectors, rather than element per element. A typical usage therefore lies in the interface to other programs, such as TRAIN. 
\\
\href{http://www.cern.ch/Hans.Grote/hansg_sign.html}{hansg}, May 8, 2001 

%%\end{document}

\section{Sectormap output}
\label{sec:sectormap}
% Begin New version: Jean-Luc Nougaret, 18-Dec-2008 

The flag \texttt{SECTORMAP} of the \texttt{TWISS} command (together with an element
selection via \texttt{SELECT, FLAG=sectormap,...}) causes a file "sectormap" to be
written.

For each user-selected element, it contains the user-selected coefficients of 
the kick vector \texttt{K} ($6$ values), of the first-order map \texttt{R} 
($6\times6$ values) and of the second-order map \texttt{T} ($6\times6\times6$ 
values)

The sector file is the output of a standard TFS table, which means that
the set of columns of interest may be selected through a SELECT command
as in the following example:  

\madxmp{
SELECT, FLAG=my\_sect\_table, COLUMN=name, pos, k1, r11, r66, t111;
}

The sectormap file contains for each selected element, one element per
line, the set of chosen K, R, and T matrix coefficients: 
\\
\\
{\footnotesize
\begin{tabular}{l|l|l}
@ NAME &              \%13s &  "MY\_SECT\_TABLE" \\ 
@ TYPE &              \%09s &  "SECTORMAP" \\ 
@ TITLE &             \%08s &  "no-title" \\ 
@ ORIGIN &           \%19s &  "MAD-X 3.04.62 Linux" \\ 
@ DATE &              \%08s &  "18/12/08" \\ 
@ TIME &              \%08s &  "10.33.58"
\end{tabular}
\\
\\
\begin{tabular}{l | l | l | l | l | l }
* NAME & POS & K1 & R11 & R66 & T111 \\ 
\$ \%s & \%le & \%le & \%le  & \%le & \%le \\ 
 "FIVECELL\$START"  & 0 & 0 & 1 & 1 & 0 \\ 
 "SEQSTART"  & 0 &  0  &  1 &  1  &  0 \\ 
 "QF.1"  & 3.1 & -1.305314637e-05 & 1.042224745 & 1 & 0 \\ 
 "DRIFT\_0" & 3.265 & 7.451656548e-21 & 1 & 1 & 0 \\ 
 "MSCBH" & 4.365 & -1.686090613e-15 & 0.9999972755 & 1 & 0.006004411526 \\ 
 "CBH.1" & 4.365 & 0 & 1 & 1 & 0 \\ 
 "DRIFT\_1" & 5.519992305 & -6.675347543e-21 & 1 & 1 & 0 \\ 
 "MB" & 19.72000769 & 2.566889547e-18 & 1.000000091 & 1 & -4.135903063e-25 \\ 
 "DRIFT\_2" & 21.17999231 & -1.757758802e-20 & 1 & 1 & 0 \\ 
 "MB" & 35.38000769 & 2.822705549e-18 & 1.000000091 & 1 & -4.135903063e-25 \\ 
 "DRIFT\_2" & 36.83999231 & 2.480880093e-20 & 1 & 1 & 0 \\ 
 "MB" & 51.04000769 & 3.006954115e-18 & 1.000000091 & 1 & -4.135903063e-25 \\ 
 "DRIFT\_3" & 52.21 & -4.886652187e-20 & 1 & 1 & 0 \\ 
... & ... & ... & ... & ... & ... \\ 
... & ... & ... & ... & ... & ... \\ 
... & ... & ... & ... & ... & ...
\end{tabular}
}
\\
\\ 
Of course, the \texttt{SELECT} statement can be combined with additional
options to filter-out the list of elements, such as in the following
statement, which for instance only retains drift-type elements:  

\madxmp{
  SELECT, \=FLAG=my\_sect\_table, CLASS=drift, \\
          \>COLUMN=name, pos, k1, r11, r66, t111;
}


\texttt{K} coefficients range: 
\texttt{K1}... 
\texttt{K6}


\texttt{R} coefficients range: 
\begin{tabular}{ccc}
\texttt{R11} & ... & \texttt{R61} \\ 
\texttt{R12} & ... & \texttt{R62} \\ 
... & ... & ... \\ 
\texttt{R16} & ... & \texttt{R66}
\end{tabular}


\texttt{T} coefficients range: 
\begin{tabular}{ccc}
\texttt{T111} & ... &\texttt{T611} \\ 
\texttt{T121} & ... & \texttt{T621} \\ 
... & ... & ... \\ 
\texttt{T161} & ... & \texttt{T661} \\ 
\texttt{T112} & ... & \texttt{T612} \\ 
... & ... & ... \\ 
\texttt{T166} & ... & \texttt{T666}
\end{tabular}

 In the above notation, 
\texttt{Rij} stands for "effect on the 
\texttt{i}-th coordinate of the 
\texttt{j}-th coordinate in phase-space", whereas 
\texttt{Tijk} stands for "combined effect on the 
\texttt{i}-th coordinate of both the 
\texttt{j}-th and 
\texttt{k}-th coordinates in phase-space" along the lattice. 
% End New Version 

%  Commented by jluc, on 18 December 2008
% The flag "sectormap" on the Twiss command (together with an element
% selection via select,flag=sectormap,...) causes a file "sectormap" to
% be written. This is a fixed format file; per selected element it
% contains:
% 
% <pre>
% end_position   element_name
% kick vector (6 values)
% first order map (6 lines with 6 values each), column-wise
% second order map (36 lines with 6 columns each, column-column-wise)
% </pre>
% 
% Or more explicitly:
% 
% <pre>
% The first line is:
% K[1] ... K[6]
% 
% Then: 
% R[1,1] ... R[6,1]
% R[1,2] ... R[6,2]
% ...
% R[1,6] ... R[6,6]
% 
% 
% Then:
% T[1,1,1] ... T[6,1,1]
% T[1,2,1] ... T[6,2,1]
% ...
% T[1,6,1] ... T[6,6,1]
% T[1,1,2] ... T[6,1,2]
% ...
% T[1,6,6] ... T[6,6,6]
% </pre>
% 
   
The maps are the accumulated maps between the selected elements. They
contain both the alignment, and field errors present. Together with the
starting value of the closed orbit (which can be obtained from the
standard twiss file) this allows the user to track particles over larger
sectors, rather than element per element. Note that effects of the higher order
are included in the Rij. In order to disable this use the flag SECTORPURE. 
This flag should be activated when used to interface TRAIN.  


%%\documentclass[a4paper,11pt]{article}
%%\usepackage{ulem}
%%\usepackage{a4wide}
%%\usepackage[dvipsnames,svgnames]{xcolor}
%%\usepackage[pdftex]{graphicx}
%%\title{Threader}
%%\usepackage{hyperref}
% commands generated by html2latex


%%\begin{document}  %%EUROPEAN ORGANIZATION FOR NUCLEAR RESEARCH 
%%\includegraphics{../icons/mx7_25.gif align=right}

\subsection{Beam Threader}  The \textbf{threader} simulates beam steering through a machine with field and alignment errors in situations where the beam does not circulate and the closed orbit cannot be measured. 

 If enabled, threading is executed whenever a trajectory or closed orbit search is carried out by the \href{../twiss/twiss.html}{TWISS} module. 

 The following MAD-X commands control the action of the threader :  


\begin{verbatim}
option, threader ;
\end{verbatim}  enables the threader 

 when set, the threader checks at all monitors the difference with respect to the stored orbit there (from \href{../twiss/twiss.html}{keeporbit}) if \href{../twiss/twiss.html}{useorbit} is present. The threader then provides kicks (if possible) to reduce the orbit difference below the maxima specified on the threader command. This procedure allows to thread with orbit bumps present  

\begin{verbatim}
threader, vector = {xmax, ymax, att} ;
\end{verbatim}  sets the parameters for the threader 
\begin{verbatim}
xmax, ymax : orbit excursion (at a monitor) at which threader acts
att        : attenuation factor for the kicks applied by the threader
defaults   : {0.005, 0.005, 1.000}
\end{verbatim}


\line(1,0){300}

 Hans Grote, 31.10.2008 

%%\end{document}

%%\title{Threader}

\section{Beam Threader}
\label{sec:threader}

In a machine with magnetic and alignment errors it can happen that the
beam does not circulate and the closed orbit cannot be established and
measured. This can also happen in \madx and the closed orbit finder does
not converge. 

The \texttt{THREADER} simulates beam steering through such a machine
with repeated measurement of trajectory over a certain number of
monitors and correction of the trajectory with upstream correctors.   

When enabled, threading is executed whenever a trajectory or closed
orbit search is carried out by the \hyperref[chap:twiss]{TWISS}
module.   

The threader is controlled as an option. 
The following \madx command enables the threader:
\madxmp{OPTION, THREADER;}
and the threader can be disabled with
\madxmp{OPTION, -THREADER;}

During the trajectory search in \hyperref[chap:twiss]{\texttt{TWISS}}, or the first 
turn of the orbit search for a closed machine, the threader calculates at each
monitor the difference between the present trajectory reading and a
reference value.
If the difference exceeds a threshold (see below), the threader searches
backwards for the first corrector that will efficiently cancel the
difference and calculates the corresponding kick. The trajectory is then
recalculated starting again from that corrector and progressing
forward. The calculated kicks are added to already existing kicks. If
Twiss is searching for a closed orbit which involves tracking this
trajectory over many turns, the threader is only active during the first
turn.  

The reference value for the trajectory difference is zero by default but
can also be obtained from a previous orbit calculation if the current
\texttt{TWISS} command has the \texttt{USEORBIT} flag enabled and a previous
\texttt{TWISS} command had the \texttt{KEEPORBIT} flag enabled. This allows
for example to thread the beam into a machine with orbit bumps present.  

The threshold values for triggering the threader correction can be set
with the command  
\madbox{
THREADER, VECTOR=\{xmax, ymax, att\};
}
where 
\begin{madlist}
  \ttitem{xmax, ymax} threshold orbit excursions beyond which the threader
  is applied. \\ Defaults: 0.005, 0.005
  \ttitem{att} attenuation factor for the kicks applied by the threader
  \\ Default: 1.0 
\end{madlist}

The attenuation factor defines the fraction of the calculated kick that
is actually applied by the threader.  
An attenuation factor of 0.5 will apply 50\% of the calculated kicks.

\section{Closed Orbit Guess}
\label{sec:coguess}

In order to help the initial finding of the closed orbit by the
\texttt{TWISS} module, it is possible to specify an initial guess for
the coordinates of the fixed point at the start of the lattice.

\madbox{
COGUESS, \=X=real, PX=real, Y=real, PY=real, T=real, PT=real, \\
         \>TOLERANCE=real, \\
         \>CLEAR=logical;
}

The \texttt{COGUESS} command has the following attributes:
\begin{madlist}
  \ttitem{X, PX, Y, PY, T, PT} each parameter specified defines a first guess for all future closed orbit
  searches in case they are different from zero.  
  
  \ttitem{TOLERANCE} sets the required convergence precision in the closed
  orbit search. \\ (Default: 1.e-6)  
  
  \ttitem{CLEAR} a flag to reset the tolerance to its default value and to
  cancel the effect of a previous \texttt{COGUESS} in defining a first
  guess for subsequent closed orbit searches. \\ (Default:~false) 

\end{madlist}


\textbf{Example}\\
\madxmp{
COGUESS, X=1.e-3; \\
... \\
TWISS; \\
... \\
COGUESS, Y=-2.e-3; \\
... \\
TWISS; \\
... \\
COGUESS, CLEAR; \\
... \\
TWISS; \\
... \\
}
\section{Coupling}
\label{sec:coupling}

In order to calculate Twiss parameters ($\beta$, $\alpha$ and $\phi$) MAD-X uses Edwards-Teng approach.

For coupled motion the matrix transformation can be written in a form $M = VUV^{-1}$, where $M$ is a symplectic 1-turn map, $U$ is a symplectic decoupled block-diagonal matrix, $V$ is a symplectic "rotation" matrix. Initial coupling parameters are calclulated based on decoupled $U$ matrix. 
  
Given that $M_1$ is the one-turn matrix at point 1, $M_2$ is the one-turn matrix at point 2 , and we know $M_{12}$, the transfer matrix between points 1 and 2, we can express the change of the one-turn map from point 1 to point 2 by {\color{blue} $M_2 = M_{12}M_1M^{-1}_{12}$}. Then $U_2 = V_2^{-1}M_2V_2 = V_2^{-1} {\color{blue}M_{12}M_1M^{-1}_{12}} V_2 = ( V_2^{-1} M_{12} {\color{blue}V_1) U_1 (V_1^{-1}} M^{-1}_{12} V_2)$.

So,  $U_2 = W_{12} U_1 W_{12}^{-1} \textrm{, where } W_{12} \equiv V_2^{-1} M_{12}  V_1 $
Knowing $V_1$ and $M_{12}$  (transfer matrix of the element), we can calculate $V_2 W_{12} = M_{12} V_1,$ where we use $V_{2}$ for computing the coupling parameters and $W_{12}$ to propagate to the end of the element. One can find $V_2$ such that $W_{12}$ is block-diagonal or off block-diagonal. 
  
For weakly coupled lattice MAD-X always uses \textbf{block-diagonal} solution, i.e. large tune (mode1) is associated with x-plane, and smaller tune (mode2) is associated with y-plane by convention. For a highly coupled lattice, it may happen at some places in the lattice that mode1 is associated with y-plane, and mode2 with x-plane. Switching the association of mode1 and mode2 with x and y-planes as one propagates the twiss parameters through the lattice is called "modes flip" (use \textbf{off-block diagonal} solution for $W_{12}$). In case the modes flip is triggered, there is an additional stability check, either the calculation of twiss parameters can be done with modes flip or not. 

\section{Intermediate values}

The \texttt{SELECT, FLAG=interpolate} command allows to select locations for
output of intermediate values for the \texttt{TWISS} command. If interpolation
points have been selected for an element, the \texttt{TWISS} command will add
the values at these locations to the TFS table rather than at the end of the
element.

\madbox{
SELECT, \=FLAG=interpolate, SEQUENCE=string, RANGE=string, CLASS=string, \\
        \>CLEAR, \\
        \>SLICE=integer, \\
        \>STEP=real, \\
        \>AT=\{real, \ldots \};
}

\textbf{Note:} This feature may behave unexpectecly with \texttt{CENTRE}.

Selected elements can optionally be filtered based on sequence, range and
class. Every selection specifies exactly one of the following actions which
overrides previous selections for the matching elements:

\begin{madlist}
  \ttitem{CLEAR} remove interpolation positions.
  \ttitem{SLICE} slice using a fixed number of slices.
  \ttitem{STEP} slice at every \texttt{STEP} meters.
  \ttitem{AT} slice at given locations (as fraction of the node length).
\end{madlist}


%% EOF
