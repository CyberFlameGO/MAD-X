%%\title{SEQUENCE}
%  Changed by: Chris ISELIN, 27-Jan-1997 
%  Changed by: Hans Grote, 30-Sep-2002 

\chapter{Sequences}
\label{chap:sequence}

\madx accepts two forms of an accelerator definition: sequences and
beam lines (See \ref{chap:beamlines}). 
However, the sequence definition is the only one used internally by \madx.

A sequence is declared with the following statements:
\madbox{
name: \=SEQUENCE, \=L=real, REFER=string, REFPOS=string, \\
      \>          \>ADD\_PASS=integer, NEXT\_SEQU=string; \\
      \>\dots \\ 
xxxxxx\=xxxxxxx\=xxxxxxxx\= \kill
%      \>label: \>       \>AT=real, FROM=string \{, additional attributes\}; \\
      \>label: \>class, \>AT=real, FROM=string \{, additional attributes\}; \\
      \>       \>class, \>AT=real, FROM=string \{, additional attributes\}; \\
      \>seqname, \>     \>AT=real, FROM=string; \\
      \>\ldots \\
ENDSEQUENCE;
}

The first statement declares a sequence, giving it a name in the form of
a label, and providing some key parameters: 

\begin{madlist}
   \ttitem{L} (Default: {\tt 0.})\\
     the total length of the sequence in meters. 
   \ttitem{REFER}{\tt = keyword} in {\tt \{entry, centre, exit\}} (Default:
          {\tt centre}) \\
     specifies which part of the element is taken as the reference point 
     at which the position along the beamline is given.
   \ttitem{REFPOS} (Default: {\tt none})\\
   specifies the name of an element in this sequence to be used as the
   reference point for the insertion of this sequence in another
   sequence.
   \ttitem{ADD\_PASS} (Default: 0)\\ 
     specifies a number of additional passes (max. 5) through the
     structure; in case of an RBEND the angle will be overwritten in  survey
     using the i-th component ($1 \le i \le {\tt add\_pass} \le 5$) of
     its {\tt array\_of\_angles} attribute (see \hyperref[sec:rbend]{\tt 
     RBEND}). 
   \ttitem{NEXT\_SEQU} (Default: {\tt none})\\
     specifies the name of a sequence to be concatenated 
     to the end of this sequence.
\end{madlist}
 
The \texttt{ENDSEQUENCE} statement terminates the declaration of a
sequence. 

Inside the \texttt{SEQUENCE \ldots ENDSEQUENCE} bracketing keywords,
several types of statements may appear:
\begin{itemize}
%% pending check...
%% \item An element already declared:
%%   \madbox{label: AT=real, FROM=string \{, attributes\} ;}
%%   where label is an already declared element, with additional \texttt{AT}
%%   attribute giving the relative element position, and an optional
%%   \texttt{FROM} attribute;   

\item An element declaration with label: 
  \madbox{label: class, AT=real, FROM=string \{, attributes\} ;}
  an element declaration as usual, with additional \texttt{AT}
  attribute giving the relative element position, and an optional
  \texttt{FROM} attribute; \\ 
  {\bf CAUTION:} an existing definition for an element with the
  same name (label) will be replaced, however, defining the same
  element twice inside the same sequence results in a fatal error,
  since a unique object (this element) would be placed at two
  different positions.

\item An element declaration from class: 
  \madbox{class, AT=real, FROM=string \{, attributes\} ;}
  a class name causing an instance of that class to be created with
  specified attributes, with additional \texttt{AT} attribute giving the
  relative element position, and an optional \texttt{FROM} attribute;\\
  For uses inside ranges, instances of the same class can then be
  accessed with an occurrence count.

\item A sub-sequence name:
  \madbox{seqname, AT=real, FROM=string ;}
  a sequence name with additional \texttt{AT} attribute giving the
  relative element position, and an optional \texttt{FROM} attribute;\\ 
  %this causes the sequence with that name (inserted sequence) to be
  %placed at the position relative to the start of the current
  %containing sequence. \\ 
  Depending upon the {\tt REFER} attribute of the current (containing)
  sequence, the \texttt{entry}, \texttt{centre}, or \texttt{exit} of
  the inserted sequence is placed at the position specified. \\
  HOWEVER, if the inserted sequence has a {\tt REFPOS} attribute containing
  the name of an element in the inserted sequence, the inserted sequence
  is placed such that the element pointed to by {\tt REFPOS} is at the location
  specified in the current sequence.
\end{itemize} 


The additional arguments that can be given in the declaration of sequence 
elements or sub-sequences are: 
\begin{madlist}
  \ttitem{AT} mandatory argument giving the location at
  which the reference point ({\tt entry, centre} or {\tt exit}) of
  the element is to be placed in the sequence. \\
  The value is absolute with the zero reference at the start of the
  sequence, unless a {\tt FROM} argument is specified.

  \ttitem{FROM} optional argument giving the name of an
  element of the same sequence. \\
  The location given for the current element or sequence in the {\tt AT}
  argument is then taken as {\bf relative}  to the position of the
  center of the element given by the {\tt FROM} argument. \\
  Only simple elements should be used in FROM; specifying a sequence 
  as the FROM reference can lead to erroneous sequence expansion at best.  
\end{madlist}


When the sequence is expanded in a \hyperref[sec:use]{\tt USE} 
command, \madx generates the drift spaces between elements according to
the following rules: 
\begin{enumerate}
\item When the distance between the exit of the previous element and the
  entrance of the next element is positive and greater than a threshold
  of $1 \mu m$ an explicit drit is generated with its own name, unless
  an already existing drift space with the same length (within $10^{-12}
  m$ tolerance) can be re-used.
\item When the absolute value of the distance between the exit of the 
  previous element and the entrance of the next element is less than a
  given tolerance of $1\mu m$, no drift space is created and the
  elements are considered as contiguous. \\
  {\bf Note that in very specific cases this can cause very small errors
    to accumulate and the actual length of the sequence can vary
    slightly from the declared length. (see second example below)} 
\item When the distance between the exit of the previous element and the 
  entrance of the next element is negative and less than $-1\mu m$,
  the elements are considered to be overlapping and \madx terminates
  with a ``{\tt negative drift length}'' fatal error. 
\end{enumerate} 

\begin{5.02.06}
When the sequence is expanded in a \hyperref[sec:use]{\tt USE} 
command, \madx also checks that the calculated sequence length 
is positive and that the lengths of all elements in the sequence 
are positive or zero. A negative sequence length or negative element 
length stops the \madx run with fatal error.
\end{5.02.06}

For efficiency reasons \madx imposes an {\bf important restriction}
on element lengths and positions: once a sequence is expanded, the
element positions and lengths are considered as fixed; in order to vary
a position or element length, a re-expansion of the sequence becomes
necessary. The \hyperref[chap:match]{\tt MATCH} command contains a special flag 
{\tt VLENGTH} to vary element lengths during matching.

{\bf Example:}
\begin{verbatim}
! Define element classes for a simple cell:
b:   sbend, l = 35.09, angle = 0.011306116;
qf:  quadrupole, l = 1.6,  k1 = -0.02268553;
qd:  quadrupole, l = 1.6,  k1 =  0.022683642;
sf:  sextupole,  l = 0.4,  k2 = -0.13129;
sd:  sextupole,  l = 0.76, k2 =  0.26328;

! define the cell as a sequence:
sequ:  sequence, l = 79;
   b1:    b,      at = 19.115;
   sf1:   sf,     at = 37.42;
   qf1:   qf,     at = 38.70;
   b2:    b,      at = 58.255, angle = b1->angle;
   sd1:   sd,     at = 76.74;
   qd1:   qd,     at = 78.20;
   endm:  marker, at = 79.0;
endsequence;
\end{verbatim}




Example of very small drift space being ignored during sequence
expansion: 
\begin{verbatim}
QTEST: QUADRUPOLE, L=1.000001;

TEST: SEQUENCE, REFER=centre, L=2.;
  QTEST, AT=1.5;
ENDSEQUENCE;

USE, SEQUENCE=TEST;
SURVEY, FILE='test';
\end{verbatim}
The above sequence will expand to a total length of $2.0000005 m$, half
a micron longer than the claimed length of $2 m$, but will not fail.
