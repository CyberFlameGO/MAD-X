\chapter{Change Log}
\label{chap:changelog}

\begin{center} 
\textbf{since version 5.02.00}
\end{center}

The following changes have been made to the code and documentation since
August 15th, 2014 in version 5.02.02

The changes are indexed by date (most recent first) and provide the \madx 
version number where the change applies as well as the SVN 
release number for the change. 

\begin{madlist}

  \ttitemn{2015-Mar-06} version 5.02.05, r5158\\
  Major change to collimator elements: the {\tt RCOLLIMATOR} and {\tt ECOLLIMATOR} 
  elements are no longer defined in \madx and replaced by a generic 
  \hyperref[sec:collimator]{\tt COLLIMATOR} element. \\
  {\bf (give reference to web page giving details for translation...)}

  \ttitemn{2015-Feb-19} version 5.02.05, r5143\\
  added the {\tt OCTAGON} in the list of predefined 
  \hyperref[chap:aperture]{\tt APERTURE} types.
  
  \ttitemn{2015-Feb-11} version 5.02.05, r5128\\
  clarified that the {\tt NMASS} constant is the unified atomic mass 
  unit and not the neutron mass. None of the constants have changed in 2014 PDG 
  publication w.r.t. 2012 version \cite{PDG2012}. Updated the reference to PDG 
  publications to include 2014 version \cite{PDG2014}.

  \ttitemn{2015-Jan-28} version 5.02.05, r5118\\
  clarified in the \hyperref[chap:elements]{definition of magnetic elements} 
  that the effect of defined magnetic strengths is always the same, 
  irrespective of the \hyperref[sec:beam]{\tt CHARGE} of the particles declared 
  in the \hyperref[sec:beam]{\tt BEAM} command. It is agreed in the literature 
  that a positive quadrupole (positive $K_1$) focuses positive particles in the 
  horizontal plane and defocuses negative particles in the same horizontal 
  plane, for the same direction of propagation. \\ 
  Currently \mad ignores the \hyperref[sec:beam]{\tt CHARGE} attribute and 
  focuses both positive and negative particles in the horizontal plane when 
  going through a quadrupole with positive $K_1$. \\
  {\bf THIS MAY CHANGE IN THE FUTURE TO CONFORM TO EXISTING CONVENTIONS}\\
  Electrostatic elements (\hyperref[sec:elseparator]{\tt ELSEPARATOR}, 
  \hyperref[sec:rfcavity]{\tt RFCAVITY}, \hyperref[sec:crabcavity]{\tt 
  CRABCAVITY}, and the RF part of the \hyperref[sec:rfmultipole]{\tt 
  RFMULTIPOLE}) handle the \hyperref[sec:beam]{\tt CHARGE} attribute 
  appropriately and provide 
  opposite effects for opposite charges travelling in the same direction. 


  \ttitemn{2014-Dec-19} version 5.02.05, r5111\\
  added the Gauss error function {\tt ERF} and the complementary error function 
  {\tt ERFC} to the list of \hyperref[subsec:operator]{available operators in 
  arithmetic expressions}. Added documentation in the same section for the 
  {\tt FLOOR}, {\tt CEIL} and {\tt ROUND} functions that were already 
  implemented. 
  
  
  \ttitemn{2014-Dec-10} version 5.02.04, r5093 and r5101\\
  clarified the global coordinate system figure~\ref{F-GLOB} with colors and 
  representations of projections of planes onto the horizontal Cartesian 
  plane as well as intersections of local coordinate planes with horizontal 
  Cartesian plane. 
  
  
  \ttitemn{2014-Nov-25} version 5.02.04, r5092\\
  removed the \hyperref[sec:global]{\tt GLOBAL} matching constraints {\tt DDQ1, 
  	DDQ2} from the documentation since they are not implemented in the code. 
  
  
  \ttitemn{2014-Nov-14} version 5.02.04, r5081\\
  added a \hyperref[sec:copyfile]{\tt COPYFILE} command. Changed the attribute 
  name for the destination for the \hyperref[sec:renamefile]{\tt RENAMEFILE} 
  command from {\tt NAME} to {\tt TO}.
  
  
  \ttitemn{2014-Nov-13} version 5.02.04, r5078\\
  fixed figure \ref{F-YSDISP} where the {\it x}-axis was pointing in
  the wrong direction and the orientation of the element for positive
  {\tt DPHI} was also not conforming to the text for the
  \hyperref[sec:ealign]{\tt EALIGN} command.
  
  
  \ttitemn{2014-Nov-13}  version 5.02.04, r5080\\
  documented a bug occurring when \hyperref[sec:line]{\tt 
  LINE} or \hyperref[sec:macro]{\tt MACRO} 
  constructs appear within a \hyperref[sec:if]{\tt IF ... ELSEIF ... ELSE} or a 
  \hyperref[sec:while]{\tt WHILE} construct. This bug will not be fixed now. \\ 
  Clarified also that {\tt IF ... ELSEIF ... ELSE} and {\tt WHILE} constructs 
  can be nested to at least six levels deep.
  
  
  \ttitemn{2014-Oct-14} version 5.02.03, r5013\\
  fixed a documented feature of \hyperref[chap:survey]{\tt SURVEY} where
  the first {\tt KSL} component of thin \hyperref[sec:multipole]{\tt
  	MULTIPOLE} elements, representing a vertical angle for a thin
  dipole, was not taken into account. Both {\tt KNL} and {\tt KSL} are
  now properly taken into account. Another change was to make {\tt
  	SURVEY} take into account the \hyperref[sec:rfmultipole]{\tt
  	RFMULTIPOLE} elements in the same way that it treats {\tt MULTIPOLE}
  elements.
  
  
  \ttitemn{2014-Aug-27} version 5.02.03, r4947 \\
  changed the behaviour of \hyperref[sec:fill]{\tt FILL} to accept as
  parameter a row number equal to the current number of rows in the
  table plus one, with the effect of creating a new row and filling it. 
  
  
  \ttitemn{2014-Aug-25} version 5.02.03, r4942 and r4943\\
  harmonized the behaviour of \hyperref[sec:fill]{\tt FILL},
  \hyperref[sec:setvars]{\tt SETVARS} and \hyperref[sec:setvars_lin]{\tt
  	SETVARS\_LIN} with respect to negative row numbers, and updated the
  default values. Added documentation sections for {\tt SETVARS} and {\tt
  	SETVARS\_LIN} that were hitherto undocumented.  
  
  
  \ttitemn{2014-Aug-18} version 5.02.03, r4932 \\
  a single element can now be repeated in a beamline expansion:
  \texttt{2*S} and \texttt{-2*S} are of course identical (single
  elements are not reversed head to tail), and also equivalent to
  \texttt{2*(S)} and \texttt{-2*(S)} if \texttt{S} is a single
  element.\\
  Documentation updated; see \ref{sec:reflect_repeat_lines}


\end{madlist}
  


\newpage

%% EOF
