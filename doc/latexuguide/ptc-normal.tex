%%\title{PTC\_NORMAL Module (Non-Linear Machine Parameters)}
%  Created by: Valery KAPIN, 21-Mar-2006 

\chapter{Non-Linear Machine Parameters}
\label{chap:ptc-normal}


The \texttt{PTC\_NORMAL} module (\cite{schmidt2005} and
\cite{damico2004-2}) of \madx is based on \ptc code. This module takes
full advantage of the \ptc Normal Form analysis which is a considerable
upgrade of what was available with the Lie Algebra technique used in
\madeight. It allows to calculate dispersions, chromaticities,
anharmonicities and Hamiltonian terms to very high order. In fact, the
order is only limited by the available computer memory and computing
time.

The number of terms per order increases with some power law. The
internal \madx tables are not adequate to keep such large amounts of
data. On the other hand, only a reduced set of this data is actually
needed by the user. Thus a much easier and flexible solution is to
gather user requirements with a series of \texttt{SELECT\_PTC\_NORMAL}
command. A special \madx table is dynamically built using those
commands and is filled by a subsequent call to \texttt{PTC\_NORMAL}.

Another essential advantage of this table is that it is
structured to facilitate exchange of Normal Form (including
Hamiltonian terms of high order) between \madx modules. The immediate
goal is to use this table to allow non-linear matching inside the
present \madx \hyperref[chap:match]{\texttt{MATCHING}} module. 

\textbf{Synopsis:}
\madbox{
PTC\_CREATE\_UNIVERSE; \\
PTC\_CREATE\_LAYOUT, MODEL=integer, METHOD=integer, NST=integer, [EXACT];\\ 
\ldots\\
SELECT\_PTC\_NORMAL, DX, \ldots , GNFU;\\
\ldots \\
PTC\_NORMAL;\\
WRITE, TABLE=normal\_results, FILE=normal\_results;\\
\ldots \\
PTC\_END; 
}

\section{SELECT\_PTC\_NORMAL} 
\label{sec:select-ptc-normal}

The \texttt{SELECT\_PTC\_NORMAL} command selects the parameters to be 
calculated by a subsequent \texttt{PTC\_NORMAL} command.

\madbox{
xxxxxxxxxxxxxxxxxx\= \kill
SELECT\_PTC\_NORMAL, \>DX=integer, DPX=integer, DY=integer, DPY=integer, \\
                     \>Q1=0, Q2=0, DQ1=integer, DQ2=integer, \\
                     \>ANHX=integerarray, ANHY=integerarray, \\
                     \>GNFU=integer,0,0, HAML=integer,0,0, \\
                     \>EIGN=integer,integer;
}

The attributes are: 
\begin{madlist}
  \ttitem{DX, DPX, DY, DPY} the dispersion paramaters specified as
  integer numbers specifying their order:  
  $D_x^{(n)} = \partial^{(n)} x_{co} / \partial \delta_p^{(n)}$, 
  $D_{px}^{(n)} = \partial^{(n)} px_{co} / \partial \delta_p^{(n)}$, 
  $D_y^{(n)} = \partial^{(n)} y_{co} / \partial \delta_p^{(n)}$, 
  $D_{py}^{(n)} = \partial^{(n)} py_{co} / \partial \delta_p^{(n)}$,
  where \texttt{co} is abbreviation of "closed orbit".

  \ttitem{Q1, Q2} horizontal and vertical tune paramaters are fixed to
  order 0: $Q_1^{(0)}$,  $Q_2^{(0)}$
  
  \ttitem{DQ1, QD2} the tune derivative parameters specified as
  integer numbers specifying their order. 
  $\partial^{(n)} Q_1 / \partial \delta_p^{(n)}$, 
  $\partial^{(n)} Q_2 / \partial \delta_{p}^{(n)}$
  
  
  \ttitem{ANHX, ANHY} the anharmonicities, each defined by three integer
  numbers: the order $n_1$ of $\epsilon_1$, the order $n_2$ of
  $\epsilon_2$ and the order $n_3$ of $\delta_p$.
  $\partial^{(n_1 + n_2 + n_3)} Q_z / (
  \partial\epsilon_1^{(n_1)} \partial\epsilon_2^{(n_2)}
  \partial\delta_p^{(n_3)})$.
  For example, \texttt{ANHX=2,0,0} represents second order
  in $\epsilon_1$:  $\partial^{(2)} Q_1 / \partial \epsilon_1^{(2)}$. 
  And \texttt{ANHY=3,1,2} represents $\partial^{(6)} Q_2 / (
  \partial\epsilon_1^{(3)} \partial\epsilon_2^{(1)}
  \partial\delta_p^{(2)})$.

  \ttitem{EIGN} components of the eigenvectors at the end of the structure
  can be specified by two integers: the eigenvector number and the
  coordinate coded in the list: \{$x, p_x, y, p_y, t, p_t$\}: \\
  The pair $n_1$,$n_2$ defines the $n_2$-th component of
  the $n_1$-th eigenvector. 
  
  \ttitem{GNFU} The Generating Function can be specified by \{$n_1$, $n_2$,
  $n_3$\}. 
  The positive and negative values of $n-1$ define the order of
  upright or skew resonances respectively. The integers $n_2$
  and $n_3$ are reserved for a future upgrade and must be set to 0. \\
  For example \texttt{GNFU=-5,0,0} calculates all Generating Function
  terms for skew decapoles. In the output table the cosine, sine and amplitude
  coefficients are denoted as "GNFC", "GNFS" and "GNFA" respectively.
  
  \ttitem{HAML} the Hamiltonian terms can be specified by \{$n_1$, $n_2$,
  $n_3$\} 
  The positive and negative values of $n_1$ define the order of
  upright or skew resonances, respectively. The integers $n_2$
  and $n_3$ are reserved for a future upgrade and must be set to 0. \\
  For example, \texttt{HAML=3,0,0} calculates all Hamiltonian terms for
  upright sextupoles. In the output table the cosine, sine and amplitude
  coefficients are denoted as "HAMC", "HAMS" and "HAMA" respectively. 
  
\end{madlist}

\textbf{Caution:} if more than one order of terms is selected only the lower one is correct 
because higher orders contain "cross terms" from the lower ones.


\section{PTC\_NORMAL}
\label{sec:ptc-normal}

The calculation of the parameters specified by the preceding
\texttt{SELECT\_PTC\_NORMAL} commands is initiated by the
\texttt{PTC\_NORMAL} command, which operates on the working beam
line defined in the latest \hyperref[sec:use]{USE} command. 

\madbox{
PTC\_NORMAL, \=ICASE=integer, NORMAL=logical, CLOSED\_ORBIT=logical,\\ 
             \>NO=integer, MAPTABLE=logical, DELTAP=double;
}

The attributes are:
\begin{madlist}
  \ttitem{ICASE} user-defined dimensionality of the phase-space 
  (4, 5 or 6). \\ (Default: 4)

  \ttitem{NO} the order of the map.  \\ (Default: 1) 
    
  \ttitem{CLOSED\_ORBIT} a logical switch to turn on the closed orbit
  calculation.  \\ (Default: false) 

  \ttitem{DELTAP} relative momentum offset for reference closed
  orbit. \\ (Default: 0.0)

  \ttitem{MAPTABLE} a logical flag to activate the storage of the
  map-table in memory. \\ 
  \texttt{MAPTABLE=true} and \texttt{NO=1} creates the one-turn matrix which
  can be used by the next \hyperref[chap:ptc-twiss]{\texttt{PTC\_TWISS}}
  command.\\ (Default: false) 

  \ttitem{NORMAL} a logical flag to activate the calculation of the
  Normal Form. \\ (Default: false) 
\end{madlist}


\textbf{Example}\\
The simple example is located on the Web-page for the
\href{http://cern.ch/frs/mad-X_examples/ptc_normal}{\texttt{PTC\_NORMAL}
  example}. 


%\href{mailto:kapin@itep.ru}{  V.Kapin}(ITEP) and 
%\href{mailto:Frank.Schmidt@cern.ch}{  F.Schmidt}, March  2006

