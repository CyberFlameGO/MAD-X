%%\title{Touschek}
%  Changed by: Frank Zimmermann, 18-Jun-2002  
%  2013-Nov-29  15:32:13  ghislain: update of documentation following extensive changes in the code

\chapter{Touschek Lifetime and Scattering Rates}

The TOUSCHEK module computes the Touschek lifetime and the scattering
rates around a lepton or hadron storage ring, based on the formalism of
Piwinski in \cite{Piwinski1998} and his article on Touschek lifetime in
\cite{chaotigner1999}.

The syntax of the \texttt{TOUSCHEK} command is: 
\madbox{
TOUSCHEK, TOLERANCE=real, FILE=filename;
}

The arguments have the following meaning: 
\begin{madlist}
   \ttitem{TOLERANCE} the tolerance for the numerical integrator
   DGAUSS. (Default:~1.e-7) 
   \ttitem{FILE} The name of the output file (Default:~'touschek') 
\end{madlist}

\texttt{TOUSCHEK} should only be called after fully qualified BEAM
command and a TWISS command. One or several cavities with rf voltages
should be defined prior to calling TWISS and TOUSCHEK. 

{\bf Warning:} Calling EMIT between the TWISS and TOUSCHEK commands
leads to TOUSCHEK using wrong beam parameters, even if the BEAM command
is reiterated.  
 
The momentum acceptance is taken from the bucket size taking into
account the  energy loss per turn \textit{U0} from synchrotron
radiation. The value of \textit{U0} is computed from the second
synchrotron radiation integral \textit{synch\_2} in the TWISS summ table
(\textit{synch\_2} is calculated only when the TWISS option 'chrom' is
invoked), using Eq. (3.61) in \cite{sands1970}, which was
generalized to the case of several harmonic rf systems. 
If \textit{synch\_2} is zero, not defined or not calculated, zero energy
loss is assumed. In the case of several rf systems with nonzero
voltages, it is assumed that the lowest frequency system defines the
phase of the outer point on the separatrix when calculating the momentum
acceptance, and that all higher-harmonic systems are either in phase or
in anti-phase to the lowest frequency system. (Note: if a storage rings
really uses a different rf scheme, one would need to change the
acceptance function in the routine \textit{cavtousch0} for that ring.) 


Example: 
\madxmp{
BEAM, \=PARTICLE = PROTON, ENERGY = 450, NPART = 1.15e11, \\
      \>EX = 7.82E-9, EY = 7.82E-9, ET = 5.302e-5; \\
USE, PERIOD = LHCb1; \\
\ldots \\ 	
VRF = 400; \\
\ldots \\
SELECT, FLAG = TWISS, CLEAR; \\
TWISS, CHROM, TABLE, FILE; \\
\ldots \\
TOUSCHEK, FILE, TOLERANCE=1.e-8;
}

The first command defines the beam parameters. It is essential that the
longitudinal emittance is set. The command
\textit{USE} selects the beam line or sequence. The next command assign
a value to the cavity rf voltage vrf  (example name). The
\textit{SELECT} clear previous assignments to the \textit{TWISS} module,
\textit{TWISS} calculates and saves the values of all twiss parameters
for all elements in the ring; the \textit{TOUSCHEK} command computes the
Touschek lifetime and writes it to the file 'touschek' (default name).   

The results are stored in the \textit{TOUSCHEK} tables, and can be
written to a file (with the default name 'touschek' in the example
above), or values can be extracted from the table using the value
command as follows  
\madxmp{
VALUE, \=table(touschek,name), table(touschek,s), table(touschek,tli),\\
       \>table(touschek,tliw), table(touschek,tlitot); 
}

where 'name' denotes the name of a beamline element, \textit{S} the
position of the center of the element, \textit{TLI} the instantaneous
Touschek loss rate within the element, and \textit{TLIW} the
instantaneous rate weighted by the length of the element divided by the
circumference (its contribution to the total loss rate), and
\textit{TLITOT} the accumulated loss rate adding the rates over all
beamline elements through the present position. The value of
\textit{TLITOT} at the end of the beamline is the inverse of the
Touschek lifetime in units of 1/s. 

All results can also be printed to a file using the command 
\madxmp{WRITE, TABLE=touschek, FILE;}

The \madx Touschek module was developed by
\href{mailto:catia.milardi@lnf.infn.it}{Catia Milardi} and 
\href{mailto:frank.zimmermann@cern.ch}{Frank Zimmermann}.

The \madx Touschek module was partially rewritten in November 2013 by
\href{mailto:ghislain.roy@cern.ch}{Ghislain Roy}  
after the discovery of a few bugs in the original code:

The first bug concerned a numerical instability in the computation of
the B2 parameter as listed in Eq. 34 in \cite{Piwinski1998}.

The initial alogorithm implemented the calculation of square root of the
difference between two expressions. It turned out that the numerical
values of both expressions could sometimes be very large and nearly
equal.
  
Because of limited precision in floating point calculations, the
difference could sometimes lead to negative values  
and the square root returned an undefined value (NaN). 
The integrator then failed to compute the integral and returned a value
of zero, with the printing of a faintly related  
message that too high accuracy was required for integrator DGAUSS. 
The algorithm didn't stop there and the end result was that the
summation over all elements in the range was wrong 
and the end results were also wrong.

This bug was eliminated by evaluation the first expression in equation
34 which calculates directly the B2 factor by taking the square root of
the sum of two squares, hence ensuring that an instability of the same
kind cannot happen.  

Another problem was that in the original algorithm the inverse Touschek
lifetime was calculated by taking the average of the twiss parameters at
both ends of the element as input. The resulting set of parameters was
no longer consistent, resulting also in poor calculation. This has been
changed by calculating the inverse Touschek lifetime at specific points,  
always considering as input the Twiss parameters given by Twiss at a
single location. This provides at least very accurate results for the
TLI parameters. 

The integration over the length of the element is now done in different
ways, depending whether the preceding TWISS command calculated the Twiss
parameters at the end of the element or at the centre (CENTRE option of
TWISS).  

In the first case (calculation at the end of the element, CENTRE=false), 
the inverse Touschek lifetime (TLI) is calculated at the end of each
element. The weighted contribution of element $i$ to the total inverse
Touschek lifetime is then given by 
\madxmp{TLIW[i] = 0.5 * (TLI[i] + TLI[i-1]) * L[i] / CIRC}

In the second case (calculation at the center of the element,
CENTRE=true), the inverse Touschek lifetime (TLI) is also calculated at
the center of each element. The weighted contribution of element $i$ to
the total inverse Touschek lifetime is then given by 
\madxmp{TLIW[i] = TLI[i] * L[i] / CIRC}

Another bug that was uncovered in the original algorithm was that the
vertical dispersion was wrongly taken into account and mostly ignored:
DY and DPY were set uniformly to half the initial values for lack of
updating in the loop over elements.  

The new algorithms have been inserted in \madx as of version 5.01.04, a
development release dated early december 2013.  

%\\\href{http://consult.cern.ch/xwho/people/62690}{frankz}   11.03.2008 
