%%\title{Physical Elements and Markers}
%  Changed by: Chris ISELIN, 24-Jan-1997 
%  Changed by: Hans Grote, 25-Sep-2002

\chapter{Definition of Elements}
\label{chap:element-definition}

\section{Element Input Format}
\label{sec:element-input}

All physical elements are defined by statements of the form 
\madbox{
label: keyword \{,attribute\};
}
where 
\begin{madlist}
  \ttitem{label} is a name to be given to the element.
  \ttitem{keyword} is an element type keyword.  
  \ttitem{attribute} normally -- with exception for multipoles -- takes
  one of the two forms: 
    \madxmp{attribute-name \== attribute-value \\ 
    attribute-name \>:= attribute-value}
    \begin{madlist}
      \ttitem{attribute-name} selects the attribute,
      as defined for the element type keyword.   
      \ttitem{attribute-value} provides a value to the
      \texttt{attribute\_name}. The value may be specified by an
      expression. 
    \end{madlist}
    The "=" assigns the value on the right to
    the attribute at the time of definition, regardless of any
    further change of the right hand side; the ":=" re-evaluates
    the expression at the right every time the attribute is being
    used. For constant right hand sides, "=" and ":=" are of
    course equivalent.  
\end{madlist}

Omitted attributes are assigned a default value.

\textbf{Example:} 
\madxmp{QF: QUADRUPOLE, L=1.8, K1=0.015832;}

A special format is used for a
\hyperref[sec:multipole]{multipole}\index{multipole}:  
\madxmp{
M: MULTIPOLE, \= KN= {kn0, kn1, kn2, ..., knmax},\\
              \> KS= {ks0, ks1, ks2, ..., ksmax};
}
where \texttt{KN} and \texttt{KS} give the integrated normal and skew strengths,
respectively. The commas are mandatory, each strength can be an
expression; their position defines the order.\\
\textbf{Example:}  
\madxmp{M: MULTIPOLE, KN={0,0,0,myoct*lrad}, KS={0,0,0,0,-1.e-5};}
defines a multipole with a normal octupole component and a skew decapole
component.  

%%%\title{Editing Element Definitions}
%  Changed by: Chris ISELIN, 24-Jan-1997 

%  Changed by: Hans Grote, 25-Sep-2002 

%%\usepackage{hyperref}
% commands generated by html2latex


%%\begin{document}

\section{Editing Element Definitions}  An element definition can be changed in two ways: 
\begin{itemize}
	\item \textbf{Entering a new definition:} The element will be replaced in the main beam line expansion. 
	\item \textbf{Entering the element name together with new attributes:} The element will be updated in place, and the new attribute values will replace the old ones. 
\end{itemize} This example shows two ways to change the strength of a quadrupole: 
\begin{verbatim}

QF: QUADRUPOLE,L=1,K1=0.01;     ! Original definition of QF
QF: QUADRUPOLE,L=1,K1=0.02;     ! Replace whole definition of QF
QF,K1=0.02;                     ! Replace value of K1
\end{verbatim} When the type of the element remains the same, replacement of an attribute is the more efficient way. 

 Element definitions can be edited freely. The changes do not affect already defined objects which belong to its \href{elm_class.html}{element class}. 

\href{http://www.cern.ch/Hans.Grote/hansg_sign.html}{hansg}, January 24, 1997 

%%\end{document}

\section{Editing Element Definitions}  
\label{sec:element-editing}

An element definition can be changed in two ways: 
\begin{itemize}
   \item \textbf{Entering a new definition:} The element will be
     replaced in the main beam line expansion.  
   \item \textbf{Entering the element name together with new
     attributes:} The element will be updated in place, and the new
     attribute values will replace the old ones.  
\end{itemize} 

This example shows two ways to change the strength of a quadrupole: 
\madxmp{
xxxxxxxxxxxxxxxxxxxxxxxxxxxxxxxxxxxxx\= \kill
QF: QUADRUPOLE, L = 1, K1 = 0.01;    \>! Original definition of QF \\
\\
QF: QUADRUPOLE, L = 2, K1 = 0.02;    \>! Replace whole definition of QF \\
\\
QF, K1 = 0.03;                       \>! Replace value of K1 for QF\\
}
When the type of the element remains the same, replacement of an
attribute is the more efficient way.  

Element definitions can be edited freely. The changes only affect
already defined objects which belong to its
\hyperref[sec:element-classes]{element class} in case the: \madxmp{option, update\_from\_parent} is used.
In case this option is used all elements which belong to that class will be updated. 
An example is given in \hyperref[sec:element-classes]{element class}.   


%%%\title{Element Classes}
%  Changed by: Chris ISELIN, 24-Jan-1997 
%  Changed by: Hans Grote, 25-Sep-2002

\section{Element Classes}  
The concept of element classes solves the problem of addressing
instances of elements in the accelerator in a convenient manner. It will
first be explained by an example. All the quadrupoles in the accelerator
form a class QUADRUPOLE. Let us define three subclasses for the
focussing quadrupoles, the defocussing quadrupoles, and the skewed
quadrupoles:  
\begin{verbatim}
MQF: QUADRUPOLE, L = LQM, K1 = KQD;     ! Focussing quadrupoles
MQD: QUADRUPOLE, L = LQM, K1 = KQF;     ! Defocussing quadrupoles
MQT: QUADRUPOLE, L = LQT;               ! Skewed quadrupoles
\end{verbatim} 

These classes can be thought of as new keywords which define elements
with specified default attributes. We now use theses classes to define
the real quadrupoles:  
\begin{verbatim}
QD1: MQD;           ! Defocussing quadrupoles
QD2: MQD;
QD3: MQD;
 ...
QF1: MQF;           ! Focussing quadrupoles
QF2: MQF;
QF3: MQF;
 ...
QT1: MQT, K1S = KQT1;   ! Skewed quadrupoles
QT2: MQT, K1S = KQT2;
 ...
\end{verbatim} 

These quadrupoles inherit all unspecified attributes from their
class. This allows to build up a hierarchy of objects with a rather
economic input structure.  

The full power of the class concept is revealed when object classes are
used to select instances of elements for various purposes. Example:  
\begin{verbatim}
select, flag = twiss, class = QUADRUPOLE; ! Select all quadrupoles for the
                                          ! Twiss TFS file
\end{verbatim}

More formally, for each element keyword MAD maintains a class of
elements with the same name. A defined element becomes itself a class
which can be used to define new objects, which will become members of
this class. A new object inherits all attributes from its class; but its
definition may override some of those values by new ones. All class and
object names can be used in range selections, providing a powerful
mechanism to specify positions for matching constraints and printing.  

 %\href{http://www.cern.ch/Hans.Grote/hansg_sign.html}{hansg}, January 24, 1997 

\section{Element Classes}  
\label{sec:element-classes}
The concept of element classes solves the problem of addressing
instances of elements in the accelerator in a convenient manner. 

It will first be explained by an example. All the quadrupoles in the
accelerator form a class QUADRUPOLE. Let us define three subclasses for the
focussing quadrupoles, the defocussing quadrupoles, and the skewed
quadrupoles:  
\madxmp{
xxxxxxxxxxxxxxxxxxxxxxxxxxxxxxxxxxxxxx\= \kill
MQF: QUADRUPOLE, L = LQM, K1 = KQD;     \>! Focussing quadrupoles \\
MQD: QUADRUPOLE, L = LQM, K1 = KQF;     \>! Defocussing quadrupoles \\
MQT: QUADRUPOLE, L = LQT;               \>! Skewed quadrupoles
}

These classes can be thought of as new keywords which define elements
with specified default attributes. We now use these classes to define
the real quadrupoles:  
\madxmp{
xxxxxxxxxxxxxxxxxxxxxxx\= \kill
QD1: MQD;              \>! Defocussing quadrupoles \\
QD2: MQD; \\
QD3: MQD; \\
 ... \\
QF1: MQF;              \>! Focussing quadrupoles \\
QF2: MQF; \\
QF3: MQF; \\
 ... \\
QT1: MQT, K1S = KQT1;   \> ! Skewed quadrupoles \\
QT2: MQT, K1S = KQT2; \\
 ...
}
These quadrupoles inherit from their class all attributes that are not
explicitly specified at time of declaration. 
This allows to build up a hierarchy of objects with a rather
economic input structure.  

The full power of the class concept is revealed when object classes are
used to select instances of elements for various purposes. Example:  


\madxmp{
SELECT, FLAG=twiss, CLASS = QUADRUPOLE; \=! Select all quadrupoles for the\\
                                        \>! Twiss TFS file
}

More formally, for each element keyword \madx maintains a class of
elements with the same name. A defined element becomes itself a class
which can be used to define new objects, which will become members of
this class. A new object inherits all attributes from its class; but its
definition may override some of those values by new ones. All class and
object names can be used in range selections, providing a powerful
mechanism to specify positions for matching constraints and printing.

The default option is that if you change the parent of an element you will
not change the children of that class. Let's see in the following example how 
it works:
\madxmp{ mb: sbend,l:=lenseq,  angle =0.001; \\
 mb2: mb;
 }
 This will create an element \texttt{mb} that then is used as the parent for \texttt{mb2}. If we now do want to give mb
 an aperture we can do the following:
 \madxmp{ apertype=ellipse, aperture:={ 1,4 }; }

 This, however, does not modify \texttt{mb2}. In case we want \texttt{mb2} to also inherent the change from \texttt{mb} we need to use the
 flag \madxmp{option, update\_from\_parent}. This should be used with care and is only changing the direct parent.
%% EOF
