%%\title{PTC Set-up Parameters}

\chapter{{\madx}-{\ptc} Auxiliaries}
\label{chap:ptc_auxiliaries}

This chapter documents the interface between \madx and \ptc and the
auxiliary commands available in the \ptc library.

{\bf Available Commands }
\begin{itemize}
   \item \hyperref[sec:ptc_setswitch]{\tt PTC\_SETSWITCH}
   \item \hyperref[sec:ptc_knob]{\tt PTC\_KNOB}
   \item \hyperref[sec:ptc_setknobvalue]{\tt PTC\_SETKNOBVALUE}
   \item \hyperref[sec:match_withptcknobs]{\tt MATCH\_WITHPTCKNOBS}
     \qquad {\bf (Under Construction)} 
   \item \hyperref[sec:ptc_printparametric]{\tt PTC\_PRINTPARAMETRIC}
   \item \hyperref[sec:ptc_eplacement]{\tt PTC\_EPLACEMENT}
   \item \hyperref[sec:ptc_printframes]{\tt PTC\_PRINTFRAMES}
   \item \hyperref[sec:ptc_select]{\tt PTC\_SELECT}
   \item \hyperref[sec:ptc_selectmoment]{\tt PTC\_SELECTMOMENT}
   \item \hyperref[sec:ptc_moments]{\tt PTC\_MOMENTS} \qquad {\bf (Under Construction)} 
   \item \hyperref[sec:ptc_dumpmaps]{\tt PTC\_DUMPMAPS}
   \item \hyperref[sec:ptc_setcavities]{\tt PTC\_SETCAVITIES}
\end{itemize}

\newpage

% add other files to the end of this file
%%\documentclass[a4paper,11pt]{article}
%%\usepackage{ulem}
%%\usepackage{a4wide}
%%\usepackage[dvipsnames,svgnames]{xcolor}
%%\usepackage[pdftex]{graphicx}
%%\title{PTC\_SETSWITCH}
%%\usepackage{hyperref}
% commands generated by html2latex


%%\begin{document}
%%\begin{center}
   %%EUROPEAN ORGANIZATION FOR NUCLEAR RESEARCH   
%%\includegraphics{http://cern.ch/madx/icons/mx7_25.gif}

\section{PTC\_SETSWITCH}

\subsection{ routine that sets the internal PTC switches }
%%\end{center}
%   ##########################################################              

%   ##########################################################              

%   ##########################################################              


\subsubsection{   USER MANUAL   }
%   ##########################################################              


\paragraph{SYNOPSIS}
\begin{verbatim}

PTC_SETSWITCH,
debuglevel = [i,0], 
maxacceleration = [l, true, true],
exact_mis = [l, false, true],
totalpath = [l, false, true],
radiation = [l false, true],
fringe = [l, false, true],
time = [l, true, true];
\end{verbatim}
%   ##########################################################              


\textbf{ Description } \\

Using this command the user can set switches of PTC and the MAD-X-PTC interface, adapting  this way the program behavior to his needs.  

\paragraph{ Command parameters and switches }
\begin{description}
	\item[\textbf{debuglevel}] \textit{ integer, default value 1, no default value if value explicitly not specified}

 Sets the level of debugging printout 0 none, 4 everything   
	\item[\textbf{maxacceleration}] \textit{ logical, default true, if value explicitly not specified then true }

 Switch saying to set cavities phases so the reference orbit is always on the crest, i.e. gains max energy   
	\item[\textbf{ exact\_mis }] \textit{ logical, default false, true if value explicitly specified }

 Switch ensures exact misalignment treatment.   
	\item[\textbf{ totalpath }] \textit{ logical, default false, true if value explicitly specified }

 If true, the 6th variable of PTC, i.e. 5th of MAD-X is the total path.       If false it is deviation from the reference particle, which is normally the closed orbit for closed       layouts.   
	\item[\textbf{ radiation }] \textit{ logical, default false, if value explicitly not specified then true}:    

 Sets the radiation switch/internal state of PTC.   
	\item[\textbf{ fringe }] \textit{ logical, default false, true if value explicitly specified}:    

 Sets the fringe switch/internal state of PTC. If true the influence of the fringe fields is evaluated for       all the elements.    

 Please note that currently fringe fields are always taken to the account for some elements       (f.g. traveling wave cavities) even if this flag is set to false.        The more detailed list of the elements will be provided later,        when the situation in this matter will be definitely settled.   
	\item[\textbf{ time }] \textit{ logical, default true, if value explicitly not specified then true }:    

 If true, Selects time of flight rather than path length. (\textit{cT} to be precise)       as the 6th variable of PTC, i.e. 5th of MAD-X.   
\end{description}
%  ############################################################ 

%  ############################################################ 

%  ############################################################ 


\subsubsection{ PROGRAMMERS MANUAL }   Values of the switches are stored in Fortran 90 module mad\_ptc\_intstate (mad\_ptc\_intstate.f90). The command is processed by pro\_ptc\_setswitch C function in file madxn.c It call an appropriate routines of the Fortran module to set each of the switches: 
\begin{itemize}
	\item  ptc\_setdebuglevel 
	\item  ptc\_setaccel\_method 
	\item  ptc\_setexactmis 
	\item  ptc\_setradiation 
	\item  ptc\_settotalpath 
	\item  ptc\_settime 
	\item  ptc\_setfringe  
\end{itemize}

%%\end{document}

%%\title{PTC\_SETSWITCH}

\section{PTC\_SETSWITCH}
\label{sec:ptc_setswitch}

Routine that sets the internal \ptc switches.

\madbox{
PTC\_SETSWITCH, \=DEBUGLEVEL=integer, \\
                \>MAXACCELERATION=logical,\\
                \>EXACT\_MIS=logical,\\
                \>TOTALPATH=logical,\\
                \>RADIATION=logical,\\
                \>FRINGE=logical,\\
                \>TIME=logical;
}

Using this command the user can set switches of \ptc and the {\madx}-{\ptc}
interface, adapting this way the program behavior to his needs.   

{\bf Command parameters and switches}
\begin{madlist}
   \ttitem{DEBUGLEVEL} (Default: 1)\\
     Sets the level of debugging printout: 0 prints none, 4 prints everything   

   \ttitem{MAXACCELERATION} (Default: true)\\
     Switch to set cavities phases so the reference orbit is always on
     the crest, i.e. gains max energy    

   \ttitem{EXACT\_MIS} (Default: .false.)\\
     Switch ensures exact misalignment treatment.   

   \ttitem{TOTALPATH} (Default: .false.)\\
     If true, the 6th variable of PTC, i.e. 5th of MAD-X, is the total
     path.  \\
     If false it is deviation from the reference particle,
     which is normally the closed orbit for closed layouts.    

   \ttitem{RADIATION} (Default: false)\\    
     Sets the radiation switch/internal state of PTC.   

   \ttitem{FRINGE} (Default: false)\\    
     Sets the fringe switch/internal state of PTC. \\ 
     If true the influence of the fringe fields is evaluated for all
     elements. \\       
     Please note that currently fringe fields are always taken into
     account for some elements (e.g. traveling wave cavities) even if
     this flag is set to false. The detailed list of elements
     will be provided later, when the situation in this matter will be
     definitely settled.    

   \ttitem{TIME} (Default: true)\\  
     If true, Selects time of flight (\textit{cT} to be precise) rather
     than path length as the 6th variable of PTC, i.e. 5th of MAD-X.     

\end{madlist}


{\bf PROGRAMMERS MANUAL}   
Values of the switches are stored in Fortran 90 module
mad\_ptc\_intstate (mad\_ptc\_intstate.f90). The command is processed by
pro\_ptc\_setswitch C function, in file madxn.c, that calls appropriate
routines of the Fortran module to set each of the switches:   
\begin{itemize}
   \item  ptc\_setdebuglevel 
   \item  ptc\_setaccel\_method 
   \item  ptc\_setexactmis 
   \item  ptc\_setradiation 
   \item  ptc\_settotalpath 
   \item  ptc\_settime 
   \item  ptc\_setfringe  
\end{itemize}



%%\documentclass[a4paper,11pt]{article}
%%\usepackage{ulem}
%%\usepackage{a4wide}
%%\usepackage[dvipsnames,svgnames]{xcolor}
%%\usepackage[pdftex]{graphicx}
%%\title{PTC\_KNOB}
%%\usepackage{hyperref}
% commands generated by html2latex


%%\begin{document}
%%\begin{center}
   %%EUROPEAN ORGANIZATION FOR NUCLEAR RESEARCH   
%%\includegraphics{http://cern.ch/madx/icons/mx7_25.gif}

\section{PTC\_KNOB}
%%\end{center}
%   ##########################################################              

%   ##########################################################              

%   ##########################################################              

%   ##########################################################              


\subsubsection{   USER MANUAL   }
%   ##########################################################              


\paragraph{SYNOPSIS}
\begin{verbatim}

PTC_KNOB, 
elementname = [s, none] , 
kn    = [i, {-1}], 
ks    = [i, {-1}], 
exactmatch = [l, true, true] ; 

\end{verbatim}
%   ##########################################################              


\textbf{ Description }\\

 Sets knobs in PTC calculations (currently ony in PTC\_TWISS, PTC\_NORMAL will follow). Knobs appear as the additional parameters of the phase space. Twiss functions are then obtained  as functions of these parameters (Taylor series).  Also map elements might be stored as functions of knobs, see  \href{PTC_Select.html}{ ptc\_select} command description to lear how to request given element to be stored as a Taylor series.  The parametric results can be further  \\
 
\begin{enumerate}
	\item  written to a file with \href{PTC_PrintParametric.html}{ptc\_printparametric}.
	\item  plotted and studied using rviewer command (rplot plugin). 
	\item  used to obtain very quickly approximate values of lattice functions for given values of knobs     (\href{PTC_SetKnobValue.html}{ptc\_setknobvalue}).     This feature is the foundation of a fast matching alorithm with PTC.     
\end{enumerate}

\textbf{ Example }\\

\href{http://cern.ch/frs/mad-X_examples/ptc_madx_interface/knobs/knobs.madx}{ dog leg chicane  } : Dipolar components of both rbends and dipolar and quadrupolar components of the focusing quads set as knobs. Some first and second order map coefficients set to be stored as parametric results. ptc\_twiss command is performed and the parametric results are written to files in two formats.  \href{http://cern.ch/frs/mad-X_examples/ptc_madx_interface/matchknobs/matchknobs.madx}{ dog leg chicane  } : Knobs values are matched to get requested lattice functions.
%   ##########################################################              

\paragraph{ Command parameters and switches }
\begin{description}
	\item \textbf{ elementname } \textit{ string in range format, }

 Specifies name of the element containing the knob(s) to be set.   
	\item \textbf{ kn,ks } \textit{ list of integers,   }

 Defines which order    
	\item \textbf{ exactmatch }
    \textbf{\textit{ logical, default true, if value explicitly not specified then true}\\\\
 Normally a knob is a property of a single element in a layout.        The specified name must match 1:1 to an element name. This is the case       when exactmatch is true.\\\\ 
 Knobs might be also set to all family of elements. In such case        the exactmatch switch must be false. A given order field component of all the elements        that name starts with the name specified by the user become a single knob.   }
	\item \textbf{initial}
\end{description}

\textbf{\_  }
%  ############################################################ 

%  ############################################################ 

%  ############################################################ 

% 
% <h3> PROGRAMMERS MANUAL </h3>
% 
% <p> 
% The command is implemented pro_ptc_knob function in madxn.c and 
% by subroutine xxxx in madx_ptc_xxx.f90.
% <p>
% Sopecified range is resolved with help of get_range command. Number of the element in the current sequence
% is resolved and passed as the parameter to the fortran routine. It allows to resolve uniquely the corresponding
% element in the PTC layout.
% <p>
% 
% 


%%\end{document}

%%\title{PTC\_KNOB}

\section{PTC\_KNOB}
\label{sec:ptc_knob}

\madbox{
PTC\_KNOB, \=ELEMENTNAME=string, \\
           \>KN=integer\{, integer\}, KS=integer\{, integer\}, \\
           \>EXACTMATCH=logical;
}

Sets knobs in \ptc calculations. This is currently valid only in
{\tt  PTC\_TWISS}, {\tt PTC\_NORMAL} will follow. 

Knobs appear as the additional parameters of
the phase space. Twiss functions are then obtained  as functions of
these parameters (Taylor series).  Also map elements might be stored as
functions of knobs, see \hyperref[sec:ptc_select]{\tt PTC\_SELECT} command
description to learn how to request given element to be stored as a
Taylor series.  \\

Further, the parametric results can be: 
\begin{enumerate}
   \item  written to a file with
     \hyperref[ptc_printparametric]{\tt PTC\_PRINTPARAMETRIC}. 
   \item  plotted and studied using rviewer command
     (\hyperref[sec:rplot]{\tt RPLOT} plugin). 
   \item  used to obtain very quickly approximate values of lattice
     functions for given values of knobs
     (\hyperref[sec:ptc_setknobvalue]{\tt PTC\_SETKNOBVALUE}). This
     feature is the foundation of a fast matching algorithm with \ptc.      
\end{enumerate}


{\bf Command parameters and switches}
\begin{madlist}
   \ttitem{ELEMENTNAME} a string in range format (Default: NULL)\\
     Specifies name of the element containing the knob(s) to be set.   
   \ttitem{KN,KS} list of integers (Default: ???)\\
     Defines which order 
   \ttitem{EXACTMATCH} (Default: .true.)\\
     Normally a knob is a property of a single element in a layout.
     The specified name must match 1:1 to an element name. This is the
     case when exactmatch is true.\\  
     Knobs might be also set to all family of elements. In such case
     the exactmatch switch must be false. A given order field
     component of all the elements that name starts with the
     name specified by the user become a single knob.
   \ttitem{INITIAL}
\end{madlist}


{\bf Example}\\
\href{http://cern.ch/frs/mad-X_examples/ptc_madx_interface/knobs/knobs.madx}{dog
  leg chicane}: Dipolar components of both rbends and dipolar and
quadrupolar components of the focusing quads set as knobs. Some first
and second order map coefficients set to be stored as parametric
results. ptc\_twiss command is performed and the parametric results are
written to files in two formats. 

\href{http://cern.ch/frs/mad-X_examples/ptc_madx_interface/matchknobs/matchknobs.madx}{dog
  leg chicane}: Knob values are matched to get requested lattice
functions.  


 
% <h3> PROGRAMMERS MANUAL </h3>
% 
% <p> 
% The command is implemented pro_ptc_knob function in madxn.c and 
% by subroutine xxxx in madx_ptc_xxx.f90.
% <p>
% Sopecified range is resolved with help of get_range command. Number of the element in the current sequence
% is resolved and passed as the parameter to the fortran routine. It allows to resolve uniquely the corresponding
% element in the PTC layout.
% <p>



%%%\title{PTC\_SETKNOBVALUE}

\section{PTC\_SETKNOBVALUE}

\begin{verbatim}
PTC_SETKNOBVALUE, 
     elementname = [s, none] , 
     kn    = [i, {-1}], 
     ks    = [i, {-1}], 
     value = [r] ; 
\end{verbatim}

With this command the user set a given knob value. In its effect all the values in 
\begin{itemize}
   \item  the twiss table used by the last ptc\_twiss command 
   \item  the columns specified with
     \href{PTC_Select.html}{ptc\_select}, parametric=true; 
\end{itemize} 
are reevaluated using the buffered parametric results.  

The parameters of the command basically contains the fields that allow
to identify uniquely the knob and the value to be set.

{\bf Command parameters and switches}
\begin{itemize}
   \item {\bf elementname}=string in range format (Default: NULL)\\
     Specifies name of the element containing the knob to be set.   

   \item {\bf kn,ks}=list of integers (Default: ???)\\
     Defines the knob   

   \item {\bf value}=real (Default: 0)\\
     Specifies the value the knob is set to.             
\end{itemize}

{\bf Example }

\href{http://cern.ch/frs/mad-X_examples/ptc_madx_interface/matchknobs/matchknobs.madx}{dog
  leg chicane}: strength of dipole field component in quadrupoles is
matched to obtain the required R56 value.    


% <h3> PROGRAMMERS MANUAL </h3>
% 
% <p> 
% The command is implemented pro_PTC_SETKNOBVALUE function in madxn.c and 



%%\title{PTC\_SETKNOBVALUE}

\section{PTC\_SETKNOBVALUE}

The {\tt PTC\_SETKNOBVALUE} command sets a given knob value.

\madbox{
PTC\_SETKNOBVALUE, \=ELEMENTNAME=string, \\
                   \>KN=integer\{,integer\}, \\
                   \>KS=integer\{,integer\}, \\
                   \>VALUE=real;
}

All values in the twiss table used by the last {\tt PTC\_TWISS} command 
and the columns specified with \hyperref[ptc_select]{\tt
  PTC\_SELECT}{\tt, PARAMETRIC=true;}  are reevaluated using the
buffered parametric results.   

The parameters of the command basically contains the fields that allow
to identify uniquely the knob and the value to be set.

{\bf Command parameters and switches}
\begin{madlist}
   \ttitem{ELEMENTNAME} a string in range format that specifies the name
   of the element containing the knob to be set. \\
     (Default: none) 
   \ttitem{KN, KS} are lists of integers that define the knob.\\
   (Default: -1)   
   \ttitem{VALUE} specifies the value to which the knob is to be set.\\
   (Default: 0)\\
\end{madlist}

{\bf Example:}\\
\href{http://cern.ch/frs/mad-X_examples/ptc_madx_interface/matchknobs/matchknobs.madx}{dog
  leg chicane}: strength of dipole field component in quadrupoles is
matched to obtain the required R56 value.    


% <h3> PROGRAMMERS MANUAL </h3>
% 
% <p> 
% The command is implemented pro_PTC_SETKNOBVALUE function in madxn.c and 





%%%\title{Match using PTC knobs}

\section{PTC\_VARYKNOBS: Matching with PTC knobs}

This matching procedure takes advantage of the parametric results that
are accessible with PTC. Namely, parameters occuring in the matching
constrains are obtained as functions (polynomials) of the matching
variables. In other words, each variable is a knob in PTC
calculation. Evaluation of the polynomials is relatively fast comparing
to the regular PTC calculation which makes findinng the minimum with the
parametrized constraints very fast.  

However, the algorithm is not faster in a general case: 
\begin{enumerate}
   \item  The calculation time dramatically increases with the number of
     parameters and at some point penalty rising from this overcomes the
     gain we get from the fast polynomial evaluation.    
   \item  A parametric result is an approximation that is valid only
     around the nominal parameter values.     
\end{enumerate}

The algorithm is described below. \\
 
\begin{verbatim}
MATCH, use_ptcknobs=true;
...
PTC_VARYKNOB: 
  initial = [s, none] , 
  element = [s, none] , 
  kn    = [i, -1], 
  ks    = [i, -1], 
  exactmatch = [l, true, true], 
  trustrange    = [r, 0.1],  
  step     = [r, 0.0], 
  lower    = [r, -1.e20],
  upper    = [r,  1.e20]; 
...
END_MATCH;
\end{verbatim}

For user convenience the limits are specified in the MAD-X units (k1,k2,
etc). This also applies to dipolar field where the user must specify
limits of k0=angle/path\_lengh. This guarantees concistency in
treatment of normal and skew dipole components.   

Important: Note that inside the code skew magnets are represented only
by  normal component and tilt, so the nominal skew component is always
zero.  Inside PTC tilt can not become a knob, while skew component can.
Remember about this fact when setting the limits of skew components in
the matching.  When the final results are exported back to MAD-X, they
are converted back to the "normal" state, so the nominal skew compoment
is zero and tilt and  normal component are modified accordingly.     

trustrange - defines the range the expansion is trusted \\



{\bf Example}\\
\href{http://cern.ch/frs/mad-X_examples/ptc_madx_interface/matchknobs/.madx}{dog leg chicane}.


{\bf Algorithm}\\
\begin{enumerate}
   \item Buffer the key commands (ptc\_varyknob, constraint,
     ptc\_setswitch, ptc\_twiss or ptc\_normal, etc) appearing between 
     match, useptcknobs=true; and any of matching actions calls
     (migrad,lmdif,jacobian, etc) 
   \item  When matching action appears,  
     \begin{enumerate}
       \item set "The Current Variables Values" (TCVV) to zero      
       \item perform THE LOOP, i.e. points 3-17 
     \end{enumerate}
   \item Prepare PTC environment (ptc\_createuniverse,
     ptc\_createlayout)  
   \item Set the user defined knobs (with ptc\_knob).  
   \item Set TCVV using ptc\_setfieldcomp command.  
   \item Run a PTC command (twiss or normal).  
   \item Run a runtime created script that performs a standard matching;
     all the user defined knobs are variables of this matching.  
   \item Evaluate constraints expressions to get the matching function
     vector (I). 
   \item Add the matched values to TCVV. 
   \item End PTC session (run ptc\_end). 
   \item If the matched values are not close enough to zeroes then goto 3.
   \item Prepare PTC environment (ptc\_createuniverse,
     ptc\_createlayout). 
   \item Set TCVV using ptc\_setfieldcomp command.
     \\   ( --- please note that knobs are not set in this case  -- )  
   \item Run a PTC command (twiss or normal).
   \item Evaluate constraints expressions to get the matching function
     vector (II). 
   \item Evaluate a penalty function that compares matching function
     vectors (I) and (II).\\     See points 7 and 14.
   \item If the matching function vectors are not similar to each other
     within requested precision then goto 3. 
   \item Print TCVV, which are the matched values. 
\end{enumerate}


 
% <h3> PROGRAMMERS MANUAL </h3>
% 
% <p> 
% The command is implemented pro_PTC_SETKNOBVALUE function in madxn.c and 
% 

%%\title{Match using PTC knobs}

\section{PTC\_VARYKNOBS (Under Construction)}
\label{sec:ptc_varyknobs}

The {\tt PTC\_VARYKNOBS} command allows matching with \ptc knobs.

\madbox{
PTC\_VARYKNOB, \=INITIAL=string, ELEMENT=string, \\
               \>KN=integer, \{integer,\} KS=integer, \{integer,\} \\
               \>EXACTMATCH=logical, TRUSTRANGE=real, \\
               \>STEP=real, LOWER=real, UPPER=real;
}

where the attributes are
\begin{madlist}
  \ttitem{INITIAL}
  \ttitem{ELEMENT}
  \ttitem{KN, KS}
  \ttitem{EXACTMATCH}
  \ttitem{TRUSTRANGE} defines the range over which the expansion is
  trusted \\ (Default:~0.1)
  \ttitem{STEP}
  \ttitem{LOWER}
  \ttitem{UPPER}
\end{madlist}

This matching procedure takes advantage of the parametric results that
are accessible with \ptc. Namely, parameters occuring in the matching
constrains are obtained as functions (polynomials) of the matching
variables. In other words, each variable is a knob in \ptc
calculation. Evaluation of the polynomials is relatively fast comparing
to the regular \ptc calculation which makes finding the minimum with the
parametrized constraints very fast.  

However, the algorithm is not faster in a general case: 
\begin{enumerate}
   \item  The calculation time dramatically increases with the number of
     parameters and at some point penalty rising from this overcomes the
     gain we get from the fast polynomial evaluation.    
   \item  A parametric result is an approximation that is valid only
     around the nominal parameter values.     
\end{enumerate}

The algorithm is described below. \\
 
\begin{verbatim}
MATCH, use_ptcknobs=true;
...
PTC_VARYKNOB: 
  initial = [s, none] , 
  element = [s, none] , 
  kn    = [i, -1], 
  ks    = [i, -1], 
  exactmatch = [l, true, true], 
  trustrange    = [r, 0.1],  
  step     = [r, 0.0], 
  lower    = [r, -1.e20],
  upper    = [r,  1.e20]; 
...
END_MATCH;
\end{verbatim}

For user convenience the limits are specified in the MAD-X units (k1,k2,
etc). This also applies to dipolar field where the user must specify
limits of k0=angle/path\_lengh. This guarantees concistency in
treatment of normal and skew dipole components.   

Important: Note that inside the code skew magnets are represented only
by  normal component and tilt, so the nominal skew component is always
zero.  Inside PTC tilt can not become a knob, while skew component can.
Remember about this fact when setting the limits of skew components in
the matching.  When the final results are exported back to MAD-X, they
are converted back to the "normal" state, so the nominal skew compoment
is zero and tilt and  normal component are modified accordingly.     





{\bf Example}\\
\href{http://cern.ch/frs/mad-X_examples/ptc_madx_interface/matchknobs/.madx}{dog leg chicane}.


{\bf Algorithm}\\
\begin{enumerate}
   \item Buffer the key commands (ptc\_varyknob, constraint,
     ptc\_setswitch, ptc\_twiss or ptc\_normal, etc) appearing between 
     match, useptcknobs=true; and any of matching actions calls
     (migrad,lmdif,jacobian, etc) 
   \item  When matching action appears,  
     \begin{enumerate}
       \item set "The Current Variables Values" (TCVV) to zero      
       \item perform THE LOOP, i.e. points 3-17 
     \end{enumerate}
   \item Prepare PTC environment (ptc\_createuniverse,
     ptc\_createlayout)  
   \item Set the user defined knobs (with ptc\_knob).  
   \item Set TCVV using ptc\_setfieldcomp command.  
   \item Run a PTC command (twiss or normal).  
   \item Run a runtime created script that performs a standard matching;
     all the user defined knobs are variables of this matching.  
   \item Evaluate constraints expressions to get the matching function
     vector (I). 
   \item Add the matched values to TCVV. 
   \item End PTC session (run ptc\_end). 
   \item If the matched values are not close enough to zeroes then goto 3.
   \item Prepare PTC environment (ptc\_createuniverse,
     ptc\_createlayout). 
   \item Set TCVV using ptc\_setfieldcomp command.
     \\   ( --- please note that knobs are not set in this case  -- )  
   \item Run a PTC command (twiss or normal).
   \item Evaluate constraints expressions to get the matching function
     vector (II). 
   \item Evaluate a penalty function that compares matching function
     vectors (I) and (II).\\     See points 7 and 14.
   \item If the matching function vectors are not similar to each other
     within requested precision then goto 3. 
   \item Print TCVV, which are the matched values. 
\end{enumerate}


 
% <h3> PROGRAMMERS MANUAL </h3>
% 
% <p> 
% The command is implemented pro_PTC_SETKNOBVALUE function in madxn.c and 
% 


%%%\title{PTC\_KNOB}

\section{PTC\_PRINTPARAMETRIC}

This command exists but is not documented. The file
PTC\_PrintParametric.html contains the same content as PTC\_Knob.html. I
presume the original file has been lost and overwritten...
 
%%\title{PTC\_KNOB}

\section{PTC\_PRINTPARAMETRIC}
\label{sec:ptc_printparametric}

{\bf Editor's Note:} This command exists but is not documented. The file 
PTC\_PrintParametric.html contains the same content as PTC\_Knob.html. I
presume the original file has been lost and overwritten...


%%%\title{PTC\_EPLACEMENT}

\section{PTC\_EPLACEMENT}

\begin{verbatim}
PTC_EPLACEMENT, 
   range = [s, none],
   x     = [r, 0],    y = [r, 0],    z = [r, 0],
   phi   = [r, 0],    theta = [r, 0], 
   onlyposition    = [l, false, true] ,
   onlyorientation = [l, false, true] ,
   autoplacedownstream = [l, true, true] ,
   refframe = [s, gcs] ; 
\end{verbatim}


Places a given element at required position and orientation.  All
rotations are made around the front face of the element.

{\bf  Command parameters and switches }
\begin{itemize}
   \item {\bf range}=string in range format (Default: ???)\\
     Specifies name of the element to be moved.   

   \item {\bf x,y,z}=real (Default: 0.0)\\
     Coordinate of the front face of the magnet.   

   \item {\bf phi, theta}=real (Default: 0.0)\\
     polar (in xz plane, around z axis) and azimuthal (around x axis)
     rotation angles, respectively.   
     
   \item {\bf refframe}=string (Default: gcs)\\
     Defines the coordinate system with respect to which coordinates and
     angles are specified. \\
     Possible values are:       
     \begin{itemize}
        \item[gcs]  global coordinate system 
        \item[current]   current position
        \item[previouselement]  end face of the previous element 
     \end{itemize}

   \item {\bf onlyposition}=logical (Default: .false.)\\
     If true, only translation are performed and orientation of the
     element is not changed.    

   \item {\bf onlyorientation}=logical (Default: .false.)\\
     If true, only rotations are performed and position of the element
     is not changed.    

   \item {\bf autoplacedownstream}=logical (Default: .true.)\\
     if true all elements downstream are placed at default positions
     with respect to the moved element, \\
     if false the rest of the layout stays untouched.    

   \item {\bf surveyall}=logical  (Default: .true.)\\
     If true, survey of all the line is performed after element
     placement at new position and orientation. It is implemented
     mainly for the software debugging purposes. If patching was
     performed correctly, the global survey should not change anything.     
\end{itemize}

\textbf{Example }\\

\href{http://cern.ch/frs/mad-X_examples/ptc_madx_interface/eplacement/chicane.madx}{Dog
  leg chicane}: postion of quadrupoless is matched to obtain required
R566 value.   


{\bf PROGRAMMER'S MANUAL}

The command is implemented pro\_ptc\_eplacement function in madxn.c and
by subroutine ptc\_eplacement() in madx\_ptc\_eplacement.f90.  

Specified range is resolved with help of get\_range command. Number of
the element in the current sequence is resolved and passed as the
parameter to the fortran routine. It allows to resolve uniquely the
corresponding element in the PTC layout.  

TRANSLATE\_Fibre and ROTATE\_Fibre routines of ptc are employed to place
and orient an element in space. These commands adds rotation and
translation from the current position. Hence, if the specified reference
frame is other then "current", the element firstly needs to be placed at
the center of the reference frame and then it is moved about the user
specified coordinates.   

After element placement at new position and orientation patch needs to
be recomputed. If autoplacedownstream is false then patch to the next
element is also recomputed. Otherwise, the layout is surveyed from the
next element on, what places all the elements downstream with default
position with respect to the moved element.  

At the end all the layout is surveyed, if surveyall flag is true, what
normally should always take place.      



%%\title{PTC\_EPLACEMENT}

\section{PTC\_EPLACEMENT}

\begin{verbatim}
PTC_EPLACEMENT, 
   range = [s, none],
   x     = [r, 0],    y = [r, 0],    z = [r, 0],
   phi   = [r, 0],    theta = [r, 0], 
   onlyposition    = [l, false, true] ,
   onlyorientation = [l, false, true] ,
   autoplacedownstream = [l, true, true] ,
   refframe = [s, gcs] ; 
\end{verbatim}


Places a given element at required position and orientation.  All
rotations are made around the front face of the element.

{\bf  Command parameters and switches }
\begin{itemize}
   \item {\bf range}=string in range format (Default: ???)\\
     Specifies name of the element to be moved.   

   \item {\bf x,y,z}=real (Default: 0.0)\\
     Coordinate of the front face of the magnet.   

   \item {\bf phi, theta}=real (Default: 0.0)\\
     polar (in xz plane, around z axis) and azimuthal (around x axis)
     rotation angles, respectively.   
     
   \item {\bf refframe}=string (Default: gcs)\\
     Defines the coordinate system with respect to which coordinates and
     angles are specified. \\
     Possible values are:       
     \begin{itemize}
        \item[gcs]  global coordinate system 
        \item[current]   current position
        \item[previouselement]  end face of the previous element 
     \end{itemize}

   \item {\bf onlyposition}=logical (Default: .false.)\\
     If true, only translation are performed and orientation of the
     element is not changed.    

   \item {\bf onlyorientation}=logical (Default: .false.)\\
     If true, only rotations are performed and position of the element
     is not changed.    

   \item {\bf autoplacedownstream}=logical (Default: .true.)\\
     if true all elements downstream are placed at default positions
     with respect to the moved element, \\
     if false the rest of the layout stays untouched.    

   \item {\bf surveyall}=logical  (Default: .true.)\\
     If true, survey of all the line is performed after element
     placement at new position and orientation. It is implemented
     mainly for the software debugging purposes. If patching was
     performed correctly, the global survey should not change anything.     
\end{itemize}

\textbf{Example }\\

\href{http://cern.ch/frs/mad-X_examples/ptc_madx_interface/eplacement/chicane.madx}{Dog
  leg chicane}: postion of quadrupoless is matched to obtain required
R566 value.   


{\bf PROGRAMMER'S MANUAL}

The command is implemented pro\_ptc\_eplacement function in madxn.c and
by subroutine ptc\_eplacement() in madx\_ptc\_eplacement.f90.  

Specified range is resolved with help of get\_range command. Number of
the element in the current sequence is resolved and passed as the
parameter to the fortran routine. It allows to resolve uniquely the
corresponding element in the PTC layout.  

TRANSLATE\_Fibre and ROTATE\_Fibre routines of ptc are employed to place
and orient an element in space. These commands adds rotation and
translation from the current position. Hence, if the specified reference
frame is other then "current", the element firstly needs to be placed at
the center of the reference frame and then it is moved about the user
specified coordinates.   

After element placement at new position and orientation patch needs to
be recomputed. If autoplacedownstream is false then patch to the next
element is also recomputed. Otherwise, the layout is surveyed from the
next element on, what places all the elements downstream with default
position with respect to the moved element.  

At the end all the layout is surveyed, if surveyall flag is true, what
normally should always take place.      




%%%\title{PTC\_PRINTFRAMES}

\section{PTC\_PRINTFRAMES}

\begin{verbatim}
PTC_PRINTFRAMES, 
    file = [s, none] ,
    format = [s, text] ; 
\end{verbatim}

Print the PTC geometry of a layout to a specified file.   \\

{\bf Command parameters and switches}
\begin{itemize}
   \item {\bf file}=string (Default: NULL)\\
     Specifies the name of the file.   
   \item {\bf format}=string (Default: text)\\
     Format of geometry.\\
     Currently two formats are accepted:       
     \begin{itemize}
	\item text: Prints a simple text file.           
	\item rootmacro: Creates \href{http://root.cern.ch}{root} macro
          that produces 3D display of the geometry.            
     \end{itemize}
\end{itemize}


{\bf Example }\\
\href{http://cern.ch/frs/mad-X_examples/ptc_madx_interface/eplacement/eplacement.madx}{Dog
  leg chicane} with some elements displaced with help of
ptc\_eplacement.

 
% <h3> PROGRAMMERS MANUAL </h3>
% 
% <p> 
% The command is implemented pro_ptc_knob function in madxn.c and 
% by subroutine xxxx in madx_ptc_xxx.f90.
% <p>
% Sopecified range is resolved with help of get_range command. Number of the element in the current sequence
% is resolved and passed as the parameter to the fortran routine. It allows to resolve uniquely the corresponding
% element in the PTC layout.
% <p>

%%\title{PTC\_PRINTFRAMES}

\section{PTC\_PRINTFRAMES}

\begin{verbatim}
PTC_PRINTFRAMES, 
    file = [s, none] ,
    format = [s, text] ; 
\end{verbatim}

Print the PTC geometry of a layout to a specified file.   \\

{\bf Command parameters and switches}
\begin{itemize}
   \item {\bf file}=string (Default: NULL)\\
     Specifies the name of the file.   
   \item {\bf format}=string (Default: text)\\
     Format of geometry.\\
     Currently two formats are accepted:       
     \begin{itemize}
	\item text: Prints a simple text file.           
	\item rootmacro: Creates \href{http://root.cern.ch}{root} macro
          that produces 3D display of the geometry.            
     \end{itemize}
\end{itemize}


{\bf Example }\\
\href{http://cern.ch/frs/mad-X_examples/ptc_madx_interface/eplacement/eplacement.madx}{Dog
  leg chicane} with some elements displaced with help of
ptc\_eplacement.

 
% <h3> PROGRAMMERS MANUAL </h3>
% 
% <p> 
% The command is implemented pro_ptc_knob function in madxn.c and 
% by subroutine xxxx in madx_ptc_xxx.f90.
% <p>
% Sopecified range is resolved with help of get_range command. Number of the element in the current sequence
% is resolved and passed as the parameter to the fortran routine. It allows to resolve uniquely the corresponding
% element in the PTC layout.
% <p>


%%%\title{PTC\_SELECT}

\section{PTC\_SELECT}

\begin{verbatim}
PTC_SELECT, 
   table      = [s, none, none], 
   column     = [s, none, none], 
   polynomial = [i, none] , 
   monomial   = [s, none] , 
   parametric = [l, false, true], 
   quantity   = [s, none] ; "
\end{verbatim}


Selects map elements to be: 
\begin{enumerate}
  \item {\bf Stored in a user specified table and column.}
    \textit{Table} and \textit{column} must be specified then, and such
    table with such column must exists.  

  \item {\bf Stored as a function (taylor series) of
    \href{PTC_Knob.html}{knobs}, if any is defined.} Then,
    \textit{parametric} should be set to true. 
\end{enumerate}
Both cases can be joined in one command.   


{\bf Command parameters and switches}

\begin{itemize}
   \item {\bf table}=string (Default: ???)\\
     Specifies name of the table where values should be stored.   
   \item {\bf column}=string (Default: ???)\\     
     Specifies name of the table where values should be stored.   
   \item {\bf polynomial}=integer (Default: ???)\\
     Specifies row of the map.   
   \item {\bf monomial}=string composed of digits (Default: ???)\\
     Defines monomial of the polynomial in PTC nomenclature. \\
     Its length should be equal to number of variables. Each digit
     corresponds to the exponent of a variable. Monomial 'ijklmn'
     defines x$^i$p$_x$$^j$y$^k$p$_y$$^l$
     $\Delta$T$^m$($\Delta$p/p)$^n$. For example, element=2 and
     monomial=1000000 defines coefficient of the second polynomial (that
     defines p$_x$) close to x, in the other words it is R21.  
   \item{\bf parametric}=logical (Default: .false.)\\
     If true, and any \href{PTC_Knob.html}{knobs} are defined, the
     map element is stored as the parametric result.            
\end{itemize}


{\bf Examples}\\

\href{http://cern.ch/frs/mad-X_examples/ptc_madx_interface/ptc_secordmatch/chicane.madx}{dog
  leg chicane}: strength of quads is matched to obtain required T112
value.    

\href{http://cern.ch/frs/mad-X_examples/ptc_madx_interface/eplacement/chicane.madx}{dog
  leg chicane}: postion of quads is matched to obtain required T566
value.   

\href{http://cern.ch/frs/mad-X_examples/ptc_madx_interface/matchwithknobs/matchwithknobs.madx}{dog
  leg chicane}: dipole and quadrupole strengths are matched with the
help of knobs to obtain required momentum compaction and Twiss
functions.   


{\bf PROGRAMMER'S MANUAL} 

The command is implemented pro\_ptc\_select function in madxn.c and  by
subroutine addpush in madx\_ptc\_knobs.f90, that is part of
madx\_ptc\_knobs\_module   

On the very beginning the existance of the table and within column is
checked. In the case of failure, error message is printed and the
function is abandoned.   

The command parameters are passed as the arguments of addpush Fortran
routine.  A selection is stored in a type called tablepush\_poly defined
madx\_ptc\_knobs.inc. A newly created object is added to array named
pushes.    

More then one element might be stored in a single table, so the module
must assure that  each of tables is augmented only ones for each magnet
(or integration slice).  For that purpose array of tables to be
augmented (named tables) is stored separately and  we assure that a
table is listed here only ones. This is simply done by checking  if a
table name is not already listed before adding a new element to the
array.   

In case the user requested an element to be stored in the paramteric
format, and column in the array of parametric results is reserved and
the index of the column is remembered in index field of tablepush\_poly
type is filled. In the other case this field is equal to zero.   

The routine ptc\_twiss (defined in file madx\_ptc\_twiss.f90), after
tracking each of magnets  in the sequence, calls putusertable
routine. This routine loops over selected elemetns defined in the pushes
table. For each of them it extracts the requested element from the map
using .sub.  operator of PTC and stores it in the defined table and
column.  If index field is not zero and any knob is defined, it extracts
the polynomial using .par. operator, and stores it in the 2D array
called results, in the row corresponding to the number of the magnet (or
integration step) and column defined by the index field.     


%%\title{PTC\_SELECT}

\section{PTC\_SELECT}

\begin{verbatim}
PTC_SELECT, 
   table      = [s, none, none], 
   column     = [s, none, none], 
   polynomial = [i, none] , 
   monomial   = [s, none] , 
   parametric = [l, false, true], 
   quantity   = [s, none] ; "
\end{verbatim}


Selects map elements to be: 
\begin{enumerate}
  \item {\bf Stored in a user specified table and column.}
    \textit{Table} and \textit{column} must be specified then, and such
    table with such column must exists.  

  \item {\bf Stored as a function (taylor series) of
    \href{PTC_Knob.html}{knobs}, if any is defined.} Then,
    \textit{parametric} should be set to true. 
\end{enumerate}
Both cases can be joined in one command.   


{\bf Command parameters and switches}

\begin{itemize}
   \item {\bf table}=string (Default: ???)\\
     Specifies name of the table where values should be stored.   
   \item {\bf column}=string (Default: ???)\\     
     Specifies name of the table where values should be stored.   
   \item {\bf polynomial}=integer (Default: ???)\\
     Specifies row of the map.   
   \item {\bf monomial}=string composed of digits (Default: ???)\\
     Defines monomial of the polynomial in PTC nomenclature. \\
     Its length should be equal to number of variables. Each digit
     corresponds to the exponent of a variable. Monomial 'ijklmn'
     defines x$^i$p$_x$$^j$y$^k$p$_y$$^l$
     $\Delta$T$^m$($\Delta$p/p)$^n$. For example, element=2 and
     monomial=1000000 defines coefficient of the second polynomial (that
     defines p$_x$) close to x, in the other words it is R21.  
   \item{\bf parametric}=logical (Default: .false.)\\
     If true, and any \href{PTC_Knob.html}{knobs} are defined, the
     map element is stored as the parametric result.            
\end{itemize}


{\bf Examples}\\

\href{http://cern.ch/frs/mad-X_examples/ptc_madx_interface/ptc_secordmatch/chicane.madx}{dog
  leg chicane}: strength of quads is matched to obtain required T112
value.    

\href{http://cern.ch/frs/mad-X_examples/ptc_madx_interface/eplacement/chicane.madx}{dog
  leg chicane}: postion of quads is matched to obtain required T566
value.   

\href{http://cern.ch/frs/mad-X_examples/ptc_madx_interface/matchwithknobs/matchwithknobs.madx}{dog
  leg chicane}: dipole and quadrupole strengths are matched with the
help of knobs to obtain required momentum compaction and Twiss
functions.   


{\bf PROGRAMMER'S MANUAL} 

The command is implemented pro\_ptc\_select function in madxn.c and  by
subroutine addpush in madx\_ptc\_knobs.f90, that is part of
madx\_ptc\_knobs\_module   

On the very beginning the existance of the table and within column is
checked. In the case of failure, error message is printed and the
function is abandoned.   

The command parameters are passed as the arguments of addpush Fortran
routine.  A selection is stored in a type called tablepush\_poly defined
madx\_ptc\_knobs.inc. A newly created object is added to array named
pushes.    

More then one element might be stored in a single table, so the module
must assure that  each of tables is augmented only ones for each magnet
(or integration slice).  For that purpose array of tables to be
augmented (named tables) is stored separately and  we assure that a
table is listed here only ones. This is simply done by checking  if a
table name is not already listed before adding a new element to the
array.   

In case the user requested an element to be stored in the paramteric
format, and column in the array of parametric results is reserved and
the index of the column is remembered in index field of tablepush\_poly
type is filled. In the other case this field is equal to zero.   

The routine ptc\_twiss (defined in file madx\_ptc\_twiss.f90), after
tracking each of magnets  in the sequence, calls putusertable
routine. This routine loops over selected elemetns defined in the pushes
table. For each of them it extracts the requested element from the map
using .sub.  operator of PTC and stores it in the defined table and
column.  If index field is not zero and any knob is defined, it extracts
the polynomial using .par. operator, and stores it in the 2D array
called results, in the row corresponding to the number of the magnet (or
integration step) and column defined by the index field.     



%%%\title{PTC\_SELECT\_MOMENT}

\section{PTC\_SELECT\_MOMENT}

\begin{verbatim}
PTC_SELECT_MOMENT, 
   table      = [s, none, none], 
   column     = [s, none, none], 
   moment_s   = [s, none] , 
   moment     = [i, {0}] , 
   parametric = [l, false, true], 
\end{verbatim}

Selects a moment to be: 
\begin{enumerate}
  \item {\bf Stored in a user specified table and column.}
    \textit{Table} and \textit{column} must be specified then, and such
    table with such column must exists.  

  \item {\bf Stored as a function (taylor series) of
    \href{PTC_Knob.html}{knobs}, if any is defined.} Then,
    \textit{parametric} should be set to true. 
\end{enumerate}
Both cases can be joined in one command.   


{\bf Command parameters and switches}
\begin{itemize}
   \item {\bf moment\_s}=list of coma separated strings composed of up
     to 6 digits (Default: ???)\\
     Defines moment of the polynomial in PTC nomenclature. String 'ijklmn'
     (where i,j,k,l,m,n are digits from 0 to 9) defines
     $<$x$^i$p$_x$$^j$y$^k$p$_y$$^l$ $\Delta$T$^m$($\Delta$p/p)$^n$$>$. 
     \\ For example, moment\_s=100000 defines $<$x$^1$$>$ 

     Note that for input we always use MAD-X notation where dp/p is always
     the 6th coordinate. Internally to PTC, dp/p is the 5th coordinate. We
     perform automatic conversion that is transparent for the user. As the
     consequence RMS in dp/p is always defined as 000002, even in 5D case.    

     This notations allows to define more then one moment with one
     command. In this case, the corresponding column names are as the passed
     strings with "mu" prefix. However, they are always extended to 6 digits,
     i.e. the trailing 0 are automatically added. For example, if specified
     moment\_s=2, the column name is mu200000.

     This method does not allow to pass bigger numbers then 9. If you need to
     define such a moment, use the command switch below.    
     
   \item {\bf moment}=list of up to 6 coma separated integers (Default: ???)\\
     Defines a moment. For example: moment=2 defines $<$x$^2$$>$ ,
     moment=0,0,2 : $<$y$^2$$>$, moment=0,14,0,2 : $<$px$^{14}$py$^2$$>$,
     etc. 

   \item {\bf table}=string (Default: moments)\\
     Specifies the name of the table where the calculated moments are stored.   

   \item {\bf column}=string (Default: ???)\\
     Ignored if \textit{ moments } is specified. Defines name of the
     column where values should be stored. If not specified then it
     is automatically generated from moment the definition
     $<$x$^i$p$_x$$^j$y$^k$p$_y$$^l$
     $\Delta$T$^m$($\Delta$p/p)$^n$$>$ =$>$ mu\_i\_j\_k\_l\_m\_n
     (numbers separated with underscores).                  

   \item {\bf parametric}=logical (Default: .false.)\\
     If it is true, and any \href{PTC_Knob.html}{knobs} are defined the map
     element is stored as the parametric result.             
\end{itemize}

{\bf Examples}\\
\href{http://cern.ch/frs/mad-X_examples/ptc_madx_interface/moments/moments.madx}{ATF2}

 
% <h3> PROGRAMMERS MANUAL </h3>
% 
% <p> 
% The command is implemented pro_ptc_SELECT function in madxn.c and 
% by subroutine xxxx in madx_ptc_xxx.f90.
% <p>
% Sopecified range is resolved with help of get_range command. Number of the element in the current sequence
% is resolved and passed as the parameter to the fortran routine. It allows to resolve uniquely the corresponding
% element in the PTC layout.
% <p>

%%\title{PTC\_SELECT\_MOMENT}
\section{PTC\_SELECT\_MOMENT}

\begin{verbatim}
PTC_SELECT_MOMENT, 
   table      = [s, none, none], 
   column     = [s, none, none], 
   moment_s   = [s, none] , 
   moment     = [i, {0}] , 
   parametric = [l, false, true], 
\end{verbatim}

Selects a moment to be: 
\begin{enumerate}
  \item {\bf Stored in a user specified table and column.}
    \textit{Table} and \textit{column} must be specified then, and such
    table with such column must exists.  

  \item {\bf Stored as a function (taylor series) of
    \href{PTC_Knob.html}{knobs}, if any is defined.} Then,
    \textit{parametric} should be set to true. 
\end{enumerate}
Both cases can be joined in one command.   


{\bf Command parameters and switches}
\begin{itemize}
   \item {\bf moment\_s}=list of coma separated strings composed of up
     to 6 digits (Default: ???)\\
     Defines moment of the polynomial in PTC nomenclature. String 'ijklmn'
     (where i,j,k,l,m,n are digits from 0 to 9) defines
     $<$x$^i$p$_x$$^j$y$^k$p$_y$$^l$ $\Delta$T$^m$($\Delta$p/p)$^n$$>$. 
     \\ For example, moment\_s=100000 defines $<$x$^1$$>$ 

     Note that for input we always use MAD-X notation where dp/p is always
     the 6th coordinate. Internally to PTC, dp/p is the 5th coordinate. We
     perform automatic conversion that is transparent for the user. As the
     consequence RMS in dp/p is always defined as 000002, even in 5D case.    

     This notations allows to define more then one moment with one
     command. In this case, the corresponding column names are as the passed
     strings with "mu" prefix. However, they are always extended to 6 digits,
     i.e. the trailing 0 are automatically added. For example, if specified
     moment\_s=2, the column name is mu200000.

     This method does not allow to pass bigger numbers then 9. If you need to
     define such a moment, use the command switch below.    
     
   \item {\bf moment}=list of up to 6 coma separated integers (Default: ???)\\
     Defines a moment. For example: moment=2 defines $<$x$^2$$>$ ,
     moment=0,0,2 : $<$y$^2$$>$, moment=0,14,0,2 : $<$px$^{14}$py$^2$$>$,
     etc. 

   \item {\bf table}=string (Default: moments)\\
     Specifies the name of the table where the calculated moments are stored.   

   \item {\bf column}=string (Default: ???)\\
     Ignored if \textit{ moments } is specified. Defines name of the
     column where values should be stored. If not specified then it
     is automatically generated from moment the definition
     $<$x$^i$p$_x$$^j$y$^k$p$_y$$^l$
     $\Delta$T$^m$($\Delta$p/p)$^n$$>$ =$>$ mu\_i\_j\_k\_l\_m\_n
     (numbers separated with underscores).                  

   \item {\bf parametric}=logical (Default: .false.)\\
     If it is true, and any \href{PTC_Knob.html}{knobs} are defined the map
     element is stored as the parametric result.             
\end{itemize}

{\bf Examples}\\
\href{http://cern.ch/frs/mad-X_examples/ptc_madx_interface/moments/moments.madx}{ATF2}

 
% <h3> PROGRAMMERS MANUAL </h3>
% 
% <p> 
% The command is implemented pro_ptc_SELECT function in madxn.c and 
% by subroutine xxxx in madx_ptc_xxx.f90.
% <p>
% Sopecified range is resolved with help of get_range command. Number of the element in the current sequence
% is resolved and passed as the parameter to the fortran routine. It allows to resolve uniquely the corresponding
% element in the PTC layout.
% <p>



%%%\title{PTC\_MOMENTS}

\section{PTC\_MOMENTS}

\begin{verbatim}
PTC_MOMENTS, 
    no = [i, 1], 
    xdistr   = [s, gauss, gauss], 
    ydistr   = [s, gauss, gauss], 
    zdistr   = [s, gauss, gauss], 
\end{verbatim}

Calculates moments previously selected with the
\href{PTC_SelectMoment.html}{ptc\_select\_moment} command.  It uses maps
saved by the ptc\_twiss command, hence, the savemaps switch of
ptc\_twiss must be set to true (default) to be able to calculate
moments.  \\ 

{\bf  Command parameters and switches }
\begin{itemize}
   \item {\bf no}=integer (Default: 1)\\
     order of the calculation, maximally twice the order of the last
     twiss.
   
   \item {\bf xdistr, ydistr, zdistr}=string (Default: gauss)\\  
     defines the distribution in x, y and z dimension respectively 
     \begin{enumerate}
	\item {\bf gauss} - Gaussian
	\item {\bf flat5} - flat distribution in the first of
          variables (dp over p) of a given dimension and Delta Dirac in
          the second one (T)  
	\item {\bf flat56} - flat rectangular distribution 
     \end{enumerate}
\end{itemize}

{\bf Examples}\\
\href{http://cern.ch/frs/mad-X_examples/ptc_madx_interface/moments/moments.madx}{ATF2}


% <h3> PROGRAMMERS MANUAL </h3>
% 
% <p> 
% The command is implemented pro_ptc_SELECT function in madxn.c and 
% by subroutine xxxx in madx_ptc_xxx.f90.
% <p>
% Sopecified range is resolved with help of get_range command. Number of the element in the current sequence
% is resolved and passed as the parameter to the fortran routine. It allows to resolve uniquely the corresponding
% element in the PTC layout.
% <p>

%%\title{PTC\_MOMENTS}

\section{PTC\_MOMENTS}

\begin{verbatim}
PTC_MOMENTS, 
    no = [i, 1], 
    xdistr   = [s, gauss, gauss], 
    ydistr   = [s, gauss, gauss], 
    zdistr   = [s, gauss, gauss], 
\end{verbatim}

Calculates moments previously selected with the
\href{PTC_SelectMoment.html}{ptc\_select\_moment} command.  It uses maps
saved by the ptc\_twiss command, hence, the savemaps switch of
ptc\_twiss must be set to true (default) to be able to calculate
moments.  \\ 

{\bf  Command parameters and switches }
\begin{itemize}
   \item {\bf no}=integer (Default: 1)\\
     order of the calculation, maximally twice the order of the last
     twiss.
   
   \item {\bf xdistr, ydistr, zdistr}=string (Default: gauss)\\  
     defines the distribution in x, y and z dimension respectively 
     \begin{enumerate}
	\item {\bf gauss} - Gaussian
	\item {\bf flat5} - flat distribution in the first of
          variables (dp over p) of a given dimension and Delta Dirac in
          the second one (T)  
	\item {\bf flat56} - flat rectangular distribution 
     \end{enumerate}
\end{itemize}

{\bf Examples}\\
\href{http://cern.ch/frs/mad-X_examples/ptc_madx_interface/moments/moments.madx}{ATF2}


% <h3> PROGRAMMERS MANUAL </h3>
% 
% <p> 
% The command is implemented pro_ptc_SELECT function in madxn.c and 
% by subroutine xxxx in madx_ptc_xxx.f90.
% <p>
% Sopecified range is resolved with help of get_range command. Number of the element in the current sequence
% is resolved and passed as the parameter to the fortran routine. It allows to resolve uniquely the corresponding
% element in the PTC layout.
% <p>



%%%\title{PTC\_DumpMaps}

\section{PTC\_DUMPMAPS}

\begin{verbatim}
PTC_DUMPMAPS, 
      file = [s, ptcmaps, ptcmaps];
\end{verbatim}

PTC\_DUMPMAPS dumps the linear part of the map for each element of the
layout into the specified file.  

{\bf Command parameters and switches}\\
\begin{itemize}
   \item {\bf file}=string  (Default: ptcmaps)\\
     Specifies the file\_name of the file to which the matrices are dumped to.   
\end{itemize}

{\bf PROGRAMMERS MANUAL} \\  
The command is implemented by subroutine ptc\_dumpmaps() in
madx\_ptc\_module.f90. The matrix for a single element is obtained by
tracking identity map through an element, that is initialized for each
element by adding identity map to the reference particle. For the
elements that change reference momentum (i.e. traveling wave cavity)  it
is tracked to the end of the following marker, that has updated
reference momentum. Hence, each cavity must be followed by a marker. If
it is not, setcavities subroutine detects error and stops the program.   
 

%%\title{PTC\_DumpMaps}

\section{PTC\_DUMPMAPS}

\begin{verbatim}
PTC_DUMPMAPS, 
      file = [s, ptcmaps, ptcmaps];
\end{verbatim}

PTC\_DUMPMAPS dumps the linear part of the map for each element of the
layout into the specified file.  

{\bf Command parameters and switches}\\
\begin{itemize}
   \item {\bf file}=string  (Default: ptcmaps)\\
     Specifies the file\_name of the file to which the matrices are dumped to.   
\end{itemize}

{\bf PROGRAMMERS MANUAL} \\  
The command is implemented by subroutine ptc\_dumpmaps() in
madx\_ptc\_module.f90. The matrix for a single element is obtained by
tracking identity map through an element, that is initialized for each
element by adding identity map to the reference particle. For the
elements that change reference momentum (i.e. traveling wave cavity)  it
is tracked to the end of the following marker, that has updated
reference momentum. Hence, each cavity must be followed by a marker. If
it is not, setcavities subroutine detects error and stops the program.   
 


%%%\title{PTC\_SetCavities}

\section{PTC\_SetCavities}

\begin{verbatim}
PTC_SetCavities;
\end{verbatim}

Tis command adjusts cavities and sets appropriate reference momenta for
a layout containing traveling wave cavities.

This command sets up the properties of a layout and traveling wave
cavities.  The main goal is to update  reference beam energy for the
elements  that follow a traveling wave cavity.  It traces the
synchronous particle, i.e. one that has  all its parameters set to zero
at the beginning of the layout under study. At the point it arrives to a
cavity,  the parameters of the latter one are adjusted according to the
switches  defined by the user. 

There are 2 cases   
\begin{enumerate}
   \item \textbf{Leaves all parameters untouched}
   \item \textbf{Phase of cavity is adjusted so it gives the maximum
     acceleration} Afterwards to the calculated phase the lag
     is added. This setting is acquired using set\_switch
     command, setting maxaccel parameter to true.   
\end{enumerate} 

Afterwards,  the synchronous particle is tracked through traveling wave
cavity  and its energy gain is known.  This energy becomes the reference
one for all the elements downstream of the cavity.  The particle is
tracked further to the next cavity,  for which the procedure described
above is repeated.    

Parameters of the cavities are dumped to the file named
twcavsettings.txt.    

At the end patches at the ends of the cavities are set,  so the
parameters after them are  calculated taking to the account reference
energy increase.   

The exact program behavior depends on the  \href{PTC_SetSwitch.html}{
  PTC switches settings}.   

Please note that in PTC phase velocity of a cavities wave is always
equal to speed of light.  Hence, if PTC internal state TIME is TRUE,
what is the most correct setting,  then voltage seen by a particle is
varying along the structure. If TIME is FALSE,  track is assumed to fly
with speed of light and in such case a particle moves together with the
wave front.    


{\bf PROGRAMMERS MANUA}

{\bf CAUTION} For the time being cavities MUST not be placed one after
another, and at least a marker must be inserted between two neighboring
accelerating structures. Otherwise, program will stop with the error
message.    

%%\title{PTC\_SetCavities}

\section{PTC\_SetCavities}

\begin{verbatim}
PTC_SetCavities;
\end{verbatim}

This command adjusts cavities and sets appropriate reference momenta for
a layout containing traveling wave cavities.

This command sets up the properties of a layout and traveling wave
cavities.  The main goal is to update  reference beam energy for the
elements  that follow a traveling wave cavity.  It traces the
synchronous particle, i.e. one that has  all its parameters set to zero
at the beginning of the layout under study. At the point it arrives to a
cavity,  the parameters of the latter one are adjusted according to the
switches  defined by the user. 

There are 2 cases   
\begin{enumerate}
   \item \textbf{Leaves all parameters untouched}
   \item \textbf{Phase of cavity is adjusted so it gives the maximum
     acceleration} Afterwards to the calculated phase the lag
     is added. This setting is acquired using set\_switch
     command, setting maxaccel parameter to true.   
\end{enumerate} 

Afterwards,  the synchronous particle is tracked through traveling wave
cavity  and its energy gain is known.  This energy becomes the reference
one for all the elements downstream of the cavity.  The particle is
tracked further to the next cavity,  for which the procedure described
above is repeated.    

Parameters of the cavities are dumped to the file named
twcavsettings.txt.    

At the end patches at the ends of the cavities are set,  so the
parameters after them are  calculated taking to the account reference
energy increase.   

The exact program behavior depends on the  \href{PTC_SetSwitch.html}{
  PTC switches settings}.   

Please note that in PTC phase velocity of a cavities wave is always
equal to speed of light.  Hence, if PTC internal state TIME is TRUE,
what is the most correct setting,  then voltage seen by a particle is
varying along the structure. If TIME is FALSE,  track is assumed to fly
with speed of light and in such case a particle moves together with the
wave front.    


{\bf PROGRAMMERS MANUAL}

{\bf CAUTION} For the time being cavities MUST not be placed one after
another, and at least a marker must be inserted between two neighboring
accelerating structures. Otherwise, program will stop with the error
message.    


%%\end{document}
