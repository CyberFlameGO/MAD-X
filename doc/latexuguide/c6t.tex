%%%\title{SIXTRACK Convertor}
%  Changed by: Mark HAYES, 19-Sep-2002 


\chapter{Conversion to \textit{SixTrack}}
\label{chap:sixtrack}

\textit{SixTrack}\cite{SixTrack}\cite{SixTrack-www} is a separate beam 
optics code that is often used for long term tracking of particles, 
\textsl{e.g.} for dynamic aperture studies, because of its speed and
controllability.  

However the input files are notoriously difficult to produce by
hand. This command may be used to generate \textit{SixTrack} input files
from a sequence loaded in \madx. 


\madbox{
  SIXTRACK, \=CAVALL=logical, \\
            \>MULT\_AUTO\_OFF=logical, MAX\_MULT\_ORD=integer, \\
            \>SPLIT=logical, APERTURE=logical, RADIUS=real, MARKALL=logical;
}
 
The parameters are defined as: 
\begin{madlist}
   \ttitem{CAVALL} (optional flag) This puts a cavity element
   (\textit{SixTrack} identifier 12) with Volt, Harmonic Number and Lag
   attributes at each location in the machine. Since for large hadron
   machines the cavities are typically all located at one particular
   spot in the machine and since many cavities slow down the tracking
   simulations considerably all cavities are lumped into one and located
   at the first appearance of a cavity. This default is enforced by
   omitting this flag.  

   \ttitem{MULT\_AUTO\_OFF} (optional flag, default = FALSE) If
   TRUE, this module does not process zero value multipoles. 
   Moreover, multipoles are prepared by \texttt{SIXTRACK}
   (output file fc.3) to be treated up to the order as specified with
   \texttt{MAX\_MULT\_ORD}.  

   \ttitem {MAX\_MULT\_ORD} (optional parameter, default = 11) Process up
   to this order for \texttt{MULT\_AUTO\_OFF = TRUE}.  

   \ttitem{SPLIT} (optional flag) OBSOLETE. This splits all the
   elements in  two. This is for backward compatibilty only. The user
   should now use the \hyperref[chap:makethin]{\texttt{MAKETHIN}} command
   instead.   

   \ttitem{APERTURE} (optional flag) flag to convert the apertures
   from \madx to \textit{SixTrack} so that \textit{SixTrack} can track
   with apertures defined. The aperture data is found in file
   \texttt{fc.3.aper}. 

   \ttitem{RADIUS} (optional, default value is 1~m). This sets the
   reference  radius for the magnets. This argument is optional but
   should normally be set. 

   \ttitem{MARKALL} (optional flag, default false) flag to convert all markers through the conversion. If false, only markers with names beginning with "ip" or "mt\_" are kept.

   \ttitem{LONG\_NAMES} (optional flag, default true) flag to increase the possible length of names (from 17 to 48) and increase the number of digits in the output in the conversion. 

   \ttitem{SIX\_VERSION} (optional, default value=0) Some new formats are only supported in recent versions of SixTrack. In order to output in these formats a version number higher or equal to the number when it was implemented needs to be set. If the needed SixTrack version is 5.02.03 then the version should be set to 50203.

\end{madlist}

\textbf{Important Notes:}
\begin{itemize} 
\item The files contain all information concerning optics, field errors
 and misalignments. Hence these should all be set and a   
\madxmp{TWISS, SAVE;}
command should always be issued before calling the \texttt{SIXTRACK} command.

\item The \hyperref[sec:bvflag]{BV flag} is presently ignored by \texttt{SIXTRACK}.

%% 2015-Oct-08  19:26:14  ghislain: change this after name mangling introduced
\item \textit{SixTrack} and the \madx command \texttt{SIXTRACK} are
  presently set up for names of a maximum of 16 characters!!!!! 
  Therefore, it is mandatory to respect this limit for \madx names.
\end{itemize}

The \texttt{SIXTRACK} command always produces at least one output file: 
\begin{itemize}
   \item  fc.2 -  the basic structure of the lattice. 
\end{itemize} 
It may also produce any or all of the following files, depending on the
sequence and command attributes:
\begin{itemize}
   \item  fc.3 -   multipole mask(s). 
   \item  fc.3.aux -  various beam parameters. 
   \item  fc.3.aper - aperture element data (units are \texttt{mm} and \texttt{degrees}).
   \item  fc.8 -   misalignments and tilts. 
   \item  fc.16 -  field errors and/or combined multipole kicks. 
   \item  fc.34 -  various optics parameters at various
     locations \\ This file is not needed by \textit{SixTrack} but may
     be used as input to the program \textit{SODD}\cite{SODD}.)   
\end{itemize}  

For a full description of these files see the \textit{SixTrack} 
website\cite{SixTrack-www}, the \textit{SixTrack} user 
manual\cite{SixTrack}; and for information on running {\textit SixTrack} 
see the \textit{SixTrack} run environment description\cite{SixTrack-RE}.

%% EOF
