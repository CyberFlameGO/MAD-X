%%\title{PTC Set-up Parameters}
%  Created by: Valery KAPIN, 21-Mar-2006 
%  Changed by: ____________, ___________ 

\chapter{\ptc Set-up Parameters}
\label{chap:ptc-setup}

The Polymorphic Tracking Code \cite{forest2002} of Etienne
Forest is a kick code, allowing a symplectic integration through all
accelerator elements giving the user full control over the precision
(number of   steps and integration type) and exactness (full or extended
Hamiltonian) of the   results. 
The degree of exactness is determined by the user and the speed of his
computer.  
The main advantage is that the code is inherently based on the map
formalism and provides users with all associated tools. 

The \ptc code is actually a library that can be used in many different
ways to create an actual module that calculates some property of
interest. 

\textbf{Attention:}
\ptc exists inside of \madx as a library. \madx offers the interface to
\ptc, \textsl{i.e.} the \madx input file is used as input for \ptc.
Internally, both \ptc and \madx have their own independent databases
which are linked via the interface. With the
\hyperref[sec:ptc-create-layout]{\texttt{PTC\_CREATE\_LAYOUT}} command,
only numerical values are transferred 
from the \madx data structures to the \ptc data structures. Any
modification to the \madx data structure is unknown to \ptc until the
next call to \hyperref[sec:ptc-create-layout]{\texttt{PTC\_CREATE\_LAYOUT}}. 
For example, a \hyperref[sec:defer]{deferred expression} of \madx is
only evaluated at the time of the
\hyperref[sec:ptc-create-layout]{\texttt{PTC\_CREATE\_LAYOUT}} command and
is ignored within \ptc afterwards. 


Several modules using the PTC code have been presently implemented in
\madx. These {\madx}-{\ptc} modules\cite{schmidt2005} are executed by
the following commands: 
\hyperref[chap:ptc-twiss]{\texttt{PTC\_TWISS}},
\hyperref[chap:ptc-normal]{\texttt{PTC\_NORMAL}},
\hyperref[chap:ptc-track]{\texttt{PTC\_TRACK}},
\hyperref[sec:ptc-trackline]{\texttt{PTC\_TRACK\_LINE}}. 

To perform calculations with these {\madx}-{\ptc} commands, the \ptc
environment must be initialized, handled and turned off by special
commands within the \madx input script. 

\section{Command Synopsis}
\label{sec:ptc-synopsis}

A typical set of commands to invoke \ptc is given below: 

\madxmp{
PTC\_CREATE\_UNIVERSE, \=SECTOR\_NMUL\_MAX= integer, SECTOR\_NMUL= integer,  \\
                       \>NTPSA= logical, SYMPRINT= logical; \\
\\
PTC\_CREATE\_LAYOUT, \=TIME= logical, MODEL= integer, \\
                     \>METHOD= integer, NST= integer, EXACT= logical, \\
                     \>OFFSET\_DELTAP= double, ERRORS\_OUT= logical, \\
                     \>MAGNET\_NAME= string, RESPLIT= logical,  \\
                     \>THIN= double, XBEND= double,  \\
                     \>EVEN = logical; \\
... \\
PTC\_MOVE\_TO\_LAYOUT, INDEX= integer; \\
...\\
PTC\_READ\_ERRORS, OVERWRITE= logical; \\
... \\
PTC\_ALIGN; \\
...\\
PTC\_END;
}

\section{PTC\_CREATE\_UNIVERSE}
\label{sec:ptc-create-universe}

The \texttt{PTC\_CREATE\_UNIVERSE} command is required to set-up the
\ptc environment.  

\madbox{
PTC\_CREATE\_UNIVERSE, \=SECTOR\_NMUL\_MAX=integer, SECTOR\_NMUL=integer, \\
                       \>NTPSA=logical, SYMPRINT=logical;
}

The attributes are:
\begin{madlist}

   \ttitem{SECTOR\_NMUL\_MAX} a global variable in \ptc needed for exact
   sector bends defining up to which order Maxwell's equation are solved
   (see \cite{forest2002}~page~76-77). 
   The value of \texttt{SECTOR\_NMUL\_MAX} must not be smaller than
   \texttt{SECTOR\_NMUL} otherwise \madx stops with an error. 
   If a negative value is passed than it is identified automatically
   by scanning the currently selected sequence.
   \\ (Default:~-1)

   \ttitem{SECTOR\_NMUL} a global variable in PTC needed for exact
   sector bends defining up to which order the multipole are included in
   solving Maxwell's equation up to order \texttt{SECTOR\_NMUL\_MAX}. 
   Multipoles of order N with N $>$ \texttt{SECTOR\_NMUL} and N $\leq$ 
   \texttt{SECTOR\_NMUL\_MAX} are treated similar to \textit{SixTrack}. 
   If a negative value of \\ 
   \texttt{SECTOR\_NMUL\_MAX} is passed than it is also identified automatically.
   However, if multipolar parameters of the bends are to be modified inside the PTC universe, 
   for example with \texttt{PTC\_READ\_ERRORS}, 
   than this parameter needs to be set to a corresponding value.
   Please note that using large values (above 10) slows down the computations,
   so the smallest required value should be used. 
   \\ 
   (Default:-1)  

   \ttitem{NTPSA} invokes the Differential Algebra (DA) package
   written in C++ and kindly provided by Lingyun Yang (lyyang@lbl.gov). \\ 
   Etienne Forest has written the wrapper to allow the use of both
   the legendary DA package written in Fortran by Martin Berz
   (default) and this new DA package of Lingyun~Yang. 
   It is expected that this DA package will allow for the efficient
   calculation of a large number of DA parameters. \\ (Default:~false)

   \ttitem{SYMPRINT} a flag to enable the printing of the check of
   symplecticity. It is recommended to leave this flag set to TRUE. \\
   (Default: true)
\end{madlist}


\section{PTC\_CREATE\_LAYOUT}
\label{sec:ptc-create-layout}

The \texttt{PTC\_CREATE\_LAYOUT} command creates the \ptc-layout
according to  the specified integration method and fills it with the
current \madx sequence defined in the latest
\hyperref[sec:use]{\texttt{USE}} command.

\madbox{
PTC\_CREATE\_LAYOUT, \=TIME=logical, MODEL=integer, METHOD=integer,  \\
                     \>NST=integer, EXACT=logical, OFFSET\_DELTAP=double, \\
                     \>ERRORS\_OUT=logical, MAGNET\_NAME=string, \\
                     \>RESPLIT=logical, THIN=double, XBEND=double, \\ 
                     \>EVEN=logical; 	
}

The attributes are:
\begin{madlist}

  \ttitem{TIME} a logical flag to control which coordinate system 
  is being used. \\ (Default=~true) \\ \\
  Please see \texttt{TIME} of \hyperref[sec:ptc-setswitch]{\texttt{PTC\_SETSWITCH}}
  command for more details.
   
  \ttitem{MODEL} an integer to switch between models:\\
  1 for "Drift-Kick-Drift";  (Default value)\\ 
  2 for "Matrix-Kick-Matrix" and \\ 
  3 for "Delta-Matrix-Kick-Matrix" (SixTrack-code model).

  \ttitem{METHOD} the integration order: 2, 4, or 6 (See
  \cite{forest2002}~Chapter~K) \\ (Default: 2)

  \ttitem{NST} the number of integration steps.  (Default:~1)\\
  The body of each element is divided into \texttt{NST} equal slices and
  Forest-Yoshida integration is carried out on each slice.
  For best results \texttt{NST} should increase with strength
  and length of elements. The optimum \texttt{NST} value corresponds to 
  the value beyond which the studied properties no longer change. 
  However, for time consuming calculations the user may
  reduce \texttt{NST}. (See below the \texttt{RESPLIT} option for automatic
  adjustment.)\\
  This attribute sets the same \texttt{NST} value  for all "thick" elements
  ($l > 0$) of a beam-line; however each individual element may also have
  its own \texttt{NST} value defined independently 
  (\hyperref[sec:add-option-PTC]{see below}).

  \ttitem{EXACT} a logical flag to turn on calculations with an exact
  Hamiltonian, otherwise the expanded Hamiltonian is used. \\
  (Default:~false)

  \ttitem{OFFSET\_DELTAP} \textbf{[ Beware: Expert attribute! ]}
  provides relative momentum deviation of the reference particle (6D case
  ONLY). This option implies \texttt{TOTALPATH=true}. \\
  (Default: 0.0)

  \ttitem{ERRORS\_OUT} a logical flag to write-out multipolar errors
  in \hyperref[sec:efcomp]{\texttt{EFCOMP}} table format. 
  \\ (Default: false) \\
  Two tables are created and filled: "errors\_field" contains only
  field errors, "errors\_total" contains also desired field
  components, which can include the strength of correctors.  
  The choice of magnets is defined by the \texttt{MAGNET\_NAME}
  attribute (see below). 
  The tables can be \hyperref[sec:write]{written} to file, and can be
  read back via the \texttt{ERRORS\_IN} flag.\\ 
  The \texttt{ERRORS\_IN} flag has precedence over this \texttt{ERRORS\_OUT} flag.

  \ttitem{MAGNET\_NAME} a string giving a simple selection for the
  names of magnet to be used for an error write-out using the
  \texttt{ERRORS\_OUT} flag (see above). The errors are recorded for all 
  magnets with names starting with the exact string given here, which
  would be equivalent to the ??? regular expression.\\
  (Default:~nil)

  \ttitem{RESPLIT} a logical flag to apply the \ptc resplit
  procedure. This is meant to create an "adaptive" setting of the
  \texttt{METHOD} and \texttt{NST} attributes according to the strengths
  of quadrupoles (using the \texttt{THIN}  attribute) and dipoles (using
  the \texttt{XBEND} attribute). The \texttt{EVEN} attribute further
  controls the number of splits.  \\
  (Default:~false)

  \ttitem{THIN} is the main \texttt{RESPLIT} attribute and is meant for
  splitting quadrupoles according to their strength. The default value
  \texttt{THIN=0.001} has shown in practice to work well without costing
  too much with respect of performance.
  
  \ttitem{XBEND} is an optional \texttt{RESPLIT} attribute and is meant for
  splitting dipoles. A value \texttt{XBEND=0.001} is also advisable for
  dipoles. \\
  (Default: -1.0 for no splitting)

  \ttitem{EVEN} a logical switch to ensure even number of splits when
  using the \texttt{RESPLIT}  procedure of \ptc, which is particularly
  useful when one attempts to calculate \texttt{PTC\_TWISS} with the
  \texttt{CENTER\_MAGNETS} option, i.e. to calculate the \texttt{TWISS} 
  parameters in the center of the element. 
  Uneven number of splits is ensured with \texttt{EVEN=false}.
  (Default:~true)
\end{madlist}

\section{PTC\_SETSWITCH}
\label{sec:ptc-setswitch}

The \texttt{PTC\_SETSWITCH} command allows to set some global \ptc states and to configure the interface between {\madx} and {\ptc}, adapting the behavior of the program to the needs.   

\madbox{
	PTC\_SETSWITCH, \=DEBUGLEVEL=integer, \\
	\>EXACT\_MIS=logical,\\
	\>TOTALPATH=logical,\\
	\>RADIATION=logical,\\
	\>ENVELOPE=logical,\\
	\>STOCHASTIC=logical,\\
	\>MODULATION=logical,\\
	\>FRINGE=logical,\\
	\>TIME=logical,\\
	\>SEED=integer,\\
	\>MAXACCELERATION=logical;
}

The command parameters and switches are:
\begin{madlist}
	\ttitem{DEBUGLEVEL} (Default: 1)\\
	Sets the level of debugging printout: 0 prints none, 4 prints everything   
	
	\ttitem{EXACT\_MIS} (Default: false)\\
	Switch ensures exact misalignment treatment.   
	
	\ttitem{TOTALPATH} (Default: false)\\
	If true, the 6th variable of PTC, i.e. 5th of MAD-X, is the total path.  \\
	If false it is deviation from the reference particle,
	which is normally the closed orbit for closed layouts. \\
	This switch changes behaviour of the RF cavities, 
	including RF multipoles and crab cavities. 
	If it is false, the time of flight effect between the cavities
	(or the ring length) is ignored because the kick is proportional to \\
	$sin(2\pi \cdot \rm{FREQ} \cdot t/c + \rm{LAG})$, where $t$ is the time coodinate.
	So only distance from the synchronous particle plays a role.
	For example, changing ring length will not affect the cavity.
	On the other hand, it gurantees that the defined LAG is observed.
	Conversely, if \texttt{TOTALPATH} is true then $t$ becomes the total time of flight
	so its effect is accounted for. In the case when cavity is detuned 
	the closed orbit momentum will change and in ray tracking phase slippage from turn
	to turn will be seen. However, \texttt{LAG} of cavities needs to be 
	carefuly calculated because its phasing will depend on its position. 
	Naturally, in cases with only one RF cavity the closed orbit search will automatically
	determine the offset. 
	
	\ttitem{RADIATION} (Default: false)\\    
	Sets the radiation switch/internal state of PTC. In PTC basically all the elements
	radiate including sextupoles, solenoids and orbit correctors.

	\ttitem{ENVELOPE} (Default: false)\\    
	Sets the envelope switch/internal state of PTC. It allows to calculate 
	dumping due to radiation and stochastic effects. 
	Warning: this makes the tracking approximately twice slower, so low order
	should be used. 

	\ttitem{STOCHASTIC} (Default: false)\\    
	Sets the stochastic switch/internal state of PTC. 
	It enables stochastic emission of photons in ray tracking,  
                    it only affects \texttt{PTC\_TRACK} and \texttt{PTC\_TRACKLINE}.
	The emission is calculated during map tracking therefore 
	\texttt{PTC\_TWISS} or \texttt{PTC\_NORMAL} 
	needs to be invoked before launching the tracking 
	(also with \texttt{RADIATION}, \texttt{ENVELOPE} and \texttt{STOCHASTIC} set to true). 
	Every tracked ray will receive the same stochastic kicks.

	\ttitem{MODULATION} (Default: false)\\    
	Sets the modulation switch/internal state of PTC. 
	It needs to be set to true to observe effect of AD dipoles.
	
	\ttitem{FRINGE} (Default: false)\\    
	Sets the fringe switch/internal state of PTC. \\ 
	If true the influence of the fringe fields is evaluated for all
	elements. \\       
	Please note that currently fringe fields are always taken into
	account for some elements (e.g. traveling wave cavities) even if
	this flag is set to false. The detailed list of elements
	will be provided later, when the situation in this matter will be
	definitely settled.    
	
	\ttitem{TIME} (Default: true)\\  
	If true, Selects time of flight (\textit{cT} to be precise) rather
	than path length as the 6th variable of PTC, i.e. 5th of MAD-X. \\
                    This option changes the canonical coordinate system depending
	whether the calculation is done in 5D or 6D: 
	 \begin{madlist}
	   \ttitem{5D} if \texttt{TIME} is true, the fifth coordinate is
	   \hyperref[subsec:tables-canon]{\texttt{PT}},
	   $p_t = \Delta E / p_0 c$ \\  
	   if \texttt{TIME} is false, the fifth coordinate is
	   \hyperref[subsec:tables-canon]{\texttt{DELTAP}},
	   $\delta_p = \Delta p / p_0$

	   \ttitem{6D} if \texttt{TIME} is true, the
	   \hyperref[subsec:tables-canon]{\madx coordinate system}
	   \{$-ct$, $p_t$\} is used. \\    
	   if \texttt{TIME} is false, the second \ptc coordinate system
	   \{-pathlength, $\delta_p$\} is used. 
	 \end{madlist}

	\textbf{Note:} at small energy ($\beta_0 << 1$),
	momentum-dependent variables like dispersion will depend  strongly on
	the choice of  the logical input variable \texttt{TIME}. In fact, the
	derivative ($\frac{\partial}{\partial \delta_p}$)  and
	($\frac{\partial}{\partial p_t}$)  are different by the
	factor $\beta_0$. One would  therefore typically  choose
	the option \texttt{TIME=false},  which sets the fifth variable to
	the relative momentum deviation $\delta_p$.

	
	\ttitem{SEED} (Default: 123456789)\\
	Sets the seed of PTC random number generator,
	which is independent from the MADX generators. 
	
	\ttitem{MAXACCELERATION} (Default: true)\\
	Switch to set cavities phases so the reference orbit is always on
	the crest, i.e. gains max energy.

	
\end{madlist}

%% \textbf{PROGRAMMERS MANUAL}   
%% Values of the switches are stored in Fortran 90 module
%% mad\_ptc\_intstate (mad\_ptc\_intstate.f90). The command is processed by
%% pro\_ptc\_setswitch C function, in file madxn.c, that calls appropriate
%% routines of the Fortran module to set each of the switches:   
%% \begin{itemize}
%%    \item  ptc\_setdebuglevel 
%%    \item  ptc\_setaccel\_method 
%%    \item  ptc\_setexactmis 
%%    \item  ptc\_setradiation 
%%    \item  ptc\_settotalpath 
%%    \item  ptc\_settime 
%%    \item  ptc\_setfringe  
%% \end{itemize}

\section{PTC\_MOVE\_TO\_LAYOUT}
\label{sec:ptc-move-to-layout}

Several \ptc layouts can be created within a single \ptc-"universe". 
The layouts are automatically numbered with sequential integers by the 
\madx code. The \texttt{PTC\_MOVE\_TO\_LAYOUT} command is used to
activate a specific layout, and the next \ptc commands will be
applied to this active \ptc layout until a new \ptc layout is created
or activated. 

\madbox{
PTC\_MOVE\_TO\_LAYOUT, INDEX=integer;
}

The only attribute is:
\begin{madlist}
	\ttitem{INDEX} is the numeric index of the \ptc layout to be
	activated.\\ (Default:~1)
\end{madlist}

\section{PTC\_READ\_ERRORS}
\label{sec:ptc-read-errors}

The \texttt{PTC\_READ\_ERRORS} command reads any number of
\textbf{"errors\_read"} table through the 
\hyperref[sec:readtable]{\texttt{READTABLE}} mechanism.

\madbox{
PTC\_READ\_ERRORS, OVERWRITE=logical; 
}

The only attribute is 
\begin{madlist}
   \ttitem{OVERWRITE} a flag to specify that the read-in errors
   overwrite previous errors instead of adding the read-in errors to
   existing errors, ie multipole components already present.\\
   (Default:~false)  
\end{madlist}

\textbf{Note:}\\
  For calculations with exact flag set to true, 
  \texttt{SECTOR\_NMUL} parameter of \texttt{PTC\_CREATE\_UNIVERSE} 
  needs to be set to a value bigger than the highest order of an error in bending magnets.

\textbf{Note:}\\
Because of the way the table is read in memory, a warning will always be
issued by default in the form:
{\small
\begin{verbatim}
warning: string_from_table_row: row out of range: errors_read->name[1>=n+1<=n]
\end{verbatim}
}
where \texttt{n} is  the number of records read from the table. 
This warning has no consequence on the errors read and the following
calculation. \\
The warning is purely the result of the way that the reading loop is
programmed with a break based on the return value of the routine
\textsl{string\_from\_table\_row}.  
But if \textsl{string\_from\_table\_row} tries to read in a row (\texttt{n+1})
past the last row (\texttt{n}) of the table, it prints a warning before
returning a value that will effectively break the loop. Of course this
will only happen if the \texttt{WARN} option is true and this can be turned
off with \madxmp{OPTION, -WARN;}

       

\section{PTC\_ALIGN}
\label{sec:ptc-align}

The \texttt{PTC\_ALIGN} command is used to apply the \madx alignment
errors to the current PTC layout, and takes no attributes.

\madbox{
PTC\_ALIGN;
}


\section{PTC\_END}
\label{sec:ptc-end}

The \texttt{PTC\_END} command turns off the \ptc environment,
which releases all memory and returns control to the \madx world proper.

\madbox{
PTC\_END;
}


\section{Additional Options for Physical Elements}
\label{sec:add-option-PTC}

For some of the \madx elements, additional attributes can be defined
that are available to \ptc only. \ptc also uses standard \madx
attributes in a slightly different way.

\madbox{
SBEND | \=RBEND | QUADRUPOLE | SEXTUPOLE | OCTUPOLE | SOLENOID ,\\
        \>L=real, ... , TILT=real, ... , NST=integer, ... ,\\
        \>KNL=\{0, real, real,...\}, KSL=\{0, real, real,...\}; 
}

These attributes are:
\begin{madlist}
  \ttitem{L} the length of the element. \\ \ptc treats bending magnets
  (\hyperref[sec:bend]{\texttt{SBEND}} or \hyperref[sec:bend]{\texttt{RBEND}})
  as \hyperref[sec:marker]{\texttt{MARKER}} if their length is equal to zero. 
     
  \ttitem{NST} gives a specific \hyperref[sec:ptc-twiss]{\texttt{NST}}
  values for a particular "thick" element ($L > 0$). \\
  \\
  For example RF cavities are represented in \madx 
  by a single kick, while \ptc splits the RF cavity into
  \texttt{NST} segments thereby taking into account properly the
  transit-time effects of the cavity. Specifying explicitly \texttt{NST=1}
  for RF cavity reproduces in \ptc the approximate results of \madx,
  ignoring transit time effects.
  \\

  \ttitem{KNL, KSL} The full range of  normal and skew multipole
  components on the bench can be  specified for the following physical
  elements: 
  \hyperref[sec:bend]{\texttt{sbend}}, 
  \hyperref[sec:bend]{\texttt{rbend}},
  \hyperref[sec:quadrupole]{\texttt{quadrupole}},
  \hyperref[sec:sextupole]{\texttt{sextupole}},  
  \hyperref[sec:octupole]{\texttt{octupole}} and 
  \hyperref[sec:solenoid]{\texttt{solenoid}}. 
  \texttt{KNL} and \texttt{KSL} multipole coefficients are specified as the
  integrated value ($\int K ds$) of the field components along the
  magnet axis. The multipole components in \ptc are spread over
  the length of thick elements. This is a considerable
  advantage of \ptc input compared  to \madx  which allows only
  \hyperref[sec:multipole]{thin multipoles}. 
  \begin{madlist}
    \ttitem{KNL} is an array representing the normal multipole
    coefficients. \\ (Default:~0~m$^{-1}$)
    \ttitem{KSL} is an array representing the skew multipole
    coefficients. \\ (Default:~0~m$^{-1}$)
  \end{madlist}
  To preserve the reference orbit of straight elements, the dipole
  components for those elements are ignored and must be specified as
  zero: \texttt{KNL(0)=0, KSL(0)=0}. 
   
  A full range of additional multipole \hyperref[sec:efcomp]{field
    errors} can be additionally specified with the
  \hyperref[sec:efcomp]{\texttt{EFCOMP}} command. Errors are added to
  the above multipole fields on the bench. 

\end{madlist}

%\href{mailto:kapin@itep.ru}{  V.Kapin}(ITEP) and 
%\href{mailto:Frank.Schmidt@cern.ch}{  F.Schmidt}, March  2006

