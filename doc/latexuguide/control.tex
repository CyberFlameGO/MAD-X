%%%\title{Range Selection}
%  Changed by: Chris ISELIN, 27-Jan-1997 
%  Changed by: Hans Grote, 15-Jan-2003 


%%%%\title{Range Selection}
%  Changed by: Chris ISELIN, 27-Jan-1997 
%  Changed by: Hans Grote, 15-Jan-2003 


%%%%\title{Range Selection}
%  Changed by: Chris ISELIN, 27-Jan-1997 
%  Changed by: Hans Grote, 15-Jan-2003 


%%%%\title{Range Selection}
%  Changed by: Chris ISELIN, 27-Jan-1997 
%  Changed by: Hans Grote, 15-Jan-2003 


%\input{control/control}
%\input{control/foot}
%\include{control/general}
%\include{control/special}


\chapter{General Control Statements} 

\madx consists of a core program, and modules for specific tasks such as
twiss parameter calculation, matching, thin lens tracking, and so on.  
 
The statements listed here are those executed by the program core. They
deal with the I/O, element and sequence declaration, sequence
manipulation, statement flow control (e.g. IF, WHILE), MACRO
declaration, saving sequences onto files in \madx or \madeight format,
and so on.  


%% Moved to TWISS chapter
%% \subsection{COGUESS}
%% \label{subsec:coguess}

%% In order to help the initial finding of the closed orbit by the
%% \texttt{TWISS} module, it is possible to specify an initial guess. 

%% \madbox{
%% COGUESS, \=TOLERANCE=real, \\
%%          \>X=real, PX=real, Y=real, PY=real, T=real, PT=real, \\
%%          \>CLEAR=logical;
%% }
%% sets the required convergence precision in the closed orbit search
%% ("tolerance", see as well Twiss command
%% \href{../twiss/twiss.html#tolerance}{tolerance}).  

%% The other parameters define a first guess for all future closed orbit
%% searches in case they are different from zero.  

%% The clear parameter in the argument list resets the tolerance to its default value 
%% and cancels the effect of coguess in defining a first guess for subsequent 
%% closed orbit searches. \\
%% Default = false, {\tt clear} alone is equivalent to {\tt clear=true}


\section{EXIT, QUIT, STOP}
\label{sec:exit}\label{sec:quit}\label{sec:stop}
Any of these three commands ends the execution of \madx:
\madbox{
EXIT;
}
\madbox{
QUIT;
}
\madbox{
STOP;
}


\section{HELP}
\label{sec:help}
The HELP command prints all parameters, and their defaults values, for
the statement given; this includes basic element types.
\madbox{
HELP, statement\_name;
}

\section{SHOW}
\label{sec:show}
The SHOW command prints the "command" (typically "beam", "beam\%sequ",
or an element name), with the actual value of all its parameters.  
\madbox{
SHOW, command;
}

\section{VALUE}
\label{sec:value}
The VALUE command evaluates the current value of all listed expressions,
constants or variables, and prints the result in the form of \madx
statements on the assigned output file. 
\madbox{
VALUE, expression\{, expression\} ;
}

Example: \\
\madxmp{
a = clight/1000.; \\
value, a, pmass, exp(sqrt(2));
}
results in 
\madxmp{
a = 299792.458         ; \\
pmass = 0.938272046        ; \\
exp(sqrt(2)) = 4.113250379        ; \\
}

\section{OPTION}
\label{sec:option}

The \texttt{OPTION} commands sets the logical value of a number of flags
that control the behavior of \madx.

\madbox{
OPTION, flag=logical;
}

Because all attributes of \texttt{OPTION} are logical flags, the
following two statements are identical:
\madxmp{
OPTION, flag = true;\\
OPTION, flag;
}
And the following two statements are also identical:
\madxmp{
OPTION, flag = false;\\
OPTION, -flag;
}

Several flags can be set in a single \texttt{OPTION} command, e.g.
\madxmp{
OPTION, ECHO, WARN=true, -INFO, VERBOSE=false;
}

The available flags, their default values and their effect on \madx if
they are set to \texttt{TRUE} are listed in table 

\begin{table}[ht]
  \caption{Flags available to \texttt{OPTION} command}
  \vspace{1ex}
  \begin{center}
   \begin{tabular}{|l|c|l|}
     \hline
     {\bf FLAG }  & {\bf default} & {\bf effect if \texttt{TRUE}} \\
     \hline
     \texttt{ECHO}  & true  & echoes the input on the standard output file \\
     \texttt{WARN}  & true  & issues warning statements\\
     \texttt{INFO}  & true  & issues information statements\\
     \texttt{DEBUG} & false & issues debugging information \\
     \texttt{ECHOMACRO} & false & issues macro expansion printout for debugging \\
     \texttt{VERBOSE} & false & issues additional printout in makethin \\
     \texttt{TRACE}   & false & prints the system time after each command \\
     \texttt{VERIFY}  & false & issues a warning if an undefined variable is used \\
     \hline
     \texttt{TELL}  & false & prints the current value of all options \\
     \texttt{RESET} & false & resets all options to their defaults \\
     \hline
     \texttt{NO\_FATAL\_STOP} & false & Prevents madx from stopping in case of a fatal error \\
                              &       & {\bf Use at your own risk !} \\
     \hline
     \texttt{RBARC}     & true & converts the RBEND straight length into the arc length \\
     \texttt{THIN\_FOC} & true & enables the 1/(rho**2) focusing of thin dipoles \\
     \texttt{BBORBIT}   & false & the closed orbit is modified by beam-beam kicks \\
     \texttt{SYMPL}     & false & all element matrices are symplectified in Twiss \\
     \texttt{TWISS\_PRINT} & true & controls whether the twiss command produces output \\
     \texttt{THREADER}  & false & enables the threader for closed orbit finding in Twiss \\ 
                        &       & (see Twiss module) \\ 
     \hline
     \texttt{BB\_ULTRA\_RELATI} & false  & To be documented \\
     \texttt{BB\_SXY\_UPDATE}   & false  & To be documented \\
     \texttt{EMITTANCE\_UPDATE} & true   & To be documented \\
     \texttt{FAST\_ERROR\_FUNC} & false  & To be documented \\
     \hline
     \end{tabular}
   \end{center}
\end{table}

%% \begin{verbatim}
%%   name           default meaning if true
%%   ====           ======= ===============
%%   echo            true   echoes the input on the standard output file
%%   warn            true   issues warnings
%%   info            true   issues information
%%   debug           false  issues debugging information
%%   echomacro       false  issues macro expansion printout for debugging
%%   verbose         false  issues additional printout in makethin
%%   trace           false  prints the system time after each command
%%   verify          false  issues a warning if an undefined variable is used
%%   tell            false  prints the current value of all options
%%   reset           false  resets all options to their defaults
%%   no_fatal_stop   false  Prevents madx from stopping in case of a fatal error. 
%%                          Use at your own risk.

%%   rbarc           true   converts the RBEND straight length into the arc 
%%                          length
%%   thin_foc        true   if false suppresses the 1(rho**2) focusing of thin 
%%                          dipoles
%%   bborbit         false  the closed orbit is modified by beam-beam kicks
%%   sympl           false  all element matrices are symplectified in Twiss
%%   twiss_print     true   controls whether the twiss command produces output.
%%   threader        false  enables the threader for closed orbit finding in Twiss.
%%                          (see Twiss module)

%%   bb_ultra_relati false  To be documented
%%   bb_sxy_update   false  To be documented 
%%   emittance_update true  To be documented
%%   fast_error_func false  To be documented
%% \end{verbatim} 

The option \texttt{RBARC} is implemented for backwards compatibility
with \madeight up to version 8.23.06 included; in this version, the
RBEND length was just taken as the arc length of an SBEND with inclined
pole faces, contrary to the \madeight manual.  



\section{EXEC}
\label{sec:exec}
Each statement may be preceded by a label, when parsed and executed the
statement is then also stored and can be executed again with
\madbox{
EXEC, label;
}
This mechanism can be invoked any number of times, and the executed
statement may be calling another \texttt{EXEC}, etc. 

Note however, that the main usage of this \madx construct is the
execution of a \href{special.html#macro}{macro}.   

\madxmp{
tw: TWISS, FILE, SAVE; ! first execution of TWISS \\
... \\
EXEC, tw; ! second execution of the same TWISS command \\
}


\section{SET}
\label{sec:set}
The SET command is used in two forms:
\madbox{
SET, FORMAT=string \{, string\} ;\\
SET, SEQUENCE=string;
}


The first form of the \texttt{SET} command defines the formats for the
output precision that \madx uses with the \texttt{SAVE}, \texttt{SHOW},
\texttt{VALUE} and \texttt{TABLE} commands. The formats can be
given in any order and stay valid until replaced. 

The formats follow the C convention and must be included in double
quotes. The allowed formats are \\
\texttt{{\it n}d} for integers with a field-width = {\it n}, \\
\texttt{{\it m}.{\it n}f} or \texttt{{\it m}.{\it n}g} or \texttt{{\it
    m}.{\it n}e} for floats with field-width = {\it m} and precision =
       {\it n}, \\
\texttt{{\it n}s} for strings with a field-width = {\it n}.\\
The default is "right adjusted", a "-" changes it to "left adjusted".

{\bf Example:}\\
\madxmp{
SET, FORMAT="12d", "-18.5e", "25s";
}

%% \begin{verbatim}
%% "nd" for integer with n = field width.
%% \end{verbatim}
%% \begin{verbatim}
%% "m.nf" or "m.ng" or "m.ne" for floating, m field width, n precision.
%% \end{verbatim}
%% \begin{verbatim}
%% "ns" for string output.
%% \end{verbatim} 


The default formats are \texttt{"10d"},\texttt{"18.10g"} and \texttt{"-18s"}.

Example: 
\begin{verbatim}
set,format="22.14e";
\end{verbatim} 
changes the current floating point format to 22.14e; the other formats remain untouched. 
\begin{verbatim}
set,format="s","d","g";
\end{verbatim} 
sets all formats to automatic adjustment according to C conventions. 



The second form of the \texttt{SET} command allows to select the
current sequence without the "USE" command, which would
bring back to a bare lattice without errors. The command only works
if the chosen sequence has been previously activated with a \texttt{USE} command,
otherwise a warning is issued and \madx continues with the
unmodified current sequence. This command is particularly useful for
commands that do not have the sequence as an argument like "EMIT" or
"IBS". 



\section{SYSTEM}
\label{sec:system}
\madbox{
SYSTEM, "string";
}
transfers the quoted string to the system for execution. The quotes are
stripped and no check of the return status is performed bt \madx.

{\bf Example:}\\ 
\madxmp{
SYSTEM,"ln -s /afs/cern.ch/user/j/joe/public/some/directory shortname";
}
makes a local link to an AFS directory with the name "shortname" on a
UNIX system. 

Attention: Although this command is very convenient, it is clearly not portable
across systems and should probably avoid it if you intend to share \madx scripts with
collaborators working on other platforms. 

\section{TITLE}
\label{sec:title}
\madbox{
TITLE, "string";
}
the string in quotes is inserted as title in various table outputs and
plot results.  


\section{USE}
\label{sec:use}
\madx operates on beamlines that must be loaded and expanded in memory
before other commands can be invoked. The \texttt{USE} allows this
loading and expansion.

\madbox{
USE, \=PERIOD=sequence\_name, SEQUENCE=sequence\_name, \\
     \>RANGE=range, \\
     \>SURVEY=logical;
}

The attributes to the \texttt{USE} command are:
\begin{madlist}
  \ttitem{SEQUENCE} name of the sequence to be loaded and expanded. 
  \ttitem{PERIOD} name of the sequence to be loaded and expanded. \\ 
  \texttt{PERIOD} is an alias to \texttt{SEQUENCE} that was kept for
  backwards compatibility with \madeight and only one of them should be
  specified in a \texttt{USE} statement. 
  \ttitem{RANGE} specifies a \hyperref[sec:range]{range}.   
  restriction so that only the specified part of the named sequence is
  loaded and  expanded.
  \ttitem{SURVEY} option to plug the survey data into the sequence elements
  nodes on the first pass (see \hyperref[chap:survey]{\tt SURVEY}).
\end{madlist}

Note that reloading a sequence with the \texttt{USE} command reloads a
bare sequence and that any \texttt{ERROR} or orbit correction previously
assigned or associated to the sequence are forgotten. 
A mechanism to select a sequence without this side effect of the
\texttt{USE} command is provided with the \texttt{SET, SEQUENCE=...} command.


\section{SELECT} 
\label{sec:select}

\begin{verbatim}
select, flag=string, range=string, class=string, pattern=string,
        sequence=string, full, clear,
        column = string{, string},  slice=integer, thick=logical;
\end{verbatim} 
selects one or several elements for special treatment in a subsequent
command based on selection criteria.

The selection criteria on a single SELECT statement are logically
ANDed, in other words, selected elements have to fulfill the {\tt RANGE},
{\tt CLASS}, and {\tt PATTERN} criteria.  
The selection criteria on different SELECT statements are logically
ORed, in other words selected elements have to fulfill any of the
selection criteria accumulated by the different statements.   
All selections for a given command remain valid until the "clear" argument
is specified; 

The "flag" argument allows a determination of the applicability of the
SELECT statement and can be one of the following: 
\begin{madlist}
   \ttitem{seqedit} selection of elements for the
     \hyperref[sec:seqedit]{seqedit} module.  
   \ttitem{error} selection of elements for the
     \hyperref[chap:error]{error} assignment module.  
   \ttitem{makethin} selection of elements for the
     \hyperref[chap:makethin]{makethin} module that
     converts the sequence into one with thin elements only.  
   \ttitem{sectormap} selection of elements for the
     \hyperref[subsec:sectormap]{sectormap} output file
     from the Twiss module.  
   \ttitem{save} selection of elements for the \hyperref[sec:save]{\tt
     SAVE} command.   
   \ttitem{table} is a table name such as {\tt twiss}, {\tt track}
     etc., and the rows and columns to be written are selected.  
\end{madlist} 

The statement
\madxmp{
SELECT, FLAG=name, FULL;
}
selects ALL positions in the sequence for the flag "name". This is the default
for flags for all tables and {\tt MAKETHIN}.

The statement 
\madxmp{
SELECT, FLAG=name, CLEAR;
}
deselects ALL positions in the sequence for the flag "name". This is the default
for flags {\tt ERROR} and {\tt SEQEDIT}.

"slice" is only used by \hyperref[chap:makethin]{makethin} and
prescribes the number of slices into which the selected elements have to
be cut (default = 1).  

"column" is only valid for tables and determines the selection of columns
to be written into the TFS file. The "name" argument is special in that
it refers to the actual name of the selected element. For an example,
see \hyperref[sec:select]{SELECT}.  

"thick" is used to determine whether the selected elements will be
treated as thick elements by the MAKETHIN command. This only applies to
QUADRUPOLES and BENDS for which thick maps have been explicitely
derived. (see ...) 
%%2014-Apr-08  17:43:44  ghislain:  A completer.

Example: 
% keep verbatim for now and until ^ is solved
\begin{verbatim} 
select, flag = error, class = quadrupole, range = mb[1]/mb[5];
select, flag = error, pattern = "^mqw.*";
\end{verbatim}
selects all quadrupoles in the range mb[1] to mb[5], as well as all
elements (in the whole sequence) with name starting with "mqw", for 
treatment by the error module.  

Example:  
\madxmp{
select, flag=save, class=variable, pattern="abc.*";
save, file=mysave;
}
will save all variables (and sequences) containing "abc" in their name.
However note that since the element class "variable" does not exist, any
element with name containing "abc" will not be saved. 

\vskip 1cm
\hrule
\vskip 1cm

%% Imported from chapter 2
%\subsection{Selection Statements}

The elements, or a range of elements, in a sequence can be selected for
various purposes. Such selections remain valid until cleared (in
difference to \madeight); it is therefore recommended to always start with a  

\begin{verbatim}
select, flag =..., clear;
\end{verbatim} 
before setting a new selection. 
\begin{verbatim}
SELECT, FLAG=name, RANGE=range, CLASS=class, PATTERN=pattern [,FULL] [,CLEAR];
\end{verbatim} 
where the name for FLAG can be one of ERROR, MAKETHIN, SEQEDIT or the
name of a twiss table which is established for all sequence positions in
general.  

Selected elements have to fulfill the \href{ranges.html#range}{RANGE},
\href{ranges.html#class}{CLASS}, and \href{wildcard.html}{PATTERN}
criteria.  

Any number of SELECT commands can be issued for the same flag and are
accumulated (logically ORed). In this context note the following:  

\begin{verbatim}
SELECT, FLAG=name, FULL;
\end{verbatim} 
selects all positions in the sequence for this flag. This is the default
for all tables and makethin, whereas for ERROR and SEQEDIT the default
is "nothing selected".  

%\href{save_select}{}
\label{save_select}
SAVE: A SELECT,FLAG=SAVE statement causes the
selected sequences, elements, and variables to be written into the save
file. A class (only used for element selection), and a pattern can be
specified. Example:  
\begin{verbatim}
select, flag=save, class=variable, pattern="abc.*";
save, file=mysave;
\end{verbatim} 
will save all variables (and sequences) containing "abc" in their name,
but not elements with names containing "abc" since the class "variable"
does not exist (astucieux, non ?).  

SECTORMAP: A SELECT,FLAG=SECTORMAP statement causes sectormaps to be
written into the file "sectormap" like in \madeight. For the file to be
written, a flag SECTORMAP must be issued on the TWISS command in
addition.  

TWISS: A SELECT,FLAG=TWISS statement causes the selected rows and
columns to be written into the Twiss TFS file (former OPTICS command in
\madeight). The column selection is done on the same select. See as well
example 2.  

%% Example 1:  
%% \begin{verbatim}
%% TITLE,'Test input for MAD-X';

%% option,rbarc=false; // use arc length of rbends
%% beam; ! sets the default beam for the following sequence
%% option,-echo;
%% call file=fv9.opt;  ! contains optics parameters
%% call file="fv9.seq"; ! contains a small sequence "fivecell"
%% OPTION,ECHO;
%% SELECT,FLAG=SECTORMAP,clear;
%% SELECT,FLAG=SECTORMAP,PATTERN="^m.*";
%% SELECT,FLAG=TWISS,clear;
%% SELECT,FLAG=TWISS,PATTERN="^m.*",column=name,s,betx,bety;
%% USE,PERIOD=FIVECELL;
%% twiss,file=optics,sectormap;
%% stop;
%% \end{verbatim} 

%% This produces a file \href{sectormap.html}{sectormap}, and a
%% twiss output file \label{tfs} (name = optics):  
%% \begin{verbatim}
%% @ TYPE             %05s "TWISS"
%% @ PARTICLE         %08s "POSITRON"
%% @ MASS             %le          0.000510998902
%% @ CHARGE           %le                       1
%% @ E0               %le                       1
%% @ PC               %le           0.99999986944
%% @ GAMMA            %le           1956.95136738
%% @ KBUNCH           %le                       1
%% @ NPART            %le                       0
%% @ EX               %le                       1
%% @ EY               %le                       1
%% @ ET               %le                       0
%% @ LENGTH           %le                   534.6
%% @ ALFA             %le        0.00044339992938
%% @ ORBIT5           %le                      -0
%% @ GAMMATR          %le           47.4900022541
%% @ Q1               %le           1.25413071556
%% @ Q2               %le           1.25485338377
%% @ DQ1              %le           1.05329608302
%% @ DQ2              %le           1.04837000224
%% @ DXMAX            %le           2.17763211131
%% @ DYMAX            %le                       0
%% @ XCOMAX           %le                       0
%% @ YCOMAX           %le                       0
%% @ BETXMAX          %le            177.70993499
%% @ BETYMAX          %le           177.671582415
%% @ XCORMS           %le                       0
%% @ YCORMS           %le                       0
%% @ DXRMS            %le           1.66004270906
%% @ DYRMS            %le                       0
%% @ DELTAP           %le                       0
%% @ TITLE            %20s "Test input for MAD-X"
%% @ ORIGIN           %16s "MAD-X 0.20 Linux"
%% @ DATE             %08s "07/06/02"
%% @ TIME             %08s "14.25.51"
%% * NAME               S                  BETX               BETY               
%% $ %s                 %le                %le                %le                
%%  "MSCBH"             4.365              171.6688159        33.31817319       
%%  "MB"                19.72              108.1309095        58.58680717       
%%  "MB"                35.38              61.96499987        102.9962313       
%%  "MB"                51.04              34.61640793        166.2227523       
%%  "MSCBV.1"           57.825             33.34442808        171.6309057       
%%  "MB"                73.18              58.61984637        108.0956006       
%%  "MB"                88.84              103.0313887        61.93159422       
%%  "MB"                104.5              166.2602486        34.58939635       
%%  "MSCBH"             111.285            171.6688159        33.31817319       
%%  "MB"                126.64             108.1309095        58.58680717       
%%  "MB"                142.3              61.96499987        102.9962313       
%%  "MB"                157.96             34.61640793        166.2227523       
%%  "MSCBV"             164.745            33.34442808        171.6309057       
%%  "MB"                180.1              58.61984637        108.0956006       
%%  "MB"                195.76             103.0313887        61.93159422       
%%  "MB"                211.42             166.2602486        34.58939635       
%%  "MSCBH"             218.205            171.6688159        33.31817319       
%%  "MB"                233.56             108.1309095        58.58680717       
%%  "MB"                249.22             61.96499987        102.9962313       
%%  "MB"                264.88             34.61640793        166.2227523       
%%  "MSCBV"             271.665            33.34442808        171.6309057       
%%  "MB"                287.02             58.61984637        108.0956006       
%%  "MB"                302.68             103.0313887        61.93159422       
%%  "MB"                318.34             166.2602486        34.58939635       
%%  "MSCBH"             325.125            171.6688159        33.31817319       
%%  "MB"                340.48             108.1309095        58.58680717       
%%  "MB"                356.14             61.96499987        102.9962313       
%%  "MB"                371.8              34.61640793        166.2227523       
%%  "MSCBV"             378.585            33.34442808        171.6309057       
%%  "MB"                393.94             58.61984637        108.0956006       
%%  "MB"                409.6              103.0313887        61.93159422       
%%  "MB"                425.26             166.2602486        34.58939635       
%%  "MSCBH"             432.045            171.6688159        33.31817319       
%%  "MB"                447.4              108.1309095        58.58680717       
%%  "MB"                463.06             61.96499987        102.9962313       
%%  "MB"                478.72             34.61640793        166.2227523       
%%  "MSCBV"             485.505            33.34442808        171.6309057       
%%  "MB"                500.86             58.61984637        108.0956006       
%%  "MB"                516.52             103.0313887        61.93159422       
%%  "MB"                532.18             166.2602486        34.58939635       
%% \end{verbatim}

 %% Example 2: 

%%  Addition of variables to (any internal) table: 
%% \begin{verbatim}
%% select, flag=table, column=name, s, betx, ..., var1, var2, ...; ! or
%% select, flag=table, full, column=var1, var2, ...; ! default col.s + new
%% \end{verbatim} 
%% will write the current value of var1 etc. into the table each time a new
%% line is added; values from the same (current) line can be accessed by
%% these variables, e.g.  
%% \begin{verbatim}
%% var1 := sqrt(beam->ex*table(twiss,betx));
%% \end{verbatim} 
%% in the case of table above being "twiss". The plot command accepts the
%% new variables.  

%% Remark: this replaces the "string" variables of MAD-8. 

%%  This example demonstrates as well the usage of a user defined table \label{ucreate}. 
%% \begin{verbatim}
%% beam,ex=1.e-6,ey=1.e-3;
%% // element definitions
%% mb:rbend, l=14.2, angle:=0,k0:=bang/14.2;
%% mq:quadrupole, l:=3.1,apertype=ellipse,aperture={1,2};
%% qft:mq, l:=0.31, k1:=kqf,tilt=-pi/4;
%% qf.1:mq, l:=3.1, k1:=kqf;
%% qf.2:mq, l:=3.1, k1:=kqf;
%% qf.3:mq, l:=3.1, k1:=kqf;
%% qf.4:mq, l:=3.1, k1:=kqf;
%% qf.5:mq, l:=3.1, k1:=kqf;
%% qd.1:mq, l:=3.1, k1:=kqd;
%% qd.2:mq, l:=3.1, k1:=kqd;
%% qd.3:mq, l:=3.1, k1:=kqd;
%% qd.4:mq, l:=3.1, k1:=kqd;
%% qd.5:mq, l:=3.1, k1:=kqd;
%% bph:hmonitor, l:=l.bpm;
%% bpv:vmonitor, l:=l.bpm;
%% cbh:hkicker;
%% cbv:vkicker;
%% cbh.1:cbh, kick:=acbh1;
%% cbh.2:cbh, kick:=acbh2;
%% cbh.3:cbh, kick:=acbh3;
%% cbh.4:cbh, kick:=acbh4;
%% cbh.5:cbh, kick:=acbh5;
%% cbv.1:cbv, kick:=acbv1;
%% cbv.2:cbv, kick:=acbv2;
%% cbv.3:cbv, kick:=acbv3;
%% cbv.4:cbv, kick:=acbv4;
%% cbv.5:cbv, kick:=acbv5;
%% !mscbh:sextupole, l:=1.1, k2:=ksf;
%% mscbh:multipole, knl:={0,0,0,ksf},tilt=-pi/8;
%% mscbv:sextupole, l:=1.1, k2:=ksd;
%% !mscbv:octupole, l:=1.1, k3:=ksd,tilt=-pi/8;

%% // sequence declaration

%% fivecell:sequence, refer=centre, l=534.6;
%%    qf.1:qf.1, at=1.550000e+00;
%%    qft:qft, at=3.815000e+00;
%% !   mscbh:mscbh, at=3.815000e+00;
%%    cbh.1:cbh.1, at=4.365000e+00;
%%    mb:mb, at=1.262000e+01;
%%    mb:mb, at=2.828000e+01;
%%    mb:mb, at=4.394000e+01;
%%    bpv:bpv, at=5.246000e+01;
%%    qd.1:qd.1, at=5.501000e+01;
%%    mscbv:mscbv, at=5.727500e+01;
%%    cbv.1:cbv.1, at=5.782500e+01;
%%    mb:mb, at=6.608000e+01;
%%    mb:mb, at=8.174000e+01;
%%    mb:mb, at=9.740000e+01;
%%    bph:bph, at=1.059200e+02;
%%    qf.2:qf.2, at=1.084700e+02;
%%    mscbh:mscbh, at=1.107350e+02;
%%    cbh.2:cbh.2, at=1.112850e+02;
%%    mb:mb, at=1.195400e+02;
%%    mb:mb, at=1.352000e+02;
%%    mb:mb, at=1.508600e+02;
%%    bpv:bpv, at=1.593800e+02;
%%    qd.2:qd.2, at=1.619300e+02;
%%    mscbv:mscbv, at=1.641950e+02;
%%    cbv.2:cbv.2, at=1.647450e+02;
%%    mb:mb, at=1.730000e+02;
%%    mb:mb, at=1.886600e+02;
%%    mb:mb, at=2.043200e+02;
%%    bph:bph, at=2.128400e+02;
%%    qf.3:qf.3, at=2.153900e+02;
%%    mscbh:mscbh, at=2.176550e+02;
%%    cbh.3:cbh.3, at=2.182050e+02;
%%    mb:mb, at=2.264600e+02;
%%    mb:mb, at=2.421200e+02;
%%    mb:mb, at=2.577800e+02;
%%    bpv:bpv, at=2.663000e+02;
%%    qd.3:qd.3, at=2.688500e+02;
%%    mscbv:mscbv, at=2.711150e+02;
%%    cbv.3:cbv.3, at=2.716650e+02;
%%    mb:mb, at=2.799200e+02;
%%    mb:mb, at=2.955800e+02;
%%    mb:mb, at=3.112400e+02;
%%    bph:bph, at=3.197600e+02;
%%    qf.4:qf.4, at=3.223100e+02;
%%    mscbh:mscbh, at=3.245750e+02;
%%    cbh.4:cbh.4, at=3.251250e+02;
%%    mb:mb, at=3.333800e+02;
%%    mb:mb, at=3.490400e+02;
%%    mb:mb, at=3.647000e+02;
%%    bpv:bpv, at=3.732200e+02;
%%    qd.4:qd.4, at=3.757700e+02;
%%    mscbv:mscbv, at=3.780350e+02;
%%    cbv.4:cbv.4, at=3.785850e+02;
%%    mb:mb, at=3.868400e+02;
%%    mb:mb, at=4.025000e+02;
%%    mb:mb, at=4.181600e+02;
%%    bph:bph, at=4.266800e+02;
%%    qf.5:qf.5, at=4.292300e+02;
%%    mscbh:mscbh, at=4.314950e+02;
%%    cbh.5:cbh.5, at=4.320450e+02;
%%    mb:mb, at=4.403000e+02;
%%    mb:mb, at=4.559600e+02;
%%    mb:mb, at=4.716200e+02;
%%    bpv:bpv, at=4.801400e+02;
%%    qd.5:qd.5, at=4.826900e+02;
%%    mscbv:mscbv, at=4.849550e+02;
%%    cbv.5:cbv.5, at=4.855050e+02;
%%    mb:mb, at=4.937600e+02;
%%    mb:mb, at=5.094200e+02;
%%    mb:mb, at=5.250800e+02;
%%    bph:bph, at=5.336000e+02;
%% end:marker, at=5.346000e+02;
%% endsequence;

%% // forces and other constants

%% l.bpm:=.3;
%% bang:=.509998807401e-2;
%% kqf:=.872651312e-2;
%% kqd:=-.872777242e-2;
%% ksf:=.0198492943;
%% ksd:=-.039621283;
%% acbv1:=1.e-4;
%% acbh1:=1.e-4;
%% !save,sequence=fivecell,file,mad8;

%% s := table(twiss,bpv[5],betx);
%% myvar := sqrt(beam->ex*table(twiss,betx));
%% use, period=fivecell;
%% select,flag=twiss,column=name,s,myvar,apertype;
%% twiss,file;
%% n = 0;
%% create,table=mytab,column=dp,mq1,mq2;
%% mq1:=table(summ,q1);
%% mq2:=table(summ,q2);
%% while ( n < 11)
%% {
%%   n = n + 1;
%%   dp = 1.e-4*(n-6);
%%   twiss,deltap=dp;
%%   fill,table=mytab;
%% }
%% write,table=mytab;
%% plot,haxis=s,vaxis=aper_1,aper_2,colour=100,range=#s/cbv.1,notitle;
%% stop;
%% \end{verbatim}
%% prints the following user table on output:

%% \begin{verbatim}
%% @ NAME             %05s "MYTAB"
%% @ TYPE             %04s "USER"
%% @ TITLE            %08s "no-title"
%% @ ORIGIN           %16s "MAD-X 1.09 Linux"
%% @ DATE             %08s "10/12/02"
%% @ TIME             %08s "10.45.25"
%% * DP                 MQ1                MQ2                
%% $ %le                %le                %le                
%%  -0.0005            1.242535951        1.270211135       
%%  -0.0004            1.242495534        1.270197018       
%%  -0.0003            1.242452432        1.270185673       
%%  -0.0002            1.242406653        1.270177093       
%%  -0.0001            1.242358206        1.270171269       
%%  0                  1.242307102        1.27016819        
%%  0.0001             1.242253353        1.270167843       
%%  0.0002             1.242196974        1.270170214       
%%  0.0003             1.24213798         1.270175288       
%%  0.0004             1.242076387        1.270183048       
%%  0.0005             1.242012214        1.270193477       
%% \end{verbatim}
%% and produces a twiss file with the additional column myvar, as well as a plot
%% file with the aperture values plotted.


%% \href{screate}{}

%% Example of joining two tables with different length into a third table
%% making use of the length of either table as given by
%% table("your\_table\_name", tablelength) and adding names by the "\_name"
%% attribute.

%% \begin{verbatim}
%% title,   "summing of offset and alignment tables";
%% set,    format="13.6f";

%% readtable, table=align,  file="align.ip2.b1.tfs";   // mesured alignment
%% readtable, table=offset, file="offset.ip2.b1.tfs";  // nominal offsets

%% n_elem  =  table(offset, tablelength);

%% create,  table=align_offset, column=_name,s_ip,x_off,dx_off,ddx_off,y_off,dy_off,ddy_off;

%% calcul(elem_name) : macro = {
%%     x_off = table(align,  elem_name, x_ali) + x_off;
%%     y_off = table(align,  elem_name, y_ali) + y_off;
%% }


%% one_elem(j_elem) : macro = {
%%     setvars, table=offset, row=j_elem;
%%     exec,  calcul(tabstring(offset, name, j_elem));
%%     fill,  table=align_offset;
%% }


%% i_elem = 0;
%% while (i_elem < n_elem) { i_elem = i_elem + 1; exec,  one_elem($i_elem); }

%% write, table=align_offset, file="align_offset.tfs";

%% stop;
%% \end{verbatim}

%%



%\input{Introduction/select}
\section{SELECT}
\label{sec:selection}
The elements, or a range of elements, in a sequence can be selected for
various purposes. Such selections remain valid until cleared (in
difference to \madeight); it is therefore recommended to always start with a  

\begin{verbatim}
select, flag =..., clear;
\end{verbatim} 
before setting a new selection. 
\begin{verbatim}
SELECT, FLAG=name, RANGE=range, CLASS=class, PATTERN=pattern [,FULL] [,CLEAR];
\end{verbatim} 
where the name for FLAG can be one of ERROR, MAKETHIN, SEQEDIT or the
name of a twiss table which is established for all sequence positions in
general.  

Selected elements have to fulfill the \href{ranges.html#range}{RANGE},
\href{ranges.html#class}{CLASS}, and \href{wildcard.html}{PATTERN}
criteria.  

Any number of SELECT commands can be issued for the same flag and are
accumulated (logically ORed). In this context note the following:  

\begin{verbatim}
SELECT, FLAG=name, FULL;
\end{verbatim} 
selects all positions in the sequence for this flag. This is the default
for all tables and makethin, whereas for ERROR and SEQEDIT the default
is "nothing selected".  

%\href{save_select}{}
%\label{save_select}
SAVE: A SELECT,FLAG=SAVE statement causes the
selected sequences, elements, and variables to be written into the save
file. A class (only used for element selection), and a pattern can be
specified. Example:  
\begin{verbatim}
select, flag=save, class=variable, pattern="abc.*";
save, file=mysave;
\end{verbatim} 
will save all variables (and sequences) containing "abc" in their name,
but not elements with names containing "abc" since the class "variable"
does not exist (astucieux, non ?).  

SECTORMAP: A SELECT,FLAG=SECTORMAP statement causes sectormaps to be
written into the file "sectormap" like in \madeight. For the file to be
written, a flag SECTORMAP must be issued on the TWISS command in
addition.  

TWISS: A SELECT,FLAG=TWISS statement causes the selected rows and
columns to be written into the Twiss TFS file (former OPTICS command in
\madeight). The column selection is done on the same select. See as well
example 2.  

Example 1:  
\begin{verbatim}
TITLE,'Test input for MAD-X';

option,rbarc=false; // use arc length of rbends
beam; ! sets the default beam for the following sequence
option,-echo;
call file=fv9.opt;  ! contains optics parameters
call file="fv9.seq"; ! contains a small sequence "fivecell"
OPTION,ECHO;
SELECT,FLAG=SECTORMAP,clear;
SELECT,FLAG=SECTORMAP,PATTERN="^m.*";
SELECT,FLAG=TWISS,clear;
SELECT,FLAG=TWISS,PATTERN="^m.*",column=name,s,betx,bety;
USE,PERIOD=FIVECELL;
twiss,file=optics,sectormap;
stop;
\end{verbatim} 

This produces a file \href{sectormap.html}{sectormap}, and a
twiss output file \label{tfs} (name = optics):  
\begin{verbatim}
@ TYPE             %05s "TWISS"
@ PARTICLE         %08s "POSITRON"
@ MASS             %le          0.000510998902
@ CHARGE           %le                       1
@ E0               %le                       1
@ PC               %le           0.99999986944
@ GAMMA            %le           1956.95136738
@ KBUNCH           %le                       1
@ NPART            %le                       0
@ EX               %le                       1
@ EY               %le                       1
@ ET               %le                       0
@ LENGTH           %le                   534.6
@ ALFA             %le        0.00044339992938
@ ORBIT5           %le                      -0
@ GAMMATR          %le           47.4900022541
@ Q1               %le           1.25413071556
@ Q2               %le           1.25485338377
@ DQ1              %le           1.05329608302
@ DQ2              %le           1.04837000224
@ DXMAX            %le           2.17763211131
@ DYMAX            %le                       0
@ XCOMAX           %le                       0
@ YCOMAX           %le                       0
@ BETXMAX          %le            177.70993499
@ BETYMAX          %le           177.671582415
@ XCORMS           %le                       0
@ YCORMS           %le                       0
@ DXRMS            %le           1.66004270906
@ DYRMS            %le                       0
@ DELTAP           %le                       0
@ TITLE            %20s "Test input for MAD-X"
@ ORIGIN           %16s "MAD-X 0.20 Linux"
@ DATE             %08s "07/06/02"
@ TIME             %08s "14.25.51"
* NAME               S                  BETX               BETY               
$ %s                 %le                %le                %le                
 "MSCBH"             4.365              171.6688159        33.31817319       
 "MB"                19.72              108.1309095        58.58680717       
 "MB"                35.38              61.96499987        102.9962313       
 "MB"                51.04              34.61640793        166.2227523       
 "MSCBV.1"           57.825             33.34442808        171.6309057       
 "MB"                73.18              58.61984637        108.0956006       
 "MB"                88.84              103.0313887        61.93159422       
 "MB"                104.5              166.2602486        34.58939635       
 "MSCBH"             111.285            171.6688159        33.31817319       
 "MB"                126.64             108.1309095        58.58680717       
 "MB"                142.3              61.96499987        102.9962313       
 "MB"                157.96             34.61640793        166.2227523       
 "MSCBV"             164.745            33.34442808        171.6309057       
 "MB"                180.1              58.61984637        108.0956006       
 "MB"                195.76             103.0313887        61.93159422       
 "MB"                211.42             166.2602486        34.58939635       
 "MSCBH"             218.205            171.6688159        33.31817319       
 "MB"                233.56             108.1309095        58.58680717       
 "MB"                249.22             61.96499987        102.9962313       
 "MB"                264.88             34.61640793        166.2227523       
 "MSCBV"             271.665            33.34442808        171.6309057       
 "MB"                287.02             58.61984637        108.0956006       
 "MB"                302.68             103.0313887        61.93159422       
 "MB"                318.34             166.2602486        34.58939635       
 "MSCBH"             325.125            171.6688159        33.31817319       
 "MB"                340.48             108.1309095        58.58680717       
 "MB"                356.14             61.96499987        102.9962313       
 "MB"                371.8              34.61640793        166.2227523       
 "MSCBV"             378.585            33.34442808        171.6309057       
 "MB"                393.94             58.61984637        108.0956006       
 "MB"                409.6              103.0313887        61.93159422       
 "MB"                425.26             166.2602486        34.58939635       
 "MSCBH"             432.045            171.6688159        33.31817319       
 "MB"                447.4              108.1309095        58.58680717       
 "MB"                463.06             61.96499987        102.9962313       
 "MB"                478.72             34.61640793        166.2227523       
 "MSCBV"             485.505            33.34442808        171.6309057       
 "MB"                500.86             58.61984637        108.0956006       
 "MB"                516.52             103.0313887        61.93159422       
 "MB"                532.18             166.2602486        34.58939635       
\end{verbatim}

 Example 2: 

 Addition of variables to (any internal) table: 
\begin{verbatim}
select, flag=table, column=name, s, betx, ..., var1, var2, ...; ! or
select, flag=table, full, column=var1, var2, ...; ! default col.s + new
\end{verbatim} 
will write the current value of var1 etc. into the table each time a new
line is added; values from the same (current) line can be accessed by
these variables, e.g.  
\begin{verbatim}
var1 := sqrt(beam->ex*table(twiss,betx));
\end{verbatim} 
in the case of table above being "twiss". The plot command accepts the
new variables.  

Remark: this replaces the "string" variables of \madeight. 

 This example demonstrates as well the usage of a user defined table \label{ucreate}. 
\begin{verbatim}
beam,ex=1.e-6,ey=1.e-3;
// element definitions
mb:rbend, l=14.2, angle:=0,k0:=bang/14.2;
mq:quadrupole, l:=3.1,apertype=ellipse,aperture={1,2};
qft:mq, l:=0.31, k1:=kqf,tilt=-pi/4;
qf.1:mq, l:=3.1, k1:=kqf;
qf.2:mq, l:=3.1, k1:=kqf;
qf.3:mq, l:=3.1, k1:=kqf;
qf.4:mq, l:=3.1, k1:=kqf;
qf.5:mq, l:=3.1, k1:=kqf;
qd.1:mq, l:=3.1, k1:=kqd;
qd.2:mq, l:=3.1, k1:=kqd;
qd.3:mq, l:=3.1, k1:=kqd;
qd.4:mq, l:=3.1, k1:=kqd;
qd.5:mq, l:=3.1, k1:=kqd;
bph:hmonitor, l:=l.bpm;
bpv:vmonitor, l:=l.bpm;
cbh:hkicker;
cbv:vkicker;
cbh.1:cbh, kick:=acbh1;
cbh.2:cbh, kick:=acbh2;
cbh.3:cbh, kick:=acbh3;
cbh.4:cbh, kick:=acbh4;
cbh.5:cbh, kick:=acbh5;
cbv.1:cbv, kick:=acbv1;
cbv.2:cbv, kick:=acbv2;
cbv.3:cbv, kick:=acbv3;
cbv.4:cbv, kick:=acbv4;
cbv.5:cbv, kick:=acbv5;
!mscbh:sextupole, l:=1.1, k2:=ksf;
mscbh:multipole, knl:={0,0,0,ksf},tilt=-pi/8;
mscbv:sextupole, l:=1.1, k2:=ksd;
!mscbv:octupole, l:=1.1, k3:=ksd,tilt=-pi/8;

// sequence declaration

fivecell:sequence, refer=centre, l=534.6;
   qf.1:qf.1, at=1.550000e+00;
   qft:qft, at=3.815000e+00;
!   mscbh:mscbh, at=3.815000e+00;
   cbh.1:cbh.1, at=4.365000e+00;
   mb:mb, at=1.262000e+01;
   mb:mb, at=2.828000e+01;
   mb:mb, at=4.394000e+01;
   bpv:bpv, at=5.246000e+01;
   qd.1:qd.1, at=5.501000e+01;
   mscbv:mscbv, at=5.727500e+01;
   cbv.1:cbv.1, at=5.782500e+01;
   mb:mb, at=6.608000e+01;
   mb:mb, at=8.174000e+01;
   mb:mb, at=9.740000e+01;
   bph:bph, at=1.059200e+02;
   qf.2:qf.2, at=1.084700e+02;
   mscbh:mscbh, at=1.107350e+02;
   cbh.2:cbh.2, at=1.112850e+02;
   mb:mb, at=1.195400e+02;
   mb:mb, at=1.352000e+02;
   mb:mb, at=1.508600e+02;
   bpv:bpv, at=1.593800e+02;
   qd.2:qd.2, at=1.619300e+02;
   mscbv:mscbv, at=1.641950e+02;
   cbv.2:cbv.2, at=1.647450e+02;
   mb:mb, at=1.730000e+02;
   mb:mb, at=1.886600e+02;
   mb:mb, at=2.043200e+02;
   bph:bph, at=2.128400e+02;
   qf.3:qf.3, at=2.153900e+02;
   mscbh:mscbh, at=2.176550e+02;
   cbh.3:cbh.3, at=2.182050e+02;
   mb:mb, at=2.264600e+02;
   mb:mb, at=2.421200e+02;
   mb:mb, at=2.577800e+02;
   bpv:bpv, at=2.663000e+02;
   qd.3:qd.3, at=2.688500e+02;
   mscbv:mscbv, at=2.711150e+02;
   cbv.3:cbv.3, at=2.716650e+02;
   mb:mb, at=2.799200e+02;
   mb:mb, at=2.955800e+02;
   mb:mb, at=3.112400e+02;
   bph:bph, at=3.197600e+02;
   qf.4:qf.4, at=3.223100e+02;
   mscbh:mscbh, at=3.245750e+02;
   cbh.4:cbh.4, at=3.251250e+02;
   mb:mb, at=3.333800e+02;
   mb:mb, at=3.490400e+02;
   mb:mb, at=3.647000e+02;
   bpv:bpv, at=3.732200e+02;
   qd.4:qd.4, at=3.757700e+02;
   mscbv:mscbv, at=3.780350e+02;
   cbv.4:cbv.4, at=3.785850e+02;
   mb:mb, at=3.868400e+02;
   mb:mb, at=4.025000e+02;
   mb:mb, at=4.181600e+02;
   bph:bph, at=4.266800e+02;
   qf.5:qf.5, at=4.292300e+02;
   mscbh:mscbh, at=4.314950e+02;
   cbh.5:cbh.5, at=4.320450e+02;
   mb:mb, at=4.403000e+02;
   mb:mb, at=4.559600e+02;
   mb:mb, at=4.716200e+02;
   bpv:bpv, at=4.801400e+02;
   qd.5:qd.5, at=4.826900e+02;
   mscbv:mscbv, at=4.849550e+02;
   cbv.5:cbv.5, at=4.855050e+02;
   mb:mb, at=4.937600e+02;
   mb:mb, at=5.094200e+02;
   mb:mb, at=5.250800e+02;
   bph:bph, at=5.336000e+02;
end:marker, at=5.346000e+02;
endsequence;

// forces and other constants

l.bpm:=.3;
bang:=.509998807401e-2;
kqf:=.872651312e-2;
kqd:=-.872777242e-2;
ksf:=.0198492943;
ksd:=-.039621283;
acbv1:=1.e-4;
acbh1:=1.e-4;
!save,sequence=fivecell,file,mad8;

s := table(twiss,bpv[5],betx);
myvar := sqrt(beam->ex*table(twiss,betx));
use, period=fivecell;
select,flag=twiss,column=name,s,myvar,apertype;
twiss,file;
n = 0;
create,table=mytab,column=dp,mq1,mq2;
mq1:=table(summ,q1);
mq2:=table(summ,q2);
while ( n < 11)
{
  n = n + 1;
  dp = 1.e-4*(n-6);
  twiss,deltap=dp;
  fill,table=mytab;
}
write,table=mytab;
plot,haxis=s,vaxis=aper_1,aper_2,colour=100,range=#s/cbv.1,notitle;
stop;
\end{verbatim}
prints the following user table on output:

\begin{verbatim}
@ NAME             %05s "MYTAB"
@ TYPE             %04s "USER"
@ TITLE            %08s "no-title"
@ ORIGIN           %16s "MAD-X 1.09 Linux"
@ DATE             %08s "10/12/02"
@ TIME             %08s "10.45.25"
* DP                 MQ1                MQ2                
$ %le                %le                %le                
 -0.0005            1.242535951        1.270211135       
 -0.0004            1.242495534        1.270197018       
 -0.0003            1.242452432        1.270185673       
 -0.0002            1.242406653        1.270177093       
 -0.0001            1.242358206        1.270171269       
 0                  1.242307102        1.27016819        
 0.0001             1.242253353        1.270167843       
 0.0002             1.242196974        1.270170214       
 0.0003             1.24213798         1.270175288       
 0.0004             1.242076387        1.270183048       
 0.0005             1.242012214        1.270193477       
\end{verbatim}
and produces a twiss file with the additional column myvar, as well as a plot
file with the aperture values plotted.


%\href{screate}{}

Example of joining two tables with different length into a third table
making use of the length of either table as given by
table("your\_table\_name", tablelength) and adding names by the "\_name"
attribute.

\begin{verbatim}
title,   "summing of offset and alignment tables";
set,    format="13.6f";

readtable, table=align,  file="align.ip2.b1.tfs";   // mesured alignment
readtable, table=offset, file="offset.ip2.b1.tfs";  // nominal offsets

n_elem  =  table(offset, tablelength);

create,  table=align_offset, column=_name,s_ip,x_off,dx_off,ddx_off,y_off,dy_off,ddy_off;

calcul(elem_name) : macro = {
    x_off = table(align,  elem_name, x_ali) + x_off;
    y_off = table(align,  elem_name, y_ali) + y_off;
}


one_elem(j_elem) : macro = {
    setvars, table=offset, row=j_elem;
    exec,  calcul(tabstring(offset, name, j_elem));
    fill,  table=align_offset;
}


i_elem = 0;
while (i_elem < n_elem) { i_elem = i_elem + 1; exec,  one_elem($i_elem); }

write, table=align_offset, file="align_offset.tfs";

stop;
\end{verbatim}


%% EOF



%%%%\title{Range Selection}
%  Changed by: Chris ISELIN, 27-Jan-1997 

%  Changed by: Hans Grote, 10-Jun-2002 

%%%\usepackage{hyperref}
% commands generated by html2latex


%%%\begin{document}
%%%\begin{center}
 %%%EUROPEAN ORGANIZATION FOR NUCLEAR RESEARCH 
%%%\includegraphics{http://cern.ch/madx/icons/mx7_25.gif}

\paragraph{Real life example for IF statements, and MACRO usage}
%%%\end{center}


\begin{verbatim}

! Creates a footprint for head-on + parasitic collisions at IP1+5 
! of lhc.6.5; both lhcb1 (for tracking) and lhcb2 (to define the
! beam-beam elements, i.e. weak-strong) are used; there are flags to
! select head-on, left, and right parasitic separately at all IPs.
! The bunch spacing can be given in nanosec and automatically creates
! the beam-beam interaction points at the correct positions.
! It is important to set the correct BEAM parameters, i.e. number
! of particles, emittances, bunch length, energy.

!--- For completeness, all files needed by this job are copied
!    to the local directory ldb. The links to the the originals
!    in offdb (official database) are commented out.

Option,  warn,info,echo;
!System,
"ln -fns /afs/cern.ch/eng/sl/MAD-X/dev/test_suite/foot/V3.01.01 ldb";
!system,"ln -fns /afs/cern.ch/eng/lhc/optics/V6.4 offdb";
Option, -echo,-info,warn;
SU=1.0;
call, file = "ldb/V6.5.seq";
call,file="ldb/slice_new.madx";
Option, echo,info,warn;

!+++++++++++++++++++++++++ Step 1 +++++++++++++++++++++++
! 	define beam constants
!++++++++++++++++++++++++++++++++++++++++++++++++++++++++

b_t_dist = 25.e-9;                  !--- bunch distance in [sec]
b_h_dist = clight * b_t_dist / 2 ;  !--- bunch half-distance in [m]
ip1_range = 58.;                     ! range for parasitic collisions
ip5_range = ip1_range;
ip2_range = 60.;
ip8_range = ip2_range;

npara_1 = ip1_range / b_h_dist;     ! # parasitic either side
npara_2 = ip2_range / b_h_dist;
npara_5 = ip5_range / b_h_dist;
npara_8 = ip8_range / b_h_dist;

value,npara_1,npara_2,npara_5,npara_8;

 eg   =  7000;
 bg   =  eg/pmass;
 en   = 3.75e-06;
 epsx = en/bg;
 epsy = en/bg;

Beam, particle = proton, sequence=lhcb1, energy = eg,
          sigt=      0.077     , 
          bv = +1, NPART=1.1E11, sige=      1.1e-4, 
          ex=epsx,   ey=epsy;

Beam, particle = proton, sequence=lhcb2, energy = eg,
          sigt=      0.077     , 
          bv = -1, NPART=1.1E11, sige=      1.1e-4, 
          ex=epsx,   ey=epsy;

beamx = beam%lhcb1->ex;   beamy%lhcb1 = beam->ey;
sigz  = beam%lhcb1->sigt; sige = beam%lhcb1->sige;

!--- split5, 4d
long_a= 0.53 * sigz/2;
long_b= 1.40 * sigz/2;
value,long_a,long_b;

ho_charge = 0.2;

!+++++++++++++++++++++++++ Step 2 +++++++++++++++++++++++
! 	slice, flatten sequence, and cycle start to ip3
!++++++++++++++++++++++++++++++++++++++++++++++++++++++++

use,sequence=lhcb1;
makethin,sequence=lhcb1;
!save,sequence=lhcb1,file=lhcb1_thin_new_seq;
use,sequence=lhcb2;
makethin,sequence=lhcb2;
!save,sequence=lhcb2,file=lhcb2_thin_new_seq;
!stop;

option,-warn,-echo,-info;
call,file="ldb/V6.5.thin.coll.str";
option,warn,echo,info;

! keep sextupoles
ksf0=ksf; ksd0=ksd;
use,period=lhcb1;
select,flag=twiss.1,column=name,x,y,betx,bety;
twiss,file;
plot,haxis=s,vaxis=x,y,colour=100,noline;

use,period=lhcb2;
select,flag=twiss.2,column=name,x,y,betx,bety;
twiss,file;
plot,haxis=s,vaxis=x,y,colour=100,noline;
seqedit,sequence=lhcb1;
flatten;
endedit;

seqedit,sequence=lhcb1;
cycle,start=ip3.b1;
endedit;

seqedit,sequence=lhcb2;
flatten;
endedit;

seqedit,sequence=lhcb2;
cycle,start=ip3.b2;
endedit;

bbmarker: marker;  /* for subsequent remove */


!+++++++++++++++++++++++++ Step 3 +++++++++++++++++++++++
! 	define the beam-beam elements
!++++++++++++++++++++++++++++++++++++++++++++++++++++++++
!
!===========================================================
! read macro definitions
option,-echo;
call,file="ldb/bb.macros";
option,echo;

!
!===========================================================
!   this sets CHARGE in the head-on beam-beam elements. 
!   set +1 * ho_charge   for parasitic on, 0 for off

 on_ho1  = +1 * ho_charge; ! ho_charge depends on split
 on_ho2  = +0 * ho_charge; ! because of the "by hand" splitting
 on_ho5  = +1 * ho_charge;
 on_ho8  = +0 * ho_charge;

!
!===========================================================
!   set CHARGE in the parasitic beam-beam elements. 
!   set +1 for parasitic on, 0 for off
 on_lr1l = +1;
 on_lr1r = +1;
 on_lr2l = +0;
 on_lr2r = +0;
 on_lr5l = +1;
 on_lr5r = +1;
 on_lr8l = +0;
 on_lr8r = +0;

!
!===========================================================
!   define markers and savebetas
assign,echo=temp.bb.install;
!--- ip1
if (on_ho1  0)
{
  exec, mkho(1);
  exec, sbhomk(1);
}
if (on_lr1l  0 || on_lr1r  0)
{
  n=1; ! counter
  while (n  0 || on_lr1l  0)
{
  n=1; ! counter
  while (n  0)
{
  exec, mkho(5);
  exec, sbhomk(5);
}
if (on_lr5l  0 || on_lr5r  0)
{
  n=1; ! counter
  while (n  0 || on_lr5l  0)
{
  n=1; ! counter
  while (n  0)
{
  exec, mkho(2);
  exec, sbhomk(2);
}
if (on_lr2l  0 || on_lr2r  0)
{
  n=1; ! counter
  while (n  0 || on_lr2l  0)
{
  n=1; ! counter
  while (n  0)
{
  exec, mkho(8);
  exec, sbhomk(8);
}
if (on_lr8l  0 || on_lr8r  0)
{
  n=1; ! counter
  while (n  0 || on_lr8l  0)
{
  n=1; ! counter
  while (n  0)
{
exec, inho(mk,1);
}
if (on_lr1l  0 || on_lr1r  0)
{
  n=1; ! counter
  while (n  0 || on_lr1l  0)
{
  n=1; ! counter
  while (n  0)
{
exec, inho(mk,5);
}
if (on_lr5l  0 || on_lr5r  0)
{
  n=1; ! counter
  while (n  0 || on_lr5l  0)
{
  n=1; ! counter
  while (n  0)
{
exec, inho(mk,2);
}
if (on_lr2l  0 || on_lr2r  0)
{
  n=1; ! counter
  while (n  0 || on_lr2l  0)
{
  n=1; ! counter
  while (n  0)
{
exec, inho(mk,8);
}
if (on_lr8l  0 || on_lr8r  0)
{
  n=1; ! counter
  while (n  0 || on_lr8l  0)
{
  n=1; ! counter
  while (n betx) / 0.0007999979093;
value,on_sep2;
!===========================================================
!   define bb elements
assign,echo=temp.bb.install;
!--- ip1
if (on_ho1  0)
{
exec, bbho(1);
}
if (on_lr1l  0)
{
  n=1; ! counter
  while (n  0)
{
  n=1; ! counter
  while (n  0)
{
exec, bbho(5);
}
if (on_lr5l  0)
{
  n=1; ! counter
  while (n  0)
{
  n=1; ! counter
  while (n  0)
{
exec, bbho(2);
}
if (on_lr2l  0)
{
  n=1; ! counter
  while (n  0)
{
  n=1; ! counter
  while (n  0)
{
exec, bbho(8);
}
if (on_lr8l  0)
{
  n=1; ! counter
  while (n  0)
{
  n=1; ! counter
  while (n  0)
{
exec, inho(bb,1);
}
if (on_lr1l  0)
{
  n=1; ! counter
  while (n  0)
{
  n=1; ! counter
  while (n  0)
{
exec, inho(bb,5);
}
if (on_lr5l  0)
{
  n=1; ! counter
  while (n  0)
{
  n=1; ! counter
  while (n  0)
{
exec, inho(bb,2);
}
if (on_lr2l  0)
{
  n=1; ! counter
  while (n  0)
{
  n=1; ! counter
  while (n  0)
{
exec, inho(bb,8);
}
if (on_lr8l  0)
{
  n=1; ! counter
  while (n  0)
{
  n=1; ! counter
  while (n  footprint";
stop;
\end{verbatim}

\paragraph{\href{macro}{Real life example of MACRO definitions}}

\begin{verbatim}

bbho(nn): macro = {
!--- macro defining head-on beam-beam elements; nn = IP number
print, text="bbipnnl2: beambeam, sigx=sqrt(rnnipnnl2->betx*epsx),";
print, text="          sigy=sqrt(rnnipnnl2->bety*epsy),";
print, text="          xma=rnnipnnl2->x,yma=rnnipnnl2->y,";
print, text="          charge:=on_honn;";
print, text="bbipnnl1: beambeam, sigx=sqrt(rnnipnnl1->betx*epsx),";
print, text="          sigy=sqrt(rnnipnnl1->bety*epsy),";
print, text="          xma=rnnipnnl1->x,yma=rnnipnnl1->y,";
print, text="          charge:=on_honn;";
print, text="bbipnn:   beambeam, sigx=sqrt(rnnipnn->betx*epsx),";
print, text="          sigy=sqrt(rnnipnn->bety*epsy),";
print, text="          xma=rnnipnn->x,yma=rnnipnn->y,";
print, text="          charge:=on_honn;";
print, text="bbipnnr1: beambeam, sigx=sqrt(rnnipnnr1->betx*epsx),";
print, text="          sigy=sqrt(rnnipnnr1->bety*epsy),";
print, text="          xma=rnnipnnr1->x,yma=rnnipnnr1->y,";
print, text="          charge:=on_honn;";
print, text="bbipnnr2: beambeam, sigx=sqrt(rnnipnnr2->betx*epsx),";
print, text="          sigy=sqrt(rnnipnnr2->bety*epsy),";
print, text="          xma=rnnipnnr2->x,yma=rnnipnnr2->y,";
print, text="          charge:=on_honn;";
};

mkho(nn): macro = {
!--- macro defining head-on markers; nn = IP number
print, text="mkipnnl2: bbmarker;";
print, text="mkipnnl1: bbmarker;";
print, text="mkipnn:   bbmarker;";
print, text="mkipnnr1: bbmarker;";
print, text="mkipnnr2: bbmarker;";
};

inho(xx,nn): macro = {
!--- macro installing bb or markers for head-on beam-beam (split into 5)
print, text="install, element= xxipnnl2, at=-long_b, from=ipnn;";
print, text="install, element= xxipnnl1, at=-long_a, from=ipnn;";
print, text="install, element= xxipnn,   at=1.e-9,   from=ipnn;";
print, text="install, element= xxipnnr1, at=+long_a, from=ipnn;"; 
print, text="install, element= xxipnnr2, at=+long_b, from=ipnn;"; 
};

sbhomk(nn): macro = {
!--- macro to create savebetas for ho markers
print, text="savebeta, label=rnnipnnl2, place=mkipnnl2;";
print, text="savebeta, label=rnnipnnl1, place=mkipnnl1;";
print, text="savebeta, label=rnnipnn,   place=mkipnn;";
print, text="savebeta, label=rnnipnnr1, place=mkipnnr1;";
print, text="savebeta, label=rnnipnnr2, place=mkipnnr2;";    
};

mkl(nn,cc): macro = {
!--- macro to create parasitic bb marker on left side of ip nn; cc = count
print, text="mkipnnplcc: bbmarker;";
};

mkr(nn,cc): macro = {
!--- macro to create parasitic bb marker on right side of ip nn; cc = count
print, text="mkipnnprcc: bbmarker;";
};

sbl(nn,cc): macro = {
!--- macro to create savebetas for left parasitic
print, text="savebeta, label=rnnipnnplcc, place=mkipnnplcc;";
};

sbr(nn,cc): macro = {
!--- macro to create savebetas for right parasitic
print, text="savebeta, label=rnnipnnprcc, place=mkipnnprcc;";
};

inl(xx,nn,cc): macro = {
!--- macro installing bb and markers for left side parasitic beam-beam
print, text="install, element= xxipnnplcc, at=-cc*b_h_dist, from=ipnn;";
};

inr(xx,nn,cc): macro = {
!--- macro installing bb and markers for right side parasitic beam-beam
print, text="install, element= xxipnnprcc, at=cc*b_h_dist, from=ipnn;";
};

bbl(nn,cc): macro = {
!--- macro defining parasitic beam-beam elements; nn = IP number
print, text="bbipnnplcc: beambeam, sigx=sqrt(rnnipnnplcc->betx*epsx),";
print, text="          sigy=sqrt(rnnipnnplcc->bety*epsy),";
print, text="          xma=rnnipnnplcc->x,yma=rnnipnnplcc->y,";
print, text="          charge:=on_lrnnl;";
};

bbr(nn,cc): macro = {
!--- macro defining parasitic beam-beam elements; nn = IP number
print, text="bbipnnprcc: beambeam, sigx=sqrt(rnnipnnprcc->betx*epsx),";
print, text="          sigy=sqrt(rnnipnnprcc->bety*epsy),";
print, text="          xma=rnnipnnprcc->x,yma=rnnipnnprcc->y,";
print, text="          charge:=on_lrnnr;";
};
\end{verbatim}\href{http://www.cern.ch/Hans.Grote/hansg_sign.html}{hansg}, June 17, 2002 

%%%\end{document}

%%%%\title{Range Selection}
%  Changed by: Chris ISELIN, 27-Jan-1997 
%  Changed by: Hans Grote, 16-Jan-2003 

\section{General Control Statements}

\subsection{ASSIGN}
\begin{verbatim}

assign, echo = "file_name", truncate;
\end{verbatim} 
where "file\_name"  is the name of an output file, or "terminal" and
trunctate specifies if the file must be trunctated when opened (ignored
for terminal). This allows switching the echo stream to a file or back,
but only for the commands value, show, and print. Allows easy
composition of future MAD-X input files. A real life example (always the
same) is to be found under \href{foot.html}{footprint example}.  

\subsection{CALL}
\begin{verbatim}

call, file = file_name;
\end{verbatim} 
where "file\_name"  is the name of an input file. This file will be read
until a "return;" statement, or until end\_of\_file; it may contain any
number of calls itself, and so on to any depth.  


\subsection{COGUESS}
\begin{verbatim}

coguess,tolerance=double,x=double,
       px=double,y=double,py=double,t=double,pt=double;
\end{verbatim} 
sets the required convergence precision in the closed orbit search
("tolerance", see as well Twiss command
\href{../twiss/twiss.html#tolerance}{tolerance}).  

The other parameters define a first guess for all future closed orbit
searches in case they are different from zero.  


\subsection{CREATE}
\begin{verbatim}

create,table=table,column=var1,var2,_name,...;
\end{verbatim} 
creates a table with the specified variables as columns. This table can
then be \hyperlink{fill}{fill}ed, and finally one can
\hyperlink{write}{write} it in TFS format. The attribute "\_name" adds
the element name to the table at the specified column, this replaces the
undocumented "withname" attribute that was not always working properly.  

See the \href{../Introduction/select.html#ucreate}{user table I}
example;  

or an example of joining 2 tables of different length in one table
including the element name:
\href{../Introduction/select.html#screate}{user table II} 




\subsection{DELETE}
\begin{verbatim}

delete,sequence=s_name,table=t_name;
\end{verbatim} 
deletes a sequence with name "s\_name" or a table with name "t\_name"
from memory. The sequence deletion is done without influence on other
sequences that may have elements that were in the deleted sequence.  


\subsection{DUMPSEQU}
\begin{verbatim}

dumpsequ, sequence = s_name, level = integer;
\end{verbatim} 
Actually a debug statement, but it may come handy at certain
occasions. Here "s\_name" is the name of an expanded (i.e. USEd)
sequence. The amount of detail is controlled by "level":  
\begin{verbatim}

level = 0:    print only the cumulative node length = sequence length
      > 0:    print all node (element) names except drifts
      > 2:    print all nodes with their attached parameters
      > 3:    print all nodes, and their elements with all parameters
\end{verbatim}


\subsection{EXEC}
\begin{verbatim}

exec, label;
\end{verbatim} 
Each statement may be preceded by a label; it is then stored and can be
executed again with "exec, label;" any number of times; the executed
statement may be another "exec", etc.; however, the major usage of this
statement is the execution of a \href{special.html#macro}{macro}.  


\subsection{EXIT}
\begin{verbatim}

exit;
\end{verbatim} 
ends the program execution. 


\subsection{FILL} 
Every command 
\begin{verbatim}

fill,table=table;
\end{verbatim} 
adds a new line with the current values of all column variables into the
user table \hyperlink{create}{create}d beforehand. This table one can
then \hyperlink{write}{write} in TFS format.  See as well the
\href{../Introduction/select.html#ucreate}{user table} example.  


\subsection{OPTION}
\begin{verbatim}

option, flag { = true | false };
option, flag | -flag;
\end{verbatim} 
sets an option as given in "flag"; the part in curly brackets is
optional: if only the name of the option is given, then the option will
be set true (see second line); a "-" sign preceding the name sets it to
"false".  

 Example: 
\begin{verbatim}

option,echo=true;
option,echo;
\end{verbatim} 
are identical, ditto 
\begin{verbatim}

option,echo=false;
option,-echo;
\end{verbatim} 
The available options are: 
\begin{verbatim}

  name           default meaning if true
  ====           ======= ===============
  bborbit         false  the closed orbit is modified by beam-beam kicks
  sympl           false  all element matrices are symplectified in Twiss
  echo            true   echoes the input on the standard output file
  trace           false  prints the system time after each command
  verify          false  issues a warning if an undefined variable is used
  warn            true   issues warnings
  info            true   issues informations
  tell            false  prints the current value of all options
  reset           false  resets all options to their defaults
  rbarc           true   converts the RBEND straight length into the arc length
  thin_foc        true   if false suppresses the 1(rho**2) focusing of thin dipoles
  no_fatal_stop   false  Prevents madx from stopping in case of a fatal error. Use at your own risk.
\end{verbatim} 
The option "rbarc" is implemented for backwards compatibility with MAD-8
up to version 8.23.06 included; in this version, the RBEND length was
just taken as the arc length of an SBEND with inclined pole faces,
contrary to the MAD-8 manual.  


\subsection{PRINT}
\begin{verbatim}

print,text="...";
\end{verbatim} 
prints the text to the current output file (see ASSIGN above). The text
can be edited with the help of a  \href{special.html#macro}{macro
  statement}. For more details, see there.  


\subsection{QUIT}
\begin{verbatim}

quit;
\end{verbatim} 
ends the program execution. 


\subsection{READTABLE}
\begin{verbatim}

readtable,file=filename;
\end{verbatim} 
reads a TFS file containing a MAD-X table back into memory. This table
can then be manipulated as any other table, i.e. its values can be
accessed, it can be plotted, written out again etc.  


\subsection{READMYTABLE}
\begin{verbatim}

readmytable,file=filename,table=internalname;
\end{verbatim} 
reads a TFS file containing a MAD-X table back into memory. This table
can then be manipulated as any other table, i.e. its values can be
accessed, it can be plotted, written out again etc. An internal name for
the table can be freely assigned while for the command READTABLE it is
taken from the information section of the table itself. This feature
allows to store multiple tables of the same type in memory without
overwriting existing ones.  


\subsection{REMOVEFILE}
\begin{verbatim}

removefile,file=filename;
\end{verbatim} 
remove a file from the disk. It is more portable than  
\begin{verbatim}

system("rm filename"); // Unix specific
\end{verbatim}


\subsection{RENAMEFILE}
\begin{verbatim}

renamefile,file=filename,name=newfilename;
\end{verbatim} 
rename the file "filename" to "newfilename" on the disk. It is more
portable than  
\begin{verbatim}

system("mv filename newfilename"); // Unix specific
\end{verbatim}


\subsection{RESBEAM}
\begin{verbatim}

resbeam,sequence=s_name;
\end{verbatim} 
resets the default values of the beam belonging to sequence s\_name, or
of the default beam if no sequence is given.  


\subsection{RETURN}
\begin{verbatim}

return;
\end{verbatim} 
ends reading from a "called" file; if encountered in the standard input
file, it ends the program execution.  


\subsection{SAVE}
\begin{verbatim}

save,beam,sequence=sequ1,sequ2,...,file=filename,beam,bare;
\end{verbatim} 
saves the sequence(s) specified with all variables and elements needed
for their expansion, onto the file "filename". If quotes are used for
the "filename" capital and low characters are kept as specified, if they
are omitted the "filename" will have lower characters only. The optional
flag can have the value "mad8" (without the quotes), in which case the
sequence(s) is/are saved in MAD-8 input format.  

The flag "beam" is optional; when given, all beams belonging to the
sequences specified are saved at the top of the save file.  

The parameter "sequence" is optional; when omitted, all sequences are
saved.  

However, it is not advisable to use "save" without the "sequence" option
unless you know what you are doing. This practice will avoid spurious
saved entries.    Any number of "select,flag=save" commands may precede
the SAVE command. In that case, the names of elements, variables, and
sequences must match the pattern(s) if given, and in addition the
elements must be of the class(es) specified. See here for a
\href{../Introduction/select.html#save_select}{SAVE with SELECT}
example.  

It is important to note that the precision of the output of the save
command depends on the output precision. Details about default
precisions and how to adjust those precisions can be found at the
\href{../Introduction/set.html#Format}{SET Format} instruction page.   
 
The Attribute 'bare' allows to save just the sequence without the
element definitions nor beam information. This allows to re-read in a
sequence with might otherwise create a stop of the program. This is
particularly useful to turn a line into a sequence to seqedit
it. Example:  
\begin{verbatim}

tl3:line=(ldl6,qtl301,mqn,qtl301,ldl7,qtl302,mqn,qtl302,ldl8,ison);
DLTL3 : LINE=(delay, tl3);
use, period=dltl3;

save,sequence=dltl3,file=t1,bare; // new parameter "bare": only sequ. saved
call,file=t1; // sequence is read in and is now a "real" sequence
// if the two preceding lines are suppressed, seqedit will print a warning
// and else do nothing
use, period=dltl3;
twiss, save, betx=bxa, alfx=alfxa, bety=bya, alfy=alfya;
plot, vaxis=betx, bety, haxis=s, colour:=100;
SEQEDIT, SEQUENCE=dltl3;
  remove,element=cx.bhe0330;
  remove,element=cd.bhe0330;
ENDEDIT;

use, period=dltl3;
twiss, save, betx=bxa, alfx=alfxa, bety=bya, alfy=alfya;
\end{verbatim}


\subsection{SAVEBETA}
\begin{verbatim}

savebeta, label=label,\href{place}{place}=place,sequence=s_name;
\end{verbatim} 
marks a place "place" in an expanded sequence "s\_name"; at the next
TWISS command execution, a  \href{../twiss/twiss.html#beta0}{beta0}
block will be saved at that place with the label "label". This is done
only once; in order to get a new beta0 block there, one has to re-issue
the command. The contents of the beta0 block can then be used in other
commands, e.g. TWISS and MATCH.  

 Example (after sequence expansion): 
\begin{verbatim}

savebeta,label=sb1,place=mb[5],sequence=fivecell;
twiss;
show,sb1;
\end{verbatim} 
will save and show the beta0 block parameters at the end (!) of the
fifth element mb in the sequence.  


\subsection{SELECT} %select</a}{SELECT}
\begin{verbatim}

select, flag=flag,range=range,class=class,pattern=pattern,
        slice=integer,column=s1,s2,s3,..,sn,sequence=s_name,
        full,clear;
\end{verbatim} 
selects one or several elements for special treatment in a subsequent
command. All selections for a given command remain valid until "clear"
is specified; the selection criteria on the same command are logically
ANDed, on different SELECT statements logically ORed.  

 Example: 
\begin{verbatim}

select,flag=error,class=quadrupole,range=mb[1]/mb[5];
select,flag=error,pattern="^mqw.*";
\end{verbatim} 
selects all quadrupoles in the range mb[1] to mb[5], and all elements
(in the whole sequence) the name of which starts with "mqw" for
treatment by the error module.  

"flag" can be one of the following:: 
\begin{itemize}
	\item seqedit: selection of elements for the
          \href{seqedit.html}{seqedit} module.  
	\item error: selection of elements for the
          \href{../error/error.html}{error} assignment module.  
	\item makethin: selection of elements for the
          \href{../makethin/makethin.html}{makethin} module that
          converts the sequence into one with thin elements only.  
	\item sectormap: selection of elements for the
          \href{../Introduction/sectormap.html}{sectormap} output file
          from the Twiss module.  
	\item table: here "table" is a table name such as twiss, track
          etc., and the rows and columns to be written are selected.  
\end{itemize} For the RANGE, CLASS, PATTERN, FULL, and CLEAR parameters
see \href{../Introduction/select.html}{SELECT}.  

"slice" is only used by \href{../makethin/makethin.html}{makethin} and
prescribes the number of slices into which the selected elements have to
be cut (default = 1).  

"column" is only valid for tables and decides the selection of columns
to be written into the TFS file. The "name" argument is special in that
it refers to the actual name of the selected element. For an example,
see \href{../Introduction/select.html}{SELECT}.  


\subsection{SHOW}
\begin{verbatim}

show,command;
\end{verbatim} 
prints the "command" (typically "beam", "beam\%sequ", or an element
name), with the actual value of all its parameters.  


\subsection{STOP}
\begin{verbatim}

stop;
\end{verbatim} 
ends the program execution. 


\subsection{SYSTEM}
\begin{verbatim}

system,"...";
\end{verbatim} 
transfers the string in quotes to the system for execution.  

Example: 
\begin{verbatim}

system,"ln -s /afs/cern.ch/user/u/user/public/some/directory short";
\end{verbatim}


\subsection{TABSTRING}
\begin{verbatim}

tabstring(arg1,arg2,arg3)
\end{verbatim}  
The"string function" tabstring(arg1,arg2,arg3) with exactly  three
arguments; arg1 is a table name (string), arg2 is a column name
(string), arg3 is a row number (integer), count starts at 0. The
function can be used in any context where a string appears; in case
there is no match, it returns \_void\_.  


\subsection{TITLE}
\begin{verbatim}

title,"...";
\end{verbatim} 
inserts the string in quotes as title in various tables and plots.  


\subsection{USE}
\begin{verbatim}

use,period=s_name,range=range,survey;
\end{verbatim} 
expands the sequence with name "s\_name", or a part of it as specified
in the \href{../Introduction/ranges.html#range}{range}. The
\texttt{survey} option plugs the survey data into the sequence elements
nodes on the first pass (see \href{../survey/survey.html}{survey}).  


\subsection{VALUE}
\begin{verbatim}

value,exp1,exp2,...;
\end{verbatim} 
prints the actual values of the expressions given. 

Example: 
\begin{verbatim}

a=clight/1000.;
value,a,pmass,exp(sqrt(2));
\end{verbatim} results in 
\begin{verbatim}

a = 299792.458         ;
pmass = 0.938271998        ;
exp(sqrt(2)) = 4.113250379        ;
\end{verbatim}


\subsection{WRITE}
\begin{verbatim}

write,table=table,file=file_name;
\end{verbatim} 
writes the table "table" onto the file "file\_name"; only the rows and
columns of a preceding select,flag=table,...; are written. If no select
has been issued for this table, the file will only contain the
header. If the FILE argument is omitted, the table is written to
standard output.  


%\href{http://www.cern.ch/Hans.Grote/hansg_sign.html}{hansg}, June 17, 2002 

%%%%\title{Range Selection}
%  Changed by: Chris ISELIN, 27-Jan-1997 
%  Changed by: Hans Grote, 30-Sep-2002 

\section{Program Flow Statements}

\subsection{IF}
\begin{verbatim}
if (logical_expression) {statement 1; statement 2; ...; statement n; }
\end{verbatim}
where \href{logical}{"logical\_expression"} is one of 
\begin{verbatim}
expr1  oper expr2
expr11 oper1 expr12 && expr21 oper2 expr22
expr11 oper1 expr12 || expr21 oper2 expr22
\end{verbatim} 
and \verb+oper+ one of 
\begin{verbatim}
==          ! equal
<>          ! not equal
<           ! less than
>           ! greater than
<=          ! less than or equal
>=          ! greater than or equal
\end{verbatim} 
The expressions are arithmetic expressions of type real. The statements
in the curly brackets are executed if the logical expression is true.  


\subsection{ELSEIF}
\begin{verbatim}
elseif (logical_expression) {statement 1; statement 2; ...; statement n; }
\end{verbatim} 
Only possible (in any number) behind an IF, or another ELSEIF; is
executed if  logical\_expression is true, and if none of the preceding
IF or ELSEIF logical conditions was true.  


\subsection{ELSE}
\begin{verbatim}
else {statement 1; statement 2; ...; statement n; }
\end{verbatim} 
Only possible (once) behind an IF, or an ELSEIF; is executed if
logical\_expression is true, and if none of the preceding IF or ELSEIF
logical conditions was true.  

For a real life example, see \href{foot.html}{ELSE example}. 


\subsection{WHILE}
\begin{verbatim}
while (logical_condition) {statement 1; statement 2; ...; statement n; }
\end{verbatim}  
executes the statements in curly brackets while the logical\_expression
is true. A simple example (in case you have forgotten the first ten
factorials) would be  
\begin{verbatim}
option, -info;   ! avoids redefiniton warnings
n = 1; m = 1;
while (n <= 10)
{
  m = m * n;  value, m;
  n = n + 1;
};
\end{verbatim}

For a real life example, see \href{foot.html}{WHILE example}.

\subsection{MACRO}
\begin{verbatim}
label: macro = {statement 1; statement 2; ...; statement n; };
label(arg1,...,argn): macro = {statement 1; statement 2; ...; statement n; };
\end{verbatim} 
The first form allows the execution of a group of statements via a
single command,  
\begin{verbatim}
exec, label;
\end{verbatim} 
that executes the statements in curly brackets exactly once. This command
can be issued any number of times.  

The second form allows to replace strings anywhere inside the statements
in curly brackets by other strings, or integer numbers prior to
execution. This is a powerful construct and should be handled with care.  

Simple example: 
\begin{verbatim}
option, -echo, -info;  ! for cleaner output
simple(xx,yy): macro = { xx = yy^2 + xx; value, xx;};
a = 3;
b = 5;
exec, simple(a,b);
\end{verbatim}


{\bf Passing arguments}\\
In the following example we use the fact that a "\$" in front of an
argument means that the truncated integer value of this argument is used
for replacement, rather than the argument string itself.  
\begin{verbatim}
tricky(xx,yy,zz): macro = {mzz.yy: xx, l = 1.yy, kzz = k.yy;};
n=0;
while (n < 3)
{
  n = n+1;
  exec, tricky(quadrupole, $n, 1);
  exec, tricky(sextupole, $n, 2);
};
\end{verbatim} 
Whereas the actual use of the preceding example is NOT recommended,
a real life example, showing the full power (!) of macros is to be
found under \href{foot.html}{macro usage} for the usage, and
under \href{foot.html#macro}{macro definition} for the
definition.


{\bf Beware of the following rules:}
\begin{itemize}
   \item Generally speaking: \textit{ special constructs } like IF,
     WHILE, MACRO will only allow one level of inclusion of another
     \textit{ special construct }.
   \item  Macros must not be called with numbers, but with strings
     (i.e. variable names in case of numerical values), i.e. {\bf NOT }
\begin{verbatim}
exec, thismacro($99, $129);
\end{verbatim}
{\bf BUT}
\begin{verbatim}
n1=99; n2=219;
exec, thismacro($n1, $n2);
\end{verbatim}
\end{itemize}

%\href{http://www.cern.ch/Hans.Grote/hansg_sign.html}{hansg}, June 17, 2002




\chapter{General Control Statements} 

\madx consists of a core program, and modules for specific tasks such as
twiss parameter calculation, matching, thin lens tracking, and so on.  
 
The statements listed here are those executed by the program core. They
deal with the I/O, element and sequence declaration, sequence
manipulation, statement flow control (e.g. IF, WHILE), MACRO
declaration, saving sequences onto files in \madx or \madeight format,
and so on.  


%% Moved to TWISS chapter
%% \subsection{COGUESS}
%% \label{subsec:coguess}

%% In order to help the initial finding of the closed orbit by the
%% \texttt{TWISS} module, it is possible to specify an initial guess. 

%% \madbox{
%% COGUESS, \=TOLERANCE=real, \\
%%          \>X=real, PX=real, Y=real, PY=real, T=real, PT=real, \\
%%          \>CLEAR=logical;
%% }
%% sets the required convergence precision in the closed orbit search
%% ("tolerance", see as well Twiss command
%% \href{../twiss/twiss.html#tolerance}{tolerance}).  

%% The other parameters define a first guess for all future closed orbit
%% searches in case they are different from zero.  

%% The clear parameter in the argument list resets the tolerance to its default value 
%% and cancels the effect of coguess in defining a first guess for subsequent 
%% closed orbit searches. \\
%% Default = false, {\tt clear} alone is equivalent to {\tt clear=true}


\section{EXIT, QUIT, STOP}
\label{sec:exit}\label{sec:quit}\label{sec:stop}
Any of these three commands ends the execution of \madx:
\madbox{
EXIT;
}
\madbox{
QUIT;
}
\madbox{
STOP;
}


\section{HELP}
\label{sec:help}
The HELP command prints all parameters, and their defaults values, for
the statement given; this includes basic element types.
\madbox{
HELP, statement\_name;
}

\section{SHOW}
\label{sec:show}
The SHOW command prints the "command" (typically "beam", "beam\%sequ",
or an element name), with the actual value of all its parameters.  
\madbox{
SHOW, command;
}

\section{VALUE}
\label{sec:value}
The VALUE command evaluates the current value of all listed expressions,
constants or variables, and prints the result in the form of \madx
statements on the assigned output file. 
\madbox{
VALUE, expression\{, expression\} ;
}

Example: \\
\madxmp{
a = clight/1000.; \\
value, a, pmass, exp(sqrt(2));
}
results in 
\madxmp{
a = 299792.458         ; \\
pmass = 0.938272046        ; \\
exp(sqrt(2)) = 4.113250379        ; \\
}

\section{OPTION}
\label{sec:option}

The \texttt{OPTION} commands sets the logical value of a number of flags
that control the behavior of \madx.

\madbox{
OPTION, flag=logical;
}

Because all attributes of \texttt{OPTION} are logical flags, the
following two statements are identical:
\madxmp{
OPTION, flag = true;\\
OPTION, flag;
}
And the following two statements are also identical:
\madxmp{
OPTION, flag = false;\\
OPTION, -flag;
}

Several flags can be set in a single \texttt{OPTION} command, e.g.
\madxmp{
OPTION, ECHO, WARN=true, -INFO, VERBOSE=false;
}

The available flags, their default values and their effect on \madx if
they are set to \texttt{TRUE} are listed in table 

\begin{table}[ht]
  \caption{Flags available to \texttt{OPTION} command}
  \vspace{1ex}
  \begin{center}
   \begin{tabular}{|l|c|l|}
     \hline
     {\bf FLAG }  & {\bf default} & {\bf effect if \texttt{TRUE}} \\
     \hline
     \texttt{ECHO}  & true  & echoes the input on the standard output file \\
     \texttt{WARN}  & true  & issues warning statements\\
     \texttt{INFO}  & true  & issues information statements\\
     \texttt{DEBUG} & false & issues debugging information \\
     \texttt{ECHOMACRO} & false & issues macro expansion printout for debugging \\
     \texttt{VERBOSE} & false & issues additional printout in makethin \\
     \texttt{TRACE}   & false & prints the system time after each command \\
     \texttt{VERIFY}  & false & issues a warning if an undefined variable is used \\
     \hline
     \texttt{TELL}  & false & prints the current value of all options \\
     \texttt{RESET} & false & resets all options to their defaults \\
     \hline
     \texttt{NO\_FATAL\_STOP} & false & Prevents madx from stopping in case of a fatal error \\
                              &       & {\bf Use at your own risk !} \\
     \hline
     \texttt{RBARC}     & true & converts the RBEND straight length into the arc length \\
     \texttt{THIN\_FOC} & true & enables the 1/(rho**2) focusing of thin dipoles \\
     \texttt{BBORBIT}   & false & the closed orbit is modified by beam-beam kicks \\
     \texttt{SYMPL}     & false & all element matrices are symplectified in Twiss \\
     \texttt{TWISS\_PRINT} & true & controls whether the twiss command produces output \\
     \texttt{THREADER}  & false & enables the threader for closed orbit finding in Twiss \\ 
                        &       & (see Twiss module) \\ 
     \hline
     \texttt{BB\_ULTRA\_RELATI} & false  & To be documented \\
     \texttt{BB\_SXY\_UPDATE}   & false  & To be documented \\
     \texttt{EMITTANCE\_UPDATE} & true   & To be documented \\
     \texttt{FAST\_ERROR\_FUNC} & false  & To be documented \\
     \hline
     \end{tabular}
   \end{center}
\end{table}

%% \begin{verbatim}
%%   name           default meaning if true
%%   ====           ======= ===============
%%   echo            true   echoes the input on the standard output file
%%   warn            true   issues warnings
%%   info            true   issues information
%%   debug           false  issues debugging information
%%   echomacro       false  issues macro expansion printout for debugging
%%   verbose         false  issues additional printout in makethin
%%   trace           false  prints the system time after each command
%%   verify          false  issues a warning if an undefined variable is used
%%   tell            false  prints the current value of all options
%%   reset           false  resets all options to their defaults
%%   no_fatal_stop   false  Prevents madx from stopping in case of a fatal error. 
%%                          Use at your own risk.

%%   rbarc           true   converts the RBEND straight length into the arc 
%%                          length
%%   thin_foc        true   if false suppresses the 1(rho**2) focusing of thin 
%%                          dipoles
%%   bborbit         false  the closed orbit is modified by beam-beam kicks
%%   sympl           false  all element matrices are symplectified in Twiss
%%   twiss_print     true   controls whether the twiss command produces output.
%%   threader        false  enables the threader for closed orbit finding in Twiss.
%%                          (see Twiss module)

%%   bb_ultra_relati false  To be documented
%%   bb_sxy_update   false  To be documented 
%%   emittance_update true  To be documented
%%   fast_error_func false  To be documented
%% \end{verbatim} 

The option \texttt{RBARC} is implemented for backwards compatibility
with \madeight up to version 8.23.06 included; in this version, the
RBEND length was just taken as the arc length of an SBEND with inclined
pole faces, contrary to the \madeight manual.  



\section{EXEC}
\label{sec:exec}
Each statement may be preceded by a label, when parsed and executed the
statement is then also stored and can be executed again with
\madbox{
EXEC, label;
}
This mechanism can be invoked any number of times, and the executed
statement may be calling another \texttt{EXEC}, etc. 

Note however, that the main usage of this \madx construct is the
execution of a \href{special.html#macro}{macro}.   

\madxmp{
tw: TWISS, FILE, SAVE; ! first execution of TWISS \\
... \\
EXEC, tw; ! second execution of the same TWISS command \\
}


\section{SET}
\label{sec:set}
The SET command is used in two forms:
\madbox{
SET, FORMAT=string \{, string\} ;\\
SET, SEQUENCE=string;
}


The first form of the \texttt{SET} command defines the formats for the
output precision that \madx uses with the \texttt{SAVE}, \texttt{SHOW},
\texttt{VALUE} and \texttt{TABLE} commands. The formats can be
given in any order and stay valid until replaced. 

The formats follow the C convention and must be included in double
quotes. The allowed formats are \\
\texttt{{\it n}d} for integers with a field-width = {\it n}, \\
\texttt{{\it m}.{\it n}f} or \texttt{{\it m}.{\it n}g} or \texttt{{\it
    m}.{\it n}e} for floats with field-width = {\it m} and precision =
       {\it n}, \\
\texttt{{\it n}s} for strings with a field-width = {\it n}.\\
The default is "right adjusted", a "-" changes it to "left adjusted".

{\bf Example:}\\
\madxmp{
SET, FORMAT="12d", "-18.5e", "25s";
}

%% \begin{verbatim}
%% "nd" for integer with n = field width.
%% \end{verbatim}
%% \begin{verbatim}
%% "m.nf" or "m.ng" or "m.ne" for floating, m field width, n precision.
%% \end{verbatim}
%% \begin{verbatim}
%% "ns" for string output.
%% \end{verbatim} 


The default formats are \texttt{"10d"},\texttt{"18.10g"} and \texttt{"-18s"}.

Example: 
\begin{verbatim}
set,format="22.14e";
\end{verbatim} 
changes the current floating point format to 22.14e; the other formats remain untouched. 
\begin{verbatim}
set,format="s","d","g";
\end{verbatim} 
sets all formats to automatic adjustment according to C conventions. 



The second form of the \texttt{SET} command allows to select the
current sequence without the "USE" command, which would
bring back to a bare lattice without errors. The command only works
if the chosen sequence has been previously activated with a \texttt{USE} command,
otherwise a warning is issued and \madx continues with the
unmodified current sequence. This command is particularly useful for
commands that do not have the sequence as an argument like "EMIT" or
"IBS". 



\section{SYSTEM}
\label{sec:system}
\madbox{
SYSTEM, "string";
}
transfers the quoted string to the system for execution. The quotes are
stripped and no check of the return status is performed bt \madx.

{\bf Example:}\\ 
\madxmp{
SYSTEM,"ln -s /afs/cern.ch/user/j/joe/public/some/directory shortname";
}
makes a local link to an AFS directory with the name "shortname" on a
UNIX system. 

Attention: Although this command is very convenient, it is clearly not portable
across systems and should probably avoid it if you intend to share \madx scripts with
collaborators working on other platforms. 

\section{TITLE}
\label{sec:title}
\madbox{
TITLE, "string";
}
the string in quotes is inserted as title in various table outputs and
plot results.  


\section{USE}
\label{sec:use}
\madx operates on beamlines that must be loaded and expanded in memory
before other commands can be invoked. The \texttt{USE} allows this
loading and expansion.

\madbox{
USE, \=PERIOD=sequence\_name, SEQUENCE=sequence\_name, \\
     \>RANGE=range, \\
     \>SURVEY=logical;
}

The attributes to the \texttt{USE} command are:
\begin{madlist}
  \ttitem{SEQUENCE} name of the sequence to be loaded and expanded. 
  \ttitem{PERIOD} name of the sequence to be loaded and expanded. \\ 
  \texttt{PERIOD} is an alias to \texttt{SEQUENCE} that was kept for
  backwards compatibility with \madeight and only one of them should be
  specified in a \texttt{USE} statement. 
  \ttitem{RANGE} specifies a \hyperref[sec:range]{range}.   
  restriction so that only the specified part of the named sequence is
  loaded and  expanded.
  \ttitem{SURVEY} option to plug the survey data into the sequence elements
  nodes on the first pass (see \hyperref[chap:survey]{\tt SURVEY}).
\end{madlist}

Note that reloading a sequence with the \texttt{USE} command reloads a
bare sequence and that any \texttt{ERROR} or orbit correction previously
assigned or associated to the sequence are forgotten. 
A mechanism to select a sequence without this side effect of the
\texttt{USE} command is provided with the \texttt{SET, SEQUENCE=...} command.


\section{SELECT} 
\label{sec:select}

\begin{verbatim}
select, flag=string, range=string, class=string, pattern=string,
        sequence=string, full, clear,
        column = string{, string},  slice=integer, thick=logical;
\end{verbatim} 
selects one or several elements for special treatment in a subsequent
command based on selection criteria.

The selection criteria on a single SELECT statement are logically
ANDed, in other words, selected elements have to fulfill the {\tt RANGE},
{\tt CLASS}, and {\tt PATTERN} criteria.  
The selection criteria on different SELECT statements are logically
ORed, in other words selected elements have to fulfill any of the
selection criteria accumulated by the different statements.   
All selections for a given command remain valid until the "clear" argument
is specified; 

The "flag" argument allows a determination of the applicability of the
SELECT statement and can be one of the following: 
\begin{madlist}
   \ttitem{seqedit} selection of elements for the
     \hyperref[sec:seqedit]{seqedit} module.  
   \ttitem{error} selection of elements for the
     \hyperref[chap:error]{error} assignment module.  
   \ttitem{makethin} selection of elements for the
     \hyperref[chap:makethin]{makethin} module that
     converts the sequence into one with thin elements only.  
   \ttitem{sectormap} selection of elements for the
     \hyperref[subsec:sectormap]{sectormap} output file
     from the Twiss module.  
   \ttitem{save} selection of elements for the \hyperref[sec:save]{\tt
     SAVE} command.   
   \ttitem{table} is a table name such as {\tt twiss}, {\tt track}
     etc., and the rows and columns to be written are selected.  
\end{madlist} 

The statement
\madxmp{
SELECT, FLAG=name, FULL;
}
selects ALL positions in the sequence for the flag "name". This is the default
for flags for all tables and {\tt MAKETHIN}.

The statement 
\madxmp{
SELECT, FLAG=name, CLEAR;
}
deselects ALL positions in the sequence for the flag "name". This is the default
for flags {\tt ERROR} and {\tt SEQEDIT}.

"slice" is only used by \hyperref[chap:makethin]{makethin} and
prescribes the number of slices into which the selected elements have to
be cut (default = 1).  

"column" is only valid for tables and determines the selection of columns
to be written into the TFS file. The "name" argument is special in that
it refers to the actual name of the selected element. For an example,
see \hyperref[sec:select]{SELECT}.  

"thick" is used to determine whether the selected elements will be
treated as thick elements by the MAKETHIN command. This only applies to
QUADRUPOLES and BENDS for which thick maps have been explicitely
derived. (see ...) 
%%2014-Apr-08  17:43:44  ghislain:  A completer.

Example: 
% keep verbatim for now and until ^ is solved
\begin{verbatim} 
select, flag = error, class = quadrupole, range = mb[1]/mb[5];
select, flag = error, pattern = "^mqw.*";
\end{verbatim}
selects all quadrupoles in the range mb[1] to mb[5], as well as all
elements (in the whole sequence) with name starting with "mqw", for 
treatment by the error module.  

Example:  
\madxmp{
select, flag=save, class=variable, pattern="abc.*";
save, file=mysave;
}
will save all variables (and sequences) containing "abc" in their name.
However note that since the element class "variable" does not exist, any
element with name containing "abc" will not be saved. 

\vskip 1cm
\hrule
\vskip 1cm

%% Imported from chapter 2
%\subsection{Selection Statements}

The elements, or a range of elements, in a sequence can be selected for
various purposes. Such selections remain valid until cleared (in
difference to \madeight); it is therefore recommended to always start with a  

\begin{verbatim}
select, flag =..., clear;
\end{verbatim} 
before setting a new selection. 
\begin{verbatim}
SELECT, FLAG=name, RANGE=range, CLASS=class, PATTERN=pattern [,FULL] [,CLEAR];
\end{verbatim} 
where the name for FLAG can be one of ERROR, MAKETHIN, SEQEDIT or the
name of a twiss table which is established for all sequence positions in
general.  

Selected elements have to fulfill the \href{ranges.html#range}{RANGE},
\href{ranges.html#class}{CLASS}, and \href{wildcard.html}{PATTERN}
criteria.  

Any number of SELECT commands can be issued for the same flag and are
accumulated (logically ORed). In this context note the following:  

\begin{verbatim}
SELECT, FLAG=name, FULL;
\end{verbatim} 
selects all positions in the sequence for this flag. This is the default
for all tables and makethin, whereas for ERROR and SEQEDIT the default
is "nothing selected".  

%\href{save_select}{}
\label{save_select}
SAVE: A SELECT,FLAG=SAVE statement causes the
selected sequences, elements, and variables to be written into the save
file. A class (only used for element selection), and a pattern can be
specified. Example:  
\begin{verbatim}
select, flag=save, class=variable, pattern="abc.*";
save, file=mysave;
\end{verbatim} 
will save all variables (and sequences) containing "abc" in their name,
but not elements with names containing "abc" since the class "variable"
does not exist (astucieux, non ?).  

SECTORMAP: A SELECT,FLAG=SECTORMAP statement causes sectormaps to be
written into the file "sectormap" like in \madeight. For the file to be
written, a flag SECTORMAP must be issued on the TWISS command in
addition.  

TWISS: A SELECT,FLAG=TWISS statement causes the selected rows and
columns to be written into the Twiss TFS file (former OPTICS command in
\madeight). The column selection is done on the same select. See as well
example 2.  

%% Example 1:  
%% \begin{verbatim}
%% TITLE,'Test input for MAD-X';

%% option,rbarc=false; // use arc length of rbends
%% beam; ! sets the default beam for the following sequence
%% option,-echo;
%% call file=fv9.opt;  ! contains optics parameters
%% call file="fv9.seq"; ! contains a small sequence "fivecell"
%% OPTION,ECHO;
%% SELECT,FLAG=SECTORMAP,clear;
%% SELECT,FLAG=SECTORMAP,PATTERN="^m.*";
%% SELECT,FLAG=TWISS,clear;
%% SELECT,FLAG=TWISS,PATTERN="^m.*",column=name,s,betx,bety;
%% USE,PERIOD=FIVECELL;
%% twiss,file=optics,sectormap;
%% stop;
%% \end{verbatim} 

%% This produces a file \href{sectormap.html}{sectormap}, and a
%% twiss output file \label{tfs} (name = optics):  
%% \begin{verbatim}
%% @ TYPE             %05s "TWISS"
%% @ PARTICLE         %08s "POSITRON"
%% @ MASS             %le          0.000510998902
%% @ CHARGE           %le                       1
%% @ E0               %le                       1
%% @ PC               %le           0.99999986944
%% @ GAMMA            %le           1956.95136738
%% @ KBUNCH           %le                       1
%% @ NPART            %le                       0
%% @ EX               %le                       1
%% @ EY               %le                       1
%% @ ET               %le                       0
%% @ LENGTH           %le                   534.6
%% @ ALFA             %le        0.00044339992938
%% @ ORBIT5           %le                      -0
%% @ GAMMATR          %le           47.4900022541
%% @ Q1               %le           1.25413071556
%% @ Q2               %le           1.25485338377
%% @ DQ1              %le           1.05329608302
%% @ DQ2              %le           1.04837000224
%% @ DXMAX            %le           2.17763211131
%% @ DYMAX            %le                       0
%% @ XCOMAX           %le                       0
%% @ YCOMAX           %le                       0
%% @ BETXMAX          %le            177.70993499
%% @ BETYMAX          %le           177.671582415
%% @ XCORMS           %le                       0
%% @ YCORMS           %le                       0
%% @ DXRMS            %le           1.66004270906
%% @ DYRMS            %le                       0
%% @ DELTAP           %le                       0
%% @ TITLE            %20s "Test input for MAD-X"
%% @ ORIGIN           %16s "MAD-X 0.20 Linux"
%% @ DATE             %08s "07/06/02"
%% @ TIME             %08s "14.25.51"
%% * NAME               S                  BETX               BETY               
%% $ %s                 %le                %le                %le                
%%  "MSCBH"             4.365              171.6688159        33.31817319       
%%  "MB"                19.72              108.1309095        58.58680717       
%%  "MB"                35.38              61.96499987        102.9962313       
%%  "MB"                51.04              34.61640793        166.2227523       
%%  "MSCBV.1"           57.825             33.34442808        171.6309057       
%%  "MB"                73.18              58.61984637        108.0956006       
%%  "MB"                88.84              103.0313887        61.93159422       
%%  "MB"                104.5              166.2602486        34.58939635       
%%  "MSCBH"             111.285            171.6688159        33.31817319       
%%  "MB"                126.64             108.1309095        58.58680717       
%%  "MB"                142.3              61.96499987        102.9962313       
%%  "MB"                157.96             34.61640793        166.2227523       
%%  "MSCBV"             164.745            33.34442808        171.6309057       
%%  "MB"                180.1              58.61984637        108.0956006       
%%  "MB"                195.76             103.0313887        61.93159422       
%%  "MB"                211.42             166.2602486        34.58939635       
%%  "MSCBH"             218.205            171.6688159        33.31817319       
%%  "MB"                233.56             108.1309095        58.58680717       
%%  "MB"                249.22             61.96499987        102.9962313       
%%  "MB"                264.88             34.61640793        166.2227523       
%%  "MSCBV"             271.665            33.34442808        171.6309057       
%%  "MB"                287.02             58.61984637        108.0956006       
%%  "MB"                302.68             103.0313887        61.93159422       
%%  "MB"                318.34             166.2602486        34.58939635       
%%  "MSCBH"             325.125            171.6688159        33.31817319       
%%  "MB"                340.48             108.1309095        58.58680717       
%%  "MB"                356.14             61.96499987        102.9962313       
%%  "MB"                371.8              34.61640793        166.2227523       
%%  "MSCBV"             378.585            33.34442808        171.6309057       
%%  "MB"                393.94             58.61984637        108.0956006       
%%  "MB"                409.6              103.0313887        61.93159422       
%%  "MB"                425.26             166.2602486        34.58939635       
%%  "MSCBH"             432.045            171.6688159        33.31817319       
%%  "MB"                447.4              108.1309095        58.58680717       
%%  "MB"                463.06             61.96499987        102.9962313       
%%  "MB"                478.72             34.61640793        166.2227523       
%%  "MSCBV"             485.505            33.34442808        171.6309057       
%%  "MB"                500.86             58.61984637        108.0956006       
%%  "MB"                516.52             103.0313887        61.93159422       
%%  "MB"                532.18             166.2602486        34.58939635       
%% \end{verbatim}

 %% Example 2: 

%%  Addition of variables to (any internal) table: 
%% \begin{verbatim}
%% select, flag=table, column=name, s, betx, ..., var1, var2, ...; ! or
%% select, flag=table, full, column=var1, var2, ...; ! default col.s + new
%% \end{verbatim} 
%% will write the current value of var1 etc. into the table each time a new
%% line is added; values from the same (current) line can be accessed by
%% these variables, e.g.  
%% \begin{verbatim}
%% var1 := sqrt(beam->ex*table(twiss,betx));
%% \end{verbatim} 
%% in the case of table above being "twiss". The plot command accepts the
%% new variables.  

%% Remark: this replaces the "string" variables of MAD-8. 

%%  This example demonstrates as well the usage of a user defined table \label{ucreate}. 
%% \begin{verbatim}
%% beam,ex=1.e-6,ey=1.e-3;
%% // element definitions
%% mb:rbend, l=14.2, angle:=0,k0:=bang/14.2;
%% mq:quadrupole, l:=3.1,apertype=ellipse,aperture={1,2};
%% qft:mq, l:=0.31, k1:=kqf,tilt=-pi/4;
%% qf.1:mq, l:=3.1, k1:=kqf;
%% qf.2:mq, l:=3.1, k1:=kqf;
%% qf.3:mq, l:=3.1, k1:=kqf;
%% qf.4:mq, l:=3.1, k1:=kqf;
%% qf.5:mq, l:=3.1, k1:=kqf;
%% qd.1:mq, l:=3.1, k1:=kqd;
%% qd.2:mq, l:=3.1, k1:=kqd;
%% qd.3:mq, l:=3.1, k1:=kqd;
%% qd.4:mq, l:=3.1, k1:=kqd;
%% qd.5:mq, l:=3.1, k1:=kqd;
%% bph:hmonitor, l:=l.bpm;
%% bpv:vmonitor, l:=l.bpm;
%% cbh:hkicker;
%% cbv:vkicker;
%% cbh.1:cbh, kick:=acbh1;
%% cbh.2:cbh, kick:=acbh2;
%% cbh.3:cbh, kick:=acbh3;
%% cbh.4:cbh, kick:=acbh4;
%% cbh.5:cbh, kick:=acbh5;
%% cbv.1:cbv, kick:=acbv1;
%% cbv.2:cbv, kick:=acbv2;
%% cbv.3:cbv, kick:=acbv3;
%% cbv.4:cbv, kick:=acbv4;
%% cbv.5:cbv, kick:=acbv5;
%% !mscbh:sextupole, l:=1.1, k2:=ksf;
%% mscbh:multipole, knl:={0,0,0,ksf},tilt=-pi/8;
%% mscbv:sextupole, l:=1.1, k2:=ksd;
%% !mscbv:octupole, l:=1.1, k3:=ksd,tilt=-pi/8;

%% // sequence declaration

%% fivecell:sequence, refer=centre, l=534.6;
%%    qf.1:qf.1, at=1.550000e+00;
%%    qft:qft, at=3.815000e+00;
%% !   mscbh:mscbh, at=3.815000e+00;
%%    cbh.1:cbh.1, at=4.365000e+00;
%%    mb:mb, at=1.262000e+01;
%%    mb:mb, at=2.828000e+01;
%%    mb:mb, at=4.394000e+01;
%%    bpv:bpv, at=5.246000e+01;
%%    qd.1:qd.1, at=5.501000e+01;
%%    mscbv:mscbv, at=5.727500e+01;
%%    cbv.1:cbv.1, at=5.782500e+01;
%%    mb:mb, at=6.608000e+01;
%%    mb:mb, at=8.174000e+01;
%%    mb:mb, at=9.740000e+01;
%%    bph:bph, at=1.059200e+02;
%%    qf.2:qf.2, at=1.084700e+02;
%%    mscbh:mscbh, at=1.107350e+02;
%%    cbh.2:cbh.2, at=1.112850e+02;
%%    mb:mb, at=1.195400e+02;
%%    mb:mb, at=1.352000e+02;
%%    mb:mb, at=1.508600e+02;
%%    bpv:bpv, at=1.593800e+02;
%%    qd.2:qd.2, at=1.619300e+02;
%%    mscbv:mscbv, at=1.641950e+02;
%%    cbv.2:cbv.2, at=1.647450e+02;
%%    mb:mb, at=1.730000e+02;
%%    mb:mb, at=1.886600e+02;
%%    mb:mb, at=2.043200e+02;
%%    bph:bph, at=2.128400e+02;
%%    qf.3:qf.3, at=2.153900e+02;
%%    mscbh:mscbh, at=2.176550e+02;
%%    cbh.3:cbh.3, at=2.182050e+02;
%%    mb:mb, at=2.264600e+02;
%%    mb:mb, at=2.421200e+02;
%%    mb:mb, at=2.577800e+02;
%%    bpv:bpv, at=2.663000e+02;
%%    qd.3:qd.3, at=2.688500e+02;
%%    mscbv:mscbv, at=2.711150e+02;
%%    cbv.3:cbv.3, at=2.716650e+02;
%%    mb:mb, at=2.799200e+02;
%%    mb:mb, at=2.955800e+02;
%%    mb:mb, at=3.112400e+02;
%%    bph:bph, at=3.197600e+02;
%%    qf.4:qf.4, at=3.223100e+02;
%%    mscbh:mscbh, at=3.245750e+02;
%%    cbh.4:cbh.4, at=3.251250e+02;
%%    mb:mb, at=3.333800e+02;
%%    mb:mb, at=3.490400e+02;
%%    mb:mb, at=3.647000e+02;
%%    bpv:bpv, at=3.732200e+02;
%%    qd.4:qd.4, at=3.757700e+02;
%%    mscbv:mscbv, at=3.780350e+02;
%%    cbv.4:cbv.4, at=3.785850e+02;
%%    mb:mb, at=3.868400e+02;
%%    mb:mb, at=4.025000e+02;
%%    mb:mb, at=4.181600e+02;
%%    bph:bph, at=4.266800e+02;
%%    qf.5:qf.5, at=4.292300e+02;
%%    mscbh:mscbh, at=4.314950e+02;
%%    cbh.5:cbh.5, at=4.320450e+02;
%%    mb:mb, at=4.403000e+02;
%%    mb:mb, at=4.559600e+02;
%%    mb:mb, at=4.716200e+02;
%%    bpv:bpv, at=4.801400e+02;
%%    qd.5:qd.5, at=4.826900e+02;
%%    mscbv:mscbv, at=4.849550e+02;
%%    cbv.5:cbv.5, at=4.855050e+02;
%%    mb:mb, at=4.937600e+02;
%%    mb:mb, at=5.094200e+02;
%%    mb:mb, at=5.250800e+02;
%%    bph:bph, at=5.336000e+02;
%% end:marker, at=5.346000e+02;
%% endsequence;

%% // forces and other constants

%% l.bpm:=.3;
%% bang:=.509998807401e-2;
%% kqf:=.872651312e-2;
%% kqd:=-.872777242e-2;
%% ksf:=.0198492943;
%% ksd:=-.039621283;
%% acbv1:=1.e-4;
%% acbh1:=1.e-4;
%% !save,sequence=fivecell,file,mad8;

%% s := table(twiss,bpv[5],betx);
%% myvar := sqrt(beam->ex*table(twiss,betx));
%% use, period=fivecell;
%% select,flag=twiss,column=name,s,myvar,apertype;
%% twiss,file;
%% n = 0;
%% create,table=mytab,column=dp,mq1,mq2;
%% mq1:=table(summ,q1);
%% mq2:=table(summ,q2);
%% while ( n < 11)
%% {
%%   n = n + 1;
%%   dp = 1.e-4*(n-6);
%%   twiss,deltap=dp;
%%   fill,table=mytab;
%% }
%% write,table=mytab;
%% plot,haxis=s,vaxis=aper_1,aper_2,colour=100,range=#s/cbv.1,notitle;
%% stop;
%% \end{verbatim}
%% prints the following user table on output:

%% \begin{verbatim}
%% @ NAME             %05s "MYTAB"
%% @ TYPE             %04s "USER"
%% @ TITLE            %08s "no-title"
%% @ ORIGIN           %16s "MAD-X 1.09 Linux"
%% @ DATE             %08s "10/12/02"
%% @ TIME             %08s "10.45.25"
%% * DP                 MQ1                MQ2                
%% $ %le                %le                %le                
%%  -0.0005            1.242535951        1.270211135       
%%  -0.0004            1.242495534        1.270197018       
%%  -0.0003            1.242452432        1.270185673       
%%  -0.0002            1.242406653        1.270177093       
%%  -0.0001            1.242358206        1.270171269       
%%  0                  1.242307102        1.27016819        
%%  0.0001             1.242253353        1.270167843       
%%  0.0002             1.242196974        1.270170214       
%%  0.0003             1.24213798         1.270175288       
%%  0.0004             1.242076387        1.270183048       
%%  0.0005             1.242012214        1.270193477       
%% \end{verbatim}
%% and produces a twiss file with the additional column myvar, as well as a plot
%% file with the aperture values plotted.


%% \href{screate}{}

%% Example of joining two tables with different length into a third table
%% making use of the length of either table as given by
%% table("your\_table\_name", tablelength) and adding names by the "\_name"
%% attribute.

%% \begin{verbatim}
%% title,   "summing of offset and alignment tables";
%% set,    format="13.6f";

%% readtable, table=align,  file="align.ip2.b1.tfs";   // mesured alignment
%% readtable, table=offset, file="offset.ip2.b1.tfs";  // nominal offsets

%% n_elem  =  table(offset, tablelength);

%% create,  table=align_offset, column=_name,s_ip,x_off,dx_off,ddx_off,y_off,dy_off,ddy_off;

%% calcul(elem_name) : macro = {
%%     x_off = table(align,  elem_name, x_ali) + x_off;
%%     y_off = table(align,  elem_name, y_ali) + y_off;
%% }


%% one_elem(j_elem) : macro = {
%%     setvars, table=offset, row=j_elem;
%%     exec,  calcul(tabstring(offset, name, j_elem));
%%     fill,  table=align_offset;
%% }


%% i_elem = 0;
%% while (i_elem < n_elem) { i_elem = i_elem + 1; exec,  one_elem($i_elem); }

%% write, table=align_offset, file="align_offset.tfs";

%% stop;
%% \end{verbatim}

%%



%%%\title{SELECT}
%  Changed by: Hans Grote, 16-Jan-2003 

\subsection{Selection Statements}

The elements, or a range of elements, in a sequence can be selected for
various purposes. Such selections remain valid until cleared (in
difference to MAD-8); it is therefore recommended to always start with a  

\begin{verbatim}
select, flag =..., clear;
\end{verbatim} 
before setting a new selection. 
\begin{verbatim}
SELECT, FLAG=name, RANGE=range, CLASS=class, PATTERN=pattern [,FULL] [,CLEAR];
\end{verbatim} 
where the name for FLAG can be one of ERROR, MAKETHIN, SEQEDIT or the
name of a twiss table which is established for all sequence positions in
general.  

Selected elements have to fulfill the \href{ranges.html#range}{RANGE},
\href{ranges.html#class}{CLASS}, and \href{wildcard.html}{PATTERN}
criteria.  

Any number of SELECT commands can be issued for the same flag and are
accumulated (logically ORed). In this context note the following:  

\begin{verbatim}
SELECT, FLAG=name, FULL;
\end{verbatim} 
selects all positions in the sequence for this flag. This is the default
for all tables and makethin, whereas for ERROR and SEQEDIT the default
is "nothing selected".  

\href{save_select}{}SAVE: A SELECT,FLAG=SAVE statement causes the
selected sequences, elements, and variables to be written into the save
file. A class (only used for element selection), and a pattern can be
specified. Example:  
\begin{verbatim}
select, flag=save, class=variable, pattern="abc.*";
save, file=mysave;
\end{verbatim} 
will save all variables (and sequences) containing "abc" in their name,
but not elements with names containing "abc" since the class "variable"
does not exist (astucieux, non ?).  

SECTORMAP: A SELECT,FLAG=SECTORMAP statement causes sectormaps to be
written into the file "sectormap" like in MAD-8. For the file to be
written, a flag SECTORMAP must be issued on the TWISS command in
addition.  

TWISS: A SELECT,FLAG=TWISS statement causes the selected rows and
columns to be written into the Twiss TFS file (former OPTICS command in
MAD-8). The column selection is done on the same select. See as well
example 2.  

Example 1:  
\begin{verbatim}
TITLE,'Test input for MAD-X';

option,rbarc=false; // use arc length of rbends
beam; ! sets the default beam for the following sequence
option,-echo;
call file=fv9.opt;  ! contains optics parameters
call file="fv9.seq"; ! contains a small sequence "fivecell"
OPTION,ECHO;
SELECT,FLAG=SECTORMAP,clear;
SELECT,FLAG=SECTORMAP,PATTERN="^m.*";
SELECT,FLAG=TWISS,clear;
SELECT,FLAG=TWISS,PATTERN="^m.*",column=name,s,betx,bety;
USE,PERIOD=FIVECELL;
twiss,file=optics,sectormap;
stop;
\end{verbatim} 

This produces a file \href{sectormap.html}{sectormap}, and a
\href{tfs}{}twiss output file (name = optics):  
\begin{verbatim}
@ TYPE             %05s "TWISS"
@ PARTICLE         %08s "POSITRON"
@ MASS             %le          0.000510998902
@ CHARGE           %le                       1
@ E0               %le                       1
@ PC               %le           0.99999986944
@ GAMMA            %le           1956.95136738
@ KBUNCH           %le                       1
@ NPART            %le                       0
@ EX               %le                       1
@ EY               %le                       1
@ ET               %le                       0
@ LENGTH           %le                   534.6
@ ALFA             %le        0.00044339992938
@ ORBIT5           %le                      -0
@ GAMMATR          %le           47.4900022541
@ Q1               %le           1.25413071556
@ Q2               %le           1.25485338377
@ DQ1              %le           1.05329608302
@ DQ2              %le           1.04837000224
@ DXMAX            %le           2.17763211131
@ DYMAX            %le                       0
@ XCOMAX           %le                       0
@ YCOMAX           %le                       0
@ BETXMAX          %le            177.70993499
@ BETYMAX          %le           177.671582415
@ XCORMS           %le                       0
@ YCORMS           %le                       0
@ DXRMS            %le           1.66004270906
@ DYRMS            %le                       0
@ DELTAP           %le                       0
@ TITLE            %20s "Test input for MAD-X"
@ ORIGIN           %16s "MAD-X 0.20 Linux"
@ DATE             %08s "07/06/02"
@ TIME             %08s "14.25.51"
* NAME               S                  BETX               BETY               
$ %s                 %le                %le                %le                
 "MSCBH"             4.365              171.6688159        33.31817319       
 "MB"                19.72              108.1309095        58.58680717       
 "MB"                35.38              61.96499987        102.9962313       
 "MB"                51.04              34.61640793        166.2227523       
 "MSCBV.1"           57.825             33.34442808        171.6309057       
 "MB"                73.18              58.61984637        108.0956006       
 "MB"                88.84              103.0313887        61.93159422       
 "MB"                104.5              166.2602486        34.58939635       
 "MSCBH"             111.285            171.6688159        33.31817319       
 "MB"                126.64             108.1309095        58.58680717       
 "MB"                142.3              61.96499987        102.9962313       
 "MB"                157.96             34.61640793        166.2227523       
 "MSCBV"             164.745            33.34442808        171.6309057       
 "MB"                180.1              58.61984637        108.0956006       
 "MB"                195.76             103.0313887        61.93159422       
 "MB"                211.42             166.2602486        34.58939635       
 "MSCBH"             218.205            171.6688159        33.31817319       
 "MB"                233.56             108.1309095        58.58680717       
 "MB"                249.22             61.96499987        102.9962313       
 "MB"                264.88             34.61640793        166.2227523       
 "MSCBV"             271.665            33.34442808        171.6309057       
 "MB"                287.02             58.61984637        108.0956006       
 "MB"                302.68             103.0313887        61.93159422       
 "MB"                318.34             166.2602486        34.58939635       
 "MSCBH"             325.125            171.6688159        33.31817319       
 "MB"                340.48             108.1309095        58.58680717       
 "MB"                356.14             61.96499987        102.9962313       
 "MB"                371.8              34.61640793        166.2227523       
 "MSCBV"             378.585            33.34442808        171.6309057       
 "MB"                393.94             58.61984637        108.0956006       
 "MB"                409.6              103.0313887        61.93159422       
 "MB"                425.26             166.2602486        34.58939635       
 "MSCBH"             432.045            171.6688159        33.31817319       
 "MB"                447.4              108.1309095        58.58680717       
 "MB"                463.06             61.96499987        102.9962313       
 "MB"                478.72             34.61640793        166.2227523       
 "MSCBV"             485.505            33.34442808        171.6309057       
 "MB"                500.86             58.61984637        108.0956006       
 "MB"                516.52             103.0313887        61.93159422       
 "MB"                532.18             166.2602486        34.58939635       
\end{verbatim}

 Example 2: 

 Addition of variables to (any internal) table: 
\begin{verbatim}
select, flag=table, column=name, s, betx, ..., var1, var2, ...; ! or
select, flag=table, full, column=var1, var2, ...; ! default col.s + new
\end{verbatim} 
will write the current value of var1 etc. into the table each time a new
line is added; values from the same (current) line can be accessed by
these variables, e.g.  
\begin{verbatim}
var1 := sqrt(beam->ex*table(twiss,betx));
\end{verbatim} 
in the case of table above being "twiss". The plot command accepts the
new variables.  

Remark: this replaces the "string" variables of MAD-8. 

\href{ucreate}{} This example demonstrates as well the usage of a user defined table. 
\begin{verbatim}
beam,ex=1.e-6,ey=1.e-3;
// element definitions
mb:rbend, l=14.2, angle:=0,k0:=bang/14.2;
mq:quadrupole, l:=3.1,apertype=ellipse,aperture={1,2};
qft:mq, l:=0.31, k1:=kqf,tilt=-pi/4;
qf.1:mq, l:=3.1, k1:=kqf;
qf.2:mq, l:=3.1, k1:=kqf;
qf.3:mq, l:=3.1, k1:=kqf;
qf.4:mq, l:=3.1, k1:=kqf;
qf.5:mq, l:=3.1, k1:=kqf;
qd.1:mq, l:=3.1, k1:=kqd;
qd.2:mq, l:=3.1, k1:=kqd;
qd.3:mq, l:=3.1, k1:=kqd;
qd.4:mq, l:=3.1, k1:=kqd;
qd.5:mq, l:=3.1, k1:=kqd;
bph:hmonitor, l:=l.bpm;
bpv:vmonitor, l:=l.bpm;
cbh:hkicker;
cbv:vkicker;
cbh.1:cbh, kick:=acbh1;
cbh.2:cbh, kick:=acbh2;
cbh.3:cbh, kick:=acbh3;
cbh.4:cbh, kick:=acbh4;
cbh.5:cbh, kick:=acbh5;
cbv.1:cbv, kick:=acbv1;
cbv.2:cbv, kick:=acbv2;
cbv.3:cbv, kick:=acbv3;
cbv.4:cbv, kick:=acbv4;
cbv.5:cbv, kick:=acbv5;
!mscbh:sextupole, l:=1.1, k2:=ksf;
mscbh:multipole, knl:={0,0,0,ksf},tilt=-pi/8;
mscbv:sextupole, l:=1.1, k2:=ksd;
!mscbv:octupole, l:=1.1, k3:=ksd,tilt=-pi/8;

// sequence declaration

fivecell:sequence, refer=centre, l=534.6;
   qf.1:qf.1, at=1.550000e+00;
   qft:qft, at=3.815000e+00;
!   mscbh:mscbh, at=3.815000e+00;
   cbh.1:cbh.1, at=4.365000e+00;
   mb:mb, at=1.262000e+01;
   mb:mb, at=2.828000e+01;
   mb:mb, at=4.394000e+01;
   bpv:bpv, at=5.246000e+01;
   qd.1:qd.1, at=5.501000e+01;
   mscbv:mscbv, at=5.727500e+01;
   cbv.1:cbv.1, at=5.782500e+01;
   mb:mb, at=6.608000e+01;
   mb:mb, at=8.174000e+01;
   mb:mb, at=9.740000e+01;
   bph:bph, at=1.059200e+02;
   qf.2:qf.2, at=1.084700e+02;
   mscbh:mscbh, at=1.107350e+02;
   cbh.2:cbh.2, at=1.112850e+02;
   mb:mb, at=1.195400e+02;
   mb:mb, at=1.352000e+02;
   mb:mb, at=1.508600e+02;
   bpv:bpv, at=1.593800e+02;
   qd.2:qd.2, at=1.619300e+02;
   mscbv:mscbv, at=1.641950e+02;
   cbv.2:cbv.2, at=1.647450e+02;
   mb:mb, at=1.730000e+02;
   mb:mb, at=1.886600e+02;
   mb:mb, at=2.043200e+02;
   bph:bph, at=2.128400e+02;
   qf.3:qf.3, at=2.153900e+02;
   mscbh:mscbh, at=2.176550e+02;
   cbh.3:cbh.3, at=2.182050e+02;
   mb:mb, at=2.264600e+02;
   mb:mb, at=2.421200e+02;
   mb:mb, at=2.577800e+02;
   bpv:bpv, at=2.663000e+02;
   qd.3:qd.3, at=2.688500e+02;
   mscbv:mscbv, at=2.711150e+02;
   cbv.3:cbv.3, at=2.716650e+02;
   mb:mb, at=2.799200e+02;
   mb:mb, at=2.955800e+02;
   mb:mb, at=3.112400e+02;
   bph:bph, at=3.197600e+02;
   qf.4:qf.4, at=3.223100e+02;
   mscbh:mscbh, at=3.245750e+02;
   cbh.4:cbh.4, at=3.251250e+02;
   mb:mb, at=3.333800e+02;
   mb:mb, at=3.490400e+02;
   mb:mb, at=3.647000e+02;
   bpv:bpv, at=3.732200e+02;
   qd.4:qd.4, at=3.757700e+02;
   mscbv:mscbv, at=3.780350e+02;
   cbv.4:cbv.4, at=3.785850e+02;
   mb:mb, at=3.868400e+02;
   mb:mb, at=4.025000e+02;
   mb:mb, at=4.181600e+02;
   bph:bph, at=4.266800e+02;
   qf.5:qf.5, at=4.292300e+02;
   mscbh:mscbh, at=4.314950e+02;
   cbh.5:cbh.5, at=4.320450e+02;
   mb:mb, at=4.403000e+02;
   mb:mb, at=4.559600e+02;
   mb:mb, at=4.716200e+02;
   bpv:bpv, at=4.801400e+02;
   qd.5:qd.5, at=4.826900e+02;
   mscbv:mscbv, at=4.849550e+02;
   cbv.5:cbv.5, at=4.855050e+02;
   mb:mb, at=4.937600e+02;
   mb:mb, at=5.094200e+02;
   mb:mb, at=5.250800e+02;
   bph:bph, at=5.336000e+02;
end:marker, at=5.346000e+02;
endsequence;

// forces and other constants

l.bpm:=.3;
bang:=.509998807401e-2;
kqf:=.872651312e-2;
kqd:=-.872777242e-2;
ksf:=.0198492943;
ksd:=-.039621283;
acbv1:=1.e-4;
acbh1:=1.e-4;
!save,sequence=fivecell,file,mad8;

s := table(twiss,bpv[5],betx);
myvar := sqrt(beam->ex*table(twiss,betx));
use, period=fivecell;
select,flag=twiss,column=name,s,myvar,apertype;
twiss,file;
n = 0;
create,table=mytab,column=dp,mq1,mq2;
mq1:=table(summ,q1);
mq2:=table(summ,q2);
while ( n < 11)
{
  n = n + 1;
  dp = 1.e-4*(n-6);
  twiss,deltap=dp;
  fill,table=mytab;
}
write,table=mytab;
plot,haxis=s,vaxis=aper_1,aper_2,colour=100,range=#s/cbv.1,notitle;
stop;
\end{verbatim}
prints the following user table on output:

\begin{verbatim}
@ NAME             %05s "MYTAB"
@ TYPE             %04s "USER"
@ TITLE            %08s "no-title"
@ ORIGIN           %16s "MAD-X 1.09 Linux"
@ DATE             %08s "10/12/02"
@ TIME             %08s "10.45.25"
* DP                 MQ1                MQ2                
$ %le                %le                %le                
 -0.0005            1.242535951        1.270211135       
 -0.0004            1.242495534        1.270197018       
 -0.0003            1.242452432        1.270185673       
 -0.0002            1.242406653        1.270177093       
 -0.0001            1.242358206        1.270171269       
 0                  1.242307102        1.27016819        
 0.0001             1.242253353        1.270167843       
 0.0002             1.242196974        1.270170214       
 0.0003             1.24213798         1.270175288       
 0.0004             1.242076387        1.270183048       
 0.0005             1.242012214        1.270193477       
\end{verbatim}
and produces a twiss file with the additional column myvar, as well as a plot
file with the aperture values plotted.


\href{screate}{}

Example of joing 2 tables with different length into a third table
making use of the length of either table as given by
table("your\_table\_name", tablelength) and adding names by the "\_name"
attribute.

\begin{verbatim}
title,   "summing of offset and alignment tables";
set,    format="13.6f";

readtable, table=align,  file="align.ip2.b1.tfs";   // mesured alignment
readtable, table=offset, file="offset.ip2.b1.tfs";  // nominal offsets

n_elem  =  table(offset, tablelength);

create,  table=align_offset, column=_name,s_ip,x_off,dx_off,ddx_off,y_off,dy_off,ddy_off;

calcul(elem_name) : macro = {
    x_off = table(align,  elem_name, x_ali) + x_off;
    y_off = table(align,  elem_name, y_ali) + y_off;
}


one_elem(j_elem) : macro = {
    setvars, table=offset, row=j_elem;
    exec,  calcul(tabstring(offset, name, j_elem));
    fill,  table=align_offset;
}


i_elem = 0;
while (i_elem < n_elem) { i_elem = i_elem + 1; exec,  one_elem($i_elem); }

write, table=align_offset, file="align_offset.tfs";

stop;
\end{verbatim}

% \href{http://www.cern.ch/Hans.Grote/hansg_sign.html}{hansg}, May 8, 2001


\section{SELECT}
\label{sec:selection}
The elements, or a range of elements, in a sequence can be selected for
various purposes. Such selections remain valid until cleared (in
difference to \madeight); it is therefore recommended to always start with a  

\begin{verbatim}
select, flag =..., clear;
\end{verbatim} 
before setting a new selection. 
\begin{verbatim}
SELECT, FLAG=name, RANGE=range, CLASS=class, PATTERN=pattern [,FULL] [,CLEAR];
\end{verbatim} 
where the name for FLAG can be one of ERROR, MAKETHIN, SEQEDIT or the
name of a twiss table which is established for all sequence positions in
general.  

Selected elements have to fulfill the \href{ranges.html#range}{RANGE},
\href{ranges.html#class}{CLASS}, and \href{wildcard.html}{PATTERN}
criteria.  

Any number of SELECT commands can be issued for the same flag and are
accumulated (logically ORed). In this context note the following:  

\begin{verbatim}
SELECT, FLAG=name, FULL;
\end{verbatim} 
selects all positions in the sequence for this flag. This is the default
for all tables and makethin, whereas for ERROR and SEQEDIT the default
is "nothing selected".  

%\href{save_select}{}
%\label{save_select}
SAVE: A SELECT,FLAG=SAVE statement causes the
selected sequences, elements, and variables to be written into the save
file. A class (only used for element selection), and a pattern can be
specified. Example:  
\begin{verbatim}
select, flag=save, class=variable, pattern="abc.*";
save, file=mysave;
\end{verbatim} 
will save all variables (and sequences) containing "abc" in their name,
but not elements with names containing "abc" since the class "variable"
does not exist (astucieux, non ?).  

SECTORMAP: A SELECT,FLAG=SECTORMAP statement causes sectormaps to be
written into the file "sectormap" like in \madeight. For the file to be
written, a flag SECTORMAP must be issued on the TWISS command in
addition.  

TWISS: A SELECT,FLAG=TWISS statement causes the selected rows and
columns to be written into the Twiss TFS file (former OPTICS command in
\madeight). The column selection is done on the same select. See as well
example 2.  

Example 1:  
\begin{verbatim}
TITLE,'Test input for MAD-X';

option,rbarc=false; // use arc length of rbends
beam; ! sets the default beam for the following sequence
option,-echo;
call file=fv9.opt;  ! contains optics parameters
call file="fv9.seq"; ! contains a small sequence "fivecell"
OPTION,ECHO;
SELECT,FLAG=SECTORMAP,clear;
SELECT,FLAG=SECTORMAP,PATTERN="^m.*";
SELECT,FLAG=TWISS,clear;
SELECT,FLAG=TWISS,PATTERN="^m.*",column=name,s,betx,bety;
USE,PERIOD=FIVECELL;
twiss,file=optics,sectormap;
stop;
\end{verbatim} 

This produces a file \href{sectormap.html}{sectormap}, and a
twiss output file \label{tfs} (name = optics):  
\begin{verbatim}
@ TYPE             %05s "TWISS"
@ PARTICLE         %08s "POSITRON"
@ MASS             %le          0.000510998902
@ CHARGE           %le                       1
@ E0               %le                       1
@ PC               %le           0.99999986944
@ GAMMA            %le           1956.95136738
@ KBUNCH           %le                       1
@ NPART            %le                       0
@ EX               %le                       1
@ EY               %le                       1
@ ET               %le                       0
@ LENGTH           %le                   534.6
@ ALFA             %le        0.00044339992938
@ ORBIT5           %le                      -0
@ GAMMATR          %le           47.4900022541
@ Q1               %le           1.25413071556
@ Q2               %le           1.25485338377
@ DQ1              %le           1.05329608302
@ DQ2              %le           1.04837000224
@ DXMAX            %le           2.17763211131
@ DYMAX            %le                       0
@ XCOMAX           %le                       0
@ YCOMAX           %le                       0
@ BETXMAX          %le            177.70993499
@ BETYMAX          %le           177.671582415
@ XCORMS           %le                       0
@ YCORMS           %le                       0
@ DXRMS            %le           1.66004270906
@ DYRMS            %le                       0
@ DELTAP           %le                       0
@ TITLE            %20s "Test input for MAD-X"
@ ORIGIN           %16s "MAD-X 0.20 Linux"
@ DATE             %08s "07/06/02"
@ TIME             %08s "14.25.51"
* NAME               S                  BETX               BETY               
$ %s                 %le                %le                %le                
 "MSCBH"             4.365              171.6688159        33.31817319       
 "MB"                19.72              108.1309095        58.58680717       
 "MB"                35.38              61.96499987        102.9962313       
 "MB"                51.04              34.61640793        166.2227523       
 "MSCBV.1"           57.825             33.34442808        171.6309057       
 "MB"                73.18              58.61984637        108.0956006       
 "MB"                88.84              103.0313887        61.93159422       
 "MB"                104.5              166.2602486        34.58939635       
 "MSCBH"             111.285            171.6688159        33.31817319       
 "MB"                126.64             108.1309095        58.58680717       
 "MB"                142.3              61.96499987        102.9962313       
 "MB"                157.96             34.61640793        166.2227523       
 "MSCBV"             164.745            33.34442808        171.6309057       
 "MB"                180.1              58.61984637        108.0956006       
 "MB"                195.76             103.0313887        61.93159422       
 "MB"                211.42             166.2602486        34.58939635       
 "MSCBH"             218.205            171.6688159        33.31817319       
 "MB"                233.56             108.1309095        58.58680717       
 "MB"                249.22             61.96499987        102.9962313       
 "MB"                264.88             34.61640793        166.2227523       
 "MSCBV"             271.665            33.34442808        171.6309057       
 "MB"                287.02             58.61984637        108.0956006       
 "MB"                302.68             103.0313887        61.93159422       
 "MB"                318.34             166.2602486        34.58939635       
 "MSCBH"             325.125            171.6688159        33.31817319       
 "MB"                340.48             108.1309095        58.58680717       
 "MB"                356.14             61.96499987        102.9962313       
 "MB"                371.8              34.61640793        166.2227523       
 "MSCBV"             378.585            33.34442808        171.6309057       
 "MB"                393.94             58.61984637        108.0956006       
 "MB"                409.6              103.0313887        61.93159422       
 "MB"                425.26             166.2602486        34.58939635       
 "MSCBH"             432.045            171.6688159        33.31817319       
 "MB"                447.4              108.1309095        58.58680717       
 "MB"                463.06             61.96499987        102.9962313       
 "MB"                478.72             34.61640793        166.2227523       
 "MSCBV"             485.505            33.34442808        171.6309057       
 "MB"                500.86             58.61984637        108.0956006       
 "MB"                516.52             103.0313887        61.93159422       
 "MB"                532.18             166.2602486        34.58939635       
\end{verbatim}

 Example 2: 

 Addition of variables to (any internal) table: 
\begin{verbatim}
select, flag=table, column=name, s, betx, ..., var1, var2, ...; ! or
select, flag=table, full, column=var1, var2, ...; ! default col.s + new
\end{verbatim} 
will write the current value of var1 etc. into the table each time a new
line is added; values from the same (current) line can be accessed by
these variables, e.g.  
\begin{verbatim}
var1 := sqrt(beam->ex*table(twiss,betx));
\end{verbatim} 
in the case of table above being "twiss". The plot command accepts the
new variables.  

Remark: this replaces the "string" variables of \madeight. 

 This example demonstrates as well the usage of a user defined table \label{ucreate}. 
\begin{verbatim}
beam,ex=1.e-6,ey=1.e-3;
// element definitions
mb:rbend, l=14.2, angle:=0,k0:=bang/14.2;
mq:quadrupole, l:=3.1,apertype=ellipse,aperture={1,2};
qft:mq, l:=0.31, k1:=kqf,tilt=-pi/4;
qf.1:mq, l:=3.1, k1:=kqf;
qf.2:mq, l:=3.1, k1:=kqf;
qf.3:mq, l:=3.1, k1:=kqf;
qf.4:mq, l:=3.1, k1:=kqf;
qf.5:mq, l:=3.1, k1:=kqf;
qd.1:mq, l:=3.1, k1:=kqd;
qd.2:mq, l:=3.1, k1:=kqd;
qd.3:mq, l:=3.1, k1:=kqd;
qd.4:mq, l:=3.1, k1:=kqd;
qd.5:mq, l:=3.1, k1:=kqd;
bph:hmonitor, l:=l.bpm;
bpv:vmonitor, l:=l.bpm;
cbh:hkicker;
cbv:vkicker;
cbh.1:cbh, kick:=acbh1;
cbh.2:cbh, kick:=acbh2;
cbh.3:cbh, kick:=acbh3;
cbh.4:cbh, kick:=acbh4;
cbh.5:cbh, kick:=acbh5;
cbv.1:cbv, kick:=acbv1;
cbv.2:cbv, kick:=acbv2;
cbv.3:cbv, kick:=acbv3;
cbv.4:cbv, kick:=acbv4;
cbv.5:cbv, kick:=acbv5;
!mscbh:sextupole, l:=1.1, k2:=ksf;
mscbh:multipole, knl:={0,0,0,ksf},tilt=-pi/8;
mscbv:sextupole, l:=1.1, k2:=ksd;
!mscbv:octupole, l:=1.1, k3:=ksd,tilt=-pi/8;

// sequence declaration

fivecell:sequence, refer=centre, l=534.6;
   qf.1:qf.1, at=1.550000e+00;
   qft:qft, at=3.815000e+00;
!   mscbh:mscbh, at=3.815000e+00;
   cbh.1:cbh.1, at=4.365000e+00;
   mb:mb, at=1.262000e+01;
   mb:mb, at=2.828000e+01;
   mb:mb, at=4.394000e+01;
   bpv:bpv, at=5.246000e+01;
   qd.1:qd.1, at=5.501000e+01;
   mscbv:mscbv, at=5.727500e+01;
   cbv.1:cbv.1, at=5.782500e+01;
   mb:mb, at=6.608000e+01;
   mb:mb, at=8.174000e+01;
   mb:mb, at=9.740000e+01;
   bph:bph, at=1.059200e+02;
   qf.2:qf.2, at=1.084700e+02;
   mscbh:mscbh, at=1.107350e+02;
   cbh.2:cbh.2, at=1.112850e+02;
   mb:mb, at=1.195400e+02;
   mb:mb, at=1.352000e+02;
   mb:mb, at=1.508600e+02;
   bpv:bpv, at=1.593800e+02;
   qd.2:qd.2, at=1.619300e+02;
   mscbv:mscbv, at=1.641950e+02;
   cbv.2:cbv.2, at=1.647450e+02;
   mb:mb, at=1.730000e+02;
   mb:mb, at=1.886600e+02;
   mb:mb, at=2.043200e+02;
   bph:bph, at=2.128400e+02;
   qf.3:qf.3, at=2.153900e+02;
   mscbh:mscbh, at=2.176550e+02;
   cbh.3:cbh.3, at=2.182050e+02;
   mb:mb, at=2.264600e+02;
   mb:mb, at=2.421200e+02;
   mb:mb, at=2.577800e+02;
   bpv:bpv, at=2.663000e+02;
   qd.3:qd.3, at=2.688500e+02;
   mscbv:mscbv, at=2.711150e+02;
   cbv.3:cbv.3, at=2.716650e+02;
   mb:mb, at=2.799200e+02;
   mb:mb, at=2.955800e+02;
   mb:mb, at=3.112400e+02;
   bph:bph, at=3.197600e+02;
   qf.4:qf.4, at=3.223100e+02;
   mscbh:mscbh, at=3.245750e+02;
   cbh.4:cbh.4, at=3.251250e+02;
   mb:mb, at=3.333800e+02;
   mb:mb, at=3.490400e+02;
   mb:mb, at=3.647000e+02;
   bpv:bpv, at=3.732200e+02;
   qd.4:qd.4, at=3.757700e+02;
   mscbv:mscbv, at=3.780350e+02;
   cbv.4:cbv.4, at=3.785850e+02;
   mb:mb, at=3.868400e+02;
   mb:mb, at=4.025000e+02;
   mb:mb, at=4.181600e+02;
   bph:bph, at=4.266800e+02;
   qf.5:qf.5, at=4.292300e+02;
   mscbh:mscbh, at=4.314950e+02;
   cbh.5:cbh.5, at=4.320450e+02;
   mb:mb, at=4.403000e+02;
   mb:mb, at=4.559600e+02;
   mb:mb, at=4.716200e+02;
   bpv:bpv, at=4.801400e+02;
   qd.5:qd.5, at=4.826900e+02;
   mscbv:mscbv, at=4.849550e+02;
   cbv.5:cbv.5, at=4.855050e+02;
   mb:mb, at=4.937600e+02;
   mb:mb, at=5.094200e+02;
   mb:mb, at=5.250800e+02;
   bph:bph, at=5.336000e+02;
end:marker, at=5.346000e+02;
endsequence;

// forces and other constants

l.bpm:=.3;
bang:=.509998807401e-2;
kqf:=.872651312e-2;
kqd:=-.872777242e-2;
ksf:=.0198492943;
ksd:=-.039621283;
acbv1:=1.e-4;
acbh1:=1.e-4;
!save,sequence=fivecell,file,mad8;

s := table(twiss,bpv[5],betx);
myvar := sqrt(beam->ex*table(twiss,betx));
use, period=fivecell;
select,flag=twiss,column=name,s,myvar,apertype;
twiss,file;
n = 0;
create,table=mytab,column=dp,mq1,mq2;
mq1:=table(summ,q1);
mq2:=table(summ,q2);
while ( n < 11)
{
  n = n + 1;
  dp = 1.e-4*(n-6);
  twiss,deltap=dp;
  fill,table=mytab;
}
write,table=mytab;
plot,haxis=s,vaxis=aper_1,aper_2,colour=100,range=#s/cbv.1,notitle;
stop;
\end{verbatim}
prints the following user table on output:

\begin{verbatim}
@ NAME             %05s "MYTAB"
@ TYPE             %04s "USER"
@ TITLE            %08s "no-title"
@ ORIGIN           %16s "MAD-X 1.09 Linux"
@ DATE             %08s "10/12/02"
@ TIME             %08s "10.45.25"
* DP                 MQ1                MQ2                
$ %le                %le                %le                
 -0.0005            1.242535951        1.270211135       
 -0.0004            1.242495534        1.270197018       
 -0.0003            1.242452432        1.270185673       
 -0.0002            1.242406653        1.270177093       
 -0.0001            1.242358206        1.270171269       
 0                  1.242307102        1.27016819        
 0.0001             1.242253353        1.270167843       
 0.0002             1.242196974        1.270170214       
 0.0003             1.24213798         1.270175288       
 0.0004             1.242076387        1.270183048       
 0.0005             1.242012214        1.270193477       
\end{verbatim}
and produces a twiss file with the additional column myvar, as well as a plot
file with the aperture values plotted.


%\href{screate}{}

Example of joining two tables with different length into a third table
making use of the length of either table as given by
table("your\_table\_name", tablelength) and adding names by the "\_name"
attribute.

\begin{verbatim}
title,   "summing of offset and alignment tables";
set,    format="13.6f";

readtable, table=align,  file="align.ip2.b1.tfs";   // mesured alignment
readtable, table=offset, file="offset.ip2.b1.tfs";  // nominal offsets

n_elem  =  table(offset, tablelength);

create,  table=align_offset, column=_name,s_ip,x_off,dx_off,ddx_off,y_off,dy_off,ddy_off;

calcul(elem_name) : macro = {
    x_off = table(align,  elem_name, x_ali) + x_off;
    y_off = table(align,  elem_name, y_ali) + y_off;
}


one_elem(j_elem) : macro = {
    setvars, table=offset, row=j_elem;
    exec,  calcul(tabstring(offset, name, j_elem));
    fill,  table=align_offset;
}


i_elem = 0;
while (i_elem < n_elem) { i_elem = i_elem + 1; exec,  one_elem($i_elem); }

write, table=align_offset, file="align_offset.tfs";

stop;
\end{verbatim}


%% EOF



%%%%\title{Range Selection}
%  Changed by: Chris ISELIN, 27-Jan-1997 

%  Changed by: Hans Grote, 10-Jun-2002 

%%%\usepackage{hyperref}
% commands generated by html2latex


%%%\begin{document}
%%%\begin{center}
 %%%EUROPEAN ORGANIZATION FOR NUCLEAR RESEARCH 
%%%\includegraphics{http://cern.ch/madx/icons/mx7_25.gif}

\paragraph{Real life example for IF statements, and MACRO usage}
%%%\end{center}


\begin{verbatim}

! Creates a footprint for head-on + parasitic collisions at IP1+5 
! of lhc.6.5; both lhcb1 (for tracking) and lhcb2 (to define the
! beam-beam elements, i.e. weak-strong) are used; there are flags to
! select head-on, left, and right parasitic separately at all IPs.
! The bunch spacing can be given in nanosec and automatically creates
! the beam-beam interaction points at the correct positions.
! It is important to set the correct BEAM parameters, i.e. number
! of particles, emittances, bunch length, energy.

!--- For completeness, all files needed by this job are copied
!    to the local directory ldb. The links to the the originals
!    in offdb (official database) are commented out.

Option,  warn,info,echo;
!System,
"ln -fns /afs/cern.ch/eng/sl/MAD-X/dev/test_suite/foot/V3.01.01 ldb";
!system,"ln -fns /afs/cern.ch/eng/lhc/optics/V6.4 offdb";
Option, -echo,-info,warn;
SU=1.0;
call, file = "ldb/V6.5.seq";
call,file="ldb/slice_new.madx";
Option, echo,info,warn;

!+++++++++++++++++++++++++ Step 1 +++++++++++++++++++++++
! 	define beam constants
!++++++++++++++++++++++++++++++++++++++++++++++++++++++++

b_t_dist = 25.e-9;                  !--- bunch distance in [sec]
b_h_dist = clight * b_t_dist / 2 ;  !--- bunch half-distance in [m]
ip1_range = 58.;                     ! range for parasitic collisions
ip5_range = ip1_range;
ip2_range = 60.;
ip8_range = ip2_range;

npara_1 = ip1_range / b_h_dist;     ! # parasitic either side
npara_2 = ip2_range / b_h_dist;
npara_5 = ip5_range / b_h_dist;
npara_8 = ip8_range / b_h_dist;

value,npara_1,npara_2,npara_5,npara_8;

 eg   =  7000;
 bg   =  eg/pmass;
 en   = 3.75e-06;
 epsx = en/bg;
 epsy = en/bg;

Beam, particle = proton, sequence=lhcb1, energy = eg,
          sigt=      0.077     , 
          bv = +1, NPART=1.1E11, sige=      1.1e-4, 
          ex=epsx,   ey=epsy;

Beam, particle = proton, sequence=lhcb2, energy = eg,
          sigt=      0.077     , 
          bv = -1, NPART=1.1E11, sige=      1.1e-4, 
          ex=epsx,   ey=epsy;

beamx = beam%lhcb1->ex;   beamy%lhcb1 = beam->ey;
sigz  = beam%lhcb1->sigt; sige = beam%lhcb1->sige;

!--- split5, 4d
long_a= 0.53 * sigz/2;
long_b= 1.40 * sigz/2;
value,long_a,long_b;

ho_charge = 0.2;

!+++++++++++++++++++++++++ Step 2 +++++++++++++++++++++++
! 	slice, flatten sequence, and cycle start to ip3
!++++++++++++++++++++++++++++++++++++++++++++++++++++++++

use,sequence=lhcb1;
makethin,sequence=lhcb1;
!save,sequence=lhcb1,file=lhcb1_thin_new_seq;
use,sequence=lhcb2;
makethin,sequence=lhcb2;
!save,sequence=lhcb2,file=lhcb2_thin_new_seq;
!stop;

option,-warn,-echo,-info;
call,file="ldb/V6.5.thin.coll.str";
option,warn,echo,info;

! keep sextupoles
ksf0=ksf; ksd0=ksd;
use,period=lhcb1;
select,flag=twiss.1,column=name,x,y,betx,bety;
twiss,file;
plot,haxis=s,vaxis=x,y,colour=100,noline;

use,period=lhcb2;
select,flag=twiss.2,column=name,x,y,betx,bety;
twiss,file;
plot,haxis=s,vaxis=x,y,colour=100,noline;
seqedit,sequence=lhcb1;
flatten;
endedit;

seqedit,sequence=lhcb1;
cycle,start=ip3.b1;
endedit;

seqedit,sequence=lhcb2;
flatten;
endedit;

seqedit,sequence=lhcb2;
cycle,start=ip3.b2;
endedit;

bbmarker: marker;  /* for subsequent remove */


!+++++++++++++++++++++++++ Step 3 +++++++++++++++++++++++
! 	define the beam-beam elements
!++++++++++++++++++++++++++++++++++++++++++++++++++++++++
!
!===========================================================
! read macro definitions
option,-echo;
call,file="ldb/bb.macros";
option,echo;

!
!===========================================================
!   this sets CHARGE in the head-on beam-beam elements. 
!   set +1 * ho_charge   for parasitic on, 0 for off

 on_ho1  = +1 * ho_charge; ! ho_charge depends on split
 on_ho2  = +0 * ho_charge; ! because of the "by hand" splitting
 on_ho5  = +1 * ho_charge;
 on_ho8  = +0 * ho_charge;

!
!===========================================================
!   set CHARGE in the parasitic beam-beam elements. 
!   set +1 for parasitic on, 0 for off
 on_lr1l = +1;
 on_lr1r = +1;
 on_lr2l = +0;
 on_lr2r = +0;
 on_lr5l = +1;
 on_lr5r = +1;
 on_lr8l = +0;
 on_lr8r = +0;

!
!===========================================================
!   define markers and savebetas
assign,echo=temp.bb.install;
!--- ip1
if (on_ho1  0)
{
  exec, mkho(1);
  exec, sbhomk(1);
}
if (on_lr1l  0 || on_lr1r  0)
{
  n=1; ! counter
  while (n  0 || on_lr1l  0)
{
  n=1; ! counter
  while (n  0)
{
  exec, mkho(5);
  exec, sbhomk(5);
}
if (on_lr5l  0 || on_lr5r  0)
{
  n=1; ! counter
  while (n  0 || on_lr5l  0)
{
  n=1; ! counter
  while (n  0)
{
  exec, mkho(2);
  exec, sbhomk(2);
}
if (on_lr2l  0 || on_lr2r  0)
{
  n=1; ! counter
  while (n  0 || on_lr2l  0)
{
  n=1; ! counter
  while (n  0)
{
  exec, mkho(8);
  exec, sbhomk(8);
}
if (on_lr8l  0 || on_lr8r  0)
{
  n=1; ! counter
  while (n  0 || on_lr8l  0)
{
  n=1; ! counter
  while (n  0)
{
exec, inho(mk,1);
}
if (on_lr1l  0 || on_lr1r  0)
{
  n=1; ! counter
  while (n  0 || on_lr1l  0)
{
  n=1; ! counter
  while (n  0)
{
exec, inho(mk,5);
}
if (on_lr5l  0 || on_lr5r  0)
{
  n=1; ! counter
  while (n  0 || on_lr5l  0)
{
  n=1; ! counter
  while (n  0)
{
exec, inho(mk,2);
}
if (on_lr2l  0 || on_lr2r  0)
{
  n=1; ! counter
  while (n  0 || on_lr2l  0)
{
  n=1; ! counter
  while (n  0)
{
exec, inho(mk,8);
}
if (on_lr8l  0 || on_lr8r  0)
{
  n=1; ! counter
  while (n  0 || on_lr8l  0)
{
  n=1; ! counter
  while (n betx) / 0.0007999979093;
value,on_sep2;
!===========================================================
!   define bb elements
assign,echo=temp.bb.install;
!--- ip1
if (on_ho1  0)
{
exec, bbho(1);
}
if (on_lr1l  0)
{
  n=1; ! counter
  while (n  0)
{
  n=1; ! counter
  while (n  0)
{
exec, bbho(5);
}
if (on_lr5l  0)
{
  n=1; ! counter
  while (n  0)
{
  n=1; ! counter
  while (n  0)
{
exec, bbho(2);
}
if (on_lr2l  0)
{
  n=1; ! counter
  while (n  0)
{
  n=1; ! counter
  while (n  0)
{
exec, bbho(8);
}
if (on_lr8l  0)
{
  n=1; ! counter
  while (n  0)
{
  n=1; ! counter
  while (n  0)
{
exec, inho(bb,1);
}
if (on_lr1l  0)
{
  n=1; ! counter
  while (n  0)
{
  n=1; ! counter
  while (n  0)
{
exec, inho(bb,5);
}
if (on_lr5l  0)
{
  n=1; ! counter
  while (n  0)
{
  n=1; ! counter
  while (n  0)
{
exec, inho(bb,2);
}
if (on_lr2l  0)
{
  n=1; ! counter
  while (n  0)
{
  n=1; ! counter
  while (n  0)
{
exec, inho(bb,8);
}
if (on_lr8l  0)
{
  n=1; ! counter
  while (n  0)
{
  n=1; ! counter
  while (n  footprint";
stop;
\end{verbatim}

\paragraph{\href{macro}{Real life example of MACRO definitions}}

\begin{verbatim}

bbho(nn): macro = {
!--- macro defining head-on beam-beam elements; nn = IP number
print, text="bbipnnl2: beambeam, sigx=sqrt(rnnipnnl2->betx*epsx),";
print, text="          sigy=sqrt(rnnipnnl2->bety*epsy),";
print, text="          xma=rnnipnnl2->x,yma=rnnipnnl2->y,";
print, text="          charge:=on_honn;";
print, text="bbipnnl1: beambeam, sigx=sqrt(rnnipnnl1->betx*epsx),";
print, text="          sigy=sqrt(rnnipnnl1->bety*epsy),";
print, text="          xma=rnnipnnl1->x,yma=rnnipnnl1->y,";
print, text="          charge:=on_honn;";
print, text="bbipnn:   beambeam, sigx=sqrt(rnnipnn->betx*epsx),";
print, text="          sigy=sqrt(rnnipnn->bety*epsy),";
print, text="          xma=rnnipnn->x,yma=rnnipnn->y,";
print, text="          charge:=on_honn;";
print, text="bbipnnr1: beambeam, sigx=sqrt(rnnipnnr1->betx*epsx),";
print, text="          sigy=sqrt(rnnipnnr1->bety*epsy),";
print, text="          xma=rnnipnnr1->x,yma=rnnipnnr1->y,";
print, text="          charge:=on_honn;";
print, text="bbipnnr2: beambeam, sigx=sqrt(rnnipnnr2->betx*epsx),";
print, text="          sigy=sqrt(rnnipnnr2->bety*epsy),";
print, text="          xma=rnnipnnr2->x,yma=rnnipnnr2->y,";
print, text="          charge:=on_honn;";
};

mkho(nn): macro = {
!--- macro defining head-on markers; nn = IP number
print, text="mkipnnl2: bbmarker;";
print, text="mkipnnl1: bbmarker;";
print, text="mkipnn:   bbmarker;";
print, text="mkipnnr1: bbmarker;";
print, text="mkipnnr2: bbmarker;";
};

inho(xx,nn): macro = {
!--- macro installing bb or markers for head-on beam-beam (split into 5)
print, text="install, element= xxipnnl2, at=-long_b, from=ipnn;";
print, text="install, element= xxipnnl1, at=-long_a, from=ipnn;";
print, text="install, element= xxipnn,   at=1.e-9,   from=ipnn;";
print, text="install, element= xxipnnr1, at=+long_a, from=ipnn;"; 
print, text="install, element= xxipnnr2, at=+long_b, from=ipnn;"; 
};

sbhomk(nn): macro = {
!--- macro to create savebetas for ho markers
print, text="savebeta, label=rnnipnnl2, place=mkipnnl2;";
print, text="savebeta, label=rnnipnnl1, place=mkipnnl1;";
print, text="savebeta, label=rnnipnn,   place=mkipnn;";
print, text="savebeta, label=rnnipnnr1, place=mkipnnr1;";
print, text="savebeta, label=rnnipnnr2, place=mkipnnr2;";    
};

mkl(nn,cc): macro = {
!--- macro to create parasitic bb marker on left side of ip nn; cc = count
print, text="mkipnnplcc: bbmarker;";
};

mkr(nn,cc): macro = {
!--- macro to create parasitic bb marker on right side of ip nn; cc = count
print, text="mkipnnprcc: bbmarker;";
};

sbl(nn,cc): macro = {
!--- macro to create savebetas for left parasitic
print, text="savebeta, label=rnnipnnplcc, place=mkipnnplcc;";
};

sbr(nn,cc): macro = {
!--- macro to create savebetas for right parasitic
print, text="savebeta, label=rnnipnnprcc, place=mkipnnprcc;";
};

inl(xx,nn,cc): macro = {
!--- macro installing bb and markers for left side parasitic beam-beam
print, text="install, element= xxipnnplcc, at=-cc*b_h_dist, from=ipnn;";
};

inr(xx,nn,cc): macro = {
!--- macro installing bb and markers for right side parasitic beam-beam
print, text="install, element= xxipnnprcc, at=cc*b_h_dist, from=ipnn;";
};

bbl(nn,cc): macro = {
!--- macro defining parasitic beam-beam elements; nn = IP number
print, text="bbipnnplcc: beambeam, sigx=sqrt(rnnipnnplcc->betx*epsx),";
print, text="          sigy=sqrt(rnnipnnplcc->bety*epsy),";
print, text="          xma=rnnipnnplcc->x,yma=rnnipnnplcc->y,";
print, text="          charge:=on_lrnnl;";
};

bbr(nn,cc): macro = {
!--- macro defining parasitic beam-beam elements; nn = IP number
print, text="bbipnnprcc: beambeam, sigx=sqrt(rnnipnnprcc->betx*epsx),";
print, text="          sigy=sqrt(rnnipnnprcc->bety*epsy),";
print, text="          xma=rnnipnnprcc->x,yma=rnnipnnprcc->y,";
print, text="          charge:=on_lrnnr;";
};
\end{verbatim}\href{http://www.cern.ch/Hans.Grote/hansg_sign.html}{hansg}, June 17, 2002 

%%%\end{document}

%%%%\title{Range Selection}
%  Changed by: Chris ISELIN, 27-Jan-1997 
%  Changed by: Hans Grote, 16-Jan-2003 

\section{General Control Statements}

\subsection{ASSIGN}
\begin{verbatim}

assign, echo = "file_name", truncate;
\end{verbatim} 
where "file\_name"  is the name of an output file, or "terminal" and
trunctate specifies if the file must be trunctated when opened (ignored
for terminal). This allows switching the echo stream to a file or back,
but only for the commands value, show, and print. Allows easy
composition of future MAD-X input files. A real life example (always the
same) is to be found under \href{foot.html}{footprint example}.  

\subsection{CALL}
\begin{verbatim}

call, file = file_name;
\end{verbatim} 
where "file\_name"  is the name of an input file. This file will be read
until a "return;" statement, or until end\_of\_file; it may contain any
number of calls itself, and so on to any depth.  


\subsection{COGUESS}
\begin{verbatim}

coguess,tolerance=double,x=double,
       px=double,y=double,py=double,t=double,pt=double;
\end{verbatim} 
sets the required convergence precision in the closed orbit search
("tolerance", see as well Twiss command
\href{../twiss/twiss.html#tolerance}{tolerance}).  

The other parameters define a first guess for all future closed orbit
searches in case they are different from zero.  


\subsection{CREATE}
\begin{verbatim}

create,table=table,column=var1,var2,_name,...;
\end{verbatim} 
creates a table with the specified variables as columns. This table can
then be \hyperlink{fill}{fill}ed, and finally one can
\hyperlink{write}{write} it in TFS format. The attribute "\_name" adds
the element name to the table at the specified column, this replaces the
undocumented "withname" attribute that was not always working properly.  

See the \href{../Introduction/select.html#ucreate}{user table I}
example;  

or an example of joining 2 tables of different length in one table
including the element name:
\href{../Introduction/select.html#screate}{user table II} 




\subsection{DELETE}
\begin{verbatim}

delete,sequence=s_name,table=t_name;
\end{verbatim} 
deletes a sequence with name "s\_name" or a table with name "t\_name"
from memory. The sequence deletion is done without influence on other
sequences that may have elements that were in the deleted sequence.  


\subsection{DUMPSEQU}
\begin{verbatim}

dumpsequ, sequence = s_name, level = integer;
\end{verbatim} 
Actually a debug statement, but it may come handy at certain
occasions. Here "s\_name" is the name of an expanded (i.e. USEd)
sequence. The amount of detail is controlled by "level":  
\begin{verbatim}

level = 0:    print only the cumulative node length = sequence length
      > 0:    print all node (element) names except drifts
      > 2:    print all nodes with their attached parameters
      > 3:    print all nodes, and their elements with all parameters
\end{verbatim}


\subsection{EXEC}
\begin{verbatim}

exec, label;
\end{verbatim} 
Each statement may be preceded by a label; it is then stored and can be
executed again with "exec, label;" any number of times; the executed
statement may be another "exec", etc.; however, the major usage of this
statement is the execution of a \href{special.html#macro}{macro}.  


\subsection{EXIT}
\begin{verbatim}

exit;
\end{verbatim} 
ends the program execution. 


\subsection{FILL} 
Every command 
\begin{verbatim}

fill,table=table;
\end{verbatim} 
adds a new line with the current values of all column variables into the
user table \hyperlink{create}{create}d beforehand. This table one can
then \hyperlink{write}{write} in TFS format.  See as well the
\href{../Introduction/select.html#ucreate}{user table} example.  


\subsection{OPTION}
\begin{verbatim}

option, flag { = true | false };
option, flag | -flag;
\end{verbatim} 
sets an option as given in "flag"; the part in curly brackets is
optional: if only the name of the option is given, then the option will
be set true (see second line); a "-" sign preceding the name sets it to
"false".  

 Example: 
\begin{verbatim}

option,echo=true;
option,echo;
\end{verbatim} 
are identical, ditto 
\begin{verbatim}

option,echo=false;
option,-echo;
\end{verbatim} 
The available options are: 
\begin{verbatim}

  name           default meaning if true
  ====           ======= ===============
  bborbit         false  the closed orbit is modified by beam-beam kicks
  sympl           false  all element matrices are symplectified in Twiss
  echo            true   echoes the input on the standard output file
  trace           false  prints the system time after each command
  verify          false  issues a warning if an undefined variable is used
  warn            true   issues warnings
  info            true   issues informations
  tell            false  prints the current value of all options
  reset           false  resets all options to their defaults
  rbarc           true   converts the RBEND straight length into the arc length
  thin_foc        true   if false suppresses the 1(rho**2) focusing of thin dipoles
  no_fatal_stop   false  Prevents madx from stopping in case of a fatal error. Use at your own risk.
\end{verbatim} 
The option "rbarc" is implemented for backwards compatibility with MAD-8
up to version 8.23.06 included; in this version, the RBEND length was
just taken as the arc length of an SBEND with inclined pole faces,
contrary to the MAD-8 manual.  


\subsection{PRINT}
\begin{verbatim}

print,text="...";
\end{verbatim} 
prints the text to the current output file (see ASSIGN above). The text
can be edited with the help of a  \href{special.html#macro}{macro
  statement}. For more details, see there.  


\subsection{QUIT}
\begin{verbatim}

quit;
\end{verbatim} 
ends the program execution. 


\subsection{READTABLE}
\begin{verbatim}

readtable,file=filename;
\end{verbatim} 
reads a TFS file containing a MAD-X table back into memory. This table
can then be manipulated as any other table, i.e. its values can be
accessed, it can be plotted, written out again etc.  


\subsection{READMYTABLE}
\begin{verbatim}

readmytable,file=filename,table=internalname;
\end{verbatim} 
reads a TFS file containing a MAD-X table back into memory. This table
can then be manipulated as any other table, i.e. its values can be
accessed, it can be plotted, written out again etc. An internal name for
the table can be freely assigned while for the command READTABLE it is
taken from the information section of the table itself. This feature
allows to store multiple tables of the same type in memory without
overwriting existing ones.  


\subsection{REMOVEFILE}
\begin{verbatim}

removefile,file=filename;
\end{verbatim} 
remove a file from the disk. It is more portable than  
\begin{verbatim}

system("rm filename"); // Unix specific
\end{verbatim}


\subsection{RENAMEFILE}
\begin{verbatim}

renamefile,file=filename,name=newfilename;
\end{verbatim} 
rename the file "filename" to "newfilename" on the disk. It is more
portable than  
\begin{verbatim}

system("mv filename newfilename"); // Unix specific
\end{verbatim}


\subsection{RESBEAM}
\begin{verbatim}

resbeam,sequence=s_name;
\end{verbatim} 
resets the default values of the beam belonging to sequence s\_name, or
of the default beam if no sequence is given.  


\subsection{RETURN}
\begin{verbatim}

return;
\end{verbatim} 
ends reading from a "called" file; if encountered in the standard input
file, it ends the program execution.  


\subsection{SAVE}
\begin{verbatim}

save,beam,sequence=sequ1,sequ2,...,file=filename,beam,bare;
\end{verbatim} 
saves the sequence(s) specified with all variables and elements needed
for their expansion, onto the file "filename". If quotes are used for
the "filename" capital and low characters are kept as specified, if they
are omitted the "filename" will have lower characters only. The optional
flag can have the value "mad8" (without the quotes), in which case the
sequence(s) is/are saved in MAD-8 input format.  

The flag "beam" is optional; when given, all beams belonging to the
sequences specified are saved at the top of the save file.  

The parameter "sequence" is optional; when omitted, all sequences are
saved.  

However, it is not advisable to use "save" without the "sequence" option
unless you know what you are doing. This practice will avoid spurious
saved entries.    Any number of "select,flag=save" commands may precede
the SAVE command. In that case, the names of elements, variables, and
sequences must match the pattern(s) if given, and in addition the
elements must be of the class(es) specified. See here for a
\href{../Introduction/select.html#save_select}{SAVE with SELECT}
example.  

It is important to note that the precision of the output of the save
command depends on the output precision. Details about default
precisions and how to adjust those precisions can be found at the
\href{../Introduction/set.html#Format}{SET Format} instruction page.   
 
The Attribute 'bare' allows to save just the sequence without the
element definitions nor beam information. This allows to re-read in a
sequence with might otherwise create a stop of the program. This is
particularly useful to turn a line into a sequence to seqedit
it. Example:  
\begin{verbatim}

tl3:line=(ldl6,qtl301,mqn,qtl301,ldl7,qtl302,mqn,qtl302,ldl8,ison);
DLTL3 : LINE=(delay, tl3);
use, period=dltl3;

save,sequence=dltl3,file=t1,bare; // new parameter "bare": only sequ. saved
call,file=t1; // sequence is read in and is now a "real" sequence
// if the two preceding lines are suppressed, seqedit will print a warning
// and else do nothing
use, period=dltl3;
twiss, save, betx=bxa, alfx=alfxa, bety=bya, alfy=alfya;
plot, vaxis=betx, bety, haxis=s, colour:=100;
SEQEDIT, SEQUENCE=dltl3;
  remove,element=cx.bhe0330;
  remove,element=cd.bhe0330;
ENDEDIT;

use, period=dltl3;
twiss, save, betx=bxa, alfx=alfxa, bety=bya, alfy=alfya;
\end{verbatim}


\subsection{SAVEBETA}
\begin{verbatim}

savebeta, label=label,\href{place}{place}=place,sequence=s_name;
\end{verbatim} 
marks a place "place" in an expanded sequence "s\_name"; at the next
TWISS command execution, a  \href{../twiss/twiss.html#beta0}{beta0}
block will be saved at that place with the label "label". This is done
only once; in order to get a new beta0 block there, one has to re-issue
the command. The contents of the beta0 block can then be used in other
commands, e.g. TWISS and MATCH.  

 Example (after sequence expansion): 
\begin{verbatim}

savebeta,label=sb1,place=mb[5],sequence=fivecell;
twiss;
show,sb1;
\end{verbatim} 
will save and show the beta0 block parameters at the end (!) of the
fifth element mb in the sequence.  


\subsection{SELECT} %select</a}{SELECT}
\begin{verbatim}

select, flag=flag,range=range,class=class,pattern=pattern,
        slice=integer,column=s1,s2,s3,..,sn,sequence=s_name,
        full,clear;
\end{verbatim} 
selects one or several elements for special treatment in a subsequent
command. All selections for a given command remain valid until "clear"
is specified; the selection criteria on the same command are logically
ANDed, on different SELECT statements logically ORed.  

 Example: 
\begin{verbatim}

select,flag=error,class=quadrupole,range=mb[1]/mb[5];
select,flag=error,pattern="^mqw.*";
\end{verbatim} 
selects all quadrupoles in the range mb[1] to mb[5], and all elements
(in the whole sequence) the name of which starts with "mqw" for
treatment by the error module.  

"flag" can be one of the following:: 
\begin{itemize}
	\item seqedit: selection of elements for the
          \href{seqedit.html}{seqedit} module.  
	\item error: selection of elements for the
          \href{../error/error.html}{error} assignment module.  
	\item makethin: selection of elements for the
          \href{../makethin/makethin.html}{makethin} module that
          converts the sequence into one with thin elements only.  
	\item sectormap: selection of elements for the
          \href{../Introduction/sectormap.html}{sectormap} output file
          from the Twiss module.  
	\item table: here "table" is a table name such as twiss, track
          etc., and the rows and columns to be written are selected.  
\end{itemize} For the RANGE, CLASS, PATTERN, FULL, and CLEAR parameters
see \href{../Introduction/select.html}{SELECT}.  

"slice" is only used by \href{../makethin/makethin.html}{makethin} and
prescribes the number of slices into which the selected elements have to
be cut (default = 1).  

"column" is only valid for tables and decides the selection of columns
to be written into the TFS file. The "name" argument is special in that
it refers to the actual name of the selected element. For an example,
see \href{../Introduction/select.html}{SELECT}.  


\subsection{SHOW}
\begin{verbatim}

show,command;
\end{verbatim} 
prints the "command" (typically "beam", "beam\%sequ", or an element
name), with the actual value of all its parameters.  


\subsection{STOP}
\begin{verbatim}

stop;
\end{verbatim} 
ends the program execution. 


\subsection{SYSTEM}
\begin{verbatim}

system,"...";
\end{verbatim} 
transfers the string in quotes to the system for execution.  

Example: 
\begin{verbatim}

system,"ln -s /afs/cern.ch/user/u/user/public/some/directory short";
\end{verbatim}


\subsection{TABSTRING}
\begin{verbatim}

tabstring(arg1,arg2,arg3)
\end{verbatim}  
The"string function" tabstring(arg1,arg2,arg3) with exactly  three
arguments; arg1 is a table name (string), arg2 is a column name
(string), arg3 is a row number (integer), count starts at 0. The
function can be used in any context where a string appears; in case
there is no match, it returns \_void\_.  


\subsection{TITLE}
\begin{verbatim}

title,"...";
\end{verbatim} 
inserts the string in quotes as title in various tables and plots.  


\subsection{USE}
\begin{verbatim}

use,period=s_name,range=range,survey;
\end{verbatim} 
expands the sequence with name "s\_name", or a part of it as specified
in the \href{../Introduction/ranges.html#range}{range}. The
\texttt{survey} option plugs the survey data into the sequence elements
nodes on the first pass (see \href{../survey/survey.html}{survey}).  


\subsection{VALUE}
\begin{verbatim}

value,exp1,exp2,...;
\end{verbatim} 
prints the actual values of the expressions given. 

Example: 
\begin{verbatim}

a=clight/1000.;
value,a,pmass,exp(sqrt(2));
\end{verbatim} results in 
\begin{verbatim}

a = 299792.458         ;
pmass = 0.938271998        ;
exp(sqrt(2)) = 4.113250379        ;
\end{verbatim}


\subsection{WRITE}
\begin{verbatim}

write,table=table,file=file_name;
\end{verbatim} 
writes the table "table" onto the file "file\_name"; only the rows and
columns of a preceding select,flag=table,...; are written. If no select
has been issued for this table, the file will only contain the
header. If the FILE argument is omitted, the table is written to
standard output.  


%\href{http://www.cern.ch/Hans.Grote/hansg_sign.html}{hansg}, June 17, 2002 

%%%%\title{Range Selection}
%  Changed by: Chris ISELIN, 27-Jan-1997 
%  Changed by: Hans Grote, 30-Sep-2002 

\section{Program Flow Statements}

\subsection{IF}
\begin{verbatim}
if (logical_expression) {statement 1; statement 2; ...; statement n; }
\end{verbatim}
where \href{logical}{"logical\_expression"} is one of 
\begin{verbatim}
expr1  oper expr2
expr11 oper1 expr12 && expr21 oper2 expr22
expr11 oper1 expr12 || expr21 oper2 expr22
\end{verbatim} 
and \verb+oper+ one of 
\begin{verbatim}
==          ! equal
<>          ! not equal
<           ! less than
>           ! greater than
<=          ! less than or equal
>=          ! greater than or equal
\end{verbatim} 
The expressions are arithmetic expressions of type real. The statements
in the curly brackets are executed if the logical expression is true.  


\subsection{ELSEIF}
\begin{verbatim}
elseif (logical_expression) {statement 1; statement 2; ...; statement n; }
\end{verbatim} 
Only possible (in any number) behind an IF, or another ELSEIF; is
executed if  logical\_expression is true, and if none of the preceding
IF or ELSEIF logical conditions was true.  


\subsection{ELSE}
\begin{verbatim}
else {statement 1; statement 2; ...; statement n; }
\end{verbatim} 
Only possible (once) behind an IF, or an ELSEIF; is executed if
logical\_expression is true, and if none of the preceding IF or ELSEIF
logical conditions was true.  

For a real life example, see \href{foot.html}{ELSE example}. 


\subsection{WHILE}
\begin{verbatim}
while (logical_condition) {statement 1; statement 2; ...; statement n; }
\end{verbatim}  
executes the statements in curly brackets while the logical\_expression
is true. A simple example (in case you have forgotten the first ten
factorials) would be  
\begin{verbatim}
option, -info;   ! avoids redefiniton warnings
n = 1; m = 1;
while (n <= 10)
{
  m = m * n;  value, m;
  n = n + 1;
};
\end{verbatim}

For a real life example, see \href{foot.html}{WHILE example}.

\subsection{MACRO}
\begin{verbatim}
label: macro = {statement 1; statement 2; ...; statement n; };
label(arg1,...,argn): macro = {statement 1; statement 2; ...; statement n; };
\end{verbatim} 
The first form allows the execution of a group of statements via a
single command,  
\begin{verbatim}
exec, label;
\end{verbatim} 
that executes the statements in curly brackets exactly once. This command
can be issued any number of times.  

The second form allows to replace strings anywhere inside the statements
in curly brackets by other strings, or integer numbers prior to
execution. This is a powerful construct and should be handled with care.  

Simple example: 
\begin{verbatim}
option, -echo, -info;  ! for cleaner output
simple(xx,yy): macro = { xx = yy^2 + xx; value, xx;};
a = 3;
b = 5;
exec, simple(a,b);
\end{verbatim}


{\bf Passing arguments}\\
In the following example we use the fact that a "\$" in front of an
argument means that the truncated integer value of this argument is used
for replacement, rather than the argument string itself.  
\begin{verbatim}
tricky(xx,yy,zz): macro = {mzz.yy: xx, l = 1.yy, kzz = k.yy;};
n=0;
while (n < 3)
{
  n = n+1;
  exec, tricky(quadrupole, $n, 1);
  exec, tricky(sextupole, $n, 2);
};
\end{verbatim} 
Whereas the actual use of the preceding example is NOT recommended,
a real life example, showing the full power (!) of macros is to be
found under \href{foot.html}{macro usage} for the usage, and
under \href{foot.html#macro}{macro definition} for the
definition.


{\bf Beware of the following rules:}
\begin{itemize}
   \item Generally speaking: \textit{ special constructs } like IF,
     WHILE, MACRO will only allow one level of inclusion of another
     \textit{ special construct }.
   \item  Macros must not be called with numbers, but with strings
     (i.e. variable names in case of numerical values), i.e. {\bf NOT }
\begin{verbatim}
exec, thismacro($99, $129);
\end{verbatim}
{\bf BUT}
\begin{verbatim}
n1=99; n2=219;
exec, thismacro($n1, $n2);
\end{verbatim}
\end{itemize}

%\href{http://www.cern.ch/Hans.Grote/hansg_sign.html}{hansg}, June 17, 2002




\chapter{General Control Statements} 

\madx consists of a core program, and modules for specific tasks such as
twiss parameter calculation, matching, thin lens tracking, and so on.  
 
The statements listed here are those executed by the program core. They
deal with the I/O, element and sequence declaration, sequence
manipulation, statement flow control (e.g. IF, WHILE), MACRO
declaration, saving sequences onto files in \madx or \madeight format,
and so on.  


%% Moved to TWISS chapter
%% \subsection{COGUESS}
%% \label{subsec:coguess}

%% In order to help the initial finding of the closed orbit by the
%% \texttt{TWISS} module, it is possible to specify an initial guess. 

%% \madbox{
%% COGUESS, \=TOLERANCE=real, \\
%%          \>X=real, PX=real, Y=real, PY=real, T=real, PT=real, \\
%%          \>CLEAR=logical;
%% }
%% sets the required convergence precision in the closed orbit search
%% ("tolerance", see as well Twiss command
%% \href{../twiss/twiss.html#tolerance}{tolerance}).  

%% The other parameters define a first guess for all future closed orbit
%% searches in case they are different from zero.  

%% The clear parameter in the argument list resets the tolerance to its default value 
%% and cancels the effect of coguess in defining a first guess for subsequent 
%% closed orbit searches. \\
%% Default = false, {\tt clear} alone is equivalent to {\tt clear=true}


\section{EXIT, QUIT, STOP}
\label{sec:exit}\label{sec:quit}\label{sec:stop}
Any of these three commands ends the execution of \madx:
\madbox{
EXIT;
}
\madbox{
QUIT;
}
\madbox{
STOP;
}


\section{HELP}
\label{sec:help}
The HELP command prints all parameters, and their defaults values, for
the statement given; this includes basic element types.
\madbox{
HELP, statement\_name;
}

\section{SHOW}
\label{sec:show}
The SHOW command prints the "command" (typically "beam", "beam\%sequ",
or an element name), with the actual value of all its parameters.  
\madbox{
SHOW, command;
}

\section{VALUE}
\label{sec:value}
The VALUE command evaluates the current value of all listed expressions,
constants or variables, and prints the result in the form of \madx
statements on the assigned output file. 
\madbox{
VALUE, expression\{, expression\} ;
}

Example: \\
\madxmp{
a = clight/1000.; \\
value, a, pmass, exp(sqrt(2));
}
results in 
\madxmp{
a = 299792.458         ; \\
pmass = 0.938272046        ; \\
exp(sqrt(2)) = 4.113250379        ; \\
}

\section{OPTION}
\label{sec:option}

The \texttt{OPTION} commands sets the logical value of a number of flags
that control the behavior of \madx.

\madbox{
OPTION, flag=logical;
}

Because all attributes of \texttt{OPTION} are logical flags, the
following two statements are identical:
\madxmp{
OPTION, flag = true;\\
OPTION, flag;
}
And the following two statements are also identical:
\madxmp{
OPTION, flag = false;\\
OPTION, -flag;
}

Several flags can be set in a single \texttt{OPTION} command, e.g.
\madxmp{
OPTION, ECHO, WARN=true, -INFO, VERBOSE=false;
}

The available flags, their default values and their effect on \madx if
they are set to \texttt{TRUE} are listed in table 

\begin{table}[ht]
  \caption{Flags available to \texttt{OPTION} command}
  \vspace{1ex}
  \begin{center}
   \begin{tabular}{|l|c|l|}
     \hline
     {\bf FLAG }  & {\bf default} & {\bf effect if \texttt{TRUE}} \\
     \hline
     \texttt{ECHO}  & true  & echoes the input on the standard output file \\
     \texttt{WARN}  & true  & issues warning statements\\
     \texttt{INFO}  & true  & issues information statements\\
     \texttt{DEBUG} & false & issues debugging information \\
     \texttt{ECHOMACRO} & false & issues macro expansion printout for debugging \\
     \texttt{VERBOSE} & false & issues additional printout in makethin \\
     \texttt{TRACE}   & false & prints the system time after each command \\
     \texttt{VERIFY}  & false & issues a warning if an undefined variable is used \\
     \hline
     \texttt{TELL}  & false & prints the current value of all options \\
     \texttt{RESET} & false & resets all options to their defaults \\
     \hline
     \texttt{NO\_FATAL\_STOP} & false & Prevents madx from stopping in case of a fatal error \\
                              &       & {\bf Use at your own risk !} \\
     \hline
     \texttt{RBARC}     & true & converts the RBEND straight length into the arc length \\
     \texttt{THIN\_FOC} & true & enables the 1/(rho**2) focusing of thin dipoles \\
     \texttt{BBORBIT}   & false & the closed orbit is modified by beam-beam kicks \\
     \texttt{SYMPL}     & false & all element matrices are symplectified in Twiss \\
     \texttt{TWISS\_PRINT} & true & controls whether the twiss command produces output \\
     \texttt{THREADER}  & false & enables the threader for closed orbit finding in Twiss \\ 
                        &       & (see Twiss module) \\ 
     \hline
     \texttt{BB\_ULTRA\_RELATI} & false  & To be documented \\
     \texttt{BB\_SXY\_UPDATE}   & false  & To be documented \\
     \texttt{EMITTANCE\_UPDATE} & true   & To be documented \\
     \texttt{FAST\_ERROR\_FUNC} & false  & To be documented \\
     \hline
     \end{tabular}
   \end{center}
\end{table}

%% \begin{verbatim}
%%   name           default meaning if true
%%   ====           ======= ===============
%%   echo            true   echoes the input on the standard output file
%%   warn            true   issues warnings
%%   info            true   issues information
%%   debug           false  issues debugging information
%%   echomacro       false  issues macro expansion printout for debugging
%%   verbose         false  issues additional printout in makethin
%%   trace           false  prints the system time after each command
%%   verify          false  issues a warning if an undefined variable is used
%%   tell            false  prints the current value of all options
%%   reset           false  resets all options to their defaults
%%   no_fatal_stop   false  Prevents madx from stopping in case of a fatal error. 
%%                          Use at your own risk.

%%   rbarc           true   converts the RBEND straight length into the arc 
%%                          length
%%   thin_foc        true   if false suppresses the 1(rho**2) focusing of thin 
%%                          dipoles
%%   bborbit         false  the closed orbit is modified by beam-beam kicks
%%   sympl           false  all element matrices are symplectified in Twiss
%%   twiss_print     true   controls whether the twiss command produces output.
%%   threader        false  enables the threader for closed orbit finding in Twiss.
%%                          (see Twiss module)

%%   bb_ultra_relati false  To be documented
%%   bb_sxy_update   false  To be documented 
%%   emittance_update true  To be documented
%%   fast_error_func false  To be documented
%% \end{verbatim} 

The option \texttt{RBARC} is implemented for backwards compatibility
with \madeight up to version 8.23.06 included; in this version, the
RBEND length was just taken as the arc length of an SBEND with inclined
pole faces, contrary to the \madeight manual.  



\section{EXEC}
\label{sec:exec}
Each statement may be preceded by a label, when parsed and executed the
statement is then also stored and can be executed again with
\madbox{
EXEC, label;
}
This mechanism can be invoked any number of times, and the executed
statement may be calling another \texttt{EXEC}, etc. 

Note however, that the main usage of this \madx construct is the
execution of a \href{special.html#macro}{macro}.   

\madxmp{
tw: TWISS, FILE, SAVE; ! first execution of TWISS \\
... \\
EXEC, tw; ! second execution of the same TWISS command \\
}


\section{SET}
\label{sec:set}
The SET command is used in two forms:
\madbox{
SET, FORMAT=string \{, string\} ;\\
SET, SEQUENCE=string;
}


The first form of the \texttt{SET} command defines the formats for the
output precision that \madx uses with the \texttt{SAVE}, \texttt{SHOW},
\texttt{VALUE} and \texttt{TABLE} commands. The formats can be
given in any order and stay valid until replaced. 

The formats follow the C convention and must be included in double
quotes. The allowed formats are \\
\texttt{{\it n}d} for integers with a field-width = {\it n}, \\
\texttt{{\it m}.{\it n}f} or \texttt{{\it m}.{\it n}g} or \texttt{{\it
    m}.{\it n}e} for floats with field-width = {\it m} and precision =
       {\it n}, \\
\texttt{{\it n}s} for strings with a field-width = {\it n}.\\
The default is "right adjusted", a "-" changes it to "left adjusted".

{\bf Example:}\\
\madxmp{
SET, FORMAT="12d", "-18.5e", "25s";
}

%% \begin{verbatim}
%% "nd" for integer with n = field width.
%% \end{verbatim}
%% \begin{verbatim}
%% "m.nf" or "m.ng" or "m.ne" for floating, m field width, n precision.
%% \end{verbatim}
%% \begin{verbatim}
%% "ns" for string output.
%% \end{verbatim} 


The default formats are \texttt{"10d"},\texttt{"18.10g"} and \texttt{"-18s"}.

Example: 
\begin{verbatim}
set,format="22.14e";
\end{verbatim} 
changes the current floating point format to 22.14e; the other formats remain untouched. 
\begin{verbatim}
set,format="s","d","g";
\end{verbatim} 
sets all formats to automatic adjustment according to C conventions. 



The second form of the \texttt{SET} command allows to select the
current sequence without the "USE" command, which would
bring back to a bare lattice without errors. The command only works
if the chosen sequence has been previously activated with a \texttt{USE} command,
otherwise a warning is issued and \madx continues with the
unmodified current sequence. This command is particularly useful for
commands that do not have the sequence as an argument like "EMIT" or
"IBS". 



\section{SYSTEM}
\label{sec:system}
\madbox{
SYSTEM, "string";
}
transfers the quoted string to the system for execution. The quotes are
stripped and no check of the return status is performed bt \madx.

{\bf Example:}\\ 
\madxmp{
SYSTEM,"ln -s /afs/cern.ch/user/j/joe/public/some/directory shortname";
}
makes a local link to an AFS directory with the name "shortname" on a
UNIX system. 

Attention: Although this command is very convenient, it is clearly not portable
across systems and should probably avoid it if you intend to share \madx scripts with
collaborators working on other platforms. 

\section{TITLE}
\label{sec:title}
\madbox{
TITLE, "string";
}
the string in quotes is inserted as title in various table outputs and
plot results.  


\section{USE}
\label{sec:use}
\madx operates on beamlines that must be loaded and expanded in memory
before other commands can be invoked. The \texttt{USE} allows this
loading and expansion.

\madbox{
USE, \=PERIOD=sequence\_name, SEQUENCE=sequence\_name, \\
     \>RANGE=range, \\
     \>SURVEY=logical;
}

The attributes to the \texttt{USE} command are:
\begin{madlist}
  \ttitem{SEQUENCE} name of the sequence to be loaded and expanded. 
  \ttitem{PERIOD} name of the sequence to be loaded and expanded. \\ 
  \texttt{PERIOD} is an alias to \texttt{SEQUENCE} that was kept for
  backwards compatibility with \madeight and only one of them should be
  specified in a \texttt{USE} statement. 
  \ttitem{RANGE} specifies a \hyperref[sec:range]{range}.   
  restriction so that only the specified part of the named sequence is
  loaded and  expanded.
  \ttitem{SURVEY} option to plug the survey data into the sequence elements
  nodes on the first pass (see \hyperref[chap:survey]{\tt SURVEY}).
\end{madlist}

Note that reloading a sequence with the \texttt{USE} command reloads a
bare sequence and that any \texttt{ERROR} or orbit correction previously
assigned or associated to the sequence are forgotten. 
A mechanism to select a sequence without this side effect of the
\texttt{USE} command is provided with the \texttt{SET, SEQUENCE=...} command.


\section{SELECT} 
\label{sec:select}

\begin{verbatim}
select, flag=string, range=string, class=string, pattern=string,
        sequence=string, full, clear,
        column = string{, string},  slice=integer, thick=logical;
\end{verbatim} 
selects one or several elements for special treatment in a subsequent
command based on selection criteria.

The selection criteria on a single SELECT statement are logically
ANDed, in other words, selected elements have to fulfill the {\tt RANGE},
{\tt CLASS}, and {\tt PATTERN} criteria.  
The selection criteria on different SELECT statements are logically
ORed, in other words selected elements have to fulfill any of the
selection criteria accumulated by the different statements.   
All selections for a given command remain valid until the "clear" argument
is specified; 

The "flag" argument allows a determination of the applicability of the
SELECT statement and can be one of the following: 
\begin{madlist}
   \ttitem{seqedit} selection of elements for the
     \hyperref[sec:seqedit]{seqedit} module.  
   \ttitem{error} selection of elements for the
     \hyperref[chap:error]{error} assignment module.  
   \ttitem{makethin} selection of elements for the
     \hyperref[chap:makethin]{makethin} module that
     converts the sequence into one with thin elements only.  
   \ttitem{sectormap} selection of elements for the
     \hyperref[subsec:sectormap]{sectormap} output file
     from the Twiss module.  
   \ttitem{save} selection of elements for the \hyperref[sec:save]{\tt
     SAVE} command.   
   \ttitem{table} is a table name such as {\tt twiss}, {\tt track}
     etc., and the rows and columns to be written are selected.  
\end{madlist} 

The statement
\madxmp{
SELECT, FLAG=name, FULL;
}
selects ALL positions in the sequence for the flag "name". This is the default
for flags for all tables and {\tt MAKETHIN}.

The statement 
\madxmp{
SELECT, FLAG=name, CLEAR;
}
deselects ALL positions in the sequence for the flag "name". This is the default
for flags {\tt ERROR} and {\tt SEQEDIT}.

"slice" is only used by \hyperref[chap:makethin]{makethin} and
prescribes the number of slices into which the selected elements have to
be cut (default = 1).  

"column" is only valid for tables and determines the selection of columns
to be written into the TFS file. The "name" argument is special in that
it refers to the actual name of the selected element. For an example,
see \hyperref[sec:select]{SELECT}.  

"thick" is used to determine whether the selected elements will be
treated as thick elements by the MAKETHIN command. This only applies to
QUADRUPOLES and BENDS for which thick maps have been explicitely
derived. (see ...) 
%%2014-Apr-08  17:43:44  ghislain:  A completer.

Example: 
% keep verbatim for now and until ^ is solved
\begin{verbatim} 
select, flag = error, class = quadrupole, range = mb[1]/mb[5];
select, flag = error, pattern = "^mqw.*";
\end{verbatim}
selects all quadrupoles in the range mb[1] to mb[5], as well as all
elements (in the whole sequence) with name starting with "mqw", for 
treatment by the error module.  

Example:  
\madxmp{
select, flag=save, class=variable, pattern="abc.*";
save, file=mysave;
}
will save all variables (and sequences) containing "abc" in their name.
However note that since the element class "variable" does not exist, any
element with name containing "abc" will not be saved. 

\vskip 1cm
\hrule
\vskip 1cm

%% Imported from chapter 2
%\subsection{Selection Statements}

The elements, or a range of elements, in a sequence can be selected for
various purposes. Such selections remain valid until cleared (in
difference to \madeight); it is therefore recommended to always start with a  

\begin{verbatim}
select, flag =..., clear;
\end{verbatim} 
before setting a new selection. 
\begin{verbatim}
SELECT, FLAG=name, RANGE=range, CLASS=class, PATTERN=pattern [,FULL] [,CLEAR];
\end{verbatim} 
where the name for FLAG can be one of ERROR, MAKETHIN, SEQEDIT or the
name of a twiss table which is established for all sequence positions in
general.  

Selected elements have to fulfill the \href{ranges.html#range}{RANGE},
\href{ranges.html#class}{CLASS}, and \href{wildcard.html}{PATTERN}
criteria.  

Any number of SELECT commands can be issued for the same flag and are
accumulated (logically ORed). In this context note the following:  

\begin{verbatim}
SELECT, FLAG=name, FULL;
\end{verbatim} 
selects all positions in the sequence for this flag. This is the default
for all tables and makethin, whereas for ERROR and SEQEDIT the default
is "nothing selected".  

%\href{save_select}{}
\label{save_select}
SAVE: A SELECT,FLAG=SAVE statement causes the
selected sequences, elements, and variables to be written into the save
file. A class (only used for element selection), and a pattern can be
specified. Example:  
\begin{verbatim}
select, flag=save, class=variable, pattern="abc.*";
save, file=mysave;
\end{verbatim} 
will save all variables (and sequences) containing "abc" in their name,
but not elements with names containing "abc" since the class "variable"
does not exist (astucieux, non ?).  

SECTORMAP: A SELECT,FLAG=SECTORMAP statement causes sectormaps to be
written into the file "sectormap" like in \madeight. For the file to be
written, a flag SECTORMAP must be issued on the TWISS command in
addition.  

TWISS: A SELECT,FLAG=TWISS statement causes the selected rows and
columns to be written into the Twiss TFS file (former OPTICS command in
\madeight). The column selection is done on the same select. See as well
example 2.  

%% Example 1:  
%% \begin{verbatim}
%% TITLE,'Test input for MAD-X';

%% option,rbarc=false; // use arc length of rbends
%% beam; ! sets the default beam for the following sequence
%% option,-echo;
%% call file=fv9.opt;  ! contains optics parameters
%% call file="fv9.seq"; ! contains a small sequence "fivecell"
%% OPTION,ECHO;
%% SELECT,FLAG=SECTORMAP,clear;
%% SELECT,FLAG=SECTORMAP,PATTERN="^m.*";
%% SELECT,FLAG=TWISS,clear;
%% SELECT,FLAG=TWISS,PATTERN="^m.*",column=name,s,betx,bety;
%% USE,PERIOD=FIVECELL;
%% twiss,file=optics,sectormap;
%% stop;
%% \end{verbatim} 

%% This produces a file \href{sectormap.html}{sectormap}, and a
%% twiss output file \label{tfs} (name = optics):  
%% \begin{verbatim}
%% @ TYPE             %05s "TWISS"
%% @ PARTICLE         %08s "POSITRON"
%% @ MASS             %le          0.000510998902
%% @ CHARGE           %le                       1
%% @ E0               %le                       1
%% @ PC               %le           0.99999986944
%% @ GAMMA            %le           1956.95136738
%% @ KBUNCH           %le                       1
%% @ NPART            %le                       0
%% @ EX               %le                       1
%% @ EY               %le                       1
%% @ ET               %le                       0
%% @ LENGTH           %le                   534.6
%% @ ALFA             %le        0.00044339992938
%% @ ORBIT5           %le                      -0
%% @ GAMMATR          %le           47.4900022541
%% @ Q1               %le           1.25413071556
%% @ Q2               %le           1.25485338377
%% @ DQ1              %le           1.05329608302
%% @ DQ2              %le           1.04837000224
%% @ DXMAX            %le           2.17763211131
%% @ DYMAX            %le                       0
%% @ XCOMAX           %le                       0
%% @ YCOMAX           %le                       0
%% @ BETXMAX          %le            177.70993499
%% @ BETYMAX          %le           177.671582415
%% @ XCORMS           %le                       0
%% @ YCORMS           %le                       0
%% @ DXRMS            %le           1.66004270906
%% @ DYRMS            %le                       0
%% @ DELTAP           %le                       0
%% @ TITLE            %20s "Test input for MAD-X"
%% @ ORIGIN           %16s "MAD-X 0.20 Linux"
%% @ DATE             %08s "07/06/02"
%% @ TIME             %08s "14.25.51"
%% * NAME               S                  BETX               BETY               
%% $ %s                 %le                %le                %le                
%%  "MSCBH"             4.365              171.6688159        33.31817319       
%%  "MB"                19.72              108.1309095        58.58680717       
%%  "MB"                35.38              61.96499987        102.9962313       
%%  "MB"                51.04              34.61640793        166.2227523       
%%  "MSCBV.1"           57.825             33.34442808        171.6309057       
%%  "MB"                73.18              58.61984637        108.0956006       
%%  "MB"                88.84              103.0313887        61.93159422       
%%  "MB"                104.5              166.2602486        34.58939635       
%%  "MSCBH"             111.285            171.6688159        33.31817319       
%%  "MB"                126.64             108.1309095        58.58680717       
%%  "MB"                142.3              61.96499987        102.9962313       
%%  "MB"                157.96             34.61640793        166.2227523       
%%  "MSCBV"             164.745            33.34442808        171.6309057       
%%  "MB"                180.1              58.61984637        108.0956006       
%%  "MB"                195.76             103.0313887        61.93159422       
%%  "MB"                211.42             166.2602486        34.58939635       
%%  "MSCBH"             218.205            171.6688159        33.31817319       
%%  "MB"                233.56             108.1309095        58.58680717       
%%  "MB"                249.22             61.96499987        102.9962313       
%%  "MB"                264.88             34.61640793        166.2227523       
%%  "MSCBV"             271.665            33.34442808        171.6309057       
%%  "MB"                287.02             58.61984637        108.0956006       
%%  "MB"                302.68             103.0313887        61.93159422       
%%  "MB"                318.34             166.2602486        34.58939635       
%%  "MSCBH"             325.125            171.6688159        33.31817319       
%%  "MB"                340.48             108.1309095        58.58680717       
%%  "MB"                356.14             61.96499987        102.9962313       
%%  "MB"                371.8              34.61640793        166.2227523       
%%  "MSCBV"             378.585            33.34442808        171.6309057       
%%  "MB"                393.94             58.61984637        108.0956006       
%%  "MB"                409.6              103.0313887        61.93159422       
%%  "MB"                425.26             166.2602486        34.58939635       
%%  "MSCBH"             432.045            171.6688159        33.31817319       
%%  "MB"                447.4              108.1309095        58.58680717       
%%  "MB"                463.06             61.96499987        102.9962313       
%%  "MB"                478.72             34.61640793        166.2227523       
%%  "MSCBV"             485.505            33.34442808        171.6309057       
%%  "MB"                500.86             58.61984637        108.0956006       
%%  "MB"                516.52             103.0313887        61.93159422       
%%  "MB"                532.18             166.2602486        34.58939635       
%% \end{verbatim}

 %% Example 2: 

%%  Addition of variables to (any internal) table: 
%% \begin{verbatim}
%% select, flag=table, column=name, s, betx, ..., var1, var2, ...; ! or
%% select, flag=table, full, column=var1, var2, ...; ! default col.s + new
%% \end{verbatim} 
%% will write the current value of var1 etc. into the table each time a new
%% line is added; values from the same (current) line can be accessed by
%% these variables, e.g.  
%% \begin{verbatim}
%% var1 := sqrt(beam->ex*table(twiss,betx));
%% \end{verbatim} 
%% in the case of table above being "twiss". The plot command accepts the
%% new variables.  

%% Remark: this replaces the "string" variables of MAD-8. 

%%  This example demonstrates as well the usage of a user defined table \label{ucreate}. 
%% \begin{verbatim}
%% beam,ex=1.e-6,ey=1.e-3;
%% // element definitions
%% mb:rbend, l=14.2, angle:=0,k0:=bang/14.2;
%% mq:quadrupole, l:=3.1,apertype=ellipse,aperture={1,2};
%% qft:mq, l:=0.31, k1:=kqf,tilt=-pi/4;
%% qf.1:mq, l:=3.1, k1:=kqf;
%% qf.2:mq, l:=3.1, k1:=kqf;
%% qf.3:mq, l:=3.1, k1:=kqf;
%% qf.4:mq, l:=3.1, k1:=kqf;
%% qf.5:mq, l:=3.1, k1:=kqf;
%% qd.1:mq, l:=3.1, k1:=kqd;
%% qd.2:mq, l:=3.1, k1:=kqd;
%% qd.3:mq, l:=3.1, k1:=kqd;
%% qd.4:mq, l:=3.1, k1:=kqd;
%% qd.5:mq, l:=3.1, k1:=kqd;
%% bph:hmonitor, l:=l.bpm;
%% bpv:vmonitor, l:=l.bpm;
%% cbh:hkicker;
%% cbv:vkicker;
%% cbh.1:cbh, kick:=acbh1;
%% cbh.2:cbh, kick:=acbh2;
%% cbh.3:cbh, kick:=acbh3;
%% cbh.4:cbh, kick:=acbh4;
%% cbh.5:cbh, kick:=acbh5;
%% cbv.1:cbv, kick:=acbv1;
%% cbv.2:cbv, kick:=acbv2;
%% cbv.3:cbv, kick:=acbv3;
%% cbv.4:cbv, kick:=acbv4;
%% cbv.5:cbv, kick:=acbv5;
%% !mscbh:sextupole, l:=1.1, k2:=ksf;
%% mscbh:multipole, knl:={0,0,0,ksf},tilt=-pi/8;
%% mscbv:sextupole, l:=1.1, k2:=ksd;
%% !mscbv:octupole, l:=1.1, k3:=ksd,tilt=-pi/8;

%% // sequence declaration

%% fivecell:sequence, refer=centre, l=534.6;
%%    qf.1:qf.1, at=1.550000e+00;
%%    qft:qft, at=3.815000e+00;
%% !   mscbh:mscbh, at=3.815000e+00;
%%    cbh.1:cbh.1, at=4.365000e+00;
%%    mb:mb, at=1.262000e+01;
%%    mb:mb, at=2.828000e+01;
%%    mb:mb, at=4.394000e+01;
%%    bpv:bpv, at=5.246000e+01;
%%    qd.1:qd.1, at=5.501000e+01;
%%    mscbv:mscbv, at=5.727500e+01;
%%    cbv.1:cbv.1, at=5.782500e+01;
%%    mb:mb, at=6.608000e+01;
%%    mb:mb, at=8.174000e+01;
%%    mb:mb, at=9.740000e+01;
%%    bph:bph, at=1.059200e+02;
%%    qf.2:qf.2, at=1.084700e+02;
%%    mscbh:mscbh, at=1.107350e+02;
%%    cbh.2:cbh.2, at=1.112850e+02;
%%    mb:mb, at=1.195400e+02;
%%    mb:mb, at=1.352000e+02;
%%    mb:mb, at=1.508600e+02;
%%    bpv:bpv, at=1.593800e+02;
%%    qd.2:qd.2, at=1.619300e+02;
%%    mscbv:mscbv, at=1.641950e+02;
%%    cbv.2:cbv.2, at=1.647450e+02;
%%    mb:mb, at=1.730000e+02;
%%    mb:mb, at=1.886600e+02;
%%    mb:mb, at=2.043200e+02;
%%    bph:bph, at=2.128400e+02;
%%    qf.3:qf.3, at=2.153900e+02;
%%    mscbh:mscbh, at=2.176550e+02;
%%    cbh.3:cbh.3, at=2.182050e+02;
%%    mb:mb, at=2.264600e+02;
%%    mb:mb, at=2.421200e+02;
%%    mb:mb, at=2.577800e+02;
%%    bpv:bpv, at=2.663000e+02;
%%    qd.3:qd.3, at=2.688500e+02;
%%    mscbv:mscbv, at=2.711150e+02;
%%    cbv.3:cbv.3, at=2.716650e+02;
%%    mb:mb, at=2.799200e+02;
%%    mb:mb, at=2.955800e+02;
%%    mb:mb, at=3.112400e+02;
%%    bph:bph, at=3.197600e+02;
%%    qf.4:qf.4, at=3.223100e+02;
%%    mscbh:mscbh, at=3.245750e+02;
%%    cbh.4:cbh.4, at=3.251250e+02;
%%    mb:mb, at=3.333800e+02;
%%    mb:mb, at=3.490400e+02;
%%    mb:mb, at=3.647000e+02;
%%    bpv:bpv, at=3.732200e+02;
%%    qd.4:qd.4, at=3.757700e+02;
%%    mscbv:mscbv, at=3.780350e+02;
%%    cbv.4:cbv.4, at=3.785850e+02;
%%    mb:mb, at=3.868400e+02;
%%    mb:mb, at=4.025000e+02;
%%    mb:mb, at=4.181600e+02;
%%    bph:bph, at=4.266800e+02;
%%    qf.5:qf.5, at=4.292300e+02;
%%    mscbh:mscbh, at=4.314950e+02;
%%    cbh.5:cbh.5, at=4.320450e+02;
%%    mb:mb, at=4.403000e+02;
%%    mb:mb, at=4.559600e+02;
%%    mb:mb, at=4.716200e+02;
%%    bpv:bpv, at=4.801400e+02;
%%    qd.5:qd.5, at=4.826900e+02;
%%    mscbv:mscbv, at=4.849550e+02;
%%    cbv.5:cbv.5, at=4.855050e+02;
%%    mb:mb, at=4.937600e+02;
%%    mb:mb, at=5.094200e+02;
%%    mb:mb, at=5.250800e+02;
%%    bph:bph, at=5.336000e+02;
%% end:marker, at=5.346000e+02;
%% endsequence;

%% // forces and other constants

%% l.bpm:=.3;
%% bang:=.509998807401e-2;
%% kqf:=.872651312e-2;
%% kqd:=-.872777242e-2;
%% ksf:=.0198492943;
%% ksd:=-.039621283;
%% acbv1:=1.e-4;
%% acbh1:=1.e-4;
%% !save,sequence=fivecell,file,mad8;

%% s := table(twiss,bpv[5],betx);
%% myvar := sqrt(beam->ex*table(twiss,betx));
%% use, period=fivecell;
%% select,flag=twiss,column=name,s,myvar,apertype;
%% twiss,file;
%% n = 0;
%% create,table=mytab,column=dp,mq1,mq2;
%% mq1:=table(summ,q1);
%% mq2:=table(summ,q2);
%% while ( n < 11)
%% {
%%   n = n + 1;
%%   dp = 1.e-4*(n-6);
%%   twiss,deltap=dp;
%%   fill,table=mytab;
%% }
%% write,table=mytab;
%% plot,haxis=s,vaxis=aper_1,aper_2,colour=100,range=#s/cbv.1,notitle;
%% stop;
%% \end{verbatim}
%% prints the following user table on output:

%% \begin{verbatim}
%% @ NAME             %05s "MYTAB"
%% @ TYPE             %04s "USER"
%% @ TITLE            %08s "no-title"
%% @ ORIGIN           %16s "MAD-X 1.09 Linux"
%% @ DATE             %08s "10/12/02"
%% @ TIME             %08s "10.45.25"
%% * DP                 MQ1                MQ2                
%% $ %le                %le                %le                
%%  -0.0005            1.242535951        1.270211135       
%%  -0.0004            1.242495534        1.270197018       
%%  -0.0003            1.242452432        1.270185673       
%%  -0.0002            1.242406653        1.270177093       
%%  -0.0001            1.242358206        1.270171269       
%%  0                  1.242307102        1.27016819        
%%  0.0001             1.242253353        1.270167843       
%%  0.0002             1.242196974        1.270170214       
%%  0.0003             1.24213798         1.270175288       
%%  0.0004             1.242076387        1.270183048       
%%  0.0005             1.242012214        1.270193477       
%% \end{verbatim}
%% and produces a twiss file with the additional column myvar, as well as a plot
%% file with the aperture values plotted.


%% \href{screate}{}

%% Example of joining two tables with different length into a third table
%% making use of the length of either table as given by
%% table("your\_table\_name", tablelength) and adding names by the "\_name"
%% attribute.

%% \begin{verbatim}
%% title,   "summing of offset and alignment tables";
%% set,    format="13.6f";

%% readtable, table=align,  file="align.ip2.b1.tfs";   // mesured alignment
%% readtable, table=offset, file="offset.ip2.b1.tfs";  // nominal offsets

%% n_elem  =  table(offset, tablelength);

%% create,  table=align_offset, column=_name,s_ip,x_off,dx_off,ddx_off,y_off,dy_off,ddy_off;

%% calcul(elem_name) : macro = {
%%     x_off = table(align,  elem_name, x_ali) + x_off;
%%     y_off = table(align,  elem_name, y_ali) + y_off;
%% }


%% one_elem(j_elem) : macro = {
%%     setvars, table=offset, row=j_elem;
%%     exec,  calcul(tabstring(offset, name, j_elem));
%%     fill,  table=align_offset;
%% }


%% i_elem = 0;
%% while (i_elem < n_elem) { i_elem = i_elem + 1; exec,  one_elem($i_elem); }

%% write, table=align_offset, file="align_offset.tfs";

%% stop;
%% \end{verbatim}

%%



%%%\title{SELECT}
%  Changed by: Hans Grote, 16-Jan-2003 

\subsection{Selection Statements}

The elements, or a range of elements, in a sequence can be selected for
various purposes. Such selections remain valid until cleared (in
difference to MAD-8); it is therefore recommended to always start with a  

\begin{verbatim}
select, flag =..., clear;
\end{verbatim} 
before setting a new selection. 
\begin{verbatim}
SELECT, FLAG=name, RANGE=range, CLASS=class, PATTERN=pattern [,FULL] [,CLEAR];
\end{verbatim} 
where the name for FLAG can be one of ERROR, MAKETHIN, SEQEDIT or the
name of a twiss table which is established for all sequence positions in
general.  

Selected elements have to fulfill the \href{ranges.html#range}{RANGE},
\href{ranges.html#class}{CLASS}, and \href{wildcard.html}{PATTERN}
criteria.  

Any number of SELECT commands can be issued for the same flag and are
accumulated (logically ORed). In this context note the following:  

\begin{verbatim}
SELECT, FLAG=name, FULL;
\end{verbatim} 
selects all positions in the sequence for this flag. This is the default
for all tables and makethin, whereas for ERROR and SEQEDIT the default
is "nothing selected".  

\href{save_select}{}SAVE: A SELECT,FLAG=SAVE statement causes the
selected sequences, elements, and variables to be written into the save
file. A class (only used for element selection), and a pattern can be
specified. Example:  
\begin{verbatim}
select, flag=save, class=variable, pattern="abc.*";
save, file=mysave;
\end{verbatim} 
will save all variables (and sequences) containing "abc" in their name,
but not elements with names containing "abc" since the class "variable"
does not exist (astucieux, non ?).  

SECTORMAP: A SELECT,FLAG=SECTORMAP statement causes sectormaps to be
written into the file "sectormap" like in MAD-8. For the file to be
written, a flag SECTORMAP must be issued on the TWISS command in
addition.  

TWISS: A SELECT,FLAG=TWISS statement causes the selected rows and
columns to be written into the Twiss TFS file (former OPTICS command in
MAD-8). The column selection is done on the same select. See as well
example 2.  

Example 1:  
\begin{verbatim}
TITLE,'Test input for MAD-X';

option,rbarc=false; // use arc length of rbends
beam; ! sets the default beam for the following sequence
option,-echo;
call file=fv9.opt;  ! contains optics parameters
call file="fv9.seq"; ! contains a small sequence "fivecell"
OPTION,ECHO;
SELECT,FLAG=SECTORMAP,clear;
SELECT,FLAG=SECTORMAP,PATTERN="^m.*";
SELECT,FLAG=TWISS,clear;
SELECT,FLAG=TWISS,PATTERN="^m.*",column=name,s,betx,bety;
USE,PERIOD=FIVECELL;
twiss,file=optics,sectormap;
stop;
\end{verbatim} 

This produces a file \href{sectormap.html}{sectormap}, and a
\href{tfs}{}twiss output file (name = optics):  
\begin{verbatim}
@ TYPE             %05s "TWISS"
@ PARTICLE         %08s "POSITRON"
@ MASS             %le          0.000510998902
@ CHARGE           %le                       1
@ E0               %le                       1
@ PC               %le           0.99999986944
@ GAMMA            %le           1956.95136738
@ KBUNCH           %le                       1
@ NPART            %le                       0
@ EX               %le                       1
@ EY               %le                       1
@ ET               %le                       0
@ LENGTH           %le                   534.6
@ ALFA             %le        0.00044339992938
@ ORBIT5           %le                      -0
@ GAMMATR          %le           47.4900022541
@ Q1               %le           1.25413071556
@ Q2               %le           1.25485338377
@ DQ1              %le           1.05329608302
@ DQ2              %le           1.04837000224
@ DXMAX            %le           2.17763211131
@ DYMAX            %le                       0
@ XCOMAX           %le                       0
@ YCOMAX           %le                       0
@ BETXMAX          %le            177.70993499
@ BETYMAX          %le           177.671582415
@ XCORMS           %le                       0
@ YCORMS           %le                       0
@ DXRMS            %le           1.66004270906
@ DYRMS            %le                       0
@ DELTAP           %le                       0
@ TITLE            %20s "Test input for MAD-X"
@ ORIGIN           %16s "MAD-X 0.20 Linux"
@ DATE             %08s "07/06/02"
@ TIME             %08s "14.25.51"
* NAME               S                  BETX               BETY               
$ %s                 %le                %le                %le                
 "MSCBH"             4.365              171.6688159        33.31817319       
 "MB"                19.72              108.1309095        58.58680717       
 "MB"                35.38              61.96499987        102.9962313       
 "MB"                51.04              34.61640793        166.2227523       
 "MSCBV.1"           57.825             33.34442808        171.6309057       
 "MB"                73.18              58.61984637        108.0956006       
 "MB"                88.84              103.0313887        61.93159422       
 "MB"                104.5              166.2602486        34.58939635       
 "MSCBH"             111.285            171.6688159        33.31817319       
 "MB"                126.64             108.1309095        58.58680717       
 "MB"                142.3              61.96499987        102.9962313       
 "MB"                157.96             34.61640793        166.2227523       
 "MSCBV"             164.745            33.34442808        171.6309057       
 "MB"                180.1              58.61984637        108.0956006       
 "MB"                195.76             103.0313887        61.93159422       
 "MB"                211.42             166.2602486        34.58939635       
 "MSCBH"             218.205            171.6688159        33.31817319       
 "MB"                233.56             108.1309095        58.58680717       
 "MB"                249.22             61.96499987        102.9962313       
 "MB"                264.88             34.61640793        166.2227523       
 "MSCBV"             271.665            33.34442808        171.6309057       
 "MB"                287.02             58.61984637        108.0956006       
 "MB"                302.68             103.0313887        61.93159422       
 "MB"                318.34             166.2602486        34.58939635       
 "MSCBH"             325.125            171.6688159        33.31817319       
 "MB"                340.48             108.1309095        58.58680717       
 "MB"                356.14             61.96499987        102.9962313       
 "MB"                371.8              34.61640793        166.2227523       
 "MSCBV"             378.585            33.34442808        171.6309057       
 "MB"                393.94             58.61984637        108.0956006       
 "MB"                409.6              103.0313887        61.93159422       
 "MB"                425.26             166.2602486        34.58939635       
 "MSCBH"             432.045            171.6688159        33.31817319       
 "MB"                447.4              108.1309095        58.58680717       
 "MB"                463.06             61.96499987        102.9962313       
 "MB"                478.72             34.61640793        166.2227523       
 "MSCBV"             485.505            33.34442808        171.6309057       
 "MB"                500.86             58.61984637        108.0956006       
 "MB"                516.52             103.0313887        61.93159422       
 "MB"                532.18             166.2602486        34.58939635       
\end{verbatim}

 Example 2: 

 Addition of variables to (any internal) table: 
\begin{verbatim}
select, flag=table, column=name, s, betx, ..., var1, var2, ...; ! or
select, flag=table, full, column=var1, var2, ...; ! default col.s + new
\end{verbatim} 
will write the current value of var1 etc. into the table each time a new
line is added; values from the same (current) line can be accessed by
these variables, e.g.  
\begin{verbatim}
var1 := sqrt(beam->ex*table(twiss,betx));
\end{verbatim} 
in the case of table above being "twiss". The plot command accepts the
new variables.  

Remark: this replaces the "string" variables of MAD-8. 

\href{ucreate}{} This example demonstrates as well the usage of a user defined table. 
\begin{verbatim}
beam,ex=1.e-6,ey=1.e-3;
// element definitions
mb:rbend, l=14.2, angle:=0,k0:=bang/14.2;
mq:quadrupole, l:=3.1,apertype=ellipse,aperture={1,2};
qft:mq, l:=0.31, k1:=kqf,tilt=-pi/4;
qf.1:mq, l:=3.1, k1:=kqf;
qf.2:mq, l:=3.1, k1:=kqf;
qf.3:mq, l:=3.1, k1:=kqf;
qf.4:mq, l:=3.1, k1:=kqf;
qf.5:mq, l:=3.1, k1:=kqf;
qd.1:mq, l:=3.1, k1:=kqd;
qd.2:mq, l:=3.1, k1:=kqd;
qd.3:mq, l:=3.1, k1:=kqd;
qd.4:mq, l:=3.1, k1:=kqd;
qd.5:mq, l:=3.1, k1:=kqd;
bph:hmonitor, l:=l.bpm;
bpv:vmonitor, l:=l.bpm;
cbh:hkicker;
cbv:vkicker;
cbh.1:cbh, kick:=acbh1;
cbh.2:cbh, kick:=acbh2;
cbh.3:cbh, kick:=acbh3;
cbh.4:cbh, kick:=acbh4;
cbh.5:cbh, kick:=acbh5;
cbv.1:cbv, kick:=acbv1;
cbv.2:cbv, kick:=acbv2;
cbv.3:cbv, kick:=acbv3;
cbv.4:cbv, kick:=acbv4;
cbv.5:cbv, kick:=acbv5;
!mscbh:sextupole, l:=1.1, k2:=ksf;
mscbh:multipole, knl:={0,0,0,ksf},tilt=-pi/8;
mscbv:sextupole, l:=1.1, k2:=ksd;
!mscbv:octupole, l:=1.1, k3:=ksd,tilt=-pi/8;

// sequence declaration

fivecell:sequence, refer=centre, l=534.6;
   qf.1:qf.1, at=1.550000e+00;
   qft:qft, at=3.815000e+00;
!   mscbh:mscbh, at=3.815000e+00;
   cbh.1:cbh.1, at=4.365000e+00;
   mb:mb, at=1.262000e+01;
   mb:mb, at=2.828000e+01;
   mb:mb, at=4.394000e+01;
   bpv:bpv, at=5.246000e+01;
   qd.1:qd.1, at=5.501000e+01;
   mscbv:mscbv, at=5.727500e+01;
   cbv.1:cbv.1, at=5.782500e+01;
   mb:mb, at=6.608000e+01;
   mb:mb, at=8.174000e+01;
   mb:mb, at=9.740000e+01;
   bph:bph, at=1.059200e+02;
   qf.2:qf.2, at=1.084700e+02;
   mscbh:mscbh, at=1.107350e+02;
   cbh.2:cbh.2, at=1.112850e+02;
   mb:mb, at=1.195400e+02;
   mb:mb, at=1.352000e+02;
   mb:mb, at=1.508600e+02;
   bpv:bpv, at=1.593800e+02;
   qd.2:qd.2, at=1.619300e+02;
   mscbv:mscbv, at=1.641950e+02;
   cbv.2:cbv.2, at=1.647450e+02;
   mb:mb, at=1.730000e+02;
   mb:mb, at=1.886600e+02;
   mb:mb, at=2.043200e+02;
   bph:bph, at=2.128400e+02;
   qf.3:qf.3, at=2.153900e+02;
   mscbh:mscbh, at=2.176550e+02;
   cbh.3:cbh.3, at=2.182050e+02;
   mb:mb, at=2.264600e+02;
   mb:mb, at=2.421200e+02;
   mb:mb, at=2.577800e+02;
   bpv:bpv, at=2.663000e+02;
   qd.3:qd.3, at=2.688500e+02;
   mscbv:mscbv, at=2.711150e+02;
   cbv.3:cbv.3, at=2.716650e+02;
   mb:mb, at=2.799200e+02;
   mb:mb, at=2.955800e+02;
   mb:mb, at=3.112400e+02;
   bph:bph, at=3.197600e+02;
   qf.4:qf.4, at=3.223100e+02;
   mscbh:mscbh, at=3.245750e+02;
   cbh.4:cbh.4, at=3.251250e+02;
   mb:mb, at=3.333800e+02;
   mb:mb, at=3.490400e+02;
   mb:mb, at=3.647000e+02;
   bpv:bpv, at=3.732200e+02;
   qd.4:qd.4, at=3.757700e+02;
   mscbv:mscbv, at=3.780350e+02;
   cbv.4:cbv.4, at=3.785850e+02;
   mb:mb, at=3.868400e+02;
   mb:mb, at=4.025000e+02;
   mb:mb, at=4.181600e+02;
   bph:bph, at=4.266800e+02;
   qf.5:qf.5, at=4.292300e+02;
   mscbh:mscbh, at=4.314950e+02;
   cbh.5:cbh.5, at=4.320450e+02;
   mb:mb, at=4.403000e+02;
   mb:mb, at=4.559600e+02;
   mb:mb, at=4.716200e+02;
   bpv:bpv, at=4.801400e+02;
   qd.5:qd.5, at=4.826900e+02;
   mscbv:mscbv, at=4.849550e+02;
   cbv.5:cbv.5, at=4.855050e+02;
   mb:mb, at=4.937600e+02;
   mb:mb, at=5.094200e+02;
   mb:mb, at=5.250800e+02;
   bph:bph, at=5.336000e+02;
end:marker, at=5.346000e+02;
endsequence;

// forces and other constants

l.bpm:=.3;
bang:=.509998807401e-2;
kqf:=.872651312e-2;
kqd:=-.872777242e-2;
ksf:=.0198492943;
ksd:=-.039621283;
acbv1:=1.e-4;
acbh1:=1.e-4;
!save,sequence=fivecell,file,mad8;

s := table(twiss,bpv[5],betx);
myvar := sqrt(beam->ex*table(twiss,betx));
use, period=fivecell;
select,flag=twiss,column=name,s,myvar,apertype;
twiss,file;
n = 0;
create,table=mytab,column=dp,mq1,mq2;
mq1:=table(summ,q1);
mq2:=table(summ,q2);
while ( n < 11)
{
  n = n + 1;
  dp = 1.e-4*(n-6);
  twiss,deltap=dp;
  fill,table=mytab;
}
write,table=mytab;
plot,haxis=s,vaxis=aper_1,aper_2,colour=100,range=#s/cbv.1,notitle;
stop;
\end{verbatim}
prints the following user table on output:

\begin{verbatim}
@ NAME             %05s "MYTAB"
@ TYPE             %04s "USER"
@ TITLE            %08s "no-title"
@ ORIGIN           %16s "MAD-X 1.09 Linux"
@ DATE             %08s "10/12/02"
@ TIME             %08s "10.45.25"
* DP                 MQ1                MQ2                
$ %le                %le                %le                
 -0.0005            1.242535951        1.270211135       
 -0.0004            1.242495534        1.270197018       
 -0.0003            1.242452432        1.270185673       
 -0.0002            1.242406653        1.270177093       
 -0.0001            1.242358206        1.270171269       
 0                  1.242307102        1.27016819        
 0.0001             1.242253353        1.270167843       
 0.0002             1.242196974        1.270170214       
 0.0003             1.24213798         1.270175288       
 0.0004             1.242076387        1.270183048       
 0.0005             1.242012214        1.270193477       
\end{verbatim}
and produces a twiss file with the additional column myvar, as well as a plot
file with the aperture values plotted.


\href{screate}{}

Example of joing 2 tables with different length into a third table
making use of the length of either table as given by
table("your\_table\_name", tablelength) and adding names by the "\_name"
attribute.

\begin{verbatim}
title,   "summing of offset and alignment tables";
set,    format="13.6f";

readtable, table=align,  file="align.ip2.b1.tfs";   // mesured alignment
readtable, table=offset, file="offset.ip2.b1.tfs";  // nominal offsets

n_elem  =  table(offset, tablelength);

create,  table=align_offset, column=_name,s_ip,x_off,dx_off,ddx_off,y_off,dy_off,ddy_off;

calcul(elem_name) : macro = {
    x_off = table(align,  elem_name, x_ali) + x_off;
    y_off = table(align,  elem_name, y_ali) + y_off;
}


one_elem(j_elem) : macro = {
    setvars, table=offset, row=j_elem;
    exec,  calcul(tabstring(offset, name, j_elem));
    fill,  table=align_offset;
}


i_elem = 0;
while (i_elem < n_elem) { i_elem = i_elem + 1; exec,  one_elem($i_elem); }

write, table=align_offset, file="align_offset.tfs";

stop;
\end{verbatim}

% \href{http://www.cern.ch/Hans.Grote/hansg_sign.html}{hansg}, May 8, 2001


\section{SELECT}
\label{sec:selection}
The elements, or a range of elements, in a sequence can be selected for
various purposes. Such selections remain valid until cleared (in
difference to \madeight); it is therefore recommended to always start with a  

\begin{verbatim}
select, flag =..., clear;
\end{verbatim} 
before setting a new selection. 
\begin{verbatim}
SELECT, FLAG=name, RANGE=range, CLASS=class, PATTERN=pattern [,FULL] [,CLEAR];
\end{verbatim} 
where the name for FLAG can be one of ERROR, MAKETHIN, SEQEDIT or the
name of a twiss table which is established for all sequence positions in
general.  

Selected elements have to fulfill the \href{ranges.html#range}{RANGE},
\href{ranges.html#class}{CLASS}, and \href{wildcard.html}{PATTERN}
criteria.  

Any number of SELECT commands can be issued for the same flag and are
accumulated (logically ORed). In this context note the following:  

\begin{verbatim}
SELECT, FLAG=name, FULL;
\end{verbatim} 
selects all positions in the sequence for this flag. This is the default
for all tables and makethin, whereas for ERROR and SEQEDIT the default
is "nothing selected".  

%\href{save_select}{}
%\label{save_select}
SAVE: A SELECT,FLAG=SAVE statement causes the
selected sequences, elements, and variables to be written into the save
file. A class (only used for element selection), and a pattern can be
specified. Example:  
\begin{verbatim}
select, flag=save, class=variable, pattern="abc.*";
save, file=mysave;
\end{verbatim} 
will save all variables (and sequences) containing "abc" in their name,
but not elements with names containing "abc" since the class "variable"
does not exist (astucieux, non ?).  

SECTORMAP: A SELECT,FLAG=SECTORMAP statement causes sectormaps to be
written into the file "sectormap" like in \madeight. For the file to be
written, a flag SECTORMAP must be issued on the TWISS command in
addition.  

TWISS: A SELECT,FLAG=TWISS statement causes the selected rows and
columns to be written into the Twiss TFS file (former OPTICS command in
\madeight). The column selection is done on the same select. See as well
example 2.  

Example 1:  
\begin{verbatim}
TITLE,'Test input for MAD-X';

option,rbarc=false; // use arc length of rbends
beam; ! sets the default beam for the following sequence
option,-echo;
call file=fv9.opt;  ! contains optics parameters
call file="fv9.seq"; ! contains a small sequence "fivecell"
OPTION,ECHO;
SELECT,FLAG=SECTORMAP,clear;
SELECT,FLAG=SECTORMAP,PATTERN="^m.*";
SELECT,FLAG=TWISS,clear;
SELECT,FLAG=TWISS,PATTERN="^m.*",column=name,s,betx,bety;
USE,PERIOD=FIVECELL;
twiss,file=optics,sectormap;
stop;
\end{verbatim} 

This produces a file \href{sectormap.html}{sectormap}, and a
twiss output file \label{tfs} (name = optics):  
\begin{verbatim}
@ TYPE             %05s "TWISS"
@ PARTICLE         %08s "POSITRON"
@ MASS             %le          0.000510998902
@ CHARGE           %le                       1
@ E0               %le                       1
@ PC               %le           0.99999986944
@ GAMMA            %le           1956.95136738
@ KBUNCH           %le                       1
@ NPART            %le                       0
@ EX               %le                       1
@ EY               %le                       1
@ ET               %le                       0
@ LENGTH           %le                   534.6
@ ALFA             %le        0.00044339992938
@ ORBIT5           %le                      -0
@ GAMMATR          %le           47.4900022541
@ Q1               %le           1.25413071556
@ Q2               %le           1.25485338377
@ DQ1              %le           1.05329608302
@ DQ2              %le           1.04837000224
@ DXMAX            %le           2.17763211131
@ DYMAX            %le                       0
@ XCOMAX           %le                       0
@ YCOMAX           %le                       0
@ BETXMAX          %le            177.70993499
@ BETYMAX          %le           177.671582415
@ XCORMS           %le                       0
@ YCORMS           %le                       0
@ DXRMS            %le           1.66004270906
@ DYRMS            %le                       0
@ DELTAP           %le                       0
@ TITLE            %20s "Test input for MAD-X"
@ ORIGIN           %16s "MAD-X 0.20 Linux"
@ DATE             %08s "07/06/02"
@ TIME             %08s "14.25.51"
* NAME               S                  BETX               BETY               
$ %s                 %le                %le                %le                
 "MSCBH"             4.365              171.6688159        33.31817319       
 "MB"                19.72              108.1309095        58.58680717       
 "MB"                35.38              61.96499987        102.9962313       
 "MB"                51.04              34.61640793        166.2227523       
 "MSCBV.1"           57.825             33.34442808        171.6309057       
 "MB"                73.18              58.61984637        108.0956006       
 "MB"                88.84              103.0313887        61.93159422       
 "MB"                104.5              166.2602486        34.58939635       
 "MSCBH"             111.285            171.6688159        33.31817319       
 "MB"                126.64             108.1309095        58.58680717       
 "MB"                142.3              61.96499987        102.9962313       
 "MB"                157.96             34.61640793        166.2227523       
 "MSCBV"             164.745            33.34442808        171.6309057       
 "MB"                180.1              58.61984637        108.0956006       
 "MB"                195.76             103.0313887        61.93159422       
 "MB"                211.42             166.2602486        34.58939635       
 "MSCBH"             218.205            171.6688159        33.31817319       
 "MB"                233.56             108.1309095        58.58680717       
 "MB"                249.22             61.96499987        102.9962313       
 "MB"                264.88             34.61640793        166.2227523       
 "MSCBV"             271.665            33.34442808        171.6309057       
 "MB"                287.02             58.61984637        108.0956006       
 "MB"                302.68             103.0313887        61.93159422       
 "MB"                318.34             166.2602486        34.58939635       
 "MSCBH"             325.125            171.6688159        33.31817319       
 "MB"                340.48             108.1309095        58.58680717       
 "MB"                356.14             61.96499987        102.9962313       
 "MB"                371.8              34.61640793        166.2227523       
 "MSCBV"             378.585            33.34442808        171.6309057       
 "MB"                393.94             58.61984637        108.0956006       
 "MB"                409.6              103.0313887        61.93159422       
 "MB"                425.26             166.2602486        34.58939635       
 "MSCBH"             432.045            171.6688159        33.31817319       
 "MB"                447.4              108.1309095        58.58680717       
 "MB"                463.06             61.96499987        102.9962313       
 "MB"                478.72             34.61640793        166.2227523       
 "MSCBV"             485.505            33.34442808        171.6309057       
 "MB"                500.86             58.61984637        108.0956006       
 "MB"                516.52             103.0313887        61.93159422       
 "MB"                532.18             166.2602486        34.58939635       
\end{verbatim}

 Example 2: 

 Addition of variables to (any internal) table: 
\begin{verbatim}
select, flag=table, column=name, s, betx, ..., var1, var2, ...; ! or
select, flag=table, full, column=var1, var2, ...; ! default col.s + new
\end{verbatim} 
will write the current value of var1 etc. into the table each time a new
line is added; values from the same (current) line can be accessed by
these variables, e.g.  
\begin{verbatim}
var1 := sqrt(beam->ex*table(twiss,betx));
\end{verbatim} 
in the case of table above being "twiss". The plot command accepts the
new variables.  

Remark: this replaces the "string" variables of \madeight. 

 This example demonstrates as well the usage of a user defined table \label{ucreate}. 
\begin{verbatim}
beam,ex=1.e-6,ey=1.e-3;
// element definitions
mb:rbend, l=14.2, angle:=0,k0:=bang/14.2;
mq:quadrupole, l:=3.1,apertype=ellipse,aperture={1,2};
qft:mq, l:=0.31, k1:=kqf,tilt=-pi/4;
qf.1:mq, l:=3.1, k1:=kqf;
qf.2:mq, l:=3.1, k1:=kqf;
qf.3:mq, l:=3.1, k1:=kqf;
qf.4:mq, l:=3.1, k1:=kqf;
qf.5:mq, l:=3.1, k1:=kqf;
qd.1:mq, l:=3.1, k1:=kqd;
qd.2:mq, l:=3.1, k1:=kqd;
qd.3:mq, l:=3.1, k1:=kqd;
qd.4:mq, l:=3.1, k1:=kqd;
qd.5:mq, l:=3.1, k1:=kqd;
bph:hmonitor, l:=l.bpm;
bpv:vmonitor, l:=l.bpm;
cbh:hkicker;
cbv:vkicker;
cbh.1:cbh, kick:=acbh1;
cbh.2:cbh, kick:=acbh2;
cbh.3:cbh, kick:=acbh3;
cbh.4:cbh, kick:=acbh4;
cbh.5:cbh, kick:=acbh5;
cbv.1:cbv, kick:=acbv1;
cbv.2:cbv, kick:=acbv2;
cbv.3:cbv, kick:=acbv3;
cbv.4:cbv, kick:=acbv4;
cbv.5:cbv, kick:=acbv5;
!mscbh:sextupole, l:=1.1, k2:=ksf;
mscbh:multipole, knl:={0,0,0,ksf},tilt=-pi/8;
mscbv:sextupole, l:=1.1, k2:=ksd;
!mscbv:octupole, l:=1.1, k3:=ksd,tilt=-pi/8;

// sequence declaration

fivecell:sequence, refer=centre, l=534.6;
   qf.1:qf.1, at=1.550000e+00;
   qft:qft, at=3.815000e+00;
!   mscbh:mscbh, at=3.815000e+00;
   cbh.1:cbh.1, at=4.365000e+00;
   mb:mb, at=1.262000e+01;
   mb:mb, at=2.828000e+01;
   mb:mb, at=4.394000e+01;
   bpv:bpv, at=5.246000e+01;
   qd.1:qd.1, at=5.501000e+01;
   mscbv:mscbv, at=5.727500e+01;
   cbv.1:cbv.1, at=5.782500e+01;
   mb:mb, at=6.608000e+01;
   mb:mb, at=8.174000e+01;
   mb:mb, at=9.740000e+01;
   bph:bph, at=1.059200e+02;
   qf.2:qf.2, at=1.084700e+02;
   mscbh:mscbh, at=1.107350e+02;
   cbh.2:cbh.2, at=1.112850e+02;
   mb:mb, at=1.195400e+02;
   mb:mb, at=1.352000e+02;
   mb:mb, at=1.508600e+02;
   bpv:bpv, at=1.593800e+02;
   qd.2:qd.2, at=1.619300e+02;
   mscbv:mscbv, at=1.641950e+02;
   cbv.2:cbv.2, at=1.647450e+02;
   mb:mb, at=1.730000e+02;
   mb:mb, at=1.886600e+02;
   mb:mb, at=2.043200e+02;
   bph:bph, at=2.128400e+02;
   qf.3:qf.3, at=2.153900e+02;
   mscbh:mscbh, at=2.176550e+02;
   cbh.3:cbh.3, at=2.182050e+02;
   mb:mb, at=2.264600e+02;
   mb:mb, at=2.421200e+02;
   mb:mb, at=2.577800e+02;
   bpv:bpv, at=2.663000e+02;
   qd.3:qd.3, at=2.688500e+02;
   mscbv:mscbv, at=2.711150e+02;
   cbv.3:cbv.3, at=2.716650e+02;
   mb:mb, at=2.799200e+02;
   mb:mb, at=2.955800e+02;
   mb:mb, at=3.112400e+02;
   bph:bph, at=3.197600e+02;
   qf.4:qf.4, at=3.223100e+02;
   mscbh:mscbh, at=3.245750e+02;
   cbh.4:cbh.4, at=3.251250e+02;
   mb:mb, at=3.333800e+02;
   mb:mb, at=3.490400e+02;
   mb:mb, at=3.647000e+02;
   bpv:bpv, at=3.732200e+02;
   qd.4:qd.4, at=3.757700e+02;
   mscbv:mscbv, at=3.780350e+02;
   cbv.4:cbv.4, at=3.785850e+02;
   mb:mb, at=3.868400e+02;
   mb:mb, at=4.025000e+02;
   mb:mb, at=4.181600e+02;
   bph:bph, at=4.266800e+02;
   qf.5:qf.5, at=4.292300e+02;
   mscbh:mscbh, at=4.314950e+02;
   cbh.5:cbh.5, at=4.320450e+02;
   mb:mb, at=4.403000e+02;
   mb:mb, at=4.559600e+02;
   mb:mb, at=4.716200e+02;
   bpv:bpv, at=4.801400e+02;
   qd.5:qd.5, at=4.826900e+02;
   mscbv:mscbv, at=4.849550e+02;
   cbv.5:cbv.5, at=4.855050e+02;
   mb:mb, at=4.937600e+02;
   mb:mb, at=5.094200e+02;
   mb:mb, at=5.250800e+02;
   bph:bph, at=5.336000e+02;
end:marker, at=5.346000e+02;
endsequence;

// forces and other constants

l.bpm:=.3;
bang:=.509998807401e-2;
kqf:=.872651312e-2;
kqd:=-.872777242e-2;
ksf:=.0198492943;
ksd:=-.039621283;
acbv1:=1.e-4;
acbh1:=1.e-4;
!save,sequence=fivecell,file,mad8;

s := table(twiss,bpv[5],betx);
myvar := sqrt(beam->ex*table(twiss,betx));
use, period=fivecell;
select,flag=twiss,column=name,s,myvar,apertype;
twiss,file;
n = 0;
create,table=mytab,column=dp,mq1,mq2;
mq1:=table(summ,q1);
mq2:=table(summ,q2);
while ( n < 11)
{
  n = n + 1;
  dp = 1.e-4*(n-6);
  twiss,deltap=dp;
  fill,table=mytab;
}
write,table=mytab;
plot,haxis=s,vaxis=aper_1,aper_2,colour=100,range=#s/cbv.1,notitle;
stop;
\end{verbatim}
prints the following user table on output:

\begin{verbatim}
@ NAME             %05s "MYTAB"
@ TYPE             %04s "USER"
@ TITLE            %08s "no-title"
@ ORIGIN           %16s "MAD-X 1.09 Linux"
@ DATE             %08s "10/12/02"
@ TIME             %08s "10.45.25"
* DP                 MQ1                MQ2                
$ %le                %le                %le                
 -0.0005            1.242535951        1.270211135       
 -0.0004            1.242495534        1.270197018       
 -0.0003            1.242452432        1.270185673       
 -0.0002            1.242406653        1.270177093       
 -0.0001            1.242358206        1.270171269       
 0                  1.242307102        1.27016819        
 0.0001             1.242253353        1.270167843       
 0.0002             1.242196974        1.270170214       
 0.0003             1.24213798         1.270175288       
 0.0004             1.242076387        1.270183048       
 0.0005             1.242012214        1.270193477       
\end{verbatim}
and produces a twiss file with the additional column myvar, as well as a plot
file with the aperture values plotted.


%\href{screate}{}

Example of joining two tables with different length into a third table
making use of the length of either table as given by
table("your\_table\_name", tablelength) and adding names by the "\_name"
attribute.

\begin{verbatim}
title,   "summing of offset and alignment tables";
set,    format="13.6f";

readtable, table=align,  file="align.ip2.b1.tfs";   // mesured alignment
readtable, table=offset, file="offset.ip2.b1.tfs";  // nominal offsets

n_elem  =  table(offset, tablelength);

create,  table=align_offset, column=_name,s_ip,x_off,dx_off,ddx_off,y_off,dy_off,ddy_off;

calcul(elem_name) : macro = {
    x_off = table(align,  elem_name, x_ali) + x_off;
    y_off = table(align,  elem_name, y_ali) + y_off;
}


one_elem(j_elem) : macro = {
    setvars, table=offset, row=j_elem;
    exec,  calcul(tabstring(offset, name, j_elem));
    fill,  table=align_offset;
}


i_elem = 0;
while (i_elem < n_elem) { i_elem = i_elem + 1; exec,  one_elem($i_elem); }

write, table=align_offset, file="align_offset.tfs";

stop;
\end{verbatim}


%% EOF



%%%%\title{Range Selection}
%  Changed by: Chris ISELIN, 27-Jan-1997 

%  Changed by: Hans Grote, 10-Jun-2002 

%%%\usepackage{hyperref}
% commands generated by html2latex


%%%\begin{document}
%%%\begin{center}
 %%%EUROPEAN ORGANIZATION FOR NUCLEAR RESEARCH 
%%%\includegraphics{http://cern.ch/madx/icons/mx7_25.gif}

\paragraph{Real life example for IF statements, and MACRO usage}
%%%\end{center}


\begin{verbatim}

! Creates a footprint for head-on + parasitic collisions at IP1+5 
! of lhc.6.5; both lhcb1 (for tracking) and lhcb2 (to define the
! beam-beam elements, i.e. weak-strong) are used; there are flags to
! select head-on, left, and right parasitic separately at all IPs.
! The bunch spacing can be given in nanosec and automatically creates
! the beam-beam interaction points at the correct positions.
! It is important to set the correct BEAM parameters, i.e. number
! of particles, emittances, bunch length, energy.

!--- For completeness, all files needed by this job are copied
!    to the local directory ldb. The links to the the originals
!    in offdb (official database) are commented out.

Option,  warn,info,echo;
!System,
"ln -fns /afs/cern.ch/eng/sl/MAD-X/dev/test_suite/foot/V3.01.01 ldb";
!system,"ln -fns /afs/cern.ch/eng/lhc/optics/V6.4 offdb";
Option, -echo,-info,warn;
SU=1.0;
call, file = "ldb/V6.5.seq";
call,file="ldb/slice_new.madx";
Option, echo,info,warn;

!+++++++++++++++++++++++++ Step 1 +++++++++++++++++++++++
! 	define beam constants
!++++++++++++++++++++++++++++++++++++++++++++++++++++++++

b_t_dist = 25.e-9;                  !--- bunch distance in [sec]
b_h_dist = clight * b_t_dist / 2 ;  !--- bunch half-distance in [m]
ip1_range = 58.;                     ! range for parasitic collisions
ip5_range = ip1_range;
ip2_range = 60.;
ip8_range = ip2_range;

npara_1 = ip1_range / b_h_dist;     ! # parasitic either side
npara_2 = ip2_range / b_h_dist;
npara_5 = ip5_range / b_h_dist;
npara_8 = ip8_range / b_h_dist;

value,npara_1,npara_2,npara_5,npara_8;

 eg   =  7000;
 bg   =  eg/pmass;
 en   = 3.75e-06;
 epsx = en/bg;
 epsy = en/bg;

Beam, particle = proton, sequence=lhcb1, energy = eg,
          sigt=      0.077     , 
          bv = +1, NPART=1.1E11, sige=      1.1e-4, 
          ex=epsx,   ey=epsy;

Beam, particle = proton, sequence=lhcb2, energy = eg,
          sigt=      0.077     , 
          bv = -1, NPART=1.1E11, sige=      1.1e-4, 
          ex=epsx,   ey=epsy;

beamx = beam%lhcb1->ex;   beamy%lhcb1 = beam->ey;
sigz  = beam%lhcb1->sigt; sige = beam%lhcb1->sige;

!--- split5, 4d
long_a= 0.53 * sigz/2;
long_b= 1.40 * sigz/2;
value,long_a,long_b;

ho_charge = 0.2;

!+++++++++++++++++++++++++ Step 2 +++++++++++++++++++++++
! 	slice, flatten sequence, and cycle start to ip3
!++++++++++++++++++++++++++++++++++++++++++++++++++++++++

use,sequence=lhcb1;
makethin,sequence=lhcb1;
!save,sequence=lhcb1,file=lhcb1_thin_new_seq;
use,sequence=lhcb2;
makethin,sequence=lhcb2;
!save,sequence=lhcb2,file=lhcb2_thin_new_seq;
!stop;

option,-warn,-echo,-info;
call,file="ldb/V6.5.thin.coll.str";
option,warn,echo,info;

! keep sextupoles
ksf0=ksf; ksd0=ksd;
use,period=lhcb1;
select,flag=twiss.1,column=name,x,y,betx,bety;
twiss,file;
plot,haxis=s,vaxis=x,y,colour=100,noline;

use,period=lhcb2;
select,flag=twiss.2,column=name,x,y,betx,bety;
twiss,file;
plot,haxis=s,vaxis=x,y,colour=100,noline;
seqedit,sequence=lhcb1;
flatten;
endedit;

seqedit,sequence=lhcb1;
cycle,start=ip3.b1;
endedit;

seqedit,sequence=lhcb2;
flatten;
endedit;

seqedit,sequence=lhcb2;
cycle,start=ip3.b2;
endedit;

bbmarker: marker;  /* for subsequent remove */


!+++++++++++++++++++++++++ Step 3 +++++++++++++++++++++++
! 	define the beam-beam elements
!++++++++++++++++++++++++++++++++++++++++++++++++++++++++
!
!===========================================================
! read macro definitions
option,-echo;
call,file="ldb/bb.macros";
option,echo;

!
!===========================================================
!   this sets CHARGE in the head-on beam-beam elements. 
!   set +1 * ho_charge   for parasitic on, 0 for off

 on_ho1  = +1 * ho_charge; ! ho_charge depends on split
 on_ho2  = +0 * ho_charge; ! because of the "by hand" splitting
 on_ho5  = +1 * ho_charge;
 on_ho8  = +0 * ho_charge;

!
!===========================================================
!   set CHARGE in the parasitic beam-beam elements. 
!   set +1 for parasitic on, 0 for off
 on_lr1l = +1;
 on_lr1r = +1;
 on_lr2l = +0;
 on_lr2r = +0;
 on_lr5l = +1;
 on_lr5r = +1;
 on_lr8l = +0;
 on_lr8r = +0;

!
!===========================================================
!   define markers and savebetas
assign,echo=temp.bb.install;
!--- ip1
if (on_ho1  0)
{
  exec, mkho(1);
  exec, sbhomk(1);
}
if (on_lr1l  0 || on_lr1r  0)
{
  n=1; ! counter
  while (n  0 || on_lr1l  0)
{
  n=1; ! counter
  while (n  0)
{
  exec, mkho(5);
  exec, sbhomk(5);
}
if (on_lr5l  0 || on_lr5r  0)
{
  n=1; ! counter
  while (n  0 || on_lr5l  0)
{
  n=1; ! counter
  while (n  0)
{
  exec, mkho(2);
  exec, sbhomk(2);
}
if (on_lr2l  0 || on_lr2r  0)
{
  n=1; ! counter
  while (n  0 || on_lr2l  0)
{
  n=1; ! counter
  while (n  0)
{
  exec, mkho(8);
  exec, sbhomk(8);
}
if (on_lr8l  0 || on_lr8r  0)
{
  n=1; ! counter
  while (n  0 || on_lr8l  0)
{
  n=1; ! counter
  while (n  0)
{
exec, inho(mk,1);
}
if (on_lr1l  0 || on_lr1r  0)
{
  n=1; ! counter
  while (n  0 || on_lr1l  0)
{
  n=1; ! counter
  while (n  0)
{
exec, inho(mk,5);
}
if (on_lr5l  0 || on_lr5r  0)
{
  n=1; ! counter
  while (n  0 || on_lr5l  0)
{
  n=1; ! counter
  while (n  0)
{
exec, inho(mk,2);
}
if (on_lr2l  0 || on_lr2r  0)
{
  n=1; ! counter
  while (n  0 || on_lr2l  0)
{
  n=1; ! counter
  while (n  0)
{
exec, inho(mk,8);
}
if (on_lr8l  0 || on_lr8r  0)
{
  n=1; ! counter
  while (n  0 || on_lr8l  0)
{
  n=1; ! counter
  while (n betx) / 0.0007999979093;
value,on_sep2;
!===========================================================
!   define bb elements
assign,echo=temp.bb.install;
!--- ip1
if (on_ho1  0)
{
exec, bbho(1);
}
if (on_lr1l  0)
{
  n=1; ! counter
  while (n  0)
{
  n=1; ! counter
  while (n  0)
{
exec, bbho(5);
}
if (on_lr5l  0)
{
  n=1; ! counter
  while (n  0)
{
  n=1; ! counter
  while (n  0)
{
exec, bbho(2);
}
if (on_lr2l  0)
{
  n=1; ! counter
  while (n  0)
{
  n=1; ! counter
  while (n  0)
{
exec, bbho(8);
}
if (on_lr8l  0)
{
  n=1; ! counter
  while (n  0)
{
  n=1; ! counter
  while (n  0)
{
exec, inho(bb,1);
}
if (on_lr1l  0)
{
  n=1; ! counter
  while (n  0)
{
  n=1; ! counter
  while (n  0)
{
exec, inho(bb,5);
}
if (on_lr5l  0)
{
  n=1; ! counter
  while (n  0)
{
  n=1; ! counter
  while (n  0)
{
exec, inho(bb,2);
}
if (on_lr2l  0)
{
  n=1; ! counter
  while (n  0)
{
  n=1; ! counter
  while (n  0)
{
exec, inho(bb,8);
}
if (on_lr8l  0)
{
  n=1; ! counter
  while (n  0)
{
  n=1; ! counter
  while (n  footprint";
stop;
\end{verbatim}

\paragraph{\href{macro}{Real life example of MACRO definitions}}

\begin{verbatim}

bbho(nn): macro = {
!--- macro defining head-on beam-beam elements; nn = IP number
print, text="bbipnnl2: beambeam, sigx=sqrt(rnnipnnl2->betx*epsx),";
print, text="          sigy=sqrt(rnnipnnl2->bety*epsy),";
print, text="          xma=rnnipnnl2->x,yma=rnnipnnl2->y,";
print, text="          charge:=on_honn;";
print, text="bbipnnl1: beambeam, sigx=sqrt(rnnipnnl1->betx*epsx),";
print, text="          sigy=sqrt(rnnipnnl1->bety*epsy),";
print, text="          xma=rnnipnnl1->x,yma=rnnipnnl1->y,";
print, text="          charge:=on_honn;";
print, text="bbipnn:   beambeam, sigx=sqrt(rnnipnn->betx*epsx),";
print, text="          sigy=sqrt(rnnipnn->bety*epsy),";
print, text="          xma=rnnipnn->x,yma=rnnipnn->y,";
print, text="          charge:=on_honn;";
print, text="bbipnnr1: beambeam, sigx=sqrt(rnnipnnr1->betx*epsx),";
print, text="          sigy=sqrt(rnnipnnr1->bety*epsy),";
print, text="          xma=rnnipnnr1->x,yma=rnnipnnr1->y,";
print, text="          charge:=on_honn;";
print, text="bbipnnr2: beambeam, sigx=sqrt(rnnipnnr2->betx*epsx),";
print, text="          sigy=sqrt(rnnipnnr2->bety*epsy),";
print, text="          xma=rnnipnnr2->x,yma=rnnipnnr2->y,";
print, text="          charge:=on_honn;";
};

mkho(nn): macro = {
!--- macro defining head-on markers; nn = IP number
print, text="mkipnnl2: bbmarker;";
print, text="mkipnnl1: bbmarker;";
print, text="mkipnn:   bbmarker;";
print, text="mkipnnr1: bbmarker;";
print, text="mkipnnr2: bbmarker;";
};

inho(xx,nn): macro = {
!--- macro installing bb or markers for head-on beam-beam (split into 5)
print, text="install, element= xxipnnl2, at=-long_b, from=ipnn;";
print, text="install, element= xxipnnl1, at=-long_a, from=ipnn;";
print, text="install, element= xxipnn,   at=1.e-9,   from=ipnn;";
print, text="install, element= xxipnnr1, at=+long_a, from=ipnn;"; 
print, text="install, element= xxipnnr2, at=+long_b, from=ipnn;"; 
};

sbhomk(nn): macro = {
!--- macro to create savebetas for ho markers
print, text="savebeta, label=rnnipnnl2, place=mkipnnl2;";
print, text="savebeta, label=rnnipnnl1, place=mkipnnl1;";
print, text="savebeta, label=rnnipnn,   place=mkipnn;";
print, text="savebeta, label=rnnipnnr1, place=mkipnnr1;";
print, text="savebeta, label=rnnipnnr2, place=mkipnnr2;";    
};

mkl(nn,cc): macro = {
!--- macro to create parasitic bb marker on left side of ip nn; cc = count
print, text="mkipnnplcc: bbmarker;";
};

mkr(nn,cc): macro = {
!--- macro to create parasitic bb marker on right side of ip nn; cc = count
print, text="mkipnnprcc: bbmarker;";
};

sbl(nn,cc): macro = {
!--- macro to create savebetas for left parasitic
print, text="savebeta, label=rnnipnnplcc, place=mkipnnplcc;";
};

sbr(nn,cc): macro = {
!--- macro to create savebetas for right parasitic
print, text="savebeta, label=rnnipnnprcc, place=mkipnnprcc;";
};

inl(xx,nn,cc): macro = {
!--- macro installing bb and markers for left side parasitic beam-beam
print, text="install, element= xxipnnplcc, at=-cc*b_h_dist, from=ipnn;";
};

inr(xx,nn,cc): macro = {
!--- macro installing bb and markers for right side parasitic beam-beam
print, text="install, element= xxipnnprcc, at=cc*b_h_dist, from=ipnn;";
};

bbl(nn,cc): macro = {
!--- macro defining parasitic beam-beam elements; nn = IP number
print, text="bbipnnplcc: beambeam, sigx=sqrt(rnnipnnplcc->betx*epsx),";
print, text="          sigy=sqrt(rnnipnnplcc->bety*epsy),";
print, text="          xma=rnnipnnplcc->x,yma=rnnipnnplcc->y,";
print, text="          charge:=on_lrnnl;";
};

bbr(nn,cc): macro = {
!--- macro defining parasitic beam-beam elements; nn = IP number
print, text="bbipnnprcc: beambeam, sigx=sqrt(rnnipnnprcc->betx*epsx),";
print, text="          sigy=sqrt(rnnipnnprcc->bety*epsy),";
print, text="          xma=rnnipnnprcc->x,yma=rnnipnnprcc->y,";
print, text="          charge:=on_lrnnr;";
};
\end{verbatim}\href{http://www.cern.ch/Hans.Grote/hansg_sign.html}{hansg}, June 17, 2002 

%%%\end{document}

%%%%\title{Range Selection}
%  Changed by: Chris ISELIN, 27-Jan-1997 
%  Changed by: Hans Grote, 16-Jan-2003 

\section{General Control Statements}

\subsection{ASSIGN}
\begin{verbatim}

assign, echo = "file_name", truncate;
\end{verbatim} 
where "file\_name"  is the name of an output file, or "terminal" and
trunctate specifies if the file must be trunctated when opened (ignored
for terminal). This allows switching the echo stream to a file or back,
but only for the commands value, show, and print. Allows easy
composition of future MAD-X input files. A real life example (always the
same) is to be found under \href{foot.html}{footprint example}.  

\subsection{CALL}
\begin{verbatim}

call, file = file_name;
\end{verbatim} 
where "file\_name"  is the name of an input file. This file will be read
until a "return;" statement, or until end\_of\_file; it may contain any
number of calls itself, and so on to any depth.  


\subsection{COGUESS}
\begin{verbatim}

coguess,tolerance=double,x=double,
       px=double,y=double,py=double,t=double,pt=double;
\end{verbatim} 
sets the required convergence precision in the closed orbit search
("tolerance", see as well Twiss command
\href{../twiss/twiss.html#tolerance}{tolerance}).  

The other parameters define a first guess for all future closed orbit
searches in case they are different from zero.  


\subsection{CREATE}
\begin{verbatim}

create,table=table,column=var1,var2,_name,...;
\end{verbatim} 
creates a table with the specified variables as columns. This table can
then be \hyperlink{fill}{fill}ed, and finally one can
\hyperlink{write}{write} it in TFS format. The attribute "\_name" adds
the element name to the table at the specified column, this replaces the
undocumented "withname" attribute that was not always working properly.  

See the \href{../Introduction/select.html#ucreate}{user table I}
example;  

or an example of joining 2 tables of different length in one table
including the element name:
\href{../Introduction/select.html#screate}{user table II} 




\subsection{DELETE}
\begin{verbatim}

delete,sequence=s_name,table=t_name;
\end{verbatim} 
deletes a sequence with name "s\_name" or a table with name "t\_name"
from memory. The sequence deletion is done without influence on other
sequences that may have elements that were in the deleted sequence.  


\subsection{DUMPSEQU}
\begin{verbatim}

dumpsequ, sequence = s_name, level = integer;
\end{verbatim} 
Actually a debug statement, but it may come handy at certain
occasions. Here "s\_name" is the name of an expanded (i.e. USEd)
sequence. The amount of detail is controlled by "level":  
\begin{verbatim}

level = 0:    print only the cumulative node length = sequence length
      > 0:    print all node (element) names except drifts
      > 2:    print all nodes with their attached parameters
      > 3:    print all nodes, and their elements with all parameters
\end{verbatim}


\subsection{EXEC}
\begin{verbatim}

exec, label;
\end{verbatim} 
Each statement may be preceded by a label; it is then stored and can be
executed again with "exec, label;" any number of times; the executed
statement may be another "exec", etc.; however, the major usage of this
statement is the execution of a \href{special.html#macro}{macro}.  


\subsection{EXIT}
\begin{verbatim}

exit;
\end{verbatim} 
ends the program execution. 


\subsection{FILL} 
Every command 
\begin{verbatim}

fill,table=table;
\end{verbatim} 
adds a new line with the current values of all column variables into the
user table \hyperlink{create}{create}d beforehand. This table one can
then \hyperlink{write}{write} in TFS format.  See as well the
\href{../Introduction/select.html#ucreate}{user table} example.  


\subsection{OPTION}
\begin{verbatim}

option, flag { = true | false };
option, flag | -flag;
\end{verbatim} 
sets an option as given in "flag"; the part in curly brackets is
optional: if only the name of the option is given, then the option will
be set true (see second line); a "-" sign preceding the name sets it to
"false".  

 Example: 
\begin{verbatim}

option,echo=true;
option,echo;
\end{verbatim} 
are identical, ditto 
\begin{verbatim}

option,echo=false;
option,-echo;
\end{verbatim} 
The available options are: 
\begin{verbatim}

  name           default meaning if true
  ====           ======= ===============
  bborbit         false  the closed orbit is modified by beam-beam kicks
  sympl           false  all element matrices are symplectified in Twiss
  echo            true   echoes the input on the standard output file
  trace           false  prints the system time after each command
  verify          false  issues a warning if an undefined variable is used
  warn            true   issues warnings
  info            true   issues informations
  tell            false  prints the current value of all options
  reset           false  resets all options to their defaults
  rbarc           true   converts the RBEND straight length into the arc length
  thin_foc        true   if false suppresses the 1(rho**2) focusing of thin dipoles
  no_fatal_stop   false  Prevents madx from stopping in case of a fatal error. Use at your own risk.
\end{verbatim} 
The option "rbarc" is implemented for backwards compatibility with MAD-8
up to version 8.23.06 included; in this version, the RBEND length was
just taken as the arc length of an SBEND with inclined pole faces,
contrary to the MAD-8 manual.  


\subsection{PRINT}
\begin{verbatim}

print,text="...";
\end{verbatim} 
prints the text to the current output file (see ASSIGN above). The text
can be edited with the help of a  \href{special.html#macro}{macro
  statement}. For more details, see there.  


\subsection{QUIT}
\begin{verbatim}

quit;
\end{verbatim} 
ends the program execution. 


\subsection{READTABLE}
\begin{verbatim}

readtable,file=filename;
\end{verbatim} 
reads a TFS file containing a MAD-X table back into memory. This table
can then be manipulated as any other table, i.e. its values can be
accessed, it can be plotted, written out again etc.  


\subsection{READMYTABLE}
\begin{verbatim}

readmytable,file=filename,table=internalname;
\end{verbatim} 
reads a TFS file containing a MAD-X table back into memory. This table
can then be manipulated as any other table, i.e. its values can be
accessed, it can be plotted, written out again etc. An internal name for
the table can be freely assigned while for the command READTABLE it is
taken from the information section of the table itself. This feature
allows to store multiple tables of the same type in memory without
overwriting existing ones.  


\subsection{REMOVEFILE}
\begin{verbatim}

removefile,file=filename;
\end{verbatim} 
remove a file from the disk. It is more portable than  
\begin{verbatim}

system("rm filename"); // Unix specific
\end{verbatim}


\subsection{RENAMEFILE}
\begin{verbatim}

renamefile,file=filename,name=newfilename;
\end{verbatim} 
rename the file "filename" to "newfilename" on the disk. It is more
portable than  
\begin{verbatim}

system("mv filename newfilename"); // Unix specific
\end{verbatim}


\subsection{RESBEAM}
\begin{verbatim}

resbeam,sequence=s_name;
\end{verbatim} 
resets the default values of the beam belonging to sequence s\_name, or
of the default beam if no sequence is given.  


\subsection{RETURN}
\begin{verbatim}

return;
\end{verbatim} 
ends reading from a "called" file; if encountered in the standard input
file, it ends the program execution.  


\subsection{SAVE}
\begin{verbatim}

save,beam,sequence=sequ1,sequ2,...,file=filename,beam,bare;
\end{verbatim} 
saves the sequence(s) specified with all variables and elements needed
for their expansion, onto the file "filename". If quotes are used for
the "filename" capital and low characters are kept as specified, if they
are omitted the "filename" will have lower characters only. The optional
flag can have the value "mad8" (without the quotes), in which case the
sequence(s) is/are saved in MAD-8 input format.  

The flag "beam" is optional; when given, all beams belonging to the
sequences specified are saved at the top of the save file.  

The parameter "sequence" is optional; when omitted, all sequences are
saved.  

However, it is not advisable to use "save" without the "sequence" option
unless you know what you are doing. This practice will avoid spurious
saved entries.    Any number of "select,flag=save" commands may precede
the SAVE command. In that case, the names of elements, variables, and
sequences must match the pattern(s) if given, and in addition the
elements must be of the class(es) specified. See here for a
\href{../Introduction/select.html#save_select}{SAVE with SELECT}
example.  

It is important to note that the precision of the output of the save
command depends on the output precision. Details about default
precisions and how to adjust those precisions can be found at the
\href{../Introduction/set.html#Format}{SET Format} instruction page.   
 
The Attribute 'bare' allows to save just the sequence without the
element definitions nor beam information. This allows to re-read in a
sequence with might otherwise create a stop of the program. This is
particularly useful to turn a line into a sequence to seqedit
it. Example:  
\begin{verbatim}

tl3:line=(ldl6,qtl301,mqn,qtl301,ldl7,qtl302,mqn,qtl302,ldl8,ison);
DLTL3 : LINE=(delay, tl3);
use, period=dltl3;

save,sequence=dltl3,file=t1,bare; // new parameter "bare": only sequ. saved
call,file=t1; // sequence is read in and is now a "real" sequence
// if the two preceding lines are suppressed, seqedit will print a warning
// and else do nothing
use, period=dltl3;
twiss, save, betx=bxa, alfx=alfxa, bety=bya, alfy=alfya;
plot, vaxis=betx, bety, haxis=s, colour:=100;
SEQEDIT, SEQUENCE=dltl3;
  remove,element=cx.bhe0330;
  remove,element=cd.bhe0330;
ENDEDIT;

use, period=dltl3;
twiss, save, betx=bxa, alfx=alfxa, bety=bya, alfy=alfya;
\end{verbatim}


\subsection{SAVEBETA}
\begin{verbatim}

savebeta, label=label,\href{place}{place}=place,sequence=s_name;
\end{verbatim} 
marks a place "place" in an expanded sequence "s\_name"; at the next
TWISS command execution, a  \href{../twiss/twiss.html#beta0}{beta0}
block will be saved at that place with the label "label". This is done
only once; in order to get a new beta0 block there, one has to re-issue
the command. The contents of the beta0 block can then be used in other
commands, e.g. TWISS and MATCH.  

 Example (after sequence expansion): 
\begin{verbatim}

savebeta,label=sb1,place=mb[5],sequence=fivecell;
twiss;
show,sb1;
\end{verbatim} 
will save and show the beta0 block parameters at the end (!) of the
fifth element mb in the sequence.  


\subsection{SELECT} %select</a}{SELECT}
\begin{verbatim}

select, flag=flag,range=range,class=class,pattern=pattern,
        slice=integer,column=s1,s2,s3,..,sn,sequence=s_name,
        full,clear;
\end{verbatim} 
selects one or several elements for special treatment in a subsequent
command. All selections for a given command remain valid until "clear"
is specified; the selection criteria on the same command are logically
ANDed, on different SELECT statements logically ORed.  

 Example: 
\begin{verbatim}

select,flag=error,class=quadrupole,range=mb[1]/mb[5];
select,flag=error,pattern="^mqw.*";
\end{verbatim} 
selects all quadrupoles in the range mb[1] to mb[5], and all elements
(in the whole sequence) the name of which starts with "mqw" for
treatment by the error module.  

"flag" can be one of the following:: 
\begin{itemize}
	\item seqedit: selection of elements for the
          \href{seqedit.html}{seqedit} module.  
	\item error: selection of elements for the
          \href{../error/error.html}{error} assignment module.  
	\item makethin: selection of elements for the
          \href{../makethin/makethin.html}{makethin} module that
          converts the sequence into one with thin elements only.  
	\item sectormap: selection of elements for the
          \href{../Introduction/sectormap.html}{sectormap} output file
          from the Twiss module.  
	\item table: here "table" is a table name such as twiss, track
          etc., and the rows and columns to be written are selected.  
\end{itemize} For the RANGE, CLASS, PATTERN, FULL, and CLEAR parameters
see \href{../Introduction/select.html}{SELECT}.  

"slice" is only used by \href{../makethin/makethin.html}{makethin} and
prescribes the number of slices into which the selected elements have to
be cut (default = 1).  

"column" is only valid for tables and decides the selection of columns
to be written into the TFS file. The "name" argument is special in that
it refers to the actual name of the selected element. For an example,
see \href{../Introduction/select.html}{SELECT}.  


\subsection{SHOW}
\begin{verbatim}

show,command;
\end{verbatim} 
prints the "command" (typically "beam", "beam\%sequ", or an element
name), with the actual value of all its parameters.  


\subsection{STOP}
\begin{verbatim}

stop;
\end{verbatim} 
ends the program execution. 


\subsection{SYSTEM}
\begin{verbatim}

system,"...";
\end{verbatim} 
transfers the string in quotes to the system for execution.  

Example: 
\begin{verbatim}

system,"ln -s /afs/cern.ch/user/u/user/public/some/directory short";
\end{verbatim}


\subsection{TABSTRING}
\begin{verbatim}

tabstring(arg1,arg2,arg3)
\end{verbatim}  
The"string function" tabstring(arg1,arg2,arg3) with exactly  three
arguments; arg1 is a table name (string), arg2 is a column name
(string), arg3 is a row number (integer), count starts at 0. The
function can be used in any context where a string appears; in case
there is no match, it returns \_void\_.  


\subsection{TITLE}
\begin{verbatim}

title,"...";
\end{verbatim} 
inserts the string in quotes as title in various tables and plots.  


\subsection{USE}
\begin{verbatim}

use,period=s_name,range=range,survey;
\end{verbatim} 
expands the sequence with name "s\_name", or a part of it as specified
in the \href{../Introduction/ranges.html#range}{range}. The
\texttt{survey} option plugs the survey data into the sequence elements
nodes on the first pass (see \href{../survey/survey.html}{survey}).  


\subsection{VALUE}
\begin{verbatim}

value,exp1,exp2,...;
\end{verbatim} 
prints the actual values of the expressions given. 

Example: 
\begin{verbatim}

a=clight/1000.;
value,a,pmass,exp(sqrt(2));
\end{verbatim} results in 
\begin{verbatim}

a = 299792.458         ;
pmass = 0.938271998        ;
exp(sqrt(2)) = 4.113250379        ;
\end{verbatim}


\subsection{WRITE}
\begin{verbatim}

write,table=table,file=file_name;
\end{verbatim} 
writes the table "table" onto the file "file\_name"; only the rows and
columns of a preceding select,flag=table,...; are written. If no select
has been issued for this table, the file will only contain the
header. If the FILE argument is omitted, the table is written to
standard output.  


%\href{http://www.cern.ch/Hans.Grote/hansg_sign.html}{hansg}, June 17, 2002 

%%%%\title{Range Selection}
%  Changed by: Chris ISELIN, 27-Jan-1997 
%  Changed by: Hans Grote, 30-Sep-2002 

\section{Program Flow Statements}

\subsection{IF}
\begin{verbatim}
if (logical_expression) {statement 1; statement 2; ...; statement n; }
\end{verbatim}
where \href{logical}{"logical\_expression"} is one of 
\begin{verbatim}
expr1  oper expr2
expr11 oper1 expr12 && expr21 oper2 expr22
expr11 oper1 expr12 || expr21 oper2 expr22
\end{verbatim} 
and \verb+oper+ one of 
\begin{verbatim}
==          ! equal
<>          ! not equal
<           ! less than
>           ! greater than
<=          ! less than or equal
>=          ! greater than or equal
\end{verbatim} 
The expressions are arithmetic expressions of type real. The statements
in the curly brackets are executed if the logical expression is true.  


\subsection{ELSEIF}
\begin{verbatim}
elseif (logical_expression) {statement 1; statement 2; ...; statement n; }
\end{verbatim} 
Only possible (in any number) behind an IF, or another ELSEIF; is
executed if  logical\_expression is true, and if none of the preceding
IF or ELSEIF logical conditions was true.  


\subsection{ELSE}
\begin{verbatim}
else {statement 1; statement 2; ...; statement n; }
\end{verbatim} 
Only possible (once) behind an IF, or an ELSEIF; is executed if
logical\_expression is true, and if none of the preceding IF or ELSEIF
logical conditions was true.  

For a real life example, see \href{foot.html}{ELSE example}. 


\subsection{WHILE}
\begin{verbatim}
while (logical_condition) {statement 1; statement 2; ...; statement n; }
\end{verbatim}  
executes the statements in curly brackets while the logical\_expression
is true. A simple example (in case you have forgotten the first ten
factorials) would be  
\begin{verbatim}
option, -info;   ! avoids redefiniton warnings
n = 1; m = 1;
while (n <= 10)
{
  m = m * n;  value, m;
  n = n + 1;
};
\end{verbatim}

For a real life example, see \href{foot.html}{WHILE example}.

\subsection{MACRO}
\begin{verbatim}
label: macro = {statement 1; statement 2; ...; statement n; };
label(arg1,...,argn): macro = {statement 1; statement 2; ...; statement n; };
\end{verbatim} 
The first form allows the execution of a group of statements via a
single command,  
\begin{verbatim}
exec, label;
\end{verbatim} 
that executes the statements in curly brackets exactly once. This command
can be issued any number of times.  

The second form allows to replace strings anywhere inside the statements
in curly brackets by other strings, or integer numbers prior to
execution. This is a powerful construct and should be handled with care.  

Simple example: 
\begin{verbatim}
option, -echo, -info;  ! for cleaner output
simple(xx,yy): macro = { xx = yy^2 + xx; value, xx;};
a = 3;
b = 5;
exec, simple(a,b);
\end{verbatim}


{\bf Passing arguments}\\
In the following example we use the fact that a "\$" in front of an
argument means that the truncated integer value of this argument is used
for replacement, rather than the argument string itself.  
\begin{verbatim}
tricky(xx,yy,zz): macro = {mzz.yy: xx, l = 1.yy, kzz = k.yy;};
n=0;
while (n < 3)
{
  n = n+1;
  exec, tricky(quadrupole, $n, 1);
  exec, tricky(sextupole, $n, 2);
};
\end{verbatim} 
Whereas the actual use of the preceding example is NOT recommended,
a real life example, showing the full power (!) of macros is to be
found under \href{foot.html}{macro usage} for the usage, and
under \href{foot.html#macro}{macro definition} for the
definition.


{\bf Beware of the following rules:}
\begin{itemize}
   \item Generally speaking: \textit{ special constructs } like IF,
     WHILE, MACRO will only allow one level of inclusion of another
     \textit{ special construct }.
   \item  Macros must not be called with numbers, but with strings
     (i.e. variable names in case of numerical values), i.e. {\bf NOT }
\begin{verbatim}
exec, thismacro($99, $129);
\end{verbatim}
{\bf BUT}
\begin{verbatim}
n1=99; n2=219;
exec, thismacro($n1, $n2);
\end{verbatim}
\end{itemize}

%\href{http://www.cern.ch/Hans.Grote/hansg_sign.html}{hansg}, June 17, 2002




\chapter{General Control Statements} 

\madx consists of a core program, and modules for specific tasks such as
\hyperref[chap:twiss]{twiss parameter calculation},
\hyperref[chap:match]{matching}, \hyperref[chap:thintrack]{thin lens
  tracking}, \textsl{etc.}   
 
The statements listed here are those executed by the program core.
They deal with the I/O, element and sequence declaration, sequence
manipulation, statement flow control (e.g. \texttt{IF, WHILE}),
\texttt{MACRO} declaration, saving sequences onto files in \madx or
\madeight format, \textsl{etc.}  


%% Moved to TWISS chapter
%% \subsection{COGUESS}
%% \label{subsec:coguess}

%% In order to help the initial finding of the closed orbit by the
%% \texttt{TWISS} module, it is possible to specify an initial guess. 

%% \madbox{
%% COGUESS, \=TOLERANCE=real, \\
%%          \>X=real, PX=real, Y=real, PY=real, T=real, PT=real, \\
%%          \>CLEAR=logical;
%% }
%% sets the required convergence precision in the closed orbit search
%% ("tolerance", see as well Twiss command
%% \href{../twiss/twiss.html#tolerance}{tolerance}).  

%% The other parameters define a first guess for all future closed orbit
%% searches in case they are different from zero.  

%% The clear parameter in the argument list resets the tolerance to its default value 
%% and cancels the effect of coguess in defining a first guess for subsequent 
%% closed orbit searches. \\
%% Default = false, \texttt{clear} alone is equivalent to \texttt{clear=true}


\section{EXIT, QUIT, STOP}
\label{sec:exit}\label{sec:quit}\label{sec:stop}
Any of these three commands ends the execution of \madx:
\madbox{
EXIT;
}
\madbox{
QUIT;
}
\madbox{
STOP;
}


\section{HELP}
\label{sec:help}
The \texttt{HELP} command prints all parameters, and their defaults
values, for the statement given; this includes basic element types.
\madbox{
HELP, statement\_name;
}

\section{SHOW}
\label{sec:show}
The \texttt{SHOW} command prints the \texttt{command} (typically
\texttt{beam}, \texttt{beam\%sequ}, or an element name), with the actual
value of all its parameters.   
\madbox{
SHOW, command;
}

\section{VALUE}
\label{sec:value}
The \texttt{VALUE} command evaluates the current value of all listed
expressions, constants or variables, and prints the result in the form
of \madx statements on the assigned output file. 
\madbox{
VALUE, expression\{, expression\} ;
}

Example: \\
\madxmp{
a = clight/1000.; \\
value, a, pmass, exp(sqrt(2));
}
results in 
\madxmp{
a = 299792.458         ; \\
pmass = 0.938272046        ; \\
exp(sqrt(2)) = 4.113250379        ; \\
}

\section{OPTION}
\label{sec:option}

The \texttt{OPTION} commands sets the logical value of a number of flags
that control the behavior of \madx.

\madbox{
OPTION, flag=logical;
}

Because all attributes of \texttt{OPTION} are logical flags, the
following two statements are identical:
\madxmp{
OPTION, flag = true;\\
OPTION, flag;
}
And the following two statements are also identical:
\madxmp{
OPTION, flag = false;\\
OPTION, -flag;
}

Several flags can be set in a single \texttt{OPTION} command, e.g.
\madxmp{
OPTION, ECHO, WARN=true, -INFO, VERBOSE=false;
}

The available flags, their default values and their effect on \madx when
they are set to \texttt{TRUE} are listed in table \ref{table:options}. Note that to obtain the proper physics in \hyperref[sec:track]{\texttt{TRACK}} with \texttt{BBORBIT} set to false, one must enable the search for the closed orbit (i.e. not use \hyperref[sec:track]{\texttt{ONEPASS}}).

\begin{table}[ht]
  \caption{Flags available to \texttt{OPTION} command}
  \vspace{1ex}
  \centering
  \label{table:options}
  \begin{tabular}{|l|c|l|}
    \hline
    \textbf{FLAG }  & \textbf{default} & \textbf{effect if \texttt{TRUE}} \\
    \hline
    \texttt{ECHO}      & true  & echoes the input on the standard output file \\
    \texttt{WARN}      & true  & issues warning statements\\
    \texttt{INFO}      & true  & issues information statements\\
    \texttt{DEBUG}     & false & issues debugging information \\
    \texttt{ECHOMACRO} & false & issues macro expansion printout for debugging \\
    \texttt{VERBOSE}   & false & issues additional printout in makethin \\
    \texttt{TRACE}     & false & prints the system time after each command \\
    \texttt{VERIFY}    & false & issues a warning if an undefined variable is used 
    \\
    \hline
    \texttt{TELL}      & false & prints the current value of all options \\
    \texttt{RESET}     & false & resets all options to their defaults \\
    \hline
    \texttt{NO\_FATAL\_STOP} & false & Prevents madx from stopping in case of a fatal error \\
    &       & \textbf{Use at your own risk !} \\
    \hline
    \texttt{RBARC}     & true & converts the RBEND straight length into the arc length \\
    \texttt{THIN\_FOC} & true & enables the $1/\rho^2$ focusing of thin dipoles \\
    \texttt{BBORBIT}   & false & the closed orbit is modified by beam-beam kicks \\
    \texttt{SYMPL}     & false & all element matrices are symplectified in Twiss \\
    \texttt{TWISS\_PRINT} & true & controls whether the twiss command produces output \\
    \texttt{THREADER}  & false & enables the threader for closed orbit finding in Twiss \\ 
    &       & (see Twiss module) \\ 
    \hline
  \end{tabular}
\end{table}

The option \texttt{RBARC} is implemented for backwards compatibility
with \madeight up to version 8.23.06 included; in this version, the
\texttt{RBEND} length was just taken as the arc length of an
\texttt{SBEND} with inclined pole faces, contrary to the \madeight manual.  



\section{EXEC}
\label{sec:exec}
Each statement may be preceded by a label, when parsed and executed the
statement is then also stored and can be executed again with
\madbox{
EXEC, label;
}
This mechanism can be invoked any number of times, and the executed
statement may be calling another \texttt{EXEC}, etc. 
\madxmp{
tw: TWISS, FILE, SAVE; ! first execution of TWISS \\
... \\
EXEC, tw; ! second execution of the same TWISS command \\
}
Note however, that the main usage of this \madx construct is the
execution of a \hyperref[sec:macro]{\texttt{MACRO}}.   

\section{SET}
\label{sec:set}
The \texttt{SET} command is used in two forms:
\madbox{
SET, FORMAT=string \{, string\} ;\\
SET, SEQUENCE=string;
}


The first form of the \texttt{SET} command defines the formats for the
output precision that \madx uses with the \texttt{SAVE}, \texttt{SHOW},
\texttt{VALUE} and \texttt{TABLE} commands. The formats can be
given in any order and stay valid until replaced. 

The formats follow the C convention and must be included in double
quotes. The allowed formats are \\
\textit{n}\texttt{d} for integers with a field-width = \textit{n}, \\
\textit{n.m}\texttt{f} or \textit{n.m}\texttt{g} or
\textit{n.m}\texttt{e} for floats with field-width = \textit{n}
and precision = \textit{m}, \\
\textit{n}\texttt{s} for strings with a field-width = \textit{n}.\\
The default is "right adjusted", a "-" changes it to "left adjusted".

\textbf{Example:}\\
\madxmp{
SET, FORMAT="12d", "-18.5e", "25s";
}

%% \begin{verbatim}
%% "nd" for integer with n = field width.
%% \end{verbatim}
%% \begin{verbatim}
%% "m.nf" or "m.ng" or "m.ne" for floating, m field width, n precision.
%% \end{verbatim}
%% \begin{verbatim}
%% "ns" for string output.
%% \end{verbatim} 


The default formats are \texttt{"10d", "18.10g"} and \texttt{"-18s"}.

Example: 
\begin{verbatim}
set, format="22.14e";
\end{verbatim} 
changes the current floating point format to \texttt{22.14e}; the other
formats remain unchanged.  
\begin{verbatim}
set, format="s","d","g";
\end{verbatim} 
sets all formats to automatic adjustment according to C conventions. 

The second form of the \texttt{SET} command allows to select the
current sequence without the \hyperref[sec:use]{\texttt{USE}} command,
which would bring back to a bare lattice without errors. The command
only works 
if the chosen sequence has been previously activated with a
\hyperref[sec:use]{\texttt{USE}} 
command, otherwise a warning is issued and \madx continues with the
unmodified current sequence. This command is particularly useful for
commands that do not have the sequence as an argument like
\hyperref[chap:emit]{\texttt{EMIT}} or \hyperref[chap:ibs]{\texttt{IBS}}. 



\section{SYSTEM}
\label{sec:system}
\madbox{
SYSTEM, "string";
}
transfers the quoted \texttt{string} to the operating system for
execution. The quotes are stripped and no check of the return status is
performed by \madx. 

\textbf{Example:} 
\madxmp{
SYSTEM, "ln -s /afs/cern.ch/user/j/joe/input shortname";
}
makes a local link to an AFS directory with the name \texttt{shortname}
on a \texttt{UNIX} system.  

\textbf{Attention:} Although this command is very convenient, it is
clearly not portable across systems and it should probably be avoided if
one intends to share \madx scripts with collaborators working on other
platforms.  

\section{TITLE}
\label{sec:title}
\madbox{
TITLE, "string";
}
defines a \texttt{string} that is inserted as title in various table
outputs and plot results.  


\section{USE}
\label{sec:use}
\madx operates on beamlines that must be loaded and expanded in memory
before other commands can be invoked. The \texttt{USE} command allows
this loading and expansion.

\madbox{
USE, \=SEQUENCE=sequence\_name, PERIOD=sequence\_name,\\
     \>RANGE=range, \\
     \>SURVEY=logical;
}

The attributes to the \texttt{USE} command are:
\begin{madlist}
  \ttitem{SEQUENCE} name of the sequence to be loaded and expanded. 
  \ttitem{PERIOD} name of the sequence to be loaded and expanded. \\ 
  \texttt{PERIOD} is an alias to \texttt{SEQUENCE} that was kept for
  backwards compatibility with \madeight and only one of them should be
  specified in a \texttt{USE} statement. 
  \ttitem{RANGE} specifies a \hyperref[sec:range]{range}.   
  restriction so that only the specified part of the named sequence is
  loaded and  expanded.
  \ttitem{SURVEY} option to plug the survey data into the sequence elements
  nodes on the first pass (see \hyperref[chap:survey]{\texttt{SURVEY}}).
\end{madlist}

Note that reloading a sequence with the \texttt{USE} command reloads a
bare sequence and that any \hyperref[chap:error]{\texttt{ERROR}} or
orbit correction previously assigned or associated to the sequence are
discarded. A mechanism to select a sequence without this side effect of the 
\texttt{USE} command is provided with the
\hyperref[sec:set]{\texttt{SET, SEQUENCE=...}} command. 


\section{SELECT} 
\label{sec:select}

Some \madx commands can perform specific operations on selected elements
or ranges of elements and can produce specific output for selected
elements or ranges of elements. 

The selection is made through the \texttt{SELECT} command and applies to
subsequent commands.

\madbox{
SELECT, \=FLAG=string, RANGE=string, CLASS=string, PATTERN=string, \\
        \>SEQUENCE=string, FULL=logical, CLEAR=logical,\\
        \>COLUMN=string\{,string\},  SLICE=integer, THICK=logical, \\
        \>STEP=real, AT=\{real, \ldots \};
} 

The attributes to the \texttt{SELECT} command are:
\begin{madlist}
  \ttitem{FLAG} determines the applicability of the \texttt{SELECT}
  statement and the attribute value can be one of the following: 
  \begin{madlist}
    \ttitem{SEQEDIT} selection of elements for the
    \hyperref[sec:seqedit]{\texttt{SEQEDIT}} module.
    
    \ttitem{ERROR} selection of elements for the
    \hyperref[chap:error]{error assignment} module.
    
    \ttitem{MAKETHIN} selection of elements for the
    \hyperref[chap:makethin]{\texttt{MAKETHIN}} command that
    converts the sequence into one with thin elements.
    
    \ttitem{SECTORMAP} selection of elements for the
    \hyperref[sec:sectormap]{\texttt{SECTORMAP}} output file
    from the \hyperref[chap:twiss]{\texttt{TWISS}} module.
    
    \ttitem{SAVE} selection of elements for the
    \hyperref[sec:save]{\texttt{SAVE}} command.

    \ttitem{INTERPOLATE} selection of interpolation points for the
    \hyperref[chap:twiss]{\texttt{TWISS}} command.

    \ttitem{tablename} is a table name such as \texttt{twiss}, 
    \texttt{track} etc., and the rows and columns to be written are
    selected.
  \end{madlist} 
  
  \ttitem{RANGE} the range of elements to be selected as defined in
  section \ref{sec:range} on \hyperref[sec:range]{range} selection.

  \ttitem{CLASS} the class of elements to be selected as defined in
  section \ref{sec:class} on \hyperref[sec:class]{class} selection.

  \ttitem{PATTERN} the regular expression pattern for the element names
  to be selected as defined in section \ref{sec:regex} on selection via
  \hyperref[sec:regex]{regular expressions}. 

  \ttitem{SEQUENCE} the name of a sequence to which the selection is applied.

  \ttitem{FULL} a logical falg to select ALL positions in the sequence
  for the named flag. \\
  For the flag \texttt{TWISS}, this attribute includes all standard
  columns for a \texttt{TWISS} table, and therefore the following two
  statements are equivalent:
  \madxmp{
SELECT, FLAG=twiss, COLUMN= name, s, betx, ..., var1; \\
SELECT, FLAG=twiss, FULL, COLUMN= var1; 
  } 
  \texttt{FULL=true} is the default for the \texttt{MAKETHIN} flag and
  for tables: \textsl{e.g.} \texttt{SELECT, FLAG=makethin;} is
  equivalent to \texttt{SELECT, FLAG=makethin, FULL;}   
  
  \ttitem{CLEAR} deselects ALL positions in the sequence for the flag
  "name". This is the default for \texttt{ERROR} and \texttt{SEQEDIT}
  flags. \\
  \textsl{e.g.} \texttt{SELECT, FLAG=error;} is equivalent to
  \texttt{SELECT, FLAG=error, CLEAR;} 

  \ttitem{COLUMN} is only valid for tables and takes as attribute value
  a list of columns to be written into the TFS file. The special "\_name"
  argument refers to the actual name of the element. 
  %For an example, see \hyperref[sec:select]{\texttt{SELECT}}.  

  \ttitem{SLICE} is the number of slices into which the selected
  elements have to be cut and is only used by
  \hyperref[chap:makethin]{\texttt{MAKETHIN}} and
  \hyperref[chap:twiss]{\texttt{FLAG=INTERPOLATE}}. (Default = 1).

  \ttitem{THICK} is a logical flag to indicate  whether the selected
  elements are treated as thick elements by the
  \hyperref[chap:makethin]{\texttt{MAKETHIN}} command. \\  
  This only applies up to now to
  \hyperref[sec:quadrupole]{\texttt{QUADRUPOLE}}s and
  \hyperref[sec:bend]{\texttt{BEND}}s for which thick maps
  have been explicitely derived. 

  \ttitem{STEP} output intermediate values every \texttt{STEP} meters
  in the \hyperref[chap:twiss]{\texttt{TWISS}} command.

  \ttitem{AT} manual specification of interpolation points for the
  \hyperref[chap:twiss]{\texttt{TWISS}} command. Specified as a
  fraction of the node length, i.e. a value of 0.5 slices at
  the centre of the element.
\end{madlist}

\vskip 5mm
\textbf{Composition of \texttt{SELECT} statements:} \\
The selection criteria provided on a single \texttt{SELECT} statement
are logically \texttt{AND}ed, \textsl{i.e.} selected elements have to
fulfill \textbf{all} provided criteria in the single \texttt{SELECT}
statement. 

The selection criteria on different \texttt{SELECT} statements are
logically \texttt{OR}ed, \textsl{i.e.} selected elements have to fulfill
\textbf{any} of the selection criteria provided by the different
\texttt{SELECT} statements. 

All selections for a given flag remain valid until a \texttt{SELECT}
statement with the \texttt{CLEAR} argument is specified for the same flag.

Note that because of these composition rules, it is considered good
practice to start by clearing the selection for a given flag before
making a new selection, eg: 
\madxmp{ 
SELECT, FLAG=twiss, CLEAR; \\
SELECT, FLAG=twiss, CLASS=MQ; \\
SELECT, FLAG=twiss, RANGE=MQ[5]/MQ[7]; \\
...
}


\vskip 5mm
\textbf{Examples:} 
\madxmp{ 
SELECT, FLAG = ERROR, CLASS = quadrupole, RANGE = mb[1]/mb[5];\\
SELECT, FLAG = ERROR, PATTERN = "\textasciicircum mqw.*";
}
selects all quadrupoles in the range mb[1] to mb[5], as well as all
elements (in the whole sequence) with name starting with "mqw", for 
treatment by the \hyperref[chap:error]{\texttt{ERROR}} module.  

\vskip 5mm
\madxmp{
SELECT, FLAG=SAVE, CLASS=variable, PATTERN="abc.*"; \\
SAVE, FILE=mysave;
}
saves all variables containing "abc" in their name,
but does not save elements with names containing "abc" since the class
"variable" does not exist.  

\vskip 5mm
\madxmp{
sig1 := sqrt(beam->ex*table(twiss,betx)); \\
SELECT, FLAG=twiss, COLUMN= \_name, s, betx, ..., sig1; ! or equivalently \\
SELECT, FLAG=twiss, FULL, COLUMN= sig1; ! default columns + new
}
writes the current value of ``sig1'' into the \texttt{TWISS} table each
time a new line is added; Note that the values from the same (current)
line can be are accessed by the variable ``sig1''.
The \hyperref[chap:plot]{\texttt{PLOT}} command also accepts the new variable 
in the table.  

%% EOF


