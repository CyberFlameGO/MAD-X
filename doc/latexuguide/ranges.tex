%%\title{Range Selection}
%  Changed by: Chris ISELIN, 27-Jan-1997 
%  Changed by: Hans Grote, 10-Jun-2002 

\chapter{Range and Class Selection Format}

\section{RANGE}
\label{sec:range}

A range can be defined starting at  a given element and ending at
another element, both elements included. Two forms exist:  

\madxmp{
RANGE = position; \\
RANGE = position1 / position2;
} 

In the first case, only one element is selected; in the second case, one
or several elements that are between \texttt{position1} and
\texttt{position2} are selected. \\
NOTE: \texttt{position1} must appear before \texttt{position2} in the
sequence.   

The \texttt{position} is composed of a single element name if that name
is unique in the sequence, or an element class name followed by the
occurrence count \texttt{n} in the sequence within square brackets to
specify the n$^{th}$ occurence of an element of that class in the sequence:
\label{range_position}
\madxmp{
xxxxxxxxxxxxxx\= \kill
mq.ir5.l6..1  \> ! unique element in the sequence\\
mb[17]        \> ! 17th occurence of an element of class mb
}

There are two special predefined psoitions in \madx: 
\begin{madlist}
   \ttitem{\#S} The start of the beam line or sequence expanded by \hyperref[sec:use]{\texttt{USE}}, 
   \ttitem{\#E} The end of the beam line or sequence expanded by \hyperref[sec:use]{\texttt{USE}}. 
\end{madlist} 

If a range is selected in a \texttt{USE} statement: 
\madxmp{
USE, PERIOD=lhcb1, RANGE=ir1/ir5;
} 
then the  \texttt{\#S} and \texttt{\#E} indices refer to the start and
end of the  range expanded by the \texttt{USE} statement. 

 Examples for ranges: 
\madxmp{
xxxxxxxxxxxxxxxxxxxxxxxx\= \kill
.., RANGE=\#S;          \>! first element \\
.., RANGE=\#S/\#E;      \>! full expansion range \\
.., RANGE=mb[5]/\#E;    \>! from 5th mb to end \\
.., RANGE=mq.ir5.l6..1; \>! first occurrence of element mq.ir5.l6..1
}

\section{CLASS}
\label{sec:class} 
The single name of a class (no occurrence counts). A class is the name
of an element (or basic type) from which other elements have been
derived. 

Example: 
\madxmp{
MQ: quadrupole;   \= ! makes the element MQ \\
Q1: MQ;           \> ! makes the element Q1 from class MQ \\
Q2: MQ;           \> ! makes the element Q2 from class MQ \\
Q1..A: Q1;        \> ! makes an element from class Q1 \\
Q2..B: Q2;        \> ! makes an element from class Q2 
}

Subsequent selection with criterion \texttt{CLASS="MQ"} will actually select 
\texttt{Q1, Q2, Q1..A}, and \texttt{Q2..B} in this example. 

%\href{http://www.cern.ch/Hans.Grote/hansg_sign.html}{hansg}, June 17, 2002 

