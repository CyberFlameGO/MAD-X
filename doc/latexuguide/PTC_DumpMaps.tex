%%\title{PTC\_DumpMaps}

\section{PTC\_DUMPMAPS}

\begin{verbatim}
PTC_DUMPMAPS, 
      file = [s, ptcmaps, ptcmaps];
\end{verbatim}

PTC\_DUMPMAPS dumps the linear part of the map for each element of the
layout into the specified file.  

{\bf Command parameters and switches}\\
\begin{itemize}
   \item {\bf file}=string  (Default: ptcmaps)\\
     Specifies the file\_name of the file to which the matrices are dumped to.   
\end{itemize}

{\bf PROGRAMMERS MANUAL} \\  
The command is implemented by subroutine ptc\_dumpmaps() in
madx\_ptc\_module.f90. The matrix for a single element is obtained by
tracking identity map through an element, that is initialized for each
element by adding identity map to the reference particle. For the
elements that change reference momentum (i.e. traveling wave cavity)  it
is tracked to the end of the following marker, that has updated
reference momentum. Hence, each cavity must be followed by a marker. If
it is not, setcavities subroutine detects error and stops the program.   
 
