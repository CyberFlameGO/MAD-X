%%\title{New MAD-X Support Rule}
%  Changed by: Chris ISELIN, 17-Jul-1997 
%  Changed by: Hans Grote, 10-Jun-2002 

\section{New MAD-X Support Rule}

Frank Schmidt 

\line(1,0){300}
\\
 At the beginning of 2006 CERN is gearing up for the LHC sector test and
 only next year from now we want to commission the LHC! It is therefore
 mandatory to relocate manpower to the most urgent needs of the LHC. For
 MAD-X this implies that we cannot continue a full-blown development of
 the code. However, the system of module keepers will stay in place and
 they will respond to your questions. Moreover, I will personally
 maintain the link to the PTC code.  

 This change of support will mean in detail:  
\begin{itemize}
   \item For any problem with a specific MAD-X module please contact
     exclusively the module keepers (\href{module/node1.html}{ find
       here}). 
   \item No changes will be made to the code to safeguard against user errors.
   \item MAD-X features will not be modified if there is a known work-around.
   \item In case of trouble please first contact an experienced MAD-X user for help.
   \item We will deal with serious bugs if they limit the use of MAD-X for the LHC.
   \item It will be appreciated if minor bugs, work-arounds, features
     are reported to the \textbf{mad-X\_bug\_report}. This is for
     reference only and might be attacked at a later stage.  
   \item  There will be no further MAD-X Day reunions.
\end{itemize}

There is still some developing work in the pipeline that will be completed soon: 
\begin{itemize}
   \item Implementation of non-linear matching.
   \item Upgrade of the thin-lens tracking code.
   \item Documentation of the PTC modules: ptc\_track, ptc\_twiss and ptc\_normal and clean-up when needed.
   \item It should also be noted that there is a well manned CLIC effort to use PTC in conjunction with MAD-X.
   \item Moreover, it is very easy to add additional modules to MAD-X if
     the code writer is willing to invest into the interfaces (to be
     discussed in each individual case). 
\end{itemize}

\begin{verbatim}
   Frank.Schmidt@cern.ch
\end{verbatim}

\line(1,0){300}

\line(1,0){300}


%%\end{document}
