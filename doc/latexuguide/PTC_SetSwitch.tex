%%\title{PTC\_SETSWITCH}

\section{PTC\_SETSWITCH}

Routine that sets the internal PTC switches.

\begin{verbatim}
PTC_SETSWITCH,
   debuglevel = [i,0], 
   maxacceleration = [l, true, true],
   exact_mis = [l, false, true],
   totalpath = [l, false, true],
   radiation = [l false, true],
   fringe = [l, false, true],
   time = [l, true, true];
\end{verbatim}

Using this command the user can set switches of PTC and the MAD-X-PTC
interface, adapting this way the program behavior to his needs.   

{\bf Command parameters and switches}
\begin{itemize}
   \item {\bf debuglevel}=integer (Default: 1)\\
     Sets the level of debugging printout: 0 prints none, 4 prints everything   

   \item {\bf maxacceleration}=logical (Default: .true.)\\
     Switch to set cavities phases so the reference orbit is always on
     the crest, i.e. gains max energy    

   \item {\bf exact\_mis}=logical (Default: .false.)\\
     Switch ensures exact misalignment treatment.   

   \item {\bf totalpath}=logical  (Default: .false.)\\
     If true, the 6th variable of PTC, i.e. 5th of MAD-X, is the total
     path.  \\
     If false it is deviation from the reference particle,
     which is normally the closed orbit for closed layouts.    

   \item {\bf radiation}=logical (Default: .false.)\\    
     Sets the radiation switch/internal state of PTC.   

   \item {\bf fringe}=logical (Default: .false.)\\    
     Sets the fringe switch/internal state of PTC. \\ 
     If true the influence of the fringe fields is evaluated for all
     elements. \\       
     Please note that currently fringe fields are always taken into
     account for some elements (e.g. traveling wave cavities) even if
     this flag is set to false. The detailed list of elements
     will be provided later, when the situation in this matter will be
     definitely settled.    

   \item {\bf time}=logical (Default: .true.)\\  
     If true, Selects time of flight (\textit{cT} to be precise) rather
     than path length as the 6th variable of PTC, i.e. 5th of MAD-X.     
\end{itemize}


{\bf PROGRAMMERS MANUAL}   
Values of the switches are stored in Fortran 90 module
mad\_ptc\_intstate (mad\_ptc\_intstate.f90). The command is processed by
pro\_ptc\_setswitch C function, in file madxn.c, that calls appropriate
routines of the Fortran module to set each of the switches:   
\begin{itemize}
   \item  ptc\_setdebuglevel 
   \item  ptc\_setaccel\_method 
   \item  ptc\_setexactmis 
   \item  ptc\_setradiation 
   \item  ptc\_settotalpath 
   \item  ptc\_settime 
   \item  ptc\_setfringe  
\end{itemize}

