%%\title{RLOT}

\section{RPLOT}

RPLOT is a MAD-X plug-in that privides additional functionality using
\href{http://root.cern.ch}{ ROOT }.  It contains several tools   

\begin{description}

\item[\textbf{ RVIEWER }] 
\textit{ plotting tool that handles the results in paramremtric form }

What makes it different      from the standard PLOT module of MAD-X is
that it is also able to      deal with the parmateric results. RPLOT
proviedes graphical user interface      that allows to choose which
functions shall be drawn, set its ranges     and adjust all the details
of the plot formatting. Of course, the result     is immendiately
visible on the screen, in contrary to the standard plot tool     that is
able to work solely in the batch mode. The user can choose several
formats to save his plot, including postscript, gif, pdf, root macro and
many      others.       

RVIEWER is able to draw the lattice functions     
\begin{enumerate}
   \item  along the layout 
   \item  at given position in function of one or two knobs  
\end{enumerate}     

It provides a convienient way to set the knob values. As the value is
set,      the plotted functions are immediately drawn for the new value.            

In order to run RVIEWER simpy issue "rviewer;" command        

\item[\textbf{ RTRACKSTORE }] 
\textit{ enables storage of the tracking data in ROOT NTuple/Tree format }

Ntuple and its modern extension called Tree are formats designed
for storing particle tracking data. It is proven to provide       the
fastest data writing and reading thanks to column wise       I/O
operations. It is commonly used for data storage by HEP
experiments. Additionally, ROOT provides automatical        ZIP data
compression that is transparent for the user algorithms.
Morover, ROOT provides wide set of very comfortable tools       for
advanced analysis and plotting of the data stored in Trees.    

Addtionally, we plan to extend RVIEWER functionality that would provide
intuitive graphical user interface to most commonly used       features
in particle tracking in accelerators. Thanks to that,       the user is
not forced to learn how to use the ROOT package.    

Currently the feature is enabled only for tracking using        the
ptc\_trackline command, however, it will be extended to       other
tracking modes.           
\end{description}

\textbf{ Download } \\
The newest version is available \href{download/}{ here }\\

\textbf{ Installation }\\
Prerequisite: ROOT must be installed beforehand compilation and whenever
the user wants to use the plug-in. See explanations on
\href{http://root.cern.ch}{ROOT webpage}.  \\ 

To install RPLOT 
\begin{enumerate}
   \item  Unpack the archive, it will create directory rplot    
\begin{verbatim}
   tar xvzf rplot-X.XX.tgz
\end{verbatim}

   \item Change to rplot directory    
\begin{verbatim}
   cd rplot
\end{verbatim}

   \item Type     
\begin{verbatim}
   make install
\end{verbatim}
\end{enumerate}

\textbf{ Examples}


\textbf{SYNOPSIS}
\begin{verbatim}
RVIEWER; 
\end{verbatim}


\textbf{ PROGRAMMERS MANUAL  } \\

 To be continued... 

