%%\title{SXF}
%  Changed by: Chris ISELIN, 27-Jan-1997 
%  Changed by: Hans Grote,  4-Dec-2002 

\chapter{SXF file format}
\label{chap:sxf}

An \texttt{SXF}\cite{SXF} lattice description is an ASCII listing that contains
one named, ``flat'', ordered list of elements, delimited as \{\ldots\}, with one
entry for each element. The list resembles a \madx ``sequence'' describing
the entire machine. The syntax is supposed to be adapted for ease of
reading by human beings and for ease of parsing by LEX and YACC. 

\section{SXFWRITE}
\label{sec:sxfwrite}

The command 
\madbox{
SXFWRITE, FILE=filename;
}
writes the current  sequence with all alignment and
field errors in \texttt{SXF}
format onto the file specified. This then represents one "instance" of
the sequence, where all parameters are given by numbers rather than
expressions; the file can be read by other programs to get a complete
picture of the sequence.  


\section{SXFREAD}
\label{sec:sxfread}

The command 
\madbox{
SXFREAD, FILE=filename;
}
reads the file "filename" in \texttt{SXF} format, stores the sequence
away and loads the sequence in memory through the USE mechanism  in
order to keep the existing errors. 


It is therefore possible to write a lattice complete with errors to a
named file and reload it later in a different \madx job:
\madxmp{
! define sequence MYSEQU \\
USE, mysequ; \\
\\
! add alignment errors and field errors \\
\\
SXFWRITE, FILE = file; \\
\\
STOP;
}
and later:
\madxmp{
SXFREAD, FILE = file; \\
! sequence mysequ is now reloaded and active, complete with errors.\\
\\
TWISS; \\
\ldots
}

%\href{http://www.cern.ch/Hans.Grote/hansg_sign.html}{hansg}, January 24, 1997 

