\chapter{File Handling Statements}
\label{chap:files}

Note that the filenames given as attribute values in File Handling statements 
must be explicit names and should not contain wildcard characters since the 
filename strings are not passed to the underlying Operating System for evaluation.


\section{ASSIGN}
\label{sec:assign}

\madbox{
ASSIGN, ECHO="filename", TRUNCATE;
}
where \texttt{filename} is either the name of an output file, or the
string \texttt{terminal}.
This allows switching the echo stream to a file or back to the terminal,
but only for the commands \hyperref[sec:value]{\texttt{VALUE}}, 
\hyperref[sec:show]{\texttt{SHOW}}, and \hyperref[sec:print]{\texttt{PRINT}}. 
Allows easy composition of future \madx input files.

\texttt{TRUNCATE} specifies whether the file is truncated when opened 
(ignored for terminal). 


\section{CALL}
\label{sec:call}
\madbox{
CALL, FILE="filename"; 
}
where \texttt{filename} is the name of an input file. The named file is
then read until a \texttt{RETURN} statement is encountered, or until the
End\_Of\_File; The file being "called" may in turn contain any number of
\texttt{CALL} statements itself, and so on to any depth.


\section{CHDIR}
\label{sec:chdir}
\madbox{
CHDIR, DIR="path/to/folder";
}
changes the current working directory to the specified folder. Both relative
and absolute pathes are possible.


\section{RETURN}
\label{Sec:return}
\madbox{
RETURN;
}
ends the reading from a "called" file. If encountered in the standard
input file, it ends the program execution.  


\section{PRINT}
\label{sec:print}
\madbox{
PRINT, TEXT="string";
}
prints the quoted text \texttt{string} to the current output file (see
\hyperref[sec:assign]{\texttt{ASSIGN}} above). The text can be edited
with the help of a \hyperref[sec:macro]{macro} statement.
%For more details, see there. [TBC]  


\section{PRINTF}
\label{sec:printf}
\madbox{
PRINTF, TEXT="string", VALUE= expr{,expr};
}
prints the numerical values specified in the \texttt{VALUE} field to the
current output file, with formatting according to the format string
provided in the \texttt{TEXT} field. 

The string format can take numeric C or \madx format specifiers for
double real values. Integer and string formats are not supported but 
can be approximated with the \texttt{\%g} format in the case of
integers, and via \hyperref[sec:macro]{\texttt{MACRO}} statements, which
perform string substitution, themselves containing a
\hyperref[sec:print]{\texttt{PRINT}} statement.   

The maximum number of values that can be printed in one
statement is limited to 100. 

If the number of format specifiers given in the string is higher 
than the number of values in the value field, undefined values are
printed where they are not explicitly provided. 

If the number of format specifiers given in the string is lower 
than the number of values in the value field, the values that 
do not correspond to a format specifier are ignored. 


\textbf{Example:}
{\small
\madxmp{
xxxxxxx\=xxxxxxxxxxxxxxxxxxxxxxxxxxxxxxxxxxxxxxxxxxxxxxxxxxxx\= \kill
a = 1.2; b = 3.4/0.3; c := 0.8*a/b; \\
PRINTF, \>TEXT="String with floats a=\%f, b=\%.3g, text and MAD float c=\%F;", \\
        \>VALUE = a,b,c; \\
PRINTF, \>TEXT="More specifiers than values: \%f, \%.3g, \%F", \>VALUE = a,b; \\
PRINTF, \>TEXT="More values than specifiers: \%f, \%.3g",      \>VALUE = a,b,c;
}}

produces the following output:
\begin{verbatim}
String with floats a=1.200000, b=11.3, text and MAD float c=  0.08470588235;
More specifiers than values: 1.200000, 11.3,   6.953222976e-310
More values than specifiers: 1.200000, 11.3
\end{verbatim}

Note that \texttt{PRINTF}, like \texttt{PRINT}, produces output that
cannot be read back by \madx. For output that can be read back by \madx,
use the command \texttt{VALUE} or TFS tables.

Note also that a percent sign (\%) can be printed using the format
\verb|text="%%"|. 


\section{RENAMEFILE}
\label{sec:renamefile}
%\madbox{RENAMEFILE, FILE="filename", NAME = "new\_filename";}
\madbox{
RENAMEFILE, FILE="filename", TO="new\_filename";
}
renames the file \texttt{filename} to \texttt{new\_filename} on disk. \\
This command is more portable than an equivalent
\hyperref[sec:system]{\texttt{SYSTEM}} call: 
\madxmp{
SYSTEM("mv filename new\_filename"); ! Unix specific
}

\section{COPYFILE}
\label{sec:copyfile}
\madbox{
COPYFILE, FILE="filename", TO="new\_filename", APPEND=logical;
}
copies the file \texttt{filename} to the file \texttt{new\_filename} on disk.

The attribute \texttt{APPEND=true} causes \texttt{COPYFILE} to append
the content of \texttt{filename} at the end of the file
\texttt{new\_filename}.\\ 
The default value \texttt{APPEND=false} causes the replacement of the
content of \texttt{new\_filename} with the content of \texttt{filename}.

\texttt{COPYFILE, APPEND=true...} is more portable than an equivalent
\hyperref[sec:system]{\texttt{SYSTEM}} call: 
\madxmp{
SYSTEM("copy /y filename new\_filename"); ! Windows specific
}

\section{REMOVEFILE}
\label{sec:removefile}
\madbox{
REMOVEFILE, FILE="filename";
}
removes the file \texttt{filename} from disk. \\
It is more portable than an equivalent \hyperref[sec:system]{\texttt{SYSTEM}} call: 
\madxmp{
SYSTEM("rm filename"); ! Unix specific
}

%% EOF
