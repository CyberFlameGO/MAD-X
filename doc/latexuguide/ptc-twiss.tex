%%\title{MAD-X PTC\_TWISS Module}
%  Created by: Valery KAPIN, 21-Mar-2006 
%  Changed by: Frank Schmidt 04-Apr-2006 
%  \href{mailto:kapin@itep.ru}{ V.Kapin}(ITEP) and 
%  \href{mailto:Frank.Schmidt@cern.ch}{ F.Schmidt}, March  2006

%  Changed by: Piotr and Ghislain 24-Jan-2014



\chapter{Ripken Optics Parameters}
\label{chap:ptc-twiss}


\section{Introduction}

The \textbf{PTC\_TWISS module}\cite{schmidt2005} of \madx is
based on the \ptc code and is supplementary to the
\hyperref[chap:twiss]{\texttt{TWISS}} module of \madx.
In \texttt{PTC\_TWISS} the Twiss parameters are calculated according to
the formalism of G.~Ripken, developped in \cite{Ripken1970} and
most accessible in \cite{Ripken1988}.

%% These parameters were available in MAD8 using the TWISS3 command. This
%% module is a typical example of the advantages when using PTC and its
%% Normal Form technique (and of course the object-oriented Fortran90
%% coding): once the rather modest programming has been performed the Twiss
%% calculation will always be automatically correct for all machine
%% conditions like closed orbit, coupling or after a new element has been
%% introduced into the code. In a traditional coding like in MAD8 this
%% depends on reprogramming and modifying the code at various places which
%% is inherently error-prone.  

\texttt{PTC\_TWISS} tracks a special representation of the beam in three
degrees of freedom. It works on the coupled lattice functions which are
essentially the projections of the lattice functions for the eigen-modes
on the three planes.  

\texttt{PTC\_TWISS} lists the projections of the ellipses of motion onto
the three planes ($x$, $p_x$), ($y$, $p_y$), ($t$, $p_t$) expressed via
Ripken's parameters $b_{k,j}$, $a_{k,j}$, $g_{k,j}$ along with the phase
advances $m_j$ in selected positions, where index $k=1 ... 3$ refers to
the plane ($x$, $y$, ...), and the index $j= 1 ... 3$ denotes the
eigen-mode.

The \texttt{PTC\_TWISS} command also calculates the dispersion values
$D_1$, ... ,$D_4$.

In \madx commands and tables, these parameters are denoted as
\texttt{beta11, ..., beta33}, \texttt{alfa11, ..., alfa33},
\texttt{gama11, ..., gama33}, \texttt{mu1, ..., mu3},
\texttt{disp1, ..., disp4}, respectively. 

The Ripken parametrization can be transformed into the Edwards-Teng
parametrization (used in the module
\hyperref[chap:twiss]{\texttt{TWISS}} of \madx) using the formulae of
Lebedev \cite{lebedev2010}. 

The parameters are noted as \texttt{betx, bety, alfx, alfy} and the
coupling matrix: \texttt{R11, R12, R21} and \texttt{R22}. In absence of
coupling, the following holds: \texttt{betx = beta11}, \texttt{bety = beta22},
\texttt{alfx = alfa11} and \texttt{alfy = alfa22}.

\texttt{PTC\_TWISS} can also compute the deltap/p 
(or deltaE/E if time=TRUE) dependency 
of the Twiss parameters. The column names \texttt{beta11p, ..., beta33p,
  alfa11p, ..., alfa33p, gama11p, ..., gama33p} denote the derivatives of
the optics parameters with respect to deltap/p (or deltaE/E if time=TRUE). 

In order to evaluate the deltap/p-dependency of the Twiss parameters,
the order (\texttt{NO}) of the map must set to at least 2.  

The derivatives of the dispersion with respect to deltap/p (or deltaE/E if time=TRUE)
have column names: \texttt{disp1p, ..., disp4p}. 
Second and third order derivatives have respective column names:  
\texttt{disp1p2, ..., disp4p2} for the second order, and 
\texttt{disp1p3, ..., disp4p3} for the third order.

In addition, \ptc computes the momentum compaction factor $\alpha_c$ (\texttt{ALPHA\_C}) and time 
slip factor $\eta_c$. Higher order derivatives of compaction factor  
with respect to deltap/p up to urder 4  (\texttt{ALPHA\_C\_P}, \texttt{ALPHA\_C\_P2} and \texttt{ALPHA\_C\_P3})
are computed  f \texttt{ICASE=56} and \texttt{time=false}, prividing the order of 
calculation as defined by \texttt{NO} switch is high enough.
The values appear in the header of the \texttt{PTC\_TWISS} output file, 
and a value of -1000000 means the value has not been computed.

%This feature is currently only available in the development
%version. \\
%\textbf{[To be checked]}

For clarification: in the 4-D case, there is the following
correspondence between MAD-X and the Ripken's notations:  
\texttt{beta11} $\equiv \beta_{xI}$, 
\texttt{beta12} $\equiv \beta_{xII}$, 
\texttt{beta21} $\equiv \beta_{yI}$, 
\texttt{beta22} $\equiv \beta_{yII}$.
In the uncoupled 4-D case, \texttt{beta11} is the same as the
classical $\beta_x$ (\texttt{betx}) and \texttt{beta22} is $\beta_y$ 
(\texttt{bety}), while \texttt{beta12} and \texttt{beta21} are zero.  
in the coupled case all \texttt{betaNN} are non-zero and \texttt{beta11},
\texttt{beta22} are distinctively different from $\beta_x$, $\beta_y$,
respectively.

\texttt{PTC\_TWISS} also tracks the eigenvectors and prints them to Twiss table
according to the \hyperref[sec:select]{\texttt{SELECT}} command  with
\texttt{FLAG=ptc\_twiss}.  
Either all 36 components or particular components of the eigenvectors
can be selected with \texttt{EIGN} or \texttt{EIGNij}, respectively
(\texttt{j} = number of eigenvector, \texttt{i} = number of coordinate
\{$x$, $p_x$, $y$, $p_y$, $t$, $p_t$\}). 

For ring lattices, \texttt{PTC\_TWISS} computes momentum compaction, transition
energy, as well as other one-turn characteristics such as the tunes
(Q1, Q2 and if \texttt{ICASE=6} with cavity, Qs) and chromaticities (for
\texttt{NO}$\geq 2$).  

\textbf{Synopsis:}
\madxmp{
PTC\_CREATE\_UNIVERSE; \\
PTC\_CREATE\_LAYOUT, MODEL=integer, METHOD=integer, NST=integer, [EXACT]; \\
...\\
SELECT, FLAG=ptc\_twiss, CLEAR; \\
SELECT, \=FLAG=ptc\_twiss, COLUMN=name, s,   \\
        \>beta11,...,beta33, alfa11,...,alfa33, gama11,...,gama33, \\
        \>beta11p,...,beta33p, alfa11p,...,alfa33p, gama11p,...,gama33p, \\
        \>mu1,...,mu3, \\
        \>disp1,...,disp4, disp1p,...,disp4p, \\
        \>disp1p2,...,disp4p2, disp1p3,...,disp4p3, \\
        \>[eign], eign11,...,eign16,...,eign61,...,eign66; \\
...\\
PTC\_TWISS; \\
...\\
PTC\_END;
}

\section{PTC\_TWISS}
\label{sec:ptc-twiss}
 
The \texttt{PTC\_TWISS} command causes computation of the Twiss
parameters in Ripken's style. It operates on the working beam line
defined in the latest \hyperref[sec:use]{\texttt{USE}} command. 

Applications for the \texttt{PTC\_TWISS} command are similar to the
\hyperref[chap:twiss]{\texttt{TWISS}} command of \madx. 
The \texttt{PTC\_TWIS}S can be applied to two basic tasks. It can calculate either a
\hyperref[sec:ptc-twiss-periodic]{periodic solution} or a
\hyperref[sec:ptc-twiss-sol-initial-cond]{solution with initial conditions}. 

\madbox{
PTC\_TWISS, \=ICASE=integer, DELTAP=real, \\
            \>CLOSED\_ORBIT=logical, DELTAP\_DEPENDENCY=logical, \\
            \>SLICE\_MAGNETS=logical, \\
            \>RANGE=string, FILE[=filename], TABLE[=tabname], \\
            \>INITIAL\_MATRIX\_TABLE=logical, INITIAL\_MATRIX\_MANUAL=logical,  \\
            \>INITIAL\_MAP\_MANUAL=logical, BETA0=string, MAPTABLE=logical, \\
            \>IGNORE\_MAP\_ORBIT=logical, RING\_PARAMETERS=logical, \\
            \>NORMAL=logical, \\
            \>BETX=real, ALFX=real, MUX=real, \\
            \>BETY=real, ALFY=real, MUY=real, \\
            \>BETZ=real, ALFZ=real,  \\
            \>DX=real,  DPX=real, DY=real, DPY=real, \\
            \>X=real, PX=real, Y=real, PY=real, T=real, PT=real, \\
            \>RE11=real, RE12=real, \ldots , RE16=real, \\
            \>... \\
            \>RE61=real, RE62=real, \ldots , RE66=real;
}


The attributes are: 
\begin{madlist}

  \ttitem{ICASE} 
  the dimensionality of the phase-space (4, 5 or
  6). (Default:~4)\\ Note that \texttt{ICASE} is internally set to 56 when
  attempting to set \texttt{ICASE=6} with no cavity, and \texttt{ICASE} is
  internally set to 4 when attempting to set \texttt{ICASE=6} with an RF
  cavity with zero voltage.

  \ttitem{NO} 
  the order of the map. (Default:~1)\\ For evaluating the
  derivatives of the Twiss parameter w.r.t. deltap/p, e.g. for
  evaluating the chromaticities, the order must be at least equal to 2.

  \ttitem{DELTAP}   
  relative momentum offset for reference closed orbit. (Default:~0.0) 

  \ttitem{CLOSED\_ORBIT}  
  a logical switch to trigger the closed orbit calculation
  (applies to \hyperref[sec:ptc-twiss-periodic]{periodic solution} ONLY). \\  
  (Default: false)

  \ttitem{DELTAP\_DEPENDENCY}  
  a logical switch to trigger the computation of the Twiss and
  dispersion derivatives w.r.t. deltap/p. Derivation formula assume that
  \texttt{ICASE $\ge$ 5}, so that deltap/p is supplied as a parameter. \\
  (Default: false)

  \ttitem{SLICE\_MAGNETS}
  %% the switch to turn on the evaluation of Twiss
  %% parameters at each thin slice inside successive magnets, instead of at
  %% the middle of each magnet. Slices are located at the so-called
  %% 'integration nodes' determined by the number of steps (nst) selected
  %% when creating the PTC layout. Note that extremities and fringes are
  %% skipped, whereas only the inner slices are kept. 
  a logical switch to activate the evaluation of Twiss parameters at each
  integration step inside magnets, in addition to the end face.  The
  number of slices is determined by the number of steps (\texttt{NST}) that
  can be separately defined for each element, or otherwise set by
  \texttt{NST} parameter when creating the PTC layout.  Note that the
  orbit  rms calculated in this mode counts as valid data points both
  the end of the previous element and the entrance of the current
  element. Since the first integration node is always at the entrance of
  the magnet (after position offset and fringe effects are calculated)
  which corresponds to the same s position (and usually optical
  functions) as the end of the previous element, the points at the
  interface between magnets are included twice in the rms calculation. \\ 
  (Default: false)

  \ttitem{CENTER\_MAGNETS}
  a logical switch to activate the evaluation of Twiss
  parameters at the middle of each magnet. This relies on internal slicing
  and 'integration nodes' as determined by the number of steps (\texttt{NST})
  selected when creating the PTC layout. This number is assumed to be even
  otherwise the program issues a warning. \\
  (Default: false)

  \ttitem{FILE} 
  if the \texttt{FILE} attribute is omitted, no output is
  written to file. \\ If the \texttt{FILE} attribute name is present, the
  optional attribute value argument is the name of the file for printing
  the \texttt{PTC\_TWISS} output. The default file name is
  "ptc\_twiss". \\ 
  (Default: false)

  \ttitem{TABLE} 
  if the \texttt{TABLE} attribute is omitted, no output is
  written to an internal table. \\ If the \texttt{TABLE} attribute name is
  present, the optional attribute value argument is the name of the
  internal table for \texttt{PTC\_TWISS} variables. The default table name
  is "ptc\_twiss". \\ 
  (Default: false)

  \ttitem{SUMMARY\_FILE} 
  if the \texttt{SUMMARY\_FILE} attribute is omitted, no summary output is
  written to file. \\  
  If the \texttt{SUMMARY\_FILE} attribute name is present, the optional
  attribute value argument is the name of the file for printing the 
  \texttt{PTC\_TWISS\_SUMMARY} table output. The default file name is
  "ptc\_twiss\_summary". \\
  (Default: false)

  \ttitem{SUMMARY\_TABLE} 
  if the \texttt{SUMMARY\_TABLE} attribute is omitted, no summary output is
  written to an internal table. \\ 
  If the \texttt{SUMMARY\_TABLE} attribute name is present, the optional
  attribute value argument is the name of the internal summary table for
  \texttt{PTC\_TWISS\_SUMMARY} variables. The default table name is 
  "ptc\_twiss\_summary".  \\ 
  (Default: false)

  \ttitem{RANGE} 
  a string in \hyperref[sec:range]{\texttt{RANGE}} format that
  specifies a segment of beam-line for the \texttt{PTC\_TWISS}
  calculation. \\ 
  (Default: $\#S/\#E$)

  \ttitem{INITIAL\_MATRIX\_TABLE} 
  a logical flag to trigger the reading of the transfer map from table
  named "map\_table" created by a preceding \texttt{PTC\_TWISS} or
  \texttt{PTC\_NORMAL} command. The table can be also read 
  before hand from files using a \hyperref[sec:readtable]{\texttt{READTABLE}}
  command.\\ 
  (Default: false)
  
  \ttitem{INITIAL\_MATRIX\_MANUAL} 
  a logical flag to trigger the use of the input variables \texttt{RE11,
    \ldots ,RE66} as the transfer matrix. \\
  (Default: false)

  \ttitem{RE11,..., RE66} 
  values of the $6\times 6$ transfer matrix. \\ 
  (Default: $6\times 6$ unit matrix) 
  
  \ttitem{INITIAL\_MAP\_MANUAL}
  a logical flag to trigger the use of an input map stored beforehand in
  file "fort.18",  e.g. by a previous initial run of
  \hyperref[chap:ptc-normal]{\texttt{PTC\_NORMAL}}). \\
  (Default: false)

  \ttitem{IGNORE\_MAP\_ORBIT} 
  a logical flag to ignore the orbit in the map and use the closed orbit
  instead if requested, or the orbit defined by the starting point
  specified with \texttt{X, PX, Y, P, T, DT} parameters otherwise. \\
  (Default: false)

  \ttitem{BETA0} 
  the name of a \texttt{BETA0} block containing the Twiss
  parameters to be used as input. When \texttt{ICASE=6} or \texttt{ICASE=56}, 
  this information  must be complemented by supplying a value for \texttt{BETZ} on the
  \texttt{PTC\_TWISS} command line. \\
  (Default: beta0)

  \ttitem{MAPTABLE} 
  a logical flag to save the one-turn-map to table
  "map\_table". The one-turn-map can then be used as starting condition
  for a subsequent \texttt{PTC\_TWISS}, see \texttt{INITIAL\_MATRIX\_TABLE}
  parameter above. \\ (Default: false)

  \ttitem{BETX, ALFX, MUX, BETY, ALFY, MUY, BETZ, ALFZ, DX, DPX, DY, DPY} : \\
  Edwards and Teng \cite{edwards1973} 
  \hyperref[chap:twiss]{Twiss and dispersion} parameters:  
  $\beta_{x,y,z}$, $\alpha_{x,y,z}$, $\mu_{x,y}$, $D_{x,y}$, $D_{px,py}$.\\
  (Default: 0) 

  \ttitem{RING\_PARAMETERS} a logical flag to force computation of ring
  parameters ($\gamma_{tr}$, $\alpha_c$, etc.). \\ 
  (Default: false)

  \ttitem{X, PX, Y, PY, T, PT} the
  \hyperref[subsec:tables-canon]{canonical} coordinates of the initial
  orbit. \\ (Default: 0.0) \\
  
  \ttitem{NORMAL} a logical flag that triggers saving of the normal form analysis results
  (the closed solution) into dedicated table called \texttt{NONLIN}. It is the same as  \texttt{ptc\_normal}.
   However, the nonlin table has format such that its coefficients can be accessed within the MADX script using 
   \texttt{table} command. This permits, for example, to perform matching of these parameters or or value extraction to a variable. 
   Also, all available orders are calculated and 
   no \texttt{select\_ptc\_normal} is required.
   Currently the following parameters are calculated: tunes (and all their derivatives), 
   dipsersions (and all their derivatives), eigen values, transfer map, 
   resonance driving terms (generating function) and the pseudo Hamiltonian.
   If radiation and envelope is switched on with \hyperref[sec:ptc-setswitch]{\texttt{PTC\_SETSWITCH}}
   then damping decrements, equilibrium emittances and beam sizes are calculated.
   Attention: the results are always the plain polynomial coefficients, 
   which is different from  \texttt{ptc\_normal} nomenclature for some variables, 
   which in turn always gives values of the partial derivatives.
   Therefore, in order to obtain values of the partial derivatives the respective factorials 
   and binomial coefficients need to be factored out by the user. 
   Also, if TIME=true the chromatic variables are NOT transformed into usual deltap/p dependence
   and in this case they are related to deltaE/E. For example, dispersions and chormaticities will be 
   different by relativistic beta factor. Similarly, the T coordinate is time times the speed of light 
   instead of path length times the speed of light.
   
  
\end{madlist}


\section{Periodic Solution}
\label{sec:ptc-twiss-periodic}	

This is the simplest form of the \texttt{PTC\_TWISS} command, which
computes the periodic solution for a specified beam line. It may
accept all basic attributes described in
\hyperref[sec:ptc-twiss]{\texttt{PTC\_TWISS}} above. 

\madbox{
PTC\_TWISS, \=ICASE=integer, DELTAP=real, CLOSED\_ORBIT=logical,\\
            \>RANGE=string, FILE[=string], TABLE[=string];
}


\section{Evaluation of Twiss parameters inside magnets}
\label{sec:ptc-twiss-slicing}

This computes the periodic solution for a specified beam
line and evaluates the Twiss parameters at each thin-slice
(a.k.a "integration-node") inside magnets. The number of such
integration-nodes is given by the number of steps (\texttt{NST})
selected when creating the \ptc layout. All other basic
attributes described in \hyperref[sec:ptc-twiss]{\texttt{PTC\_TWISS}}
above may be selected.

\madbox{
PTC\_TWISS, \=ICASE=integer, DELTAP=real, CLOSED\_ORBIT=logical,\\
            \>RANGE=string, FILE[=string], TABLE[=string],\\ 
            \>SLICE\_MAGNETS=logical;
}


\textbf{Example:} \\
An example is found in the
\href{http://madx.web.cern.ch/madx/madX/examples/ptc_twiss/SliceMagnets/}
{\texttt{PTC\_TWISS} Examples} repository. 



\section{Solution with Initial Conditions}
\label{sec:ptc-twiss-sol-initial-cond}


Initial conditions can be supplied in different ways.  Naturally only
one of the methods below can be used at a time, and they can not be
mixed.  In this mode it is assumed that the lattice is a line and no
ring parameters are evaluated (their values are set to -1000000), unless
\texttt{RING\_PARAMETERS=true}, which forces computation of closed solution
for the resulting map. If a closed solution does not exist, \ptc
reports an error and exits.

The following logic is programmed in \ptc to identify the source of
initial conditions: 

\madbox{
xxxxxxxx\=xx\= \kill
IF     \>( \=INITIAL\_MATRIX\_TABLE=true  \&\&  (a map-table exists))  THEN \\
        \>  \>use \hyperref[subsec:from-map-table]{initial values from a Map-Table}\\
\\
ELSEIF \>( \>INITIAL\_MAP\_MANUAL=true ) THEN \\
        \>  \>use \hyperref[subsec:from-map-file]{initial values from a Given Map File}\\ 
\\
ELSEIF \>( \>INITIAL\_MATRIX\_MANUAL=true ) THEN \\
        \>  \>use \hyperref[subsec:from-given-matrix]{initial values from a Given Matrix}\\ 
\\
ELSEIF \>( \>BETA0 block is given ) THEN \\
       \>  \>use \hyperref[subsec:from-beta0-block]{initial values from a BETA0 block}\\ 
\\
ELSE \\
       \>  \> use \hyperref[subsec:from-twiss-command]{initial values from Given Twiss parameters}\\ 
\\
ENDIF
}


\subsection{Initial Values from the Given Twiss Parameters}
\label{subsec:from-twiss-command}

\texttt{PTC\_TWISS} calculates a solution with initial conditions
given by the Twiss parameters, which are explicitly typed as attributes
to the command. 
This case is also limited to uncoupled motion of the preceding ring or
beam-line.  

\madbox{
PTC\_TWISS, \=ICASE=integer, DELTAP=real, \\ %CLOSED\_ORBIT, \\
            \>RANGE=string, FILE[=string], TABLE[=string], \\
            \>BETX=real, ALFX=real, MUX=real, \\
            \>BETY=real, ALFY=real, MUY=real, \\
            \>BETZ=real, ALFZ=real, \\
            \>DX=real, DPX=real, DY=real, DPY=real, \\
            \>X=real, PX=real, Y=real, PY=real, T=real, PT=real;
}


\textbf{Example:} \\
An example is found in the
\href{http://madx.web.cern.ch/madx/madX/examples/ptc_twiss/}
{\texttt{PTC\_TWISS} Examples} 
in the folder
\href{http://madx.web.cern.ch/madx/madX/examples/ptc_twiss/Example2}
{"Example2"}.  


\subsection{Initial Values from a Map-Table}
\label{subsec:from-map-table}

\texttt{PTC\_TWISS} calculates a solution with initial conditions given
as a map-table of preceding ring or beam-line. It requires the input
option \texttt{INITIAL\_MATRIX\_TABLE} and an existing map-table
in memory, as generated by a preceding
\hyperref[sec:ptc-normal]{\texttt{PTC\_NORMAL}} command.  

\madbox{
PTC\_TWISS, \=ICASE=integer, DELTAP=real, \\ %CLOSED\_ORBIT, \\
            \>RANGE=string, FILE[=string], TABLE[=string], \\
            \>INITIAL\_MATRIX\_TABLE;
}

\textbf{Example:} \\
An example is found in the
\href{http://madx.web.cern.ch/madx/madX/examples/ptc_twiss/}
{\texttt{PTC\_TWISS} Examples} in the folder
\href{http://madx.web.cern.ch/madx/madX/examples/ptc_twiss/Example3}
{"Example3"}. 


\subsection{Initial Values from a Map-File}
\label{subsec:from-map-file}

\texttt{PTC\_TWISS} calculates a solution with initial
conditions given as a map-file (fort.18) obtained from a
preceding ring or beam-line. It requires the input option
\texttt{INITIAL\_MAP\_MANUAL} and an existing map-file
in file "fort.18", as generated by a preceding
\hyperref[sec:ptc-normal]{\texttt{PTC\_NORMAL}} command. 

\madbox{
PTC\_TWISS, \=ICASE=integer, DELTAP=real, \\ % \\ CLOSED\_ORBIT, \\
            \>RANGE=string, FILE[=string], TABLE[=string],\\
            \>INITIAL\_MAP\_MANUAL;
}

\textbf{Example:} \\
An example is found in the
\href{http://madx.web.cern.ch/madx/madX/examples/ptc_twiss/}
{\texttt{PTC\_TWISS} Examples} in the folder
\href{http://madx.web.cern.ch/madx/madX/examples/ptc_twiss/Example3}
{"Example3"}. 


\subsection{Initial Values from a Given Matrix}
\label{subsec:from-given-matrix}

\texttt{PTC\_TWISS} calculates a solution with initial conditions given
by a matrix explicitly given as attribute to the command. 
It requires the option \texttt{INITIAL\_MATRIX\_MANUAL}. 
\madx expects a symplectic 6x6 transfer matrix as input. 

\madbox{
PTC\_TWISS, \=ICASE=integer, DELTAP=real, \\ % CLOSED\_ORBIT, \\
            \>RANGE=string, FILE[=string], TABLE[=string], \\
            \>INITIAL\_MATRIX\_MANUAL,  \\
            \>RE11=real, RE12=real, ... , RE16=real, \\
            \>...\\
            \>RE61=real, RE62=real, ... , RE66=real;
}

\textbf{Example:} \\
An example is found in the
\href{http://madx.web.cern.ch/madx/madX/examples/ptc_twiss/}
{\texttt{PTC\_TWISS} Examples} in the folder
\href{http://madx.web.cern.ch/madx/madX/examples/ptc_twiss/Example4}
{"Example4"}. 


\subsection{Initial Values from Twiss Parameters via BETA0-block}
\label{subsec:from-beta0-block}

\texttt{PTC\_TWISS} calculates a solution with initial conditions given by
Twiss parameters, which are transferred from the \texttt{BETA0} block.  The
data in the the \texttt{BETA0} block have to be filled by a combination of
the \hyperref[sec:savebeta]{\texttt{SAVEBETA}} and
\hyperref[chap:twiss]{\texttt{TWISS}} commands of \madx for a preceding
ring or beam-line. This case is limited to uncoupled motion of the
preceding machine. 

\madbox{
PTC\_TWISS, \=ICASE=integer, DELTAP=real, \\ % CLOSED\_ORBIT, \\
            \>RANGE=string, FILE[=string], TABLE[=string], \\
            \>BETA0=string ;
}
      
\textbf{Example:} \\
An example is found in the
\href{http://madx.web.cern.ch/madx/madX/examples/ptc_twiss/}
{\texttt{PTC\_TWISS} Examples} in the folder
\href{http://madx.web.cern.ch/madx/madX/examples/ptc_twiss/Example1}
{"Example1"}. 


%% EOF
