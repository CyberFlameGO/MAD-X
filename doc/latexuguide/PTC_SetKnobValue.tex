%%\title{PTC\_SETKNOBVALUE}

\section{PTC\_SETKNOBVALUE}

\begin{verbatim}
PTC_SETKNOBVALUE, 
     elementname = [s, none] , 
     kn    = [i, {-1}], 
     ks    = [i, {-1}], 
     value = [r] ; 
\end{verbatim}

With this command the user set a given knob value. In its effect all the values in 
\begin{itemize}
   \item  the twiss table used by the last ptc\_twiss command 
   \item  the columns specified with
     \href{PTC_Select.html}{ptc\_select}, parametric=true; 
\end{itemize} 
are reevaluated using the buffered parametric results.  

The parameters of the command basically contains the fields that allow
to identify uniquely the knob and the value to be set.

{\bf Command parameters and switches}
\begin{itemize}
   \item {\bf elementname}=string in range format (Default: NULL)\\
     Specifies name of the element containing the knob to be set.   

   \item {\bf kn,ks}=list of integers (Default: ???)\\
     Defines the knob   

   \item {\bf value}=real (Default: 0)\\
     Specifies the value the knob is set to.             
\end{itemize}

{\bf Example }

\href{http://cern.ch/frs/mad-X_examples/ptc_madx_interface/matchknobs/matchknobs.madx}{dog
  leg chicane}: strength of dipole field component in quadrupoles is
matched to obtain the required R56 value.    


% <h3> PROGRAMMERS MANUAL </h3>
% 
% <p> 
% The command is implemented pro_PTC_SETKNOBVALUE function in madxn.c and 


