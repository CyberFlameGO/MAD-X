%%\title{Geometric Layout}
%  Created by: Andre VERDIER, 21-Jun-2002 
%  Changed by: Andre Verdier, 25-Jun-2002 
%  Changed by: Hans Grote, 30-Sep-2002 
%  Changed by: Andre Verdier, 14-Jan-2003 
%  Directory :  /afs/cern.ch/eng/sl/MAD-X/pro/docu/survey/ 
 
\chapter{SURVEY}
\label{chap:survey}

The \texttt{SURVEY} command computes the geometrical layout,
\textsl{i.e.} the coordinates of all machine elements in a 
\hyperref[sec:global-ref]{global reference system}. These
coordinates can be used for installation. In order to produce
coordinates in a particular system, the initial coordinates and angles
can be specified. 

\madbox{
SURVEY, \=SEQUENCE=string, FILE= string, \\
        \>X0=real, Y0=real, Z0=real, \\
        \>THETA0=real, PHI0=real, PSI0=real;}

The \texttt{SURVEY} command has the following attributes:
\begin{madlist}
  \ttitem{SEQUENCE} the name of sequence to be surveyed. By default the
  last sequence expanded with the \texttt{USE} command will be surveyed.  
  \ttitem{FILE}  the name of external file to which the results are
  written.\\ (Default:~survey)
  \ttitem{X0, Y0, Z0} the initial horizontal, vertical and longitudinal
  coordinates in meters.\\ (Default: 0.0, 0.0, 0.0) 
  \ttitem{THETA0, PHI0, PSI0} the initial horizontal and vertical angles
  and transverse tilt in radians.\\ (Default: 0.0, 0.0, 0.0)
\end{madlist}

The computation results are written on the internal table named
"survey". Results can also be written on an external file. Each line
contains the global coordinates of an element taken at the end of the
element.

The computation takes into account the length of each element, as well
as the rotation angles defined for \hyperref[bend-sbend]{\texttt{SBEND}},
\hyperref[bend-rbend]{\texttt{RBEND}}, thin
\hyperref[sec:multipole]{\texttt{MULTIPOLE}} and thin
\hyperref[sec:rfmultipole]{\texttt{RFMULTIPOLE}} 
elements exclusively.  Rotation angles introduced via the \texttt{KNL},
\texttt{KSL} mechanism for other elements are ignored by
\texttt{SURVEY}, other \madx commands, as well as \ptc commands. 

\textbf{Example:} average LHC ring with CERN coordinates.
\madxmp{
REAL CONST R0 = 1.0; ! to obtain the average ring \\
USE, SEQUENCE=lhcb1;\\
SURVEY, \=X0=-2202.21027, Y0=2359.00656, Z0=2710.63882,  \\
        \>THETA0=-4.315508007, PHI0=0.0124279564, PSI0=-0.0065309236, \\
        \>FILE=survey.lhcb1;
}


\textbf{WARNING :}\\
In the case a machine geometry is constructed with thick lenses, the
circumference changes when the structure is converted into thin lenses
via the \hyperref[chap:makethin]{\texttt{MAKETHIN}} command. This is an
unavoidable feature of \texttt{MAKETHIN}. ONLY the structure with thick
lenses must be used for practical purposes.


%% {\bf INFORMATION:}\\
%% The skew dipole component of a MULTIPOLE element
%% (MULTIPOLE, KSL=\{real\}) is NOT taken into account in the \texttt{SURVEY}
%% calculation. You should use a tilted normal MULTIPOLE or BEND instead.


%\textit{F.Tecker, March 2006}
