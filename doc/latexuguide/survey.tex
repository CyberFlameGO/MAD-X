%%\title{Geometric Layout}
%  Created by: Andre VERDIER, 21-Jun-2002 
%  Changed by: Andre Verdier, 25-Jun-2002 
%  Changed by: Hans Grote, 30-Sep-2002 
%  Changed by: Andre Verdier, 14-Jan-2003 
%  Directory :  /afs/cern.ch/eng/sl/MAD-X/pro/docu/survey/ 
 
\chapter{SURVEY}
\label{chap:survey}

The \texttt{SURVEY} command computes the geometrical layout, i.e. the
coordinates of all machine elements in a 
\href{../Introduction/conventions.html}{global reference system}. These
coordinates can be used for installation. In order to produce
coordinates in a particular system, the initial coordinates and angles
can be specified. 

\madbox{
SURVEY, \=SEQUENCE=string, FILE= string, \\
        \>X0=real, Y0=real, Z0=real, \\
        \>THETA0=real, PHI0=real, PSI0=real;}
%        \>FILE=string, TABLE=string, SEQUENCE=string;

The command has the following attributes:
%% \begin{madlist}
%%    \ttitem{X0} The initial X coordinate [m]. 
%%    \ttitem{Y0} The initial Y coordinate [m]. 
%%    \ttitem{Z0} The initial Z coordinate [m]. 
%%    \ttitem{THETA0} The initial angle theta [rad]. 
%%    \ttitem{PHI0} The initial angle phi [rad]. 
%%    \ttitem{PSI0} The initial angle psi [rad]. 
%% \end{madlist}
\begin{madlist}
  \ttitem{SEQUENCE}   name of sequence to be surveyed. By default the
  last sequence expanded with the \texttt{USE} command will be surveyed.  
  \ttitem{FILE}   name of external file to which the results are
  written.\\ (Default:~survey)
  \ttitem{X0, Y0, Z0} initial horizontal, vertical and longitudinal
  coordinates.\\ (Default: 0.0, 0.0, 0.0) 
  \ttitem{THETA0, PHI0, PSI0} initial horizontal and vertical angles and
  transverse tilt.\\ (Default: 0.0, 0.0, 0.0)
  %% \ttitem{TABLE}   name of internal table containing survey
  %% results.\\ (Default: survey) \\
  %% {\bf NOTE: this parameter is ignored and default value 'survey' is always used.}
\end{madlist}


The computation results are written on the internal
table with name "survey". Results can also be written on an external
file. Each line contains the global coordinates of an element taken at
the end of the element.  

{\bf Example : average LHC ring with CERN coordinates.}
\madxmp{
REAL CONST R0 = 1.0; ! to obtain the average ring \\
USE, SEQUENCE=lhcb1;\\
SURVEY, \=X0=-2202.21027, Y0=2359.00656, Z0=2710.63882,  \\
        \>THETA0=-4.315508007, PHI0=0.0124279564, PSI0=-0.0065309236, \\
        \>FILE=survey.lhcb1;
}


{\bf WARNING :}\\
In the case a machine geometry is constructed with thick
lenses, the circumference will change if the structure is converted into
thin lenses (via the \href{../makethin/makethin.html}{makethin}
command). This is an unavoidable feature. ONLY the structure with thick
lenses must be used for practical purposes.


{\bf INFORMATION:}\\
The skew dipole component of a MULTIPOLE element
(MULTIPOLE, KSL=\{real\}) is NOT taken into account in the \texttt{SURVEY}
calculation. You should use a tilted normal MULTIPOLE or BEND instead.


%\textit{F.Tecker, March 2006}
