\chapter{Table Handling Statements} 
\label{chap:tables}

\section{CREATE}
\label{sec:create}
\madbox{
CREATE, TABLE=tabname, COLUMN= var\{, var\} \{, \_name\} ;
}
creates a table with the specified variables as columns. 
The table created is initially empty and can be subsequently
\hyperref[sec:fill]{filled}, and eventually
\hyperref[sec:write]{written} to file in
\hyperref[chap:tfs]{\texttt{TFS}} format.

The special variable name attribute \texttt{\_name} (name preceded by
underscore) adds the element name to the table at the specified column. 


\section{DELETE}
\label{sec:delete}
\madbox{
DELETE, SEQUENCE=seqname, TABLE=tabname;
}
deletes a sequence with name \texttt{seqname} or a table with name
\texttt{tabname} from memory. The sequence deletion is done without
influence on other sequences that may have elements that werein common
with the deleted sequence.

\section{READTABLE}
\label{sec:readtable}
\madbox{
READTABLE, FILE="filename", TABLE=tabname;
}
reads the \texttt{TFS} file \texttt{filename} containing a \mad table
loads the table into memory with the name \texttt{tabname}. If the table
name is not specified, use the name specified in the
information section of the TFS file. The table can then be manipulated
as any other table, \textsl{i.e.} its values can be accessed, its data
can be plotted or changed, and it can be written out again. 

\section{READMYTABLE}
\label{sec:readmytable}
\madbox{
READMYTABLE, FILE="filename", TABLE=tabname;
}
deprecated alias for \texttt{READTABLE}.

\section{WRITE}
\label{sec:write}
\madbox{
WRITE, TABLE=tabname, FILE="filename";
}
writes the table "tabname" onto the file "filename"; only the rows and
columns of a preceding \texttt{SELECT, FLAG=table,...;} are written. 
If no \texttt{SELECT} has been issued for this table, only the header is
written to file.
If the FILE argument is omitted, the table is written to standard output.  


%\href{http://www.cern.ch/Hans.Grote/hansg_sign.html}{hansg}, June 17, 2002 

\section{SETVARS}
\label{sec:setvars}

The \texttt{SETVARS} command sets the variables with values extracted
from the row of a table.

\madbox{
SETVARS, TABLE=tabname, ROW=integer;
}

The attributes of \texttt{SETVARS} are:
\begin{madlist}
  \ttitem{TABLE} the name of the table. (Default: none)
  \ttitem{ROW} the row number containing the values. (Default: -1)
\end{madlist}

Negative \texttt{ROW} values are allowed and count the row numbers from
the last row, allowing access to the table in reverse order of rows:
\texttt{ROW~=~-1} accesses the last row of the table,
\texttt{ROW~=~-2} accesses the penultimate (one before last) row,
etc\ldots  

Trying to access the table forward beyond the last row, i.e. \texttt{ROW}
strictly greater than \texttt{nrow} the number of rows in the table, or
trying to access the table backwards before the first row, i.e. \texttt{ROW}
strictly lower than \texttt{-nrow}, or trying to access the illegal
\texttt{ROW=0}, all result in a ``row out of bound'' message and no
variable values are returned and set.  

%% If \texttt{nrow} is the number of rows in the table, the conditions
%% \texttt{ROW $>$ nrow}, ie trying to access the table forward beyond the
%% last row, or \texttt{ROW $<$ -nrow}, ie trying to access the table
%% backwards before the first row, or \texttt{ROW=0} 
%% result in a ``row out of bound'' message and no variable values are
%% returned and set. 


\section{SETVARS\_LIN}
\label{sec:setvars-lin}

The \texttt{SETVARS\_LIN} command sets the variables with values calculated
by linear interpolation, or extrapolation, between two rows of a table. 

\madbox{
  SETVARS\_LIN, \=TABLE=tabname, \\
                \>ROW1=integer, ROW2=integer, PARAM=string;
}

The attributes of \texttt{SETVARS\_LIN} are:
\begin{madlist}
  \ttitem{TABLE} the name of the table. (Default: none)
  \ttitem{ROW1} a first row number with values for interpolation. (Default: 0)
  \ttitem{ROW2} a second row number with values for interpolation. (Default: 0)
  \ttitem{PARAM} a string containing the linear interpolation factor or
  the name of a variable or expression containing the interpolation
  factor. If the resulting value of \texttt{PARAM} is outside the
  $[0,1]$ interval, the result is a linear extrapolation. \\
  (Default: "interp", itself defaulting to a value of 0.0 when evaluated)
\end{madlist}

\texttt{SETVARS\_LIN} sets the variables with values calculated through
the following formula that \madx constructs internally as a deferred
expression which is immediately evaluated:
\madxmp{value := value(row1)*(1-param) + value(row2)*param;}
Both the expression and the value of the expression are available to the
user through respectively the commands \hyperref[sec:show]{\texttt{SHOW}} 
and \hyperref[sec:value]{\texttt{VALUE}}.

When the values are represented as strings, \textsl{e.g.} the name or 
keyword of elements, the resulting value is the string in \texttt{ROW1}.

Negative \texttt{ROW\textit{i}} values are allowed and count the row
numbers from the last row, allowing access to the table in reverse order
of rows: 
\texttt{ROW\textit{i}~=~-1} accesses the last row of the table,
\texttt{ROW\textit{i}~=~-2} accesses the penultimate (one before last)
row, etc\ldots  

Trying to access the table forward beyond the last row,
i.e. \texttt{ROW\textit{i}} strictly greater than \texttt{nrow} the
number of rows in the table, or trying to access the table backwards
before the first row, i.e. \texttt{ROW\textit{i}} strictly lower than
\texttt{-nrow}, or trying to access the illegal
\texttt{ROW\textit{i}=0}, all result in a ``row out of bound'' message
and the expression is not constructed or evaluated. 

\textbf{Example:}
\madxmp{
!  extracts the position of the centre of each element from a standard \\
!  TWISS table giving positions at end of elements: \\
len = table(twiss,tablelength); \\
interpolate = 0.5; \\
i = 2;\\
WHILE \=(i < len) \{ \\
      \>SETVARS\_LIN, TABLE=twiss, ROW1=i-1, ROW2=i, PARAM=interpolate; \\
      \>! now variables are interpolated at the center of the elements. \\
      \>! in particular S holds the position of the center of the element. \\
      \>SHOW, s; VALUE, s; \\
      \>\ldots \\ 
      \>i = i + 1; \};
}


\section{SETVARS\_KNOB}
\label{sec:setvars-knob}

The \texttt{SETVARS\_KNOB} command creates or manipulates the expression for the variables by adding a term \texttt{"+ val * knob"} where val is  extracted from the row of a table and knob from the command.

\madbox{
SETVARS\_KNOB, TABLE=tabname, ROW=integer, KNOB=knobname;
}

The attributes of \texttt{SETVARS} are:
\begin{madlist}
  \ttitem{TABLE} the name of the table. (Default: none)
  \ttitem{ROW} the row number containing the values. (Default: -1)
  \ttitem{KNOB} the name of the knob. (Default: none)
  \ttitem{NOAPPEND} when true the term is not appended but replace the exisint value or expression.
\end{madlist}

Negative \texttt{ROW} values are allowed and count the row numbers from
the last row, allowing access to the table in reverse order of rows:
\texttt{ROW~=~-1} accesses the last row of the table,
\texttt{ROW~=~-2} accesses the penultimate (one before last) row,
etc\ldots  


\section{SETVARS\_CONST}
\label{sec:setvars-const}

The \texttt{SETVARS\_CONST} sets the value for the variables to constant.

\madbox{
	SETVARS\_CONST, TABLE=tabname, CONST=value;
}

The attributes of \texttt{SETVARS} are:
\begin{madlist}
	\ttitem{TABLE} the name of the table. (Default: none)
	\ttitem{CONST} the value used to set the variables. (Default: 0)
\end{madlist}

\section{FILL} 
\label{sec:fill}
The \texttt{FILL} command fills a row of a table with the current values 
of all declared column variables of the table.
\madbox{
FILL, TABLE=tabname, ROW=integer;
}

The \texttt{FILL} command takes two arguments:
\begin{madlist}
  \ttitem{TABLE} is the name of the table to be filled. The table must
  have been \hyperref[sec:create]{created} beforehand. 
  The table can then be \hyperref[sec:write]{written} to file in
  \texttt{TFS} format. 

  \ttitem{ROW} is the row number to be filled with the current values of 
  all column variables. \\ 
  \texttt{ROW=0}, or \texttt{ROW=nrow + 1}, where \texttt{nrow} is the
  current number of rows in the table, causes \texttt{FILL} to add a row at
  the end of the table and fill it with the current values of all
  column variables. \\ (Default: 0) 
\end{madlist}

Negative \texttt{ROW} values are allowed and count the row numbers from
the last row, allowing access to the table in reverse order of rows:
\texttt{ROW = -1} accesses the last row of the table,
\texttt{ROW = -2} accesses the penultimate (one before last) row,
etc\ldots  

Trying to access the table forward beyond the last row, i.e. \texttt{ROW}
strictly greater than \texttt{nrow + 1}, where \texttt{nrow} is the number of
rows in the table, or trying to access the table backwards before the
first row, i.e. \texttt{ROW} strictly lower than \texttt{-nrow}, both
result in a ``row out of bound'' message and no values are filled in the
table. 

%% If \texttt{nrow} is the number of rows in the table, the conditions
%% \texttt{ROW $>$ nrow}, ie trying to access the table forward beyond the
%% last row, or \texttt{ROW $<$ -nrow}, ie trying to access the table
%% backwards before the first row, result in a ``row out of bound'' message
%% and no values are filled in the table.

\textbf{Reminder:} One can get access to the current number of rows in a
table using the variable \madxmp{TABLE(tablenanme, TABLELENGTH)}

%% See as well the \href{../Introduction/select.html#ucreate}{user
%% table} example.   
\section{FILL\_KNOB} 
\label{sec:fill-knob}
The \texttt{FILL\_KNOB} command fills a row of a table with the variation (multiplied by scaling factor) of all declared column variables of the table when the knob value change by 1.
\madbox{
FILL\_KNOB, TABLE=tabname, ROW=integer, KNOB=knobname, SCALE=scaling;
}

The \texttt{FILL} command takes two arguments:
\begin{madlist}
  \ttitem{TABLE} is the name of the table to be filled. The table must
  have been \hyperref[sec:create]{created} beforehand.

  \ttitem{ROW} is the row number to be filled with the current values of 
  all column variables. \\ 
  \texttt{ROW=0}, or \texttt{ROW=nrow + 1}, where \texttt{nrow} is the
  current number of rows in the table, causes \texttt{FILL} to add a row at
  the end of the table and fill it with the current values of all
  column variables. \\ (Default: 0) 

  \ttitem{KNOB} the name of the variable that is varied to calculate the values to fill a table.

  \ttitem{SCALE} scaling factor applied to all the variation calculated by varying the knob.

\end{madlist}


\section{SHRINK} 
\label{sec:shrink}
The \texttt{SHRINK} command removes a number of rows at the end of a table.
\madbox{
SHRINK, TABLE=tabname, ROW=integer;
}

The \texttt{SHRINK} command takes two arguments:
\begin{madlist}
  \ttitem{TABLE} is the name of the table from which rows should be removed. 
  The table must have been previously \hyperref[sec:create]{created} and 
  \hyperref[sec:fill]{filled} or read from file with
  \hyperref[sec:readtable]{\texttt{READTABLE}}.
	   
  \ttitem{ROW} is the number of the last row to be kept in the table. 
  All rows beyond the given row number are removed. \\
  Negative values are allowed and count the row numbers from
  the last row, allowing access to the table in reverse order of rows:
  \texttt{ROW~=~-1} removes the last row of the table,
  \texttt{ROW~=~-2} removes the last two rows of the table,
  etc\ldots  \\ 
  (Default: -1) 
\end{madlist}

Trying to access the table forward beyond the last row, i.e. \texttt{ROW}
strictly greater than \texttt{nrow}, where \texttt{nrow} is the number of
rows in the table, or trying to access the table backwards before the
first row, \textsl{i.e.} \texttt{ROW} strictly lower than \texttt{-nrow},
both result in a ``row out of bound'' message and no values are filled
in the table. 


%% \section{SELECT} 
%% \label{sec:select}

%% \begin{verbatim}
%% select, flag=string, range=string, class=string, pattern=string,
%%         sequence=string, full, clear,
%%         column = string{, string},  slice=integer, thick=logical;
%% \end{verbatim} 
%% selects one or several elements for special treatment in a subsequent
%% command based on selection criteria.

%% The selection criteria on a single SELECT statement are logically
%% ANDed, in other words, selected elements have to fulfill the \texttt{RANGE},
%% \texttt{CLASS}, and \texttt{PATTERN} criteria.  
%% The selection criteria on different SELECT statements are logically
%% ORed, in other words selected elements have to fulfill any of the
%% selection criteria accumulated by the different statements.   
%% All selections for a given command remain valid until the "clear" argument
%% is specified; 

%% The "flag" argument allows a determination of the applicability of the
%% SELECT statement and can be one of the following: 
%% \begin{madlist}
%%    \ttitem{seqedit} selection of elements for the
%%      \href{seqedit.html}{seqedit} module.  
%%    \ttitem{error} selection of elements for the
%%      \href{../error/error.html}{error} assignment module.  
%%    \ttitem{makethin} selection of elements for the
%%      \href{../makethin/makethin.html}{makethin} module that
%%      converts the sequence into one with thin elements only.  
%%    \ttitem{sectormap} selection of elements for the
%%      \href{../Introduction/sectormap.html}{sectormap} output file
%%      from the Twiss module.  
%%    \ttitem{save} selection of elements for the \texttt{SAVE} command.  
%%    \ttitem{table} is a table name such as \texttt{twiss}, \texttt{track}
%%      etc., and the rows and columns to be written are selected.  
%% \end{madlist} 

%% The statement
%% \begin{verbatim}
%% SELECT, FLAG=name, FULL;
%% \end{verbatim} 
%% selects ALL positions in the sequence for the flag "name". This is the default
%% for flags for all tables and \texttt{MAKETHIN}.

%% The statement 
%% \begin{verbatim}
%% SELECT, FLAG=name, CLEAR;
%% \end{verbatim} 
%% deselects ALL positions in the sequence for the flag "name". This is the default
%% for flags \texttt{ERROR} and \texttt{SEQEDIT}.

%% "slice" is only used by \href{../makethin/makethin.html}{makethin} and
%% prescribes the number of slices into which the selected elements have to
%% be cut (default = 1).  

%% "column" is only valid for tables and determines the selection of columns
%% to be written into the TFS file. The "name" argument is special in that
%% it refers to the actual name of the selected element. For an example,
%% see \href{../Introduction/select.html}{SELECT}.  

%% "thick" is used to determine whether the selected elements will be
%% treated as thick elements by the MAKETHIN command. This only applies to
%% QUADRUPOLES and BENDS for which thick maps have been explicitely
%% derived. (see ...) 
%% %%2014-Apr-08  17:43:44  ghislain:  A completer.

%% Example: 
%% \begin{verbatim}
%% select, flag = error, class = quadrupole, range = mb[1]/mb[5];
%% select, flag = error, pattern = "^mqw.*";
%% \end{verbatim} 
%% selects all quadrupoles in the range mb[1] to mb[5], as well as all
%% elements (in the whole sequence) with name starting with "mqw", for 
%% treatment by the error module.  

%% Example:  
%% \begin{verbatim}
%% select, flag=save, class=variable, pattern="abc.*";
%% save, file=mysave;
%% \end{verbatim} 
%% will save all variables containing "abc" in their name.
%% However note that since the element class "variable" does not exist, any
%% element with name containing "abc" will not be saved. 

%% \vskip 1cm
%% \hrule
%% \vskip 1cm

%% %% Imported from chapter 2
%% %\subsection{Selection Statements}

%% The elements, or a range of elements, in a sequence can be selected for
%% various purposes. Such selections remain valid until cleared (in
%% difference to \madeight); it is therefore recommended to always start with a  

%% \begin{verbatim}
%% select, flag =..., clear;
%% \end{verbatim} 
%% before setting a new selection. 
%% \begin{verbatim}
%% SELECT, FLAG=name, RANGE=range, CLASS=class, PATTERN=pattern [,FULL] [,CLEAR];
%% \end{verbatim} 
%% where the name for FLAG can be one of ERROR, MAKETHIN, SEQEDIT or the
%% name of a twiss table which is established for all sequence positions in
%% general.  

%% Selected elements have to fulfill the \href{ranges.html#range}{RANGE},
%% \href{ranges.html#class}{CLASS}, and \href{wildcard.html}{PATTERN}
%% criteria.  

%% Any number of SELECT commands can be issued for the same flag and are
%% accumulated (logically ORed). In this context note the following:  

%% \begin{verbatim}
%% SELECT, FLAG=name, FULL;
%% \end{verbatim} 
%% selects all positions in the sequence for this flag. This is the default
%% for all tables and makethin, whereas for ERROR and SEQEDIT the default
%% is "nothing selected".  

%% %\href{save_select}{}
%% \label{save_select}
%% SAVE: A SELECT,FLAG=SAVE statement causes the
%% selected sequences, elements, and variables to be written into the save
%% file. A class (only used for element selection), and a pattern can be
%% specified. Example:  
%% \begin{verbatim}
%% select, flag=save, class=variable, pattern="abc.*";
%% save, file=mysave;
%% \end{verbatim} 
%% will save all variables containing "abc" in their name,
%% but not elements with names containing "abc" since the class "variable"
%% does not exist (astucieux, non ?).  

%% SECTORMAP: A SELECT,FLAG=SECTORMAP statement causes sectormaps to be
%% written into the file "sectormap" like in \madeight. For the file to be
%% written, a flag SECTORMAP must be issued on the TWISS command in
%% addition.  

%% TWISS: A SELECT,FLAG=TWISS statement causes the selected rows and
%% columns to be written into the Twiss TFS file (former OPTICS command in
%% \madeight). The column selection is done on the same select. See as well
%% example 2.  

%% %% Example 1:  
%% %% \begin{verbatim}
%% %% TITLE,'Test input for MAD-X';

%% %% option,rbarc=false; // use arc length of rbends
%% %% beam; ! sets the default beam for the following sequence
%% %% option,-echo;
%% %% call file=fv9.opt;  ! contains optics parameters
%% %% call file="fv9.seq"; ! contains a small sequence "fivecell"
%% %% OPTION,ECHO;
%% %% SELECT,FLAG=SECTORMAP,clear;
%% %% SELECT,FLAG=SECTORMAP,PATTERN="^m.*";
%% %% SELECT,FLAG=TWISS,clear;
%% %% SELECT,FLAG=TWISS,PATTERN="^m.*",column=name,s,betx,bety;
%% %% USE,PERIOD=FIVECELL;
%% %% twiss,file=optics,sectormap;
%% %% stop;
%% %% \end{verbatim} 

%% %% This produces a file \href{sectormap.html}{sectormap}, and a
%% %% twiss output file \label{tfs} (name = optics):  
%% %% \begin{verbatim}
%% %% @ TYPE             %05s "TWISS"
%% %% @ PARTICLE         %08s "POSITRON"
%% %% @ MASS             %le          0.000510998902
%% %% @ CHARGE           %le                       1
%% %% @ E0               %le                       1
%% %% @ PC               %le           0.99999986944
%% %% @ GAMMA            %le           1956.95136738
%% %% @ KBUNCH           %le                       1
%% %% @ NPART            %le                       0
%% %% @ EX               %le                       1
%% %% @ EY               %le                       1
%% %% @ ET               %le                       0
%% %% @ LENGTH           %le                   534.6
%% %% @ ALFA             %le        0.00044339992938
%% %% @ ORBIT5           %le                      -0
%% %% @ GAMMATR          %le           47.4900022541
%% %% @ Q1               %le           1.25413071556
%% %% @ Q2               %le           1.25485338377
%% %% @ DQ1              %le           1.05329608302
%% %% @ DQ2              %le           1.04837000224
%% %% @ DXMAX            %le           2.17763211131
%% %% @ DYMAX            %le                       0
%% %% @ XCOMAX           %le                       0
%% %% @ YCOMAX           %le                       0
%% %% @ BETXMAX          %le            177.70993499
%% %% @ BETYMAX          %le           177.671582415
%% %% @ XCORMS           %le                       0
%% %% @ YCORMS           %le                       0
%% %% @ DXRMS            %le           1.66004270906
%% %% @ DYRMS            %le                       0
%% %% @ DELTAP           %le                       0
%% %% @ TITLE            %20s "Test input for MAD-X"
%% %% @ ORIGIN           %16s "MAD-X 0.20 Linux"
%% %% @ DATE             %08s "07/06/02"
%% %% @ TIME             %08s "14.25.51"
%% %% * NAME               S                  BETX               BETY               
%% %% $ %s                 %le                %le                %le                
%% %%  "MSCBH"             4.365              171.6688159        33.31817319       
%% %%  "MB"                19.72              108.1309095        58.58680717       
%% %%  "MB"                35.38              61.96499987        102.9962313       
%% %%  "MB"                51.04              34.61640793        166.2227523       
%% %%  "MSCBV.1"           57.825             33.34442808        171.6309057       
%% %%  "MB"                73.18              58.61984637        108.0956006       
%% %%  "MB"                88.84              103.0313887        61.93159422       
%% %%  "MB"                104.5              166.2602486        34.58939635       
%% %%  "MSCBH"             111.285            171.6688159        33.31817319       
%% %%  "MB"                126.64             108.1309095        58.58680717       
%% %%  "MB"                142.3              61.96499987        102.9962313       
%% %%  "MB"                157.96             34.61640793        166.2227523       
%% %%  "MSCBV"             164.745            33.34442808        171.6309057       
%% %%  "MB"                180.1              58.61984637        108.0956006       
%% %%  "MB"                195.76             103.0313887        61.93159422       
%% %%  "MB"                211.42             166.2602486        34.58939635       
%% %%  "MSCBH"             218.205            171.6688159        33.31817319       
%% %%  "MB"                233.56             108.1309095        58.58680717       
%% %%  "MB"                249.22             61.96499987        102.9962313       
%% %%  "MB"                264.88             34.61640793        166.2227523       
%% %%  "MSCBV"             271.665            33.34442808        171.6309057       
%% %%  "MB"                287.02             58.61984637        108.0956006       
%% %%  "MB"                302.68             103.0313887        61.93159422       
%% %%  "MB"                318.34             166.2602486        34.58939635       
%% %%  "MSCBH"             325.125            171.6688159        33.31817319       
%% %%  "MB"                340.48             108.1309095        58.58680717       
%% %%  "MB"                356.14             61.96499987        102.9962313       
%% %%  "MB"                371.8              34.61640793        166.2227523       
%% %%  "MSCBV"             378.585            33.34442808        171.6309057       
%% %%  "MB"                393.94             58.61984637        108.0956006       
%% %%  "MB"                409.6              103.0313887        61.93159422       
%% %%  "MB"                425.26             166.2602486        34.58939635       
%% %%  "MSCBH"             432.045            171.6688159        33.31817319       
%% %%  "MB"                447.4              108.1309095        58.58680717       
%% %%  "MB"                463.06             61.96499987        102.9962313       
%% %%  "MB"                478.72             34.61640793        166.2227523       
%% %%  "MSCBV"             485.505            33.34442808        171.6309057       
%% %%  "MB"                500.86             58.61984637        108.0956006       
%% %%  "MB"                516.52             103.0313887        61.93159422       
%% %%  "MB"                532.18             166.2602486        34.58939635       
%% %% \end{verbatim}

%%  %% Example 2: 

%% %%  Addition of variables to (any internal) table: 
%% %% \begin{verbatim}
%% %% select, flag=table, column=name, s, betx, ..., var1, var2, ...; ! or
%% %% select, flag=table, full, column=var1, var2, ...; ! default col.s + new
%% %% \end{verbatim} 
%% %% will write the current value of var1 etc. into the table each time a new
%% %% line is added; values from the same (current) line can be accessed by
%% %% these variables, e.g.  
%% %% \begin{verbatim}
%% %% var1 := sqrt(beam->ex*table(twiss,betx));
%% %% \end{verbatim} 
%% %% in the case of table above being "twiss". The plot command accepts the
%% %% new variables.  

%% %% Remark: this replaces the "string" variables of MAD-8. 

%% %%  This example demonstrates as well the usage of a user defined table \label{ucreate}. 
%% %% \begin{verbatim}
%% %% beam,ex=1.e-6,ey=1.e-3;
%% %% // element definitions
%% %% mb:rbend, l=14.2, angle:=0,k0:=bang/14.2;
%% %% mq:quadrupole, l:=3.1,apertype=ellipse,aperture={1,2};
%% %% qft:mq, l:=0.31, k1:=kqf,tilt=-pi/4;
%% %% qf.1:mq, l:=3.1, k1:=kqf;
%% %% qf.2:mq, l:=3.1, k1:=kqf;
%% %% qf.3:mq, l:=3.1, k1:=kqf;
%% %% qf.4:mq, l:=3.1, k1:=kqf;
%% %% qf.5:mq, l:=3.1, k1:=kqf;
%% %% qd.1:mq, l:=3.1, k1:=kqd;
%% %% qd.2:mq, l:=3.1, k1:=kqd;
%% %% qd.3:mq, l:=3.1, k1:=kqd;
%% %% qd.4:mq, l:=3.1, k1:=kqd;
%% %% qd.5:mq, l:=3.1, k1:=kqd;
%% %% bph:hmonitor, l:=l.bpm;
%% %% bpv:vmonitor, l:=l.bpm;
%% %% cbh:hkicker;
%% %% cbv:vkicker;
%% %% cbh.1:cbh, kick:=acbh1;
%% %% cbh.2:cbh, kick:=acbh2;
%% %% cbh.3:cbh, kick:=acbh3;
%% %% cbh.4:cbh, kick:=acbh4;
%% %% cbh.5:cbh, kick:=acbh5;
%% %% cbv.1:cbv, kick:=acbv1;
%% %% cbv.2:cbv, kick:=acbv2;
%% %% cbv.3:cbv, kick:=acbv3;
%% %% cbv.4:cbv, kick:=acbv4;
%% %% cbv.5:cbv, kick:=acbv5;
%% %% !mscbh:sextupole, l:=1.1, k2:=ksf;
%% %% mscbh:multipole, knl:={0,0,0,ksf},tilt=-pi/8;
%% %% mscbv:sextupole, l:=1.1, k2:=ksd;
%% %% !mscbv:octupole, l:=1.1, k3:=ksd,tilt=-pi/8;

%% %% // sequence declaration

%% %% fivecell:sequence, refer=centre, l=534.6;
%% %%    qf.1:qf.1, at=1.550000e+00;
%% %%    qft:qft, at=3.815000e+00;
%% %% !   mscbh:mscbh, at=3.815000e+00;
%% %%    cbh.1:cbh.1, at=4.365000e+00;
%% %%    mb:mb, at=1.262000e+01;
%% %%    mb:mb, at=2.828000e+01;
%% %%    mb:mb, at=4.394000e+01;
%% %%    bpv:bpv, at=5.246000e+01;
%% %%    qd.1:qd.1, at=5.501000e+01;
%% %%    mscbv:mscbv, at=5.727500e+01;
%% %%    cbv.1:cbv.1, at=5.782500e+01;
%% %%    mb:mb, at=6.608000e+01;
%% %%    mb:mb, at=8.174000e+01;
%% %%    mb:mb, at=9.740000e+01;
%% %%    bph:bph, at=1.059200e+02;
%% %%    qf.2:qf.2, at=1.084700e+02;
%% %%    mscbh:mscbh, at=1.107350e+02;
%% %%    cbh.2:cbh.2, at=1.112850e+02;
%% %%    mb:mb, at=1.195400e+02;
%% %%    mb:mb, at=1.352000e+02;
%% %%    mb:mb, at=1.508600e+02;
%% %%    bpv:bpv, at=1.593800e+02;
%% %%    qd.2:qd.2, at=1.619300e+02;
%% %%    mscbv:mscbv, at=1.641950e+02;
%% %%    cbv.2:cbv.2, at=1.647450e+02;
%% %%    mb:mb, at=1.730000e+02;
%% %%    mb:mb, at=1.886600e+02;
%% %%    mb:mb, at=2.043200e+02;
%% %%    bph:bph, at=2.128400e+02;
%% %%    qf.3:qf.3, at=2.153900e+02;
%% %%    mscbh:mscbh, at=2.176550e+02;
%% %%    cbh.3:cbh.3, at=2.182050e+02;
%% %%    mb:mb, at=2.264600e+02;
%% %%    mb:mb, at=2.421200e+02;
%% %%    mb:mb, at=2.577800e+02;
%% %%    bpv:bpv, at=2.663000e+02;
%% %%    qd.3:qd.3, at=2.688500e+02;
%% %%    mscbv:mscbv, at=2.711150e+02;
%% %%    cbv.3:cbv.3, at=2.716650e+02;
%% %%    mb:mb, at=2.799200e+02;
%% %%    mb:mb, at=2.955800e+02;
%% %%    mb:mb, at=3.112400e+02;
%% %%    bph:bph, at=3.197600e+02;
%% %%    qf.4:qf.4, at=3.223100e+02;
%% %%    mscbh:mscbh, at=3.245750e+02;
%% %%    cbh.4:cbh.4, at=3.251250e+02;
%% %%    mb:mb, at=3.333800e+02;
%% %%    mb:mb, at=3.490400e+02;
%% %%    mb:mb, at=3.647000e+02;
%% %%    bpv:bpv, at=3.732200e+02;
%% %%    qd.4:qd.4, at=3.757700e+02;
%% %%    mscbv:mscbv, at=3.780350e+02;
%% %%    cbv.4:cbv.4, at=3.785850e+02;
%% %%    mb:mb, at=3.868400e+02;
%% %%    mb:mb, at=4.025000e+02;
%% %%    mb:mb, at=4.181600e+02;
%% %%    bph:bph, at=4.266800e+02;
%% %%    qf.5:qf.5, at=4.292300e+02;
%% %%    mscbh:mscbh, at=4.314950e+02;
%% %%    cbh.5:cbh.5, at=4.320450e+02;
%% %%    mb:mb, at=4.403000e+02;
%% %%    mb:mb, at=4.559600e+02;
%% %%    mb:mb, at=4.716200e+02;
%% %%    bpv:bpv, at=4.801400e+02;
%% %%    qd.5:qd.5, at=4.826900e+02;
%% %%    mscbv:mscbv, at=4.849550e+02;
%% %%    cbv.5:cbv.5, at=4.855050e+02;
%% %%    mb:mb, at=4.937600e+02;
%% %%    mb:mb, at=5.094200e+02;
%% %%    mb:mb, at=5.250800e+02;
%% %%    bph:bph, at=5.336000e+02;
%% %% end:marker, at=5.346000e+02;
%% %% endsequence;

%% %% // forces and other constants

%% %% l.bpm:=.3;
%% %% bang:=.509998807401e-2;
%% %% kqf:=.872651312e-2;
%% %% kqd:=-.872777242e-2;
%% %% ksf:=.0198492943;
%% %% ksd:=-.039621283;
%% %% acbv1:=1.e-4;
%% %% acbh1:=1.e-4;
%% %% !save,sequence=fivecell,file,mad8;

%% %% s := table(twiss,bpv[5],betx);
%% %% myvar := sqrt(beam->ex*table(twiss,betx));
%% %% use, period=fivecell;
%% %% select,flag=twiss,column=name,s,myvar,apertype;
%% %% twiss,file;
%% %% n = 0;
%% %% create,table=mytab,column=dp,mq1,mq2;
%% %% mq1:=table(summ,q1);
%% %% mq2:=table(summ,q2);
%% %% while ( n < 11)
%% %% {
%% %%   n = n + 1;
%% %%   dp = 1.e-4*(n-6);
%% %%   twiss,deltap=dp;
%% %%   fill,table=mytab;
%% %% }
%% %% write,table=mytab;
%% %% plot,haxis=s,vaxis=aper_1,aper_2,colour=100,range=#s/cbv.1,notitle;
%% %% stop;
%% %% \end{verbatim}
%% %% prints the following user table on output:

%% %% \begin{verbatim}
%% %% @ NAME             %05s "MYTAB"
%% %% @ TYPE             %04s "USER"
%% %% @ TITLE            %08s "no-title"
%% %% @ ORIGIN           %16s "MAD-X 1.09 Linux"
%% %% @ DATE             %08s "10/12/02"
%% %% @ TIME             %08s "10.45.25"
%% %% * DP                 MQ1                MQ2                
%% %% $ %le                %le                %le                
%% %%  -0.0005            1.242535951        1.270211135       
%% %%  -0.0004            1.242495534        1.270197018       
%% %%  -0.0003            1.242452432        1.270185673       
%% %%  -0.0002            1.242406653        1.270177093       
%% %%  -0.0001            1.242358206        1.270171269       
%% %%  0                  1.242307102        1.27016819        
%% %%  0.0001             1.242253353        1.270167843       
%% %%  0.0002             1.242196974        1.270170214       
%% %%  0.0003             1.24213798         1.270175288       
%% %%  0.0004             1.242076387        1.270183048       
%% %%  0.0005             1.242012214        1.270193477       
%% %% \end{verbatim}
%% %% and produces a twiss file with the additional column myvar, as well as a plot
%% %% file with the aperture values plotted.


%% %% \href{screate}{}

%% %% Example of joining two tables with different length into a third table
%% %% making use of the length of either table as given by
%% %% table("your\_table\_name", tablelength) and adding names by the "\_name"
%% %% attribute.

%% %% \begin{verbatim}
%% %% title,   "summing of offset and alignment tables";
%% %% set,    format="13.6f";

%% %% readtable, table=align,  file="align.ip2.b1.tfs";   // mesured alignment
%% %% readtable, table=offset, file="offset.ip2.b1.tfs";  // nominal offsets

%% %% n_elem  =  table(offset, tablelength);

%% %% create,  table=align_offset, column=_name,s_ip,x_off,dx_off,ddx_off,y_off,dy_off,ddy_off;

%% %% calcul(elem_name) : macro = {
%% %%     x_off = table(align,  elem_name, x_ali) + x_off;
%% %%     y_off = table(align,  elem_name, y_ali) + y_off;
%% %% }


%% %% one_elem(j_elem) : macro = {
%% %%     setvars, table=offset, row=j_elem;
%% %%     exec,  calcul(tabstring(offset, name, j_elem));
%% %%     fill,  table=align_offset;
%% %% }


%% %% i_elem = 0;
%% %% while (i_elem < n_elem) { i_elem = i_elem + 1; exec,  one_elem($i_elem); }

%% %% write, table=align_offset, file="align_offset.tfs";

%% %% stop;
%% %% \end{verbatim}

%% %%






%% %%%\title{SELECT}
%  Changed by: Hans Grote, 16-Jan-2003 

\subsection{Selection Statements}

The elements, or a range of elements, in a sequence can be selected for
various purposes. Such selections remain valid until cleared (in
difference to MAD-8); it is therefore recommended to always start with a  

\begin{verbatim}
select, flag =..., clear;
\end{verbatim} 
before setting a new selection. 
\begin{verbatim}
SELECT, FLAG=name, RANGE=range, CLASS=class, PATTERN=pattern [,FULL] [,CLEAR];
\end{verbatim} 
where the name for FLAG can be one of ERROR, MAKETHIN, SEQEDIT or the
name of a twiss table which is established for all sequence positions in
general.  

Selected elements have to fulfill the \href{ranges.html#range}{RANGE},
\href{ranges.html#class}{CLASS}, and \href{wildcard.html}{PATTERN}
criteria.  

Any number of SELECT commands can be issued for the same flag and are
accumulated (logically ORed). In this context note the following:  

\begin{verbatim}
SELECT, FLAG=name, FULL;
\end{verbatim} 
selects all positions in the sequence for this flag. This is the default
for all tables and makethin, whereas for ERROR and SEQEDIT the default
is "nothing selected".  

\href{save_select}{}SAVE: A SELECT,FLAG=SAVE statement causes the
selected sequences, elements, and variables to be written into the save
file. A class (only used for element selection), and a pattern can be
specified. Example:  
\begin{verbatim}
select, flag=save, class=variable, pattern="abc.*";
save, file=mysave;
\end{verbatim} 
will save all variables (and sequences) containing "abc" in their name,
but not elements with names containing "abc" since the class "variable"
does not exist (astucieux, non ?).  

SECTORMAP: A SELECT,FLAG=SECTORMAP statement causes sectormaps to be
written into the file "sectormap" like in MAD-8. For the file to be
written, a flag SECTORMAP must be issued on the TWISS command in
addition.  

TWISS: A SELECT,FLAG=TWISS statement causes the selected rows and
columns to be written into the Twiss TFS file (former OPTICS command in
MAD-8). The column selection is done on the same select. See as well
example 2.  

Example 1:  
\begin{verbatim}
TITLE,'Test input for MAD-X';

option,rbarc=false; // use arc length of rbends
beam; ! sets the default beam for the following sequence
option,-echo;
call file=fv9.opt;  ! contains optics parameters
call file="fv9.seq"; ! contains a small sequence "fivecell"
OPTION,ECHO;
SELECT,FLAG=SECTORMAP,clear;
SELECT,FLAG=SECTORMAP,PATTERN="^m.*";
SELECT,FLAG=TWISS,clear;
SELECT,FLAG=TWISS,PATTERN="^m.*",column=name,s,betx,bety;
USE,PERIOD=FIVECELL;
twiss,file=optics,sectormap;
stop;
\end{verbatim} 

This produces a file \href{sectormap.html}{sectormap}, and a
\href{tfs}{}twiss output file (name = optics):  
\begin{verbatim}
@ TYPE             %05s "TWISS"
@ PARTICLE         %08s "POSITRON"
@ MASS             %le          0.000510998902
@ CHARGE           %le                       1
@ E0               %le                       1
@ PC               %le           0.99999986944
@ GAMMA            %le           1956.95136738
@ KBUNCH           %le                       1
@ NPART            %le                       0
@ EX               %le                       1
@ EY               %le                       1
@ ET               %le                       0
@ LENGTH           %le                   534.6
@ ALFA             %le        0.00044339992938
@ ORBIT5           %le                      -0
@ GAMMATR          %le           47.4900022541
@ Q1               %le           1.25413071556
@ Q2               %le           1.25485338377
@ DQ1              %le           1.05329608302
@ DQ2              %le           1.04837000224
@ DXMAX            %le           2.17763211131
@ DYMAX            %le                       0
@ XCOMAX           %le                       0
@ YCOMAX           %le                       0
@ BETXMAX          %le            177.70993499
@ BETYMAX          %le           177.671582415
@ XCORMS           %le                       0
@ YCORMS           %le                       0
@ DXRMS            %le           1.66004270906
@ DYRMS            %le                       0
@ DELTAP           %le                       0
@ TITLE            %20s "Test input for MAD-X"
@ ORIGIN           %16s "MAD-X 0.20 Linux"
@ DATE             %08s "07/06/02"
@ TIME             %08s "14.25.51"
* NAME               S                  BETX               BETY               
$ %s                 %le                %le                %le                
 "MSCBH"             4.365              171.6688159        33.31817319       
 "MB"                19.72              108.1309095        58.58680717       
 "MB"                35.38              61.96499987        102.9962313       
 "MB"                51.04              34.61640793        166.2227523       
 "MSCBV.1"           57.825             33.34442808        171.6309057       
 "MB"                73.18              58.61984637        108.0956006       
 "MB"                88.84              103.0313887        61.93159422       
 "MB"                104.5              166.2602486        34.58939635       
 "MSCBH"             111.285            171.6688159        33.31817319       
 "MB"                126.64             108.1309095        58.58680717       
 "MB"                142.3              61.96499987        102.9962313       
 "MB"                157.96             34.61640793        166.2227523       
 "MSCBV"             164.745            33.34442808        171.6309057       
 "MB"                180.1              58.61984637        108.0956006       
 "MB"                195.76             103.0313887        61.93159422       
 "MB"                211.42             166.2602486        34.58939635       
 "MSCBH"             218.205            171.6688159        33.31817319       
 "MB"                233.56             108.1309095        58.58680717       
 "MB"                249.22             61.96499987        102.9962313       
 "MB"                264.88             34.61640793        166.2227523       
 "MSCBV"             271.665            33.34442808        171.6309057       
 "MB"                287.02             58.61984637        108.0956006       
 "MB"                302.68             103.0313887        61.93159422       
 "MB"                318.34             166.2602486        34.58939635       
 "MSCBH"             325.125            171.6688159        33.31817319       
 "MB"                340.48             108.1309095        58.58680717       
 "MB"                356.14             61.96499987        102.9962313       
 "MB"                371.8              34.61640793        166.2227523       
 "MSCBV"             378.585            33.34442808        171.6309057       
 "MB"                393.94             58.61984637        108.0956006       
 "MB"                409.6              103.0313887        61.93159422       
 "MB"                425.26             166.2602486        34.58939635       
 "MSCBH"             432.045            171.6688159        33.31817319       
 "MB"                447.4              108.1309095        58.58680717       
 "MB"                463.06             61.96499987        102.9962313       
 "MB"                478.72             34.61640793        166.2227523       
 "MSCBV"             485.505            33.34442808        171.6309057       
 "MB"                500.86             58.61984637        108.0956006       
 "MB"                516.52             103.0313887        61.93159422       
 "MB"                532.18             166.2602486        34.58939635       
\end{verbatim}

 Example 2: 

 Addition of variables to (any internal) table: 
\begin{verbatim}
select, flag=table, column=name, s, betx, ..., var1, var2, ...; ! or
select, flag=table, full, column=var1, var2, ...; ! default col.s + new
\end{verbatim} 
will write the current value of var1 etc. into the table each time a new
line is added; values from the same (current) line can be accessed by
these variables, e.g.  
\begin{verbatim}
var1 := sqrt(beam->ex*table(twiss,betx));
\end{verbatim} 
in the case of table above being "twiss". The plot command accepts the
new variables.  

Remark: this replaces the "string" variables of MAD-8. 

\href{ucreate}{} This example demonstrates as well the usage of a user defined table. 
\begin{verbatim}
beam,ex=1.e-6,ey=1.e-3;
// element definitions
mb:rbend, l=14.2, angle:=0,k0:=bang/14.2;
mq:quadrupole, l:=3.1,apertype=ellipse,aperture={1,2};
qft:mq, l:=0.31, k1:=kqf,tilt=-pi/4;
qf.1:mq, l:=3.1, k1:=kqf;
qf.2:mq, l:=3.1, k1:=kqf;
qf.3:mq, l:=3.1, k1:=kqf;
qf.4:mq, l:=3.1, k1:=kqf;
qf.5:mq, l:=3.1, k1:=kqf;
qd.1:mq, l:=3.1, k1:=kqd;
qd.2:mq, l:=3.1, k1:=kqd;
qd.3:mq, l:=3.1, k1:=kqd;
qd.4:mq, l:=3.1, k1:=kqd;
qd.5:mq, l:=3.1, k1:=kqd;
bph:hmonitor, l:=l.bpm;
bpv:vmonitor, l:=l.bpm;
cbh:hkicker;
cbv:vkicker;
cbh.1:cbh, kick:=acbh1;
cbh.2:cbh, kick:=acbh2;
cbh.3:cbh, kick:=acbh3;
cbh.4:cbh, kick:=acbh4;
cbh.5:cbh, kick:=acbh5;
cbv.1:cbv, kick:=acbv1;
cbv.2:cbv, kick:=acbv2;
cbv.3:cbv, kick:=acbv3;
cbv.4:cbv, kick:=acbv4;
cbv.5:cbv, kick:=acbv5;
!mscbh:sextupole, l:=1.1, k2:=ksf;
mscbh:multipole, knl:={0,0,0,ksf},tilt=-pi/8;
mscbv:sextupole, l:=1.1, k2:=ksd;
!mscbv:octupole, l:=1.1, k3:=ksd,tilt=-pi/8;

// sequence declaration

fivecell:sequence, refer=centre, l=534.6;
   qf.1:qf.1, at=1.550000e+00;
   qft:qft, at=3.815000e+00;
!   mscbh:mscbh, at=3.815000e+00;
   cbh.1:cbh.1, at=4.365000e+00;
   mb:mb, at=1.262000e+01;
   mb:mb, at=2.828000e+01;
   mb:mb, at=4.394000e+01;
   bpv:bpv, at=5.246000e+01;
   qd.1:qd.1, at=5.501000e+01;
   mscbv:mscbv, at=5.727500e+01;
   cbv.1:cbv.1, at=5.782500e+01;
   mb:mb, at=6.608000e+01;
   mb:mb, at=8.174000e+01;
   mb:mb, at=9.740000e+01;
   bph:bph, at=1.059200e+02;
   qf.2:qf.2, at=1.084700e+02;
   mscbh:mscbh, at=1.107350e+02;
   cbh.2:cbh.2, at=1.112850e+02;
   mb:mb, at=1.195400e+02;
   mb:mb, at=1.352000e+02;
   mb:mb, at=1.508600e+02;
   bpv:bpv, at=1.593800e+02;
   qd.2:qd.2, at=1.619300e+02;
   mscbv:mscbv, at=1.641950e+02;
   cbv.2:cbv.2, at=1.647450e+02;
   mb:mb, at=1.730000e+02;
   mb:mb, at=1.886600e+02;
   mb:mb, at=2.043200e+02;
   bph:bph, at=2.128400e+02;
   qf.3:qf.3, at=2.153900e+02;
   mscbh:mscbh, at=2.176550e+02;
   cbh.3:cbh.3, at=2.182050e+02;
   mb:mb, at=2.264600e+02;
   mb:mb, at=2.421200e+02;
   mb:mb, at=2.577800e+02;
   bpv:bpv, at=2.663000e+02;
   qd.3:qd.3, at=2.688500e+02;
   mscbv:mscbv, at=2.711150e+02;
   cbv.3:cbv.3, at=2.716650e+02;
   mb:mb, at=2.799200e+02;
   mb:mb, at=2.955800e+02;
   mb:mb, at=3.112400e+02;
   bph:bph, at=3.197600e+02;
   qf.4:qf.4, at=3.223100e+02;
   mscbh:mscbh, at=3.245750e+02;
   cbh.4:cbh.4, at=3.251250e+02;
   mb:mb, at=3.333800e+02;
   mb:mb, at=3.490400e+02;
   mb:mb, at=3.647000e+02;
   bpv:bpv, at=3.732200e+02;
   qd.4:qd.4, at=3.757700e+02;
   mscbv:mscbv, at=3.780350e+02;
   cbv.4:cbv.4, at=3.785850e+02;
   mb:mb, at=3.868400e+02;
   mb:mb, at=4.025000e+02;
   mb:mb, at=4.181600e+02;
   bph:bph, at=4.266800e+02;
   qf.5:qf.5, at=4.292300e+02;
   mscbh:mscbh, at=4.314950e+02;
   cbh.5:cbh.5, at=4.320450e+02;
   mb:mb, at=4.403000e+02;
   mb:mb, at=4.559600e+02;
   mb:mb, at=4.716200e+02;
   bpv:bpv, at=4.801400e+02;
   qd.5:qd.5, at=4.826900e+02;
   mscbv:mscbv, at=4.849550e+02;
   cbv.5:cbv.5, at=4.855050e+02;
   mb:mb, at=4.937600e+02;
   mb:mb, at=5.094200e+02;
   mb:mb, at=5.250800e+02;
   bph:bph, at=5.336000e+02;
end:marker, at=5.346000e+02;
endsequence;

// forces and other constants

l.bpm:=.3;
bang:=.509998807401e-2;
kqf:=.872651312e-2;
kqd:=-.872777242e-2;
ksf:=.0198492943;
ksd:=-.039621283;
acbv1:=1.e-4;
acbh1:=1.e-4;
!save,sequence=fivecell,file,mad8;

s := table(twiss,bpv[5],betx);
myvar := sqrt(beam->ex*table(twiss,betx));
use, period=fivecell;
select,flag=twiss,column=name,s,myvar,apertype;
twiss,file;
n = 0;
create,table=mytab,column=dp,mq1,mq2;
mq1:=table(summ,q1);
mq2:=table(summ,q2);
while ( n < 11)
{
  n = n + 1;
  dp = 1.e-4*(n-6);
  twiss,deltap=dp;
  fill,table=mytab;
}
write,table=mytab;
plot,haxis=s,vaxis=aper_1,aper_2,colour=100,range=#s/cbv.1,notitle;
stop;
\end{verbatim}
prints the following user table on output:

\begin{verbatim}
@ NAME             %05s "MYTAB"
@ TYPE             %04s "USER"
@ TITLE            %08s "no-title"
@ ORIGIN           %16s "MAD-X 1.09 Linux"
@ DATE             %08s "10/12/02"
@ TIME             %08s "10.45.25"
* DP                 MQ1                MQ2                
$ %le                %le                %le                
 -0.0005            1.242535951        1.270211135       
 -0.0004            1.242495534        1.270197018       
 -0.0003            1.242452432        1.270185673       
 -0.0002            1.242406653        1.270177093       
 -0.0001            1.242358206        1.270171269       
 0                  1.242307102        1.27016819        
 0.0001             1.242253353        1.270167843       
 0.0002             1.242196974        1.270170214       
 0.0003             1.24213798         1.270175288       
 0.0004             1.242076387        1.270183048       
 0.0005             1.242012214        1.270193477       
\end{verbatim}
and produces a twiss file with the additional column myvar, as well as a plot
file with the aperture values plotted.


\href{screate}{}

Example of joing 2 tables with different length into a third table
making use of the length of either table as given by
table("your\_table\_name", tablelength) and adding names by the "\_name"
attribute.

\begin{verbatim}
title,   "summing of offset and alignment tables";
set,    format="13.6f";

readtable, table=align,  file="align.ip2.b1.tfs";   // mesured alignment
readtable, table=offset, file="offset.ip2.b1.tfs";  // nominal offsets

n_elem  =  table(offset, tablelength);

create,  table=align_offset, column=_name,s_ip,x_off,dx_off,ddx_off,y_off,dy_off,ddy_off;

calcul(elem_name) : macro = {
    x_off = table(align,  elem_name, x_ali) + x_off;
    y_off = table(align,  elem_name, y_ali) + y_off;
}


one_elem(j_elem) : macro = {
    setvars, table=offset, row=j_elem;
    exec,  calcul(tabstring(offset, name, j_elem));
    fill,  table=align_offset;
}


i_elem = 0;
while (i_elem < n_elem) { i_elem = i_elem + 1; exec,  one_elem($i_elem); }

write, table=align_offset, file="align_offset.tfs";

stop;
\end{verbatim}

% \href{http://www.cern.ch/Hans.Grote/hansg_sign.html}{hansg}, May 8, 2001


%% \section{SELECT}
%% \label{sec:selection}
%% The elements, or a range of elements, in a sequence can be selected for
%% various purposes. Such selections remain valid until cleared (in
%% difference to \madeight); it is therefore recommended to always start with a  

%% \begin{verbatim}
%% select, flag =..., clear;
%% \end{verbatim} 
%% before setting a new selection. 
%% \begin{verbatim}
%% SELECT, FLAG=name, RANGE=range, CLASS=class, PATTERN=pattern [,FULL] [,CLEAR];
%% \end{verbatim} 
%% where the name for FLAG can be one of ERROR, MAKETHIN, SEQEDIT or the
%% name of a twiss table which is established for all sequence positions in
%% general.  

%% Selected elements have to fulfill the \href{ranges.html#range}{RANGE},
%% \href{ranges.html#class}{CLASS}, and \href{wildcard.html}{PATTERN}
%% criteria.  

%% Any number of SELECT commands can be issued for the same flag and are
%% accumulated (logically ORed). In this context note the following:  

%% \begin{verbatim}
%% SELECT, FLAG=name, FULL;
%% \end{verbatim} 
%% selects all positions in the sequence for this flag. This is the default
%% for all tables and makethin, whereas for ERROR and SEQEDIT the default
%% is "nothing selected".  

%% %\href{save_select}{}
%% \label{save_select}
%% SAVE: A SELECT,FLAG=SAVE statement causes the
%% selected sequences, elements, and variables to be written into the save
%% file. A class (only used for element selection), and a pattern can be
%% specified. Example:  
%% \begin{verbatim}
%% select, flag=save, class=variable, pattern="abc.*";
%% save, file=mysave;
%% \end{verbatim} 
%% will save all variables containing "abc" in their name,
%% but not elements with names containing "abc" since the class "variable"
%% does not exist (astucieux, non ?).  

%% SECTORMAP: A SELECT,FLAG=SECTORMAP statement causes sectormaps to be
%% written into the file "sectormap" like in \madeight. For the file to be
%% written, a flag SECTORMAP must be issued on the TWISS command in
%% addition.  

%% TWISS: A SELECT,FLAG=TWISS statement causes the selected rows and
%% columns to be written into the Twiss TFS file (former OPTICS command in
%% \madeight). The column selection is done on the same select. See as well
%% example 2.  

%% Example 1:  
%% \begin{verbatim}
%% TITLE,'Test input for MAD-X';

%% option,rbarc=false; // use arc length of rbends
%% beam; ! sets the default beam for the following sequence
%% option,-echo;
%% call file=fv9.opt;  ! contains optics parameters
%% call file="fv9.seq"; ! contains a small sequence "fivecell"
%% OPTION,ECHO;
%% SELECT,FLAG=SECTORMAP,clear;
%% SELECT,FLAG=SECTORMAP,PATTERN="^m.*";
%% SELECT,FLAG=TWISS,clear;
%% SELECT,FLAG=TWISS,PATTERN="^m.*",column=name,s,betx,bety;
%% USE,PERIOD=FIVECELL;
%% twiss,file=optics,sectormap;
%% stop;
%% \end{verbatim} 

%% This produces a file \href{sectormap.html}{sectormap}, and a
%% twiss output file \label{tfs} (name = optics):  
%% \begin{verbatim}
%% @ TYPE             %05s "TWISS"
%% @ PARTICLE         %08s "POSITRON"
%% @ MASS             %le          0.000510998902
%% @ CHARGE           %le                       1
%% @ E0               %le                       1
%% @ PC               %le           0.99999986944
%% @ GAMMA            %le           1956.95136738
%% @ KBUNCH           %le                       1
%% @ NPART            %le                       0
%% @ EX               %le                       1
%% @ EY               %le                       1
%% @ ET               %le                       0
%% @ LENGTH           %le                   534.6
%% @ ALFA             %le        0.00044339992938
%% @ ORBIT5           %le                      -0
%% @ GAMMATR          %le           47.4900022541
%% @ Q1               %le           1.25413071556
%% @ Q2               %le           1.25485338377
%% @ DQ1              %le           1.05329608302
%% @ DQ2              %le           1.04837000224
%% @ DXMAX            %le           2.17763211131
%% @ DYMAX            %le                       0
%% @ XCOMAX           %le                       0
%% @ YCOMAX           %le                       0
%% @ BETXMAX          %le            177.70993499
%% @ BETYMAX          %le           177.671582415
%% @ XCORMS           %le                       0
%% @ YCORMS           %le                       0
%% @ DXRMS            %le           1.66004270906
%% @ DYRMS            %le                       0
%% @ DELTAP           %le                       0
%% @ TITLE            %20s "Test input for MAD-X"
%% @ ORIGIN           %16s "MAD-X 0.20 Linux"
%% @ DATE             %08s "07/06/02"
%% @ TIME             %08s "14.25.51"
%% * NAME               S                  BETX               BETY               
%% $ %s                 %le                %le                %le                
%%  "MSCBH"             4.365              171.6688159        33.31817319       
%%  "MB"                19.72              108.1309095        58.58680717       
%%  "MB"                35.38              61.96499987        102.9962313       
%%  "MB"                51.04              34.61640793        166.2227523       
%%  "MSCBV.1"           57.825             33.34442808        171.6309057       
%%  "MB"                73.18              58.61984637        108.0956006       
%%  "MB"                88.84              103.0313887        61.93159422       
%%  "MB"                104.5              166.2602486        34.58939635       
%%  "MSCBH"             111.285            171.6688159        33.31817319       
%%  "MB"                126.64             108.1309095        58.58680717       
%%  "MB"                142.3              61.96499987        102.9962313       
%%  "MB"                157.96             34.61640793        166.2227523       
%%  "MSCBV"             164.745            33.34442808        171.6309057       
%%  "MB"                180.1              58.61984637        108.0956006       
%%  "MB"                195.76             103.0313887        61.93159422       
%%  "MB"                211.42             166.2602486        34.58939635       
%%  "MSCBH"             218.205            171.6688159        33.31817319       
%%  "MB"                233.56             108.1309095        58.58680717       
%%  "MB"                249.22             61.96499987        102.9962313       
%%  "MB"                264.88             34.61640793        166.2227523       
%%  "MSCBV"             271.665            33.34442808        171.6309057       
%%  "MB"                287.02             58.61984637        108.0956006       
%%  "MB"                302.68             103.0313887        61.93159422       
%%  "MB"                318.34             166.2602486        34.58939635       
%%  "MSCBH"             325.125            171.6688159        33.31817319       
%%  "MB"                340.48             108.1309095        58.58680717       
%%  "MB"                356.14             61.96499987        102.9962313       
%%  "MB"                371.8              34.61640793        166.2227523       
%%  "MSCBV"             378.585            33.34442808        171.6309057       
%%  "MB"                393.94             58.61984637        108.0956006       
%%  "MB"                409.6              103.0313887        61.93159422       
%%  "MB"                425.26             166.2602486        34.58939635       
%%  "MSCBH"             432.045            171.6688159        33.31817319       
%%  "MB"                447.4              108.1309095        58.58680717       
%%  "MB"                463.06             61.96499987        102.9962313       
%%  "MB"                478.72             34.61640793        166.2227523       
%%  "MSCBV"             485.505            33.34442808        171.6309057       
%%  "MB"                500.86             58.61984637        108.0956006       
%%  "MB"                516.52             103.0313887        61.93159422       
%%  "MB"                532.18             166.2602486        34.58939635       
%% \end{verbatim}

%%  Example 2: 

%%  Addition of variables to (any internal) table: 
%% \begin{verbatim}
%% select, flag=table, column=name, s, betx, ..., var1, var2, ...; ! or
%% select, flag=table, full, column=var1, var2, ...; ! default col.s + new
%% \end{verbatim} 
%% will write the current value of var1 etc. into the table each time a new
%% line is added; values from the same (current) line can be accessed by
%% these variables, e.g.  
%% \begin{verbatim}
%% var1 := sqrt(beam->ex*table(twiss,betx));
%% \end{verbatim} 
%% in the case of table above being "twiss". The plot command accepts the
%% new variables.  

%% Remark: this replaces the "string" variables of \madeight. 

%%  This example demonstrates as well the usage of a user defined table \label{ucreate}. 
%% \begin{verbatim}
%% beam,ex=1.e-6,ey=1.e-3;
%% // element definitions
%% mb:rbend, l=14.2, angle:=0,k0:=bang/14.2;
%% mq:quadrupole, l:=3.1,apertype=ellipse,aperture={1,2};
%% qft:mq, l:=0.31, k1:=kqf,tilt=-pi/4;
%% qf.1:mq, l:=3.1, k1:=kqf;
%% qf.2:mq, l:=3.1, k1:=kqf;
%% qf.3:mq, l:=3.1, k1:=kqf;
%% qf.4:mq, l:=3.1, k1:=kqf;
%% qf.5:mq, l:=3.1, k1:=kqf;
%% qd.1:mq, l:=3.1, k1:=kqd;
%% qd.2:mq, l:=3.1, k1:=kqd;
%% qd.3:mq, l:=3.1, k1:=kqd;
%% qd.4:mq, l:=3.1, k1:=kqd;
%% qd.5:mq, l:=3.1, k1:=kqd;
%% bph:hmonitor, l:=l.bpm;
%% bpv:vmonitor, l:=l.bpm;
%% cbh:hkicker;
%% cbv:vkicker;
%% cbh.1:cbh, kick:=acbh1;
%% cbh.2:cbh, kick:=acbh2;
%% cbh.3:cbh, kick:=acbh3;
%% cbh.4:cbh, kick:=acbh4;
%% cbh.5:cbh, kick:=acbh5;
%% cbv.1:cbv, kick:=acbv1;
%% cbv.2:cbv, kick:=acbv2;
%% cbv.3:cbv, kick:=acbv3;
%% cbv.4:cbv, kick:=acbv4;
%% cbv.5:cbv, kick:=acbv5;
%% !mscbh:sextupole, l:=1.1, k2:=ksf;
%% mscbh:multipole, knl:={0,0,0,ksf},tilt=-pi/8;
%% mscbv:sextupole, l:=1.1, k2:=ksd;
%% !mscbv:octupole, l:=1.1, k3:=ksd,tilt=-pi/8;

%% // sequence declaration

%% fivecell:sequence, refer=centre, l=534.6;
%%    qf.1:qf.1, at=1.550000e+00;
%%    qft:qft, at=3.815000e+00;
%% !   mscbh:mscbh, at=3.815000e+00;
%%    cbh.1:cbh.1, at=4.365000e+00;
%%    mb:mb, at=1.262000e+01;
%%    mb:mb, at=2.828000e+01;
%%    mb:mb, at=4.394000e+01;
%%    bpv:bpv, at=5.246000e+01;
%%    qd.1:qd.1, at=5.501000e+01;
%%    mscbv:mscbv, at=5.727500e+01;
%%    cbv.1:cbv.1, at=5.782500e+01;
%%    mb:mb, at=6.608000e+01;
%%    mb:mb, at=8.174000e+01;
%%    mb:mb, at=9.740000e+01;
%%    bph:bph, at=1.059200e+02;
%%    qf.2:qf.2, at=1.084700e+02;
%%    mscbh:mscbh, at=1.107350e+02;
%%    cbh.2:cbh.2, at=1.112850e+02;
%%    mb:mb, at=1.195400e+02;
%%    mb:mb, at=1.352000e+02;
%%    mb:mb, at=1.508600e+02;
%%    bpv:bpv, at=1.593800e+02;
%%    qd.2:qd.2, at=1.619300e+02;
%%    mscbv:mscbv, at=1.641950e+02;
%%    cbv.2:cbv.2, at=1.647450e+02;
%%    mb:mb, at=1.730000e+02;
%%    mb:mb, at=1.886600e+02;
%%    mb:mb, at=2.043200e+02;
%%    bph:bph, at=2.128400e+02;
%%    qf.3:qf.3, at=2.153900e+02;
%%    mscbh:mscbh, at=2.176550e+02;
%%    cbh.3:cbh.3, at=2.182050e+02;
%%    mb:mb, at=2.264600e+02;
%%    mb:mb, at=2.421200e+02;
%%    mb:mb, at=2.577800e+02;
%%    bpv:bpv, at=2.663000e+02;
%%    qd.3:qd.3, at=2.688500e+02;
%%    mscbv:mscbv, at=2.711150e+02;
%%    cbv.3:cbv.3, at=2.716650e+02;
%%    mb:mb, at=2.799200e+02;
%%    mb:mb, at=2.955800e+02;
%%    mb:mb, at=3.112400e+02;
%%    bph:bph, at=3.197600e+02;
%%    qf.4:qf.4, at=3.223100e+02;
%%    mscbh:mscbh, at=3.245750e+02;
%%    cbh.4:cbh.4, at=3.251250e+02;
%%    mb:mb, at=3.333800e+02;
%%    mb:mb, at=3.490400e+02;
%%    mb:mb, at=3.647000e+02;
%%    bpv:bpv, at=3.732200e+02;
%%    qd.4:qd.4, at=3.757700e+02;
%%    mscbv:mscbv, at=3.780350e+02;
%%    cbv.4:cbv.4, at=3.785850e+02;
%%    mb:mb, at=3.868400e+02;
%%    mb:mb, at=4.025000e+02;
%%    mb:mb, at=4.181600e+02;
%%    bph:bph, at=4.266800e+02;
%%    qf.5:qf.5, at=4.292300e+02;
%%    mscbh:mscbh, at=4.314950e+02;
%%    cbh.5:cbh.5, at=4.320450e+02;
%%    mb:mb, at=4.403000e+02;
%%    mb:mb, at=4.559600e+02;
%%    mb:mb, at=4.716200e+02;
%%    bpv:bpv, at=4.801400e+02;
%%    qd.5:qd.5, at=4.826900e+02;
%%    mscbv:mscbv, at=4.849550e+02;
%%    cbv.5:cbv.5, at=4.855050e+02;
%%    mb:mb, at=4.937600e+02;
%%    mb:mb, at=5.094200e+02;
%%    mb:mb, at=5.250800e+02;
%%    bph:bph, at=5.336000e+02;
%% end:marker, at=5.346000e+02;
%% endsequence;

%% // forces and other constants

%% l.bpm:=.3;
%% bang:=.509998807401e-2;
%% kqf:=.872651312e-2;
%% kqd:=-.872777242e-2;
%% ksf:=.0198492943;
%% ksd:=-.039621283;
%% acbv1:=1.e-4;
%% acbh1:=1.e-4;
%% !save,sequence=fivecell,file,mad8;

%% s := table(twiss,bpv[5],betx);
%% myvar := sqrt(beam->ex*table(twiss,betx));
%% use, period=fivecell;
%% select,flag=twiss,column=name,s,myvar,apertype;
%% twiss,file;
%% n = 0;
%% create,table=mytab,column=dp,mq1,mq2;
%% mq1:=table(summ,q1);
%% mq2:=table(summ,q2);
%% while ( n < 11)
%% {
%%   n = n + 1;
%%   dp = 1.e-4*(n-6);
%%   twiss,deltap=dp;
%%   fill,table=mytab;
%% }
%% write,table=mytab;
%% plot,haxis=s,vaxis=aper_1,aper_2,colour=100,range=#s/cbv.1,notitle;
%% stop;
%% \end{verbatim}
%% prints the following user table on output:

%% \begin{verbatim}
%% @ NAME             %05s "MYTAB"
%% @ TYPE             %04s "USER"
%% @ TITLE            %08s "no-title"
%% @ ORIGIN           %16s "MAD-X 1.09 Linux"
%% @ DATE             %08s "10/12/02"
%% @ TIME             %08s "10.45.25"
%% * DP                 MQ1                MQ2                
%% $ %le                %le                %le                
%%  -0.0005            1.242535951        1.270211135       
%%  -0.0004            1.242495534        1.270197018       
%%  -0.0003            1.242452432        1.270185673       
%%  -0.0002            1.242406653        1.270177093       
%%  -0.0001            1.242358206        1.270171269       
%%  0                  1.242307102        1.27016819        
%%  0.0001             1.242253353        1.270167843       
%%  0.0002             1.242196974        1.270170214       
%%  0.0003             1.24213798         1.270175288       
%%  0.0004             1.242076387        1.270183048       
%%  0.0005             1.242012214        1.270193477       
%% \end{verbatim}
%% and produces a twiss file with the additional column myvar, as well as a plot
%% file with the aperture values plotted.


%% \href{screate}{}

%% Example of joining two tables with different length into a third table
%% making use of the length of either table as given by
%% table("your\_table\_name", tablelength) and adding names by the "\_name"
%% attribute.

%% \begin{verbatim}
%% title,   "summing of offset and alignment tables";
%% set,    format="13.6f";

%% readtable, table=align,  file="align.ip2.b1.tfs";   // mesured alignment
%% readtable, table=offset, file="offset.ip2.b1.tfs";  // nominal offsets

%% n_elem  =  table(offset, tablelength);

%% create,  table=align_offset, column=_name,s_ip,x_off,dx_off,ddx_off,y_off,dy_off,ddy_off;

%% calcul(elem_name) : macro = {
%%     x_off = table(align,  elem_name, x_ali) + x_off;
%%     y_off = table(align,  elem_name, y_ali) + y_off;
%% }


%% one_elem(j_elem) : macro = {
%%     setvars, table=offset, row=j_elem;
%%     exec,  calcul(tabstring(offset, name, j_elem));
%%     fill,  table=align_offset;
%% }


%% i_elem = 0;
%% while (i_elem < n_elem) { i_elem = i_elem + 1; exec,  one_elem($i_elem); }

%% write, table=align_offset, file="align_offset.tfs";

%% stop;
%% \end{verbatim}

%% EOF


