%%\title{Conventions}
%  Changed by: Chris ISELIN, 27-Mar-1997 
%  Changed by: Hans Grote, 25-Sep-2002 
%  Changed by: Ghislain ROY, 30-Jul-2015

\chapter{Conventions}
\label{chap:conventions}
\index{conventions}


\section{Reference System}
\label{sec:reference}
\index{reference!system}
\index{reference!orbit}
\index{local coordinates}
\index{coordinates!local}
The accelerator and/or beam line to be studied is described as a
sequence of beam elements placed sequentially along a reference
orbit. 
The reference orbit is the path of a charged particle having the
central design momentum of the accelerator through idealised magnets
with no fringe fields (see Figure~\ref{F-REF}).

\begin{figure}[htb]
\centering
\setlength{\unitlength}{1pt}
\begin{picture}(400,250)
% axes
\thicklines
\put(180,60){\vector(0,1){130}}
\put(188,182){\makebox(0,0){\(y\)}}
\put(280,0){\vector(-1,1){160}}
\put(120,150){\makebox(0,0){\(x\)}}
\put(140,80){\vector(2,1){100}}
\put(225,130){\makebox(0,0){\(z\)}}
\thinlines
\put(180,100){\circle*{4}}
% coordinates
\put(150,130){\line(0,1){60}}
\put(180,160){\line(-1,1){30}}
\put(150,130){\circle*{4}}
\put(150,190){\circle*{4}}
\put(180,160){\circle*{4}}
% radii of curvature and centre
\put(280,0){\vector(-3,1){172}}
\put(160,30){\makebox(0,0){\(\rho\)}}
\put(280,0){\vector(0,1){142}}
\put(290,60){\makebox(0,0){\(\rho\)}}
\put(280,0){\circle*{4}}
\put(295,15){\vector(-1,-1){12}}
\put(295,15){\makebox(0,0)[bl]{\shortstack{centre of \\curvature}}}
% actual orbit
\thicklines
\color{red}
\bezier{150}(0,100)(75,165)(150,190)
\bezier{150}(150,190)(240,220)(320,220)
\put(320,220){\vector(1,0){4}}
\color{black}
\put(80,180){\vector(1,-1){12}}
\put(80,180){\makebox(0,0)[br]{\shortstack{actual \\orbit}}}
\put(150,190){\vector(3,1){18}}
\put(159,200){\makebox(0,0)[br]{\(d\vec r\)}}
% actual orbit
\color{blue}
\bezier{150}(40,0)(100,60)(180,100)
\bezier{150}(180,100)(260,140)(340,160)
\put(340,160){\vector(4,1){4}}
\color{black}
\put(328,162){\makebox{\(s\)}}
\put(320,135){\vector(-1,1){12}}
\put(320,135){\makebox(0,0)[tl]{\shortstack{reference \\orbit}}}
\end{picture}
\caption[Local Reference System]{Local Reference System}
\label{F-REF}
\end{figure}


The reference orbit consists of a series of straight line segments and
circular arcs. It is defined under the assumption that all elements are
perfectly aligned. The accompanying tripod of the reference orbit spans
a local curvilinear right handed coordinate system \textit{(x,y,s)} The
local \textit{s}-axis is the tangent to the reference orbit. The two
other axes are perpendicular to the reference orbit and are labelled
\textit{x} (in the bend plane) and \textit{y} (perpendicular to the bend
plane).  


\section{Closed Orbit}
\label{sec:closed-orbit}

Due to various errors like misalignment errors, field errors, fringe
fields etc., the closed orbit does not coincide with the reference
orbit. The closed orbit also changes with the momentum error. 
The closed orbit is described with respect to the reference orbit, using
the local reference system (\textit{x, y, s}). It is evaluated including
any nonlinear effects.  

\madx also computes the betatron and synchrotron oscillations with respect
to the closed orbit. Results are given in the local (\textit{x, y,
  s})-system defined by the reference orbit. 


\section{Global Reference System}
\label{sec:global-ref}

The global reference orbit of the accelerator is
uniquely defined by the sequence of physical elements. The local
reference system (\textit{x, y, s}) may thus be
referred to a global Cartesian coordinate system (\textit{X, Y, Z}) 
(see Figure~\ref{F-GLOB}). 

The positions between beam elements are indexed with $i=0,\ldots n$. 
The local reference system  ({$x_i$, $y_i$, $s_i$}) at position
\textit{i}, i.e. the displacement and direction of the reference orbit
with respect to the system (\textit{X, Y, Z}) are
defined by three displacements  ($X_i$, $Y_i$, $Z_i$) and three angles
($\theta_i$, $\phi_i$, $\psi_i$) 

%% \begin{figure}[htb]% 1.2
%% \centering
%% \setlength{\unitlength}{1pt}
%% \begin{picture}(400,270)
%% % global axes
%% \thicklines
%% \put(20,150){\vector(2,-1){280}}
%% \put(290,35){\makebox(0,0){\(Z\)}}
%% \put(20,100){\vector(3,1){360}}
%% \put(370,200){\makebox(0,0){\(X\)}}
%% \put(80,0){\vector(0,1){270}}
%% \put(70,240){\makebox(0,0){\(Y\)}}
%% %local axes
%% \put(133.3,0){\vector(1,3){90}}
%% \put(213.3,260){\makebox(0,0){\(x\)}}
%% \put(300,150){\vector(-2,1){180}}
%% \put(150,215){\makebox(0,0){\(y\)}}
%% \put(0,100){\vector(2,1){270}}
%% \put(260,240){\makebox(0,0){\(s\)}}
%% % projection of s onto ZX
%% \thinlines
%% \put(80,120){\circle*{4}}
%% \put(200,200){\circle*{4}}
%% \put(0,110){\line(1,0){290}}
%% \put(300,110){\makebox(0,0)[l]{\shortstack{projection of \(s\) \\
%% onto \(ZX\)-plane}}}
%% \put(20,110){\circle*{4}}
%% \put(50,110){\circle*{4}}
%% \put(100,110){\circle*{4}}
%% % displacement of local system
%% \put(140,140){\line(2,-1){60}}
%% \put(160,128){\makebox(0,0)[tr]{\(Z\)}}
%% \put(140,90){\line(3,1){60}}
%% \put(190,100){\makebox(0,0)[tl]{\(X\)}}
%% \put(200,110){\line(0,1){90}}
%% \put(205,140){\makebox(0,0)[l]{\(Y\)}}
%% \put(140,90){\circle*{4}}
%% \put(140,140){\circle*{4}}
%% \put(200,110){\circle*{4}}
%% % intersection of xy and ZX
%% \put(130,0){\line(1,1){230}}
%% \put(135,5){\circle*{4}}
%% \put(193.3,63.3){\circle*{4}}
%% \put(240,110){\circle*{4}}
%% \put(286.7,156.7){\circle*{4}}
%% \put(335,205){\circle*{4}}
%% \thicklines
%% \put(180,20){\vector(-1,1){12}}
%% \put(180,20){\makebox(0,0)[tl]{\shortstack{intersection of \\
%% \(xy\) and \(ZX\) planes}}}
%% % reference orbit
%% \bezier{80}(140,150)(170,185)(200,200)
%% \bezier{80}(200,200)(230,215)(260,220)
%% \put(260,220){\makebox(0,0)[l]{\shortstack{reference \\orbit}}}
%% % roll angle
%% \bezier{30}(160,30)(160,40)(150,50)
%% \put(152,48){\vector(-1,1){2}}
%% \put(150,30){\makebox(0,0){\(\psi\)}}
%% \put(140,30){\makebox(0,0)[br]{roll angle}}
%% % pitch angle
%% \bezier{20}(60,110)(60,120)(55,125)
%% \put(57,123){\vector(-1,2){2}}
%% \put(50,118){\makebox(0,0){\(\phi\)}}
%% \put(40,125){\makebox(0,0)[br]{pitch angle}}
%% % azimuth
%% \bezier{20}(130,95)(140,100)(140,110)
%% \put(140,105){\vector(0,1){5}}
%% \put(130,105){\makebox(0,0){\(\theta\)}}
%% \put(115,95){\makebox(0,0)[t]{azimuth}}
%% \end{picture}
%% \caption{Global Reference System}
%% \label{F-GLOB}
%% \end{figure}

%% alternate take for more clarity
\begin{figure}[htb]% 1.2
\centering
\setlength{\unitlength}{1pt}
\begin{picture}(400,270)

% global axes
\color{black}
\thicklines
\put(20,150){\vector(2,-1){280}}
\put(300,20){\makebox(0,0){\(Z\)}}
\put(20,100){\vector(3,1){360}}
\put(380,210){\makebox(0,0){\(X\)}}
\put(80,0){\vector(0,1){270}}
\put(70,265){\makebox(0,0){\(Y\)}}

%local axes
\color{red}
\thinlines
\put(133.3,0){\vector(1,3){90}}
\put(213.3,265){\makebox(0,0){\(x\)}}
\put(300,150){\vector(-2,1){180}}
\put(125,230){\makebox(0,0){\(y\)}}
\put(0,100){\vector(2,1){270}}
\put(260,240){\makebox(0,0){\(s\)}}

% displacement of local system
\color{black}
\put(140,140){\line(2,-1){60}}
\put(160,128){\makebox(0,0)[tr]{\(Z_i\)}}
\put(140,90){\line(3,1){60}}
\put(180,100){\makebox(0,0)[tl]{\(X_i\)}}
\put(200,110){\line(0,1){90}}
\put(205,140){\makebox(0,0)[l]{\(Y_i\)}}

% roll angle
\bezier{30}(160,30)(160,40)(150,50)
\put(151,49){\vector(-1,1){1}}
\put(150,30){\makebox(0,0){\(\psi_i\)}}

% pitch angle
\bezier{20}(60,110)(60,120)(55,125)
\put(57,123){\vector(-1,2){2}}
\put(50,118){\makebox(0,0){\(\phi_i\)}}

% azimuth angle
\bezier{20}(130,95)(140,100)(140,110)
\put(140,105){\vector(0,1){5}}
\put(130,103){\makebox(0,0){\(\theta_i\)}}

% reference orbit
\color{black}
\thicklines
\bezier{80}(140,155)(170,185)(200,200)
\bezier{80}(200,200)(230,215)(260,210)
\put(260,213){\makebox(0,0)[l]{\shortstack{reference \\orbit}}}

% intersections of planes
\color{green}
\thinlines
\multiput(130,0)(2,2){115}{\line(0,1){1}}
\thicklines
\put(180,20){\vector(-1,1){12}}
\put(172,20){\makebox(0,0)[tl]{\shortstack{intersection of \\
			\(xy\) and \(ZX\) planes}}}

\thinlines
\multiput(0,128)(2,-1.8){73}{\line(0,-1){1}}
\thicklines
\put(60,50){\vector(1,1){12}}
\put(0,50){\makebox(0,0)[tl]{\shortstack{intersection of \\
			\(xs\) and \(ZX\) \\ planes}}}

\thinlines
\multiput(0,107)(3,0.522){120}{\line(0,1){1}}
\thicklines
\put(318,149){\vector(-1,1){12}}
\put(320,150){\makebox(0,0)[tl]{\shortstack{intersection of \\
			\(ys\) and \(ZX\) \\ planes}}}

% projection of axes onto ZX (horizontal plane)
\color{blue}
\thinlines
\multiput(0,110)(2,0){180}{\line(1,0){1}}
\put(300,95){\makebox(0,0)[l]{\shortstack{projection of \(s\) \\
onto \(ZX\)-plane}}}

\put(50,110){\circle*{4}}
\put(100,110){\circle*{4}}

\multiput(131,0)(1,1.6){123}{\line(1,0){1}}
\put(177,71){\circle*{4}}
\put(240,173){\circle*{4}}

\multiput(147,80)(1.79,1){100}{\line(1,0){1}}
\put(154,83){\circle*{4}}

% dots for intersections of planes
\color{green}
\put(20,110){\circle*{4}}
\put(28,102){\circle*{4}}
\put(135,5){\circle*{4}}
\put(193.3,63.3){\circle*{4}}
\put(240,110){\circle*{4}}
\put(286.7,156.7){\circle*{4}}
\put(335,205){\circle*{4}}

% dots for displacements
\color{black}
\put(140,90){\circle*{4}}
\put(140,140){\circle*{4}}
\put(200,110){\circle*{4}}

\end{picture}
\caption[Global Reference System]{Global Reference System showing the global 
Cartesian system ($X, Y, Z$) in black and the local reference system ($x, y, 
s$) in red after translation ($X_i , Y_i, Z_i$) and rotation ($\theta_i, 
\phi_i, \psi_i$). The projections of the local reference system axes onto the 
horizontal $ZX$ plane of the Cartesian system are figured with blue dashed 
lines. The intersections of planes $ys$, $xy$ and $xs$ of the local reference 
system with the horizontal $ZX$ plane of the Cartesian system are figured in 
green dashed lines. }
\label{F-GLOB}
\end{figure}



The above quantities are defined more precisely as follows:  
\begin{madlist}
   \ttitem{X} Displacement of the local origin in \textit{X}-direction. 
   \ttitem{Y} Displacement of the local origin in \textit{Y}-direction. 
   \ttitem{Z} Displacement of the local origin in \textit{Z}-direction. 
   \ttitem{THETA} $\theta$ is the angle of rotation (azimuth) about the
     global \textit{Y}-axis, between the global \textit{Z}-axis and the
     projection of the reference orbit onto the (\textit{Z},
     \textit{X})-plane. A positive angle \texttt{THETA} forms a
     right-hand screw with the \textit{Y}-axis. 
   \ttitem{PHI} $\phi$ is the elevation angle, i.e. the angle between the
     reference orbit and its projection onto the (\textit{Z},
     \textit{X})-plane. A positive angle \texttt{PHI} corresponds to
     increasing \textit{Y}. \\ 
     If only horizontal bends are present, the reference
     orbit remains in the (\textit{Z}, \textit{X})-plane and
     \texttt{PHI} is always zero. 
   \ttitem{PSI} $\psi$ is the roll angle about the local \textit{s}-axis,
     i.e. the angle between the line defined by the intersection of the 
     (\textit{x, y})-plane and (\textit{Z, X})-plane on one hand, and
     the local \textit{x}-axis on the other hand. 
     A positive angle \texttt{PSI} forms a right-hand screw with the
     \textit{s}-axis. 
\end{madlist} 

The angles ($\theta$, $\phi$, $\psi$) are \textbf{not} the Euler
angles. The reference orbit starts at the origin and points by default
in the direction of the positive \textit{Z}-axis. The initial local axes
(\textit{x}, \textit{y}, \textit{s})  coincide with the global axes
(\textit{X}, \textit{Y}, \textit{Z}) in this order. The initial values
($X_0$, $Y_0$, $Z_0$, $\theta_0$, $\phi_0$, $\psi_0$) are therefore all zero
unless the user specifies different initial conditions.  

Internally the displacement is described by a vector \textit{V} and the
orientation by a unitary matrix \textit{W}. The column vectors of
\textit{W} are the unit vectors spanning  the local coordinate axes in
the order (\textit{x, y, s}). \textit{V} and \textit{W} have the values:  

\begin{equation}
V =
 \begin{pmatrix}
  X \\
  Y \\
  Z
 \end{pmatrix}
, \qquad
W=\Theta \quad \Phi \quad \Psi
\end{equation}
 where 
\begin{equation}
\Theta =
 \begin{pmatrix}
  \cos \theta  & 0 &  \sin \theta \\
  0            & 1 &  0 \\
  -\sin \theta & 0 &  \cos \theta
 \end{pmatrix}
, \quad
\Phi =
 \begin{pmatrix}
  1 & 0          &  0 \\
  0 & \cos \phi  &  \sin \phi \\
  0 & -\sin \phi &  \cos \phi
 \end{pmatrix}
, \quad
\Psi =
 \begin{pmatrix}
  \cos \psi &  -\sin \psi & 0 \\
  \sin \psi &  \cos \psi  & 0 \\
  0	    &	0	  & 1 
 \end{pmatrix}
\end{equation}

The reference orbit should be closed, and it should not be twisted. 
This means that the displacement of the local reference system must be
periodic with the revolution frequency of the accelerator, while the
position angles must be periodic (modulo $2\pi$) with the revolution
frequency. If $\psi$ is not periodic (modulo $2\pi$), coupling effects are
introduced. 
When advancing through a beam element, \madx computes
$V_i$ and $W_i$ by the recurrence relations  
\begin{equation}
   V_i=W_{i-1}R_i+V_{i-1},
   \qquad
   W_i=W_{i-1}S_i
\end{equation}
The vector $R_i$ is the displacement and the matrix $S_i$ is the
rotation of the local reference system  at the exit of the element
\textit{i} with respect to the entrance of the same element. The values
of $R_i$ and $S_i$ are listed below for different physical element types.  


\section{Local Reference Systems}
\label{sec:local-ref}

\subsection{Reference System for Straight Beam Elements} 
\label{subsec:local-straight}
In straight elements the local reference system is simply translated by
the length of the element along the local \textit{s}-axis. 
This is true for  
\hyperref[sec:drift]{Drift spaces},\ttindex{drift} 
\hyperref[sec:quadrupole]{Quadrupoles},\ttindex{quadrupole} 
\hyperref[sec:sextupole]{Sextupoles},\ttindex{sextupole} 
\hyperref[sec:octupole]{Octupoles},\ttindex{octupole} 
\hyperref[sec:solenoid]{Solenoids},\ttindex{solenoid} 
\hyperref[sec:crab-cavity]{Crab cavities},\ttindex{crab cavity}
\hyperref[sec:rf-cavity]{RF cavities},\ttindex{cavity}\ttindex{RF cavity}
\hyperref[sec:separator]{Electrostatic separators},\ttindex{separator}\ttindex{electrostatic separator}
\hyperref[sec:kicker]{Closed orbit correctors}\ttindex{corrector} and
\hyperref[sec:monitor]{Beam position monitors}\ttindex{monitor}.


The corresponding \textit{R}, \textit{S} are 
\begin{equation}
R =
 \begin{pmatrix}
  0 \\
  0 \\
  L
 \end{pmatrix}
, \quad
S =
 \begin{pmatrix}
  1 & 0 &  0 \\
  0 & 1 &  0 \\
  0 & 0 &  1
 \end{pmatrix}
.
\end{equation}
A rotation of the element about the \textit{S}-axis has no effect on
\textit{R} and \textit{S}, since the rotations of the reference system
before and after the element cancel.  

%% % picture prepared by ghislain to reproduce the png graphics in html.
%% \begin{figure}[htb]
%%   \centering
%%   \setlength{\unitlength}{1pt}
%%   \begin{picture}(200,200)
%%     \thinlines
%%     % s-axis
%%     \put(0,100){\line(1,0){20}}
%%     \put(30,100){\line(1,0){140}}
%%     \put(180,100){\vector(1,0){20}}
%%     \put(200,90){\makebox(0,0){$s$}}
%%     % y-axes
%%     \put(25,100){\circle{10}}
%%     \put(25,100){\circle*{2}}
%%     \put(18,110){\makebox(0,0){$y_1$}}
%%     \put(175,100){\circle{10}}
%%     \put(175,100){\circle*{2}}
%%     \put(182,110){\makebox(0,0){$y_2$}}
%%     % x-axes
%%     \put(25,15){\line(0,1){80}}
%%     \put(25,105){\vector(0,1){80}}
%%     \put(16,180){\makebox(0,0){$x_1$}}
%%     \put(175,15){\line(0,1){80}}
%%     \put(175,105){\vector(0,1){80}}
%%     \put(184,180){\makebox(0,0){$x_2$}}
%%     % length
%%     \put(25,30){\vector(1,0){150}}
%%     \put(175,30){\vector(-1,0){150}}
%%     \put(100,25){\makebox(0,0){$L$}}
%%     % element box
%%     \thicklines
%%     \put(25,50){\line(0,1){45}}
%%     \put(25,105){\line(0,1){45}}
%%     \put(25,150){\line(1,0){150}}
%%     \put(175,150){\line(0,-1){45}}
%%     \put(175,95){\line(0,-1){45}}
%%     \put(175,50){\line(-1,0){150}}
%%   \end{picture}
%%   \caption{Reference System for Straight Beam Elements}
%%   \label{F-REF2}
%% \end{figure}

%% Original picture from MAD-8 manual 
\begin{figure}[ht]
\centering
\setlength{\unitlength}{1pt}
\begin{picture}(400,100)
\thinlines
% axes
\put(150,50){\circle{8}}\put(150,50){\circle*{2}}
\put(140,40){\makebox(0,0){\(y_1\)}}
\put(250,50){\circle{8}}\put(250,50){\circle*{2}}
\put(260,40){\makebox(0,0){\(y_2\)}}
\put(100,50){\line(1,0){46}}
\put(154,50){\line(1,0){92}}
\put(254,50){\vector(1,0){46}}
\put(290,40){\makebox(0,0){\(s\)}}
\put(150,0){\line(0,1){46}}
\put(150,54){\vector(0,1){46}}
\put(140,90){\makebox(0,0){\(x_1\)}}
\put(250,0){\line(0,1){46}}
\put(250,54){\vector(0,1){46}}
\put(260,90){\makebox(0,0){\(x_2\)}}
% magnet outline
\thicklines
\color{blue}
\put(150,54){\line(0,1){26}}
\put(150,46){\line(0,-1){26}}
\put(250,54){\line(0,1){26}}
\put(250,46){\line(0,-1){26}}
\put(150,20){\line(1,0){100}}
\put(150,80){\line(1,0){100}}
\color{black}
\put(200,2){\vector(1,0){50}}
\put(200,2){\vector(-1,0){50}}
\put(200,10){\makebox(0,0){L}}
\end{picture}
\caption{Reference System for Straight Beam Elements}
\label{F-DRF}
\end{figure}
 

\subsection{Reference System for Bending Magnets}
\label{subsec:local-rbend}
\hyperref[sec:bend]{Bending magnets} have a curved reference orbit. 
For both rectangular and sector bending magnets, the \textit{R} and
\textit{S} are expressed as function the bend angle $\alpha$: 
\begin{equation}
R =
 \begin{pmatrix}
  \rho\,(\cos \alpha - 1) \\
  0 \\
  \rho\,\sin \alpha
 \end{pmatrix}
, \quad
S =
 \begin{pmatrix}
  \cos \alpha & 0 &  -\sin \alpha \\
  0 & 1 &  0 \\
  \sin \alpha & 0 &  \cos \alpha
 \end{pmatrix}
\end{equation}
 
A positive bend angle represents a bend to the right, i.e. towards
negative \textit{x} values. 
For sector bending magnets, the bend radius is given by $\rho$, and for
rectangular bending magnets it has the value $\rho = L / (2 \sin(\alpha/2))$. 
If the magnet is rotated about the \textit{s}-axis by an angle $\psi$,
\textit{R} and \textit{S} are transformed by  
\begin{equation}
   \overline{R}=TR,
   \qquad
   \overline{S}=TST^{-1}
\end{equation}
where \textit{T} is the orthogonal rotation matrix 
\begin{equation}
T =
 \begin{pmatrix}
  \cos \psi &  -\sin \psi & 0 \\
  \sin \psi &  \cos \psi  & 0 \\
  0	    &	0	  & 1 
 \end{pmatrix}
\end{equation}
The special value $\psi = \pi/2$ represents a bend down.  

\begin{figure}[htb]
\centering
\setlength{\unitlength}{1pt}
\begin{picture}(400,215)
% axes
\thinlines
\put(150,150){\circle{8}}\put(150,150){\circle*{2}}
\put(160,140){\makebox(0,0){\(y_1\)}}
\put(250,150){\circle{8}}\put(250,150){\circle*{2}}
\put(240,140){\makebox(0,0){\(y_2\)}}
\put(74,124.7){\vector(3,1){72}}
\put(84,135){\makebox(0,0){\(s_1\)}}
\put(254,148.7){\vector(3,-1){72}}
\put(316,135){\makebox(0,0){\(s_2\)}}
\put(200,0){\vector(-1,3){48.7}}
\put(165,75){\makebox(0,0){\(\rho\)}}
\put(148.7,154){\vector(-1,3){18}}
\put(118,206){\makebox(0,0){\(x_1\)}}
\put(200,0){\vector(1,3){48.7}}
\put(235,75){\makebox(0,0){\(\rho\)}}
\put(251.3,154){\vector(1,3){18}}
\put(282,206){\makebox(0,0){\(x_2\)}}
\bezier{20}(190.5,28.5)(200,31.7)(209.5,28.5)
\put(200,20){\makebox(0,0){\(\alpha\)}}
\put(154,150){\line(1,0){92}}
\put(200,150){\circle*{4}}
\put(200,150){\vector(0,1){60}}
\put(210,200){\makebox(0,0){\(x\)}}
\put(150,154){\line(0,1){44}}
\put(150,146){\line(0,-1){46}}
\put(151,154){\line(1,4){11}}
\put(250,154){\line(0,1){44}}
\put(250,146){\line(0,-1){46}}
\put(249,154){\line(-1,4){11}}
% magnet outline
\thicklines
\put(200,102){\vector(-1,0){50}}
\put(200,102){\vector(1,0){50}}
\put(200,110){\makebox(0,0){L}}
\color{blue}
\put(151,154){\line(1,4){6}}
\put(149,146){\line(-1,-4){6}}
\put(249,154){\line(-1,4){6}}
\put(251,146){\line(1,-4){6}}
\put(157,178){\line(1,0){86}}
\put(143,122){\line(1,0){114}}
\color{black}
\bezier{10}(150,195)(155.5,195)(160.9,193.7)
\put(155.5,195){\vector(3,-1){5.4}}
\put(150,205){\makebox(0,0)[l]{\(e_1\)}}
\bezier{10}(250,195)(244.5,195)(239.1,193.7)
\put(244.5,195){\vector(-3,-1){5.4}}
\put(250,205){\makebox(0,0)[r]{\(e_2\)}}
\end{picture}
\caption[Reference System for a Rectangular Bending Magnet]%
{Reference System for a Rectangular Bending Magnet;
the signs of pole-face rotations are positive as shown.}
\label{F-RBND}
\end{figure}
 
\begin{figure}[htb]
\centering
\setlength{\unitlength}{1pt}
\begin{picture}(400,215)
% axes
\thinlines
\put(150,150){\circle{8}}\put(150,150){\circle*{2}}
\put(160,140){\makebox(0,0){\(y_1\)}}
\put(250,150){\circle{8}}\put(250,150){\circle*{2}}
\put(240,140){\makebox(0,0){\(y_2\)}}
\put(74,124.7){\vector(3,1){72}}
\put(84,135){\makebox(0,0){\(s_1\)}}
\put(254,148.7){\vector(3,-1){72}}
\put(316,135){\makebox(0,0){\(s_2\)}}
\put(200,0){\vector(-1,3){48.7}}
\put(165,75){\makebox(0,0){\(\rho\)}}
\put(148.7,154){\vector(-1,3){18}}
\put(118,206){\makebox(0,0){\(x_1\)}}
\put(200,0){\vector(1,3){48.7}}
\put(235,75){\makebox(0,0){\(\rho\)}}
\put(251.3,154){\vector(1,3){18}}
\put(282,206){\makebox(0,0){\(x_2\)}}
\bezier{20}(190.5,28.5)(200,31.7)(209.5,28.5)
\put(200,20){\makebox(0,0){\(\alpha\)}}
\put(200,158.8){\circle*{4}}
\put(200,158.8){\vector(0,1){50}}
\put(210,200){\makebox(0,0){\(r\)}}
\put(151,154){\line(1,4){10}}
\put(249,154){\line(-1,4){10}}
% magnet outline
\thicklines
\bezier{100}(154,151.3)(200,166.7)(246,151.3)
\put(162,154){\vector(-3,-1){8}}
\put(238,154){\vector(3,-1){8}}
\put(210,168){\makebox(0,0){L}}
\color{blue}
\put(151,154){\line(1,4){6}}
\put(149,146){\line(-1,-4){6}}
\put(249,154){\line(-1,4){6}}
\put(251,146){\line(1,-4){6}}
\bezier{90}(157,178)(200,188.4)(243,178)
\bezier{110}(143,122)(200,148.6)(257,122)
\color{black}
\bezier{20}(137.4,187.9)(149.1,191.5)(159.7,188.8)
\put(153.7,190.8){\vector(3,-1){6}}
\put(150,180){\makebox(0,0){\(e_1\)}}
\bezier{20}(262.6,187.9)(250.9,191.5)(240.3,188.8)
\put(246.3,190.8){\vector(-3,-1){6}}
\put(250,180){\makebox(0,0){\(e_2\)}}
\end{picture}
\caption[Reference System for a Sector Bending Magnet]%
{Reference System for a Sector Bending Magnet;
the signs of pole-face rotations are positive as shown.}
\label{F-SBND}
\end{figure}

\section{Sign Conventions for Magnetic Fields}
\label{sec:sign-convention}
The \madx program uses the following Taylor expansion for the field on the
mid-plane $y=0$, described in \cite{slac75} (Note the factorial in the
denominator): 
\begin{equation}
B_y(x,0)=\sum_{n=0}^{\infty} \frac{B_n\,x^n}{n!}
\end{equation}

The field coefficients have the following meaning: 
\begin{madlist}
   \item[$B_0$] 
     Dipole field, with a positive value in the
     positive $y$ direction; a positive field bends a positively
     charged particle to the right.  
   \item[$B_1$] 
     Quadrupole coefficient
     \( B_1 = ( \partial B_y / \partial x ) \);
     a positive value corresponds to horizontal focussing of a
     positively charged particle. 
   \item[$B_2$] 
     Sextupole coefficient
     \( B_2 =  ( \partial^2 B_y / \partial x^2 ) \). 
   \item[$B_3$] 
     Octupole coefficient
     \( B_3 =  ( \partial^3 B_y / \partial x^3 ) \). 
   \item[\ldots] etc.
\end{madlist} 

Using this expansion and the curvature \textit{h} of the reference
orbit, the longitudinal component of the vector potential to order 4 is:  
\begin{equation}
  \begin{aligned}
    A_s = &+ B_0\,\Big(x-\frac{hx^2}{2(1+hx)}\Big) \\
    &+ B_1\,\Big(\frac{1}{2}(x^2-y^2) - \frac{h}{6}x^3 + \frac{h^2}{24}(4x^4-y^4)+\cdots\Big) \\
    &+ B_2\,\Big(\frac{1}{6}(x^3-3xy^2) - \frac{h}{24}(x^4-y^4)+\cdots\Big) \\
    &+ B_3\,\Big(\frac{1}{24}(x^4-6x^2y^2+y^4) \cdots \Big) \\
    &+\cdots
  \end{aligned}
\end{equation}
Taking \(\vec{B} = \nabla \times \vec{A}\) in curvilinear coordinates,
the field components can be computed as  
\begin{equation}\label{eq:field-components}
  \begin{aligned}
B_x(x,y) = &+ B_1\,\Big(y+\frac{h^2}{6}y^3+\cdots\Big) \\
           &+ B_2\,\Big(xy - \frac{h}{6}y^3+\cdots \Big) \\
           &+ B_3\,\Big(\frac{1}{6}(3x^2y-y^3)+ \cdots \Big)\\
           &+\cdots\\
& \\
B_y(x,y)=  &+ B_0   \\
           &+ B_1\,\Big(x-\frac{h}{2}y^2+\frac{h^2}{2}xy^2+\cdots \Big)\\
           &+ B_2\,\Big(\frac{1}{2}(x^2-y^2)-\frac{h}{2}xy^2+\cdots \Big)\\
           &+ B_3\,\Big(\frac{1}{6}(x^3-3xy^2)+ \cdots \Big)\\
           &+\cdots
  \end{aligned}
\end{equation}

It can be easily verified that both \(\nabla \times \vec{B}\)
and \(\nabla . \vec{B}\) are zero to the order of the
\(B_3\) term.  

Introducing the magnetic rigidity \(B \rho = p_s / q\) as the
momentum of the particle divided by its charge, the multipole
coefficients are computed as
\begin{equation}\label{eq:kn}
K_n = q B_n / p_s  =  B_n / B \rho 
\end{equation}

\section{Generalisation to normal and skew components}
\label{sec:normalskew}
The previous section assumed an expansion at the mid-plane ($y=0$), 
symmetry around the mid-plane and considered only the vertical 
component of the field. 

An extension using complex notation for the position ($x + i y$) 
and the field is given as

\begin{equation}
B_y +  i B_x =\sum_{n=0}^{\infty} (b_n\,+ia_n) \frac{(x+iy)^n}{n^{n-1}}
\end{equation}

By introducing the normal and skew multipole coefficients 
$KN$ and $KS$ at order $n$ as
\begin{equation}\label{eq:knn}
KN_n = q\,b_n / p_s  =  b_n / B \rho 
\end{equation}
and
\begin{equation}\label{eq:kns}
KS_n = q\,a_n / p_s  =  a_n / B \rho 
\end{equation}
the kicks received in each plane can be expressed as the summation 
over all components
\begin{equation}
\Delta P_x - i \Delta P_y = \sum_{n=0}^{\infty} -(KN_n + i KS_n) \frac{(x+iy)^n}{n!}
\end{equation}

\textbf{Remark:} need to add references to the $(a_n,b_n)$ field conventions 
in the magnet world. 


\section{Variables}
\label{sec:variables}

For each variable listed in this section, the physical units are given
between square brackets, where [1] denotes a dimensionless variable.

\subsection{Canonical Variables Describing Orbits}
\label{subsec:tables-canon}
\madx uses the following canonical variables to describe the motion of particles: 

\begin{madlist}
	\ttitem{X} Horizontal position $x$ of the (closed) orbit,
	referred to the ideal orbit [m].    
	\ttitem{PX} Horizontal canonical momentum $p_x$ of the
	(closed) orbit referred to the ideal orbit, divided by the
	reference momentum: $\textrm{PX} = p_x / p_0$, \cite{slac75}.   
	\ttitem{Y} Vertical position $y$ of the (closed) orbit, referred
	to the ideal orbit [m].   
	\ttitem{PY} Vertical canonical momentum $p_y$ of the (closed)
	orbit referred to the ideal orbit, divided by the reference
	momentum: $\textrm{PY} = p_y / p_0$, \cite{slac75}.   
	\ttitem{T} Velocity of light times the negative time difference with
	respect to the reference particle: $\textrm{T} =  -  c\Delta t$, [m]. A
	positive T means that the particle arrives ahead of the reference
	particle.   
	\ttitem{PT} Energy error, divided by the reference momentum times the
	velocity of light: $\textrm{PT} = \Delta E / p_s c$ where $p_s = p_0 (1+$DELTAP), \cite{slac75}. 
	This value is only non-zero when synchrotron motion is
	present. It describes the deviation of the particle from the orbit
	of a particle with the momentum error DELTAP.   
	\ttitem{DELTAP} \textbf{Difference between the reference momentum and the design
	momentum, divided by the design momentum: DELTAP =
	$\Delta p / p_0$, \cite{slac75}.} This quantity is used to
	\hyperref[chap:differences]{normalize} all element strengths.   
\end{madlist} 

The independent variable is: 
\begin{madlist}
  \ttitem{S} \index{arc length} Arc length \textit{s} along the
  reference orbit, [m].    
\end{madlist} 

In the limit of fully relativistic particles ($\gamma \gg 1$, $v = c$,
$p c = E$), the variables T and PT used here agree with the
longitudinal variables used in \cite{TRANSPORT}. This means that T
becomes the negative path length difference, while PT becomes the
fractional momentum error. The reference momentum $p_s$ must be
constant in order to keep the system canonical.  

\subsection{Normalised Variables and other Derived Quantities}
\label{subsec:tables-normal}
\begin{madlist}
  \ttitem{XN} The normalised horizontal displacement, [sqrt(m)]\\
  $x_n = Re ( E_1^T \, S\, Z )$
  \ttitem{PXN} The normalised horizontal transverse momentum, [sqrt(m)]\\
  $p_{xn} = Im ( E_1^T\, S\, Z )$
  \ttitem{WX} The horizontal Courant-Snyder invariant, [m]\\
  WX = $x_n^2 + p_{xn}^2$
  \ttitem{PHIX} The horizontal phase, [1]\\
  $\phi_x = -\arctan ( p_{xn} / x_n ) / 2 \pi$
  \ttitem{YN} The normalised vertical displacement, [sqrt(m)]\\
  $y_n = Re ( E_2^T \,S\, Z )$
  \ttitem{PYN} The normalised vertical transverse momentum, [sqrt(m)]\\
  $p_{yn} = Im ( E_2^T\, S\, Z )$
  \ttitem{WY} The vertical Courant-Snyder invariant, [m]\\
  WY = $y_n^2 + p_{yn}^2$
  \ttitem{PHIY} The vertical phase, [1]\\
  $\phi_y = -\arctan ( p_{yn} / y_n ) / 2 \pi$
  \ttitem{TN} The normalised longitudinal displacement, [sqrt(m)]\\
  $t_n = Re ( E_3^T \,S\, Z )$
  \ttitem{PTN} The normalised longitudinal transverse momentum, [sqrt(m)]\\
  $p_{tn} = Im ( E_3^T\, S\, Z )$
  \ttitem{WT} The longitudinal invariant, [m]\\
  WT = $t_n^2 + p_{tn}^2$
  \ttitem{PHIT} The longitudinal phase, [1]\\
  $\phi_t = - \arctan ( p_{tn} / t_n ) / 2 \pi$
\end{madlist} 

In the above formulas the vectors \(E_i\) are the three complex
eigenvectors, \(Z\) is the phase space vector, and the matrix \textit{S}
is the ``symplectic unit matrix'':   

\begin{equation}
Z = \left(
\begin{array}{l} x \\ p_x \\ y \\ p_y \\ t \\ p_t
\end{array} \right), \quad
S =
 \begin{pmatrix}
  0 & 1 & 0 & 0 & 0 & 0 \\
  -1 & 0 & 0 & 0 & 0 & 0 \\
  0 & 0 & 0 & 1 & 0 & 0 \\
  0 & 0 & -1 & 0 & 0 & 0 \\
  0 & 0 & 0 & 0 & 0 & 1 \\
  0 & 0 & 0 & 0 & -1 & 0 \\
 \end{pmatrix}
\end{equation}


\subsection{Linear Lattice Functions (Optical Functions)}
\label{subsec:tables-linear}
Several \madx commands refer to linear lattice functions or optical
functions.  

Because \madx uses the canonical momenta ($p_x$, $p_y$) instead of the
slopes ($x'$, $y'$), the definitions of the linear lattice functions
differ slightly from those in Courant and Snyder\cite{Courant-Snyder1958}.

Notice that in \madx, PT substitutes DELTAP as longitudinal
variable. 
Dispersive and chromatic functions are hence derivatives with
respect to PT. 
And since PT=BETA*DELTAP, where BETA is the relativistic Lorentz 
factor, those functions given by \madx must be multiplied by BETA a
number of time equal to the order of the derivative to find the
functions given in the literature. 

The linear lattice functions are known to \madx under the following names:
\begin{madlist}
  \ttitem{BETX} Amplitude function $\beta_x$, [m].   
  \ttitem{ALFX} Correlation function 
  $\alpha_x = - \frac{1}{2} (\partial \beta_x / \partial s)$, [1]   
  \ttitem{MUX} Phase function $\mu_x = \int ds / \beta_x$, [$2 \pi$]
  \ttitem{DX} Dispersion of $x$: $D_x = (\partial x / \partial p_t)$, [m] 
  \ttitem{DPX} Dispersion of $p_x$: $D_{px} = (\partial p_x / \partial p_t) / p_s$, [1] 
  \ttitem{BETY} Amplitude function $\beta_y$, [m]   
  \ttitem{ALFY} Correlation function 
  $\alpha_y = - \frac{1}{2} ( \partial \beta_y / \partial s)$, [1] 
  \ttitem{MUY} Phase function $\mu_y = \int ds / \beta_y$, [$2 \pi$]
  \ttitem{DY} Dispersion of $y$: $D_y = (\partial y / \partial p_t)$, [m] 
  \ttitem{DPY} Dispersion of $p_y$: $D_{py} = ( \partial p_y / \partial p_t) / p_s$, [1] 
  \ttitem{R11, R12, R21, R22} : Coupling Matrix     
\end{madlist}

%  The TWISS table also defines the following expressions which 
%  can be used in plots:
% \begin{itemize}
%   \item  GAMX = (1 + ALFX*ALFX) / BETX, 
%   \item  GAMY = (1 + ALFY*ALFY) / BETY, 
%   \item  SIGX = SQRT(BETX * EX), the vertical r.m.s. half-width of the beam, 
%   \item  SIGY = SQRT(BETY * EY), the vertical r.m.s. half-height of the beam. 
% \end{itemize}


\subsection{Chromatic Functions} 
\label{subsec:tables-chrom}
Several \madx commands refer to the chromatic functions. 

Because \madx uses the canonical momenta ($p_x$, $p_y$) instead of the
slopes ($x'$, $y'$), the definitions of the chromatic functions differ
slightly from those in \cite{Montague1979}.

Notice also that in \madx, PT substitutes DELTAP as longitudinal
variable. Dispersive and chromatic functions are hence derivatives with
respect to PT. 
Since PT=BETA*DELTAP, where BETA is the relativistic Lorentz 
factor, those functions given by \madx must be multiplied by BETA a
number of times equal to the order of the derivative to find the
functions given in the literature. 

The chromatic functions are known to \madx under the following names:

\begin{madlist}
  \ttitem{WX} Chromatic amplitude function $W_x = \sqrt{a_x^2 + b_x^2}$ ,
         [1], where \\
         \[
         b_x = \frac{1}{\beta_x} \frac{\partial \beta_x}{\partial p_t}  ,\qquad
         a_x = \frac{\partial \alpha_x}{\partial p_t} -
         \frac{\alpha_x}{\beta_x}\frac{\partial \beta_x}{\partial p_t}
         \]
         %% the equation used to be the following, which is in contradiction
         %% with MAD8 users' guide and Montague (LEP Note 165) where a and b
         %% are exactly inverted. 
         %% \[
         %% a_x = \frac{1}{\beta_x} \frac{\partial \beta_x}{\partial p_t}  ,\qquad
         %% b_x = \frac{\partial \alpha_x}{\partial p_t} -
         %% \frac{\alpha_x}{\beta_x}\frac{\partial \beta_x}{\partial p_t}
         %% \]
  \ttitem{PHIX} Chromatic phase function $\Phi_x = \arctan (a_x / b_x)$, [$2 \pi$] 
  \ttitem{DMUX} Chromatic derivative of phase function: 
  $DMUX = (\partial \mu_x / \partial p_t)$,  [$2 \pi$]
  \ttitem{DDX} Chromatic derivative of dispersion $D_x$ :  
  $DDX = \frac{1}{2} (\partial^2 x / \partial p_t^2)$, [m]     
  \ttitem{DDPX} Chromatic derivative of dispersion $D_{px}$ : 
  $DDPX = \frac{1}{2} ( \partial^2 p_x / \partial p_t^2 ) / p_s $, [1]
  \ttitem{WY} Chromatic amplitude function $W_y = \sqrt{a_y^2 + b_y^2}$ ,
         [1], where \\
         \[
         b_y = \frac{1}{\beta_y} \frac{\partial \beta_y}{\partial p_t} ,\qquad
         a_y = \frac{\partial \alpha_y}{\partial p_t} -
         \frac{\alpha_y}{\beta_y}\frac{\partial \beta_y}{\partial p_t}
         \]
  \ttitem{PHIY} Chromatic phase function $\Phi_y = \arctan (a_y / b_y)$, [$2 \pi$]
  \ttitem{DMUY} Chromatic derivative of phase function:
  $DMUY = (\partial \mu_y / \partial p_t)$,  [$2 \pi$] 
  \ttitem{DDY} Chromatic derivative of dispersion $D_y$ : 
  $DDY = \frac{1}{2} (\partial^2 y / \partial p_t^2)$, [m]      
  \ttitem{DDPY} Chromatic derivative of dispersion $D_{py}$ : 
  $DDPY = \frac{1}{2} ( \partial^2 p_y / \partial p_t^2 ) / p_s $, [1] 
\end{madlist}

\subsection{Variables in the SUMM Table}
\label{subsec:tables-summ}
After a successful TWISS command a summary table, with name SUMM, is created which
contains the following variables:  

\begin{madlist}
	\ttitem{LENGTH} The length of the machine, [m].     
	\ttitem{ORBIT5} The (ORBIT5 = $-$T $= c \Delta t$, [m]) component of the closed orbit.     
	\ttitem{ALFA} The momentum compaction factor $\alpha_c$, [1].     
	\ttitem{GAMMATR} The transition energy $\gamma_{tr}$, [1].     
	\ttitem{Q1} The horizontal tune $Q_1$ [1].     
	\ttitem{DQ1} The horizontal chromaticity $dq_1 = \partial Q_1 / \partial p_t$, [1]
	\ttitem{BETXMAX} The largest horizontal $\beta_x$, [m].     
	\ttitem{DXMAX} The maximum of the absolute horizontal dispersion $D_x$, [m].     
	\ttitem{DXRMS} The r.m.s. of the horizontal dispersion $D_x$, [m].     
	\ttitem{XCOMAX} The maximum of the absolute horizontal closed orbit deviation [m].     
	\ttitem{XRMS} The r.m.s. of the horizontal closed orbit deviation [m].     
	\ttitem{Q2} The vertical tune $Q_2$ [1].     
	\ttitem{DQ2} The vertical chromaticity $dq_2 = \partial Q_2 / \partial p_t$, [1]
	\ttitem{BETYMAX} The largest vertical $\beta_y$, [m].     
	\ttitem{DYMAX} The maximum of the absolute vertical dispersion $D_y$, [m].     
	\ttitem{DYRMS} The r.m.s. of the vertical dispersion $D_y$, [m].     
	\ttitem{YCOMAX} The maximum of the absolute vertical closed orbit deviation [m].     
	\ttitem{YCORMS} The r.m.s. of the vertical closed orbit deviation [m].     
	\ttitem{DELTAP} \textbf{Momentum difference, divided by the reference
	momentum [1]. \\ DELTAP = $\Delta p / p_0$}
	\ttitem{SYNCH\_1} First synchrotron radiation integral  
	\ttitem{SYNCH\_2} Second synchrotron radiation integral  
	\ttitem{SYNCH\_3} Third synchrotron radiation integral  
	\ttitem{SYNCH\_4} Fourth synchrotron radiation integral  
	\ttitem{SYNCH\_5} Fifth synchrotron radiation integral  
\end{madlist}

Notice that in \madx, PT substitutes DELTAP as longitudinal
variable. Dispersive and chromatic functions are hence derivatives with
respect to PT.
And since PT=BETA*DELTAP, where BETA is the relativistic Lorentz 
factor, those functions given by \madx must be multiplied by BETA a
number of time equal to the order of the derivative to find the
functions given in the literature. 

\subsection{Variables in the TRACK Table}
\label{subsec:tables-track}
The command RUN writes tables with the following variables: 
\begin{madlist}
  \ttitem{X} Horizontal position $x$ of the orbit, referred to the
  ideal orbit [m].    
  \ttitem{PX} Horizontal canonical momentum $p_x$ of the orbit
  referred to the ideal orbit, divided by the reference momentum.    
  \ttitem{Y} Vertical position $y$ of the orbit, referred to the
  ideal orbit [m].    
  \ttitem{PY} Vertical canonical momentum $p_y$ of the orbit
  referred to the ideal orbit, divided by the reference momentum.    
  \ttitem{T} Velocity of light times the negative time difference with
  respect to the reference particle, $\mathtt{T}=-c\Delta t$, [m]. 
  A positive T means that the particle arrives ahead of the reference
  particle.    
  \ttitem{PT} Energy difference, divided by the reference momentum times
  the velocity of light, [1].
\end{madlist} 

When tracking Lyapunov companions, the TRACK table defines the following
dependent expressions:  
\begin{madlist}
  \ttitem{DISTANCE} the relative Lyapunov distance between the two particles.    
  \ttitem{LYAPUNOV} the estimated Lyapunov Exponent.   
  \ttitem{LOGDIST} the natural logarithm of the relative distance.   
  \ttitem{LOGTURNS} the natural logarithm of the turn number.   
\end{madlist}





\section{Physical Units}
\label{sec:units}
\madx uses units derived from the ``Syst\`eme
International'' (SI). These units are summarised in the
\hyperlink{table}{Units table}.  

\begin{table}[ht]
  \caption{Physical Units used by \madx}
  \vspace{1ex}
  \centering  
  \index{length}
  \index{angle}
  \index{quadrupole}
  \index{multipole}
  \index{voltage}
  \index{field}
  \index{electric field}
  \index{frequency}
  \index{RF}
  \index{energy}
  \index{mass}
  \index{momentum}
  \index{current}
  \index{charge}
  \index{impedance}
  \index{emittance}
  \index{power}
  \index{high order modes} 
  \begin{tabular}{|l|l|}
    \hline
    \textbf{Quantity}       & \textbf{Unit} \\
    \hline
    Length                  & m (metres) \\ 
    Angle                   & rad (radians) \\ 
    Quadrupole coefficient  & m$^{\tt{-2}}$ \\ 
    Multipole coefficient, 2n poles   & m$^{\tt{-n}}$ \\ 
    Electric voltage        & MV (megavolts) \\ 
    Electric field strength & MV/m \\ 
    Frequency               & MHz (megahertz) \\ 
    Phase angles            & $2\pi$ \\ 
    Particle energy         & GeV \\ 
    Particle mass           & GeV/$c^2$ \\ 
    Particle momentum       & GeV/$c$ \\ 
    Beam current            & A (amp\`eres) \\ 
    Particle charge         & e (elementary charges) \\ 
    Impedance               & M$\Omega$ (Megohms) \\ 
    Emittance               & $\pi * 10^{-3}$ m.rad \\ % To be checked (Ghislain)
    RF power                & MW (megawatts) \\ 
    Higher order mode loss factor & V/pc \\
    \hline
  \end{tabular}
\end{table}

%% End of Chapter 'Conventions'
