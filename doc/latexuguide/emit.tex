%%\title{EMIT}
%  Changed by: Chris ISELIN, 27-Jan-1997 
%  Changed by: Hans Grote, 15-Oct-2002 
%  Changed by: Ralph Assmann, 02-Sep-2003 

\chapter{EMIT: Equilibrium emittances} 
\label{chap:emit}
%Fully Coupled Motion and Radiation}

\texttt{EMIT} calculates the equilibrium emittances: 

\madbox{
EMIT, DELTAP=real, TOL=real;
}\vspace{5mm}

The attributes for \texttt{EMIT} command are: 

\begin{madlist}
  \ttitem{DELTAP} specified average energy error. \\
  EMIT adjusts the RF frequencies such as to obtain this
  average energy error. More precisely, the revolution frequency $f_0$
  is determined for a fictitious particle with constant momentum error  
  %DELTAP = delta$_\textit{s}$ = delta(\textit{E}) / \textit{p$_s$ c}
  DELTAP = $\delta_s = \delta(E) / p_s c$ which travels along the design
  orbit. The RF frequencies are then set to   
  %\textit{f$_{RF}$ = h f$_0$}. 
  $f_{RF} = h f_0$. 

  \ttitem{TOL} The tolerance attribute is for the distinction between
  static and dynamic cases: \\
  if for the eigenvalues of the one-turn matrix, $|$e\_val\_5$|$ \textless
  tol and $|$e\_val\_6$|$ \textless tol, then the longitudinal motion is
  not considered, otherwise it is. \\
  (Default:~1.000001)  
\end{madlist}


If the machine contains at least one RF cavity, and if synchrotron
radiation is \href{../Introduction/beam.html#radiate}{ON}, the
\texttt{EMIT} command computes the equilibrium emittances and other
electron beam parameters using the method of 
\href{../Introduction/bibliography.html#chao}{[Chao]}. 

In this calculation the effects of quadrupoles, sextupoles, and
octupoles along the closed orbit is also considered. Thin multipoles are
used only if they have a fictitious length \texttt{LRAD} different from zero.  

If the machine contains no RF cavity, if synchrotron radiation is
\href{../Introduction/beam.html#radiate}{OFF}, or if the longitudinal
motion is not stable, \texttt{EMIT} only computes the parameters that
are not related to radiation.  

In the current implementation, the BEAM values of the emittances and
beam sizes are only updated for deltap = zero.  

{\bf Example:}\\ 
\madxmp{
RFC: RFCAVITY, HARMON..., VOLT=...; \\
\ldots \\
BEAM, ENERGY = 100.0, RADIATE; \\
EMIT, DELTAP = 0.01;
}

{\bf Remark:}\\
 This module assumes nearly constant lattice functions
inside elements. This assumption works for many machines, like LEP
(\href{http://cern.ch/frs/mad-X_examples/emit/LEP/}{see example}), but
it fails when the lattice functions largely vary inside single
elements. In the later case it is advised to slice the elements as shown
in the example pertaining to
\href{http://cern.ch/frs/mad-X_examples/emit/ALBA/}{ALBA}.     

% \href{Rogelio HREF=http://consult.cern.ch/xwho/people/69118}{R.Tom\'as} 
% \textbf{Last updated:} 03/13/2013 13:47:20

