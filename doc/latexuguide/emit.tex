%%\title{EMIT}
%  Changed by: Chris ISELIN, 27-Jan-1997 
%  Changed by: Hans Grote, 15-Oct-2002 
%  Changed by: Ralph Assmann, 02-Sep-2003 

\chapter{EMIT: Equilibrium emittances} 
\label{chap:emit}
%Fully Coupled Motion and Radiation}

\texttt{EMIT} calculates the equilibrium emittances: 

\madbox{
EMIT, DELTAP=real, TOL=real;
}\vspace{5mm}

The attributes for the \texttt{EMIT} command are: 

\begin{madlist}
  \ttitem{DELTAP} the average energy error. \\
  {\tt EMIT} adjusts the RF frequencies in order to obtain this
  average energy error: the revolution frequency $f_0$
  is determined for a fictitious particle with constant momentum error  
  %DELTAP = delta$_\textit{s}$ = delta(\textit{E}) / \textit{p$_s$ c}
  {\tt DELTAP} = $\delta_s = \delta(E) / p_s c$ travelling along the design
  orbit. The RF frequencies are then set to   
  %\textit{f$_{RF}$ = h f$_0$}. 
  $f_{RF} = h f_0$. 

  \ttitem{TOL} The tolerance attribute is for the distinction between
  static and dynamic cases: \\
  if for the eigenvalues of the one-turn matrix, $|$e\_val\_5$|$ \textless
  tol and $|$e\_val\_6$|$ \textless tol, then the longitudinal motion is
  not considered, otherwise it is. \\
  (Default:~1.000001)  
\end{madlist}


If the machine contains at least one RF cavity, and if synchrotron
radiation is enabled with \hyperref[sec:beam]{\tt BEAM, RADIATE=true;}, the
{\tt EMIT} command computes the equilibrium emittances and other
electron beam parameters using the method in \cite{chao1979}.

In this calculation the effects of quadrupoles, sextupoles and
octupoles along the closed orbit are also considered. Thin multipoles are
used only if they have a fictitious length {\tt LRAD} different from zero.  

If the machine does not contain any RF cavity, if synchrotron radiation is
turned off (\hyperref[sec:beam]{\tt BEAM, RADIATE=false;}, or if the longitudinal
motion is not stable, {\tt EMIT} only computes the parameters that
are not related to radiation and does not update the {\tt BEAM} values.

If synchrotron radiation is enabled (\hyperref[sec:beam]{\tt BEAM, RADIATE=true;}, 
and the {\tt DELTAP} attribute is zero, and the longitudinal motion is stable, 
{\tt EMIT} calculates and updates the following values for the {\tt BEAM} attached 
to the current sequence: both geometric and normalized transverse emittances, 
longitudinal emittance and beam sizes ($\sigma_E$ and $\sigma_t$), 
damping partition numbers, energy loss per turn and synchrotron tune.


{\bf Example:}
\madxmp{
RFC: RFCAVITY, HARMON..., VOLT=...; \\
\ldots \\
BEAM, ENERGY = 100.0, RADIATE=true; \\
EMIT, DELTAP = 0.01;
}

{\bf Remark:}\\
This module assumes nearly constant lattice functions
inside elements. This assumption works for many machines, like LEP
(\href{http://cern.ch/frs/mad-X_examples/emit/LEP/}{see example}), but
it fails when the lattice functions largely vary inside single
elements. In the later case it is advised to slice the elements as shown
in the example pertaining to
\href{http://cern.ch/frs/mad-X_examples/emit/ALBA/}{ALBA}.     

% \href{Rogelio HREF=http://consult.cern.ch/xwho/people/69118}{R.Tom\'as} 
% \textbf{Last updated:} 03/13/2013 13:47:20

