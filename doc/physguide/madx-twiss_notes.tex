\documentclass{cern-art} % Specifies the document style.
%
\usepackage{hyperref}
\usepackage{xspace}

\usepackage{vmargin,times,graphicx,amsmath,amssymb,color} % ,draftcopy use draftcopy for experiments
\usepackage{verbatim} % to allow for verbatim and comment
\usepackage{here}
\usepackage{wrapfig}
\usepackage{fancyref}
\usepackage{listings}

\usepackage{color}
\definecolor{grey}{rgb}{0.4,0.4,0.4}
\definecolor{darkgreen}{rgb}{0,0.4,0}
\definecolor{comment}{rgb}{0.1,0.50,0.56}
\definecolor{strings}{rgb}{0.25,0.44,0.63}

% programming language
\lstset{ basicstyle=\footnotesize\ttfamily, % Standardschrift
         %numbers=left,               % Line numbers..
         numberstyle=\color{grey}\tiny,          % Numbers style..
         %stepnumber=2,               % Distance between line numbers
         %numbersep=5pt,              % Distance from line number to text
         tabsize=2,                  % Tabs size
         extendedchars=true,         %
         breaklines=true,            % Break long lines
         keywordstyle=\color{darkgreen}\bfseries,
         stringstyle=\color{strings}\ttfamily, % String color
         commentstyle=\it\color{comment}\ttfamily,
%          showspaces=false,           % Show spaces
%          showtabs=false,             % Show tabs
         xleftmargin=0pt,
         framexleftmargin=17pt,
         framexrightmargin=5pt,
         framexbottommargin=4pt,
         showstringspaces=false
}

\everymath{\displaystyle}

\lstset{language=Fortran}
\lstset{keywordstyle=\color{blue}\bfseries}
\lstset{numbers=left,stepnumber=2}
\lstset{commentstyle=\color{red}\bfseries}

\setmarginsrb{15mm}{8mm}{15mm}{10mm}{12pt}{10mm}{0pt}{10mm}

% shortcut
\renewcommand{\L}[1]{\lstinline[firstnumber=last]{#1}}
\newcommand{\T}[1]{{\tt #1}}
\newcommand{\madx}{\mbox{MAD-X}\xspace}
\newcommand{\madeight}{\mbox{MAD8}\xspace}

%%%%%%%%%%%%%%%%%%%%%%%%%%%%%%%%%%%%%%%%%%%%%%%%%%%%%%%%%%%%%%%%%%%%%%%%
%%%%%%%%%%%%%%%%%%%%%%%%%%%%%%%%%%%%%%%%%%%%%%%%%%%%%%%%%%%%%%%%%%%%%%%%
\begin{document}

\PSLogo{fig/cern-logo.pdf}
\DocReference{CERN-ACC-NOTE-2014-XXXX}

\Date{April 2014}
% \RevisionDate{April 15, 2014}

\Author{L. Deniau and A. Latina}
\Institute{CERN -- BE/ABP}
\Email{\{laurent.deniau, andrea.latina\}@cern.ch}

\Title{Ndiff Reference Manual}
\Title{MAD-X Technical Notes on Twiss Module}

\Keywords{\madx, beam dynamics, thick tracking, optical functions.}
%\Distribution{ABP group}

\Maketitle

\Summary{%
	This technical note describes in details the transport maps used by the Twiss module of \madx, including the commented code. It also proposes some alternatives formulations for potential future implementations.
}

%%%%%%%%%%%%%%%%%%%%%%%%%%%%%%%%%
\cleardoublepage
\tableofcontents

%%%%%%%%%%%%%%%%%%%%%%%%%%%%%%%%%
\cleardoublepage
\section{Motivation}

The implementation of the Twiss module was inherited from \madeight and adapted to the LHC studies during the crash program for the development of its successor \madx. The root of this code goes back into the 80's and some approximations and adaptations were never documented. Hence, the purpose of this work is to recover the physics of this module using reverse engineer, and to document the approximations and assumptions made by the code. It will be also the opportunity to clarify some features and inconsistencies, and potentially correct some known bugs.

%%%%%%%%%%%%%%%%%%%%%%%%%%%%%%%%%
\section{Bending magnets}

\madx does not support the so called ``true \T{RBEND}'' element. This kind of magnet is internally represented as a sector bend (i.e. \T{SBEND}) with the pole faces made parallel by setting the pole face angles to $\T{e1}=\T{e2}=\alpha/2$, where $\alpha$ is the \T{SBEND} angle.


\subsection{The routine \L{tmbend}}
\subsubsection{Preamble}
Declarations and preamble:
\begin{lstlisting}[firstnumber=auto]
SUBROUTINE tmbend(ftrk,orbit,fmap,el,ek,re,te)

  use twtrrfi
  use twisslfi
  use twiss_elpfi
  implicit none

  !----------------------------------------------------------------------*
  !     Purpose:                                                         *
  !     TRANSPORT map for sector bending magnets                         *
  !     Input:                                                           *
  !     ftrk      (logical) if true, track orbit.                        *
  !     Input/output:                                                    *
  !     orbit(6)  (double)  closed orbit.                                *
  !     Output:                                                          *
  !     fmap      (logical) if true, element has a map.                  *
  !     el        (double)  element length.                              *
  !     ek(6)     (double)  kick due to element.                         *
  !     re(6,6)   (double)  transfer matrix.                             *
  !     te(6,6,6) (double)  second-order terms.                          *
  !----------------------------------------------------------------------*
  logical ftrk,fmap,cplxy,dorad
  integer nd,n_ferr,node_fd_errors,code
  double precision orbit(6),f_errors(0:maxferr),ek(6),re(6,6),           &
       te(6,6,6),rw(6,6),tw(6,6,6),x,y,deltap,field(2,0:maxmul),fintx,   &
       el,tilt,e1,e2,sk1,sk2,h1,h2,hgap,fint,sks,an,h,dh,corr,ek0(6),ct, &
       st,hx,hy,rfac,arad,gamma,pt,rhoinv,blen,node_value,get_value,bvk, &
       el0,orbit0(6)
  double precision orbit00(6),ek00(6),re00(6,6),te00(6,6,6)
  integer, external :: el_par_vector ! function from the C core
  integer elpar_vl
\end{lstlisting}
The following statements retrieve the \madeight elements \L{code} and change them from \L{tkicker} (39) to \L{kicker} (15), and from \L{placeholder} (38) to \L{instrument} (24), and initializes the matrix \L{rw} to the identity, and the tensor \L{tw} and the vector \L{ek0} to zero.
\begin{lstlisting}[firstnumber=last]
  !---- Initialize.
  ct=0
  st=0
  code = node_value('mad8_type ') ! function from the C core
  if(code.eq.39) code=15
  if(code.eq.38) code=24
  deltap=0
  call dzero(ek0,6)
  call m66one(rw)
  call dzero(tw,216)
\end{lstlisting}
If the element has nonzero length the logical variable \L{fmap} is set to \L{true}.
If the element has zero length it simply jumps to the end. [{\em Suggestion: to replace by \L{if (el.eq.0) return}}.]

\begin{lstlisting}[firstnumber=last]
  !---- Test for non-zero length.
  fmap = el.ne.0
  if (fmap) then
  \end{lstlisting}
TODO
It reads the bend parameters from the command line and some dynamic variables from \L{probe}:
  \begin{lstlisting}[firstnumber=last]
     call dzero(f_errors,maxferr+1)
     n_ferr = node_fd_errors(f_errors)
     !-- get element parameters
     elpar_vl = el_par_vector(b_k3s, g_elpar) ! get strengths up to k3s
     bvk = node_value('other_bv ')
     arad = get_value('probe ','arad ')
     deltap = get_value('probe ','deltap ')
     gamma = get_value('probe ','gamma ')
     dorad = get_value('probe ','radiate ') .ne. 0
     an = bvk * g_elpar(b_angle)
     tilt = g_elpar(b_tilt)
     e1 = g_elpar(b_e1)
     e2 = g_elpar(b_e2)

     if(code.eq.2) then
        e1 = e1 + an / 2
        e2 = e2 + an / 2
     endif

     !---  bvk also applied further down

     sk1 = g_elpar(b_k1)
     sk2 = g_elpar(b_k2)
     h1 = g_elpar(b_h1)
     h2 = g_elpar(b_h2)
     hgap = g_elpar(b_hgap)
     fint = g_elpar(b_fint)
     fintx = g_elpar(b_fintx)
     sks = g_elpar(b_k1s)
     h = an / el
  \end{lstlisting}
\L{h} is $h=1/\rho$. I guess the following \L{dh} is how much does the inverse of the bending radius change for a particle with momentum difference \L{deltap} is w.r.t. the nominal value. The first branch of this \L{if} takes into account the field errors, also.
  \begin{lstlisting}[firstnumber=last]
     !---- Apply field errors and change coefficients using DELTAP.
     if (n_ferr .gt. 0) then
        nd = n_ferr
        call dzero(field,nd)
        call dcopy(f_errors,field,n_ferr)
        dh = (- h * deltap + bvk * field(1,0) / el) / (1 + deltap)
        sk1 = (sk1 + field(1,1) / el) / (1 + deltap)
        sk2 = (sk2 + field(1,2) / el) / (1 + deltap)
        sks = (sks + field(2,1) / el) / (1 + deltap)
     else
        dh = - h * deltap / (1 + deltap)
        sk1 = sk1 / (1 + deltap)
        sk2 = sk2 / (1 + deltap)
        sks = sks / (1 + deltap)
     endif
  \end{lstlisting}
Applies \L{bvk}-flag.
 \begin{lstlisting}[firstnumber=last]
     sk1 = bvk * sk1
     sk2 = bvk * sk2
     sks = bvk * sks
  \end{lstlisting}
  if the flags \L{(ftrk .and. dorad)} are set, then it applies half radiation at the entrance. Also, it applies the tilt rotation.
 \begin{lstlisting}[firstnumber=last]
     !---- Half radiation effects at entrance.
     if (ftrk .and. dorad) then
        ct = cos(tilt)
        st = sin(tilt)
        x =   orbit(1) * ct + orbit(3) * st
        y = - orbit(1) * st + orbit(3) * ct
        hx = h + dh + sk1*(x - h*y**2/2) + sks*y +                  &
             sk2*(x**2 - y**2)/2
        hy = sks * x - sk1*y - sk2*x*y
        rfac = (arad * gamma**3 * el / 3)                         &
             * (hx**2 + hy**2) * (1 + h*x) * (1 - tan(e1)*x)
        pt = orbit(6)
        orbit(2) = orbit(2) - rfac * (1 + pt) * orbit(2)
        orbit(4) = orbit(4) - rfac * (1 + pt) * orbit(4)
        orbit(6) = orbit(6) - rfac * (1 + pt) ** 2
     endif
  \end{lstlisting}


%%%%%%%%%%%%%%%%%%%%%%%%%%%%%%%%%
\subsubsection{Option ``centre''}
If \L{centre} option
   \begin{lstlisting}[firstnumber=last]
     !---- Body of the dipole.
     !---- centre option
     if(centre_cptk.or.centre_bttk) then
        call dcopy(orbit,orbit00,6)
        call dcopy(ek,ek00,6)
        call dcopy(re,re00,36)
        call dcopy(te,te00,216)
  \end{lstlisting}
It halves the length of the element.
   \begin{lstlisting}[firstnumber=last]
        el0=el/2
\end{lstlisting}
It stores the element's \textit{kick}, \textit{transfer matrix}, and \textit{tensor} for such a halved sector bend in \L{ek,re,te}:
\begin{lstlisting}[firstnumber=last]
        call tmsect(.true.,el0,h,dh,sk1,sk2,ek,re,te)
\end{lstlisting}
It stores the $R$ and $T$ for the fringing field in \L{rw} and \L{tw} respectively; then it concatenates the transport maps in such that the fringe fields precede the half-bending magnet. The resulting transport map is stored in   \L{ek},  \L{re}, and \L{te}:
\begin{lstlisting}[firstnumber=last]
        !---- Fringe fields.
        corr = (h + h) * hgap * fint
        call tmfrng(.true.,h,sk1,e1,h1,1,corr,rw,tw)
        call tmcat1(.true.,ek,re,te,ek0,rw,tw,ek,re,te)
\end{lstlisting}
It rotates the map:
\begin{lstlisting}[firstnumber=last]
        !---- Apply tilt.
        if (tilt .ne. 0) then
           call tmtilt(.true.,tilt,ek,re,te)
           cplxy = .true.
        endif
\end{lstlisting}
If the flag \L{ftrk} is set, tracks the orbit through such a halved bend then zeroes \L{ek},  \L{re}, and \L{te}:
\begin{lstlisting}[firstnumber=last]
        !---- Track orbit.
        call dcopy(orbit,orbit0,6)
        if (ftrk) call tmtrak(ek,re,te,orbit0,orbit0)
        if(centre_cptk) call twcptk(re,orbit0)
        if(centre_bttk) call twbttk(re,te)
        call dcopy(orbit00,orbit,6)
        call dcopy(ek00,ek,6)
        call dcopy(re00,re,36)
        call dcopy(te00,te,216)
     endif
     !---- End
  \end{lstlisting}

%%%%%%%%%%%%%%%%%%%%%%%%%%%%%%%%%
\subsubsection{Body of the dipole}
  It fills  \L{ek},  \L{re}, and \L{te} with kick, matrix and tensor for the sector bend. Then fills  \L{rw}, and \L{tw} with the fringing fields and combines the maps such that the body of the bend is sandwiched between the two fringing fields. Notice that if the option ``centre'' is selected, the initial fringing field has  already been applied, thing which not seems to be correct.  
   \begin{lstlisting}[firstnumber=last]
     call tmsect(.true.,el,h,dh,sk1,sk2,ek,re,te)

     !---- Fringe fields.
     corr = (h + h) * hgap * fint
     call tmfrng(.true.,h,sk1,e1,h1,1,corr,rw,tw)
     call tmcat1(.true.,ek,re,te,ek0,rw,tw,ek,re,te)
     !---- Tor: use FINTX if set
     if (fintx .ge. 0) then
        corr = (h + h) * hgap * fintx
     else
        corr = (h + h) * hgap * fint
     endif
     call tmfrng(.true.,h,sk1,e2,h2,-1,corr,rw,tw)
     call tmcat1(.true.,ek0,rw,tw,ek,re,te,ek,re,te)
\end{lstlisting}
It rotates the map:
\begin{lstlisting}[firstnumber=last]
     !---- Apply tilt.
     if (tilt .ne. 0) then
        call tmtilt(.true.,tilt,ek,re,te)
        cplxy = .true.
     endif
\end{lstlisting}
Tracks the orbit:
\begin{lstlisting}[firstnumber=last]
     !---- Track orbit.
     if (ftrk) then
        call tmtrak(ek,re,te,orbit,orbit)
\end{lstlisting}
Radiation effects at the exit:
\begin{lstlisting}[firstnumber=last]
        !---- Half radiation effects at exit.
        if (ftrk .and. dorad) then
           x =   orbit(1) * ct + orbit(3) * st
           y = - orbit(1) * st + orbit(3) * ct
           hx = h + dh + sk1*(x - h*y**2/2) + sks*y +                &
                sk2*(x**2 - y**2)/2
           hy = sks * x - sk1*y - sk2*x*y
           rfac = (arad * gamma**3 * el / 3)                       &
                * (hx**2 + hy**2) * (1 + h*x) * (1 - tan(e2)*x)
           pt = orbit(6)
           orbit(2) = orbit(2) - rfac * (1 + pt) * orbit(2)
           orbit(4) = orbit(4) - rfac * (1 + pt) * orbit(4)
           orbit(6) = orbit(6) - rfac * (1 + pt) ** 2
        endif
     endif
  endif ! if (fmap) line 44

  !---- Tor: set parameters for sychrotron integral calculations
  ! LD: useless code?
  rhoinv = h
  blen = el

end SUBROUTINE tmbend
\end{lstlisting}

%%%%%%%%%%%%%%%%%%%%%%%%%%%%%%%%%
\subsection{The routine {tmsect}}
\L{tmsect(fsec,el,h,dh,sk1,sk2,ek,re,te)} fills \L{ek,re,te} with kick, matrix and tensor for this sector bend.

%%%%%%%%%%%%%%%%%%%%%%%%%%%%%%%%%
%%%%%%%%%%%%%%%%%%%%%%%%%%%%%%%%%
\section{Quadrupole magnets}

%%%%%%%%%%%%%%%%%%%%%%%%%%%%%%%%%
%%%%%%%%%%%%%%%%%%%%%%%%%%%%%%%%%
\section{Multipoles magnets}

%%%%%%%%%%%%%%%%%%%%%%%%%%%%%%%%%
%%%%%%%%%%%%%%%%%%%%%%%%%%%%%%%%%
\section{Solenoide magnets}

%%%%%%%%%%%%%%%%%%%%%%%%%%%%%%%%%
%%%%%%%%%%%%%%%%%%%%%%%%%%%%%%%%%
\section{RF cavities}

%%%%%%%%%%%%%%%%%%%%%%%%%%%%%%%%%
%%%%%%%%%%%%%%%%%%%%%%%%%%%%%%%%%
\appendix
\section{Appendix}
\begin{itemize}
\item[-] \L{dcopy(in,out,n)}
\item[-] \L{tmcat1(fsec,eb,rb,tb,ea,ra,ta,ed,rd,td)}$$\left\{ \begin{array}{c}
e_{d}\\
R_{d}\\
T_{d}
\end{array}\right\} =\left\{ \begin{array}{c}
e_{b}\\
R_{b}\\
T_{b}
\end{array}\right\} \circ \left\{ \begin{array}{c}
e_{a}\\
R_{a}\\
T_{a}
\end{array}\right\} $$
\end{itemize}

\end{document}